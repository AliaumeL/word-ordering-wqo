\newcommand{\klogo}{%
\begin{tikzpicture}[scale=0.2,line/.style={draw, line width=0.2pt, line cap=round, line join=round}]
\coordinate (A00) at (0,0);
\coordinate (A01) at (0,1);
\coordinate (A10) at (1,0);
\coordinate (B10) at (1,0.2);
\coordinate (B01) at (0.2,1);

\coordinate (C01) at (0.4,0.7);
\coordinate (C10) at (0.7,0.4);
\coordinate (C12) at (0.4,1.2);
\coordinate (C21) at (1.2, 0.4);
\coordinate (C22) at (1.2, 1.2);

\coordinate (D00) at (C10);
\coordinate (D01) at (0.8,0.5);
\coordinate (D10) at (0.8,0.3);

\coordinate (E01) at (0.3,0.7);
\coordinate (E10) at (0.5,0.7);

\draw[line] (B01) -- (A01) -- (A00) -- (A10) -- (B10);
\draw[line] (C01) -- (C12) -- (C22) -- (C21) -- (C10);

\draw[line] (D01) -- (D00) -- (D10);
\draw[line] (E01) -- (E10);

\end{tikzpicture}%
}

% Upgreek letters
\makeatletter
\newcommand\mathgr[1]{\tokcycle
  {\addcytoks{##1}}
  {\processtoks{##1}}
  {\ifcsname up\expandafter\@gobble\string##1\endcsname
   \addcytoks[1]{\csname up\expandafter\@gobble\string##1\endcsname}%
    \else\addcytoks{##1}\fi}
  {\addcytoks{##1}}{#1}%
  \expandafter\mathrm\expandafter{\the\cytoks}%
}
\makeatother



% Helper to sanitize labels: replace : and - with _
\makeatletter
\newcommand{\SanitizeLabel}[1]{%
  \expandafter\@sanitize@label#1\@nil
}
\def\@sanitize@label#1#2\@nil{%
  \ifx\relax#1\relax
  \else
    \if:#1\_\else
    \if-#1\_\else
    #1\fi\fi
    \@sanitize@label#2\@nil
  \fi
}
\makeatother

% Create a new macro proofof
% taking as input a label of a theorem
% and creating a proof with a reference to that
% label
\NewDocumentEnvironment{proofof}{ m O{appendix} }{
    % Convert label to command name by replacing : and - with _
    % For example: antichain-branches-prefix:lem -> antichain_branches_prefix_lem
    \edef\cmdname{\SanitizeLabel{#1}}%
    % if the command \cmdname exists, then 
    % call \cmdname* to restate the theorem
    \ifcsname\cmdname\endcsname
        \def\isInsideRestatedTheorem{1}%
        \csname\cmdname\endcsname*%
    \fi
    \begin{proof}[Proof of {\cref{#1}} as stated on page {\pageref{#1}}]
        \phantomsection
        \label{#1:proof}
}{
        % if the optional argument is "appendix" 
        % then printout a "backlink"
        % and otherwise do nothing
        \ifthenelse{\equal{#2}{appendix}}{
        % Some link to go back to the theorem
        \marginpar{\vspace{-2em}\texttt{\tiny{\hyperref[#1]{$\triangleright$ Back to p.\pageref{#1}}}}}
        }{}
    \end{proof}
}

% Create a new macro proofref
% that takes as input a label of a theorem
% and creates a reference to its proof
\NewDocumentCommand{\proofref}{ m }{
    % checks if the label #1:proof exists, if yes
    % it creates a link to it, otherwise it writes nothing
    \IfRefUndefinedExpandable{#1:proof}{}{
        % Checks if we are inside a restated theorem
        % if yes, we do not print anything
        \ifdefined\isInsideRestatedTheorem
        \else
            \marginpar{\vspace{0.6em}\texttt{\tiny{\hyperref[#1:proof]{$\triangleright$ Proven p.\pageref{#1:proof}}}}}
        \fi
    }
}



% Automate the creation of new orderings
% based on a given symbol.
% For instance,
% \NewDocumentOrdering{\pref}{\preceq}{\prec}
% will create the following commands:
% \prefleq and \preflt
% that will respectively expand to
% \mathrel{\kl[\pref]{\preceq}} and \mathrel{\kl[\pref]{\prec}}
\NewDocumentCommand{\NewDocumentOrdering}{ m m m }{
    \expandafter\newcommand\csname #1leq\endcsname{
        \mathrel{\kl[#1]{#2}}
    }
    \expandafter\newcommand\csname #1lt\endcsname{
        \mathrel{\kl[#1]{#3}}
    }
    \knowledge{#1}{notion}
}

% Little math macros
\NewDocumentCommand{\set}{ m }{\{ #1 \}}
\NewDocumentCommand{\setof}{ m m }{\{ #1 \mid #2 \}}
\NewDocumentCommand{\card}{ m }{\left| #1 \right|}
\NewDocumentCommand{\Nat}{ }{\mathbb{N}}
\NewDocumentCommand{\Rel}{ }{\mathbb{Z}}
\NewDocumentCommand{\seqof}{ m O{n \in \Nat} }{\left( #1 \right)_{#2}}
\NewDocumentCommand{\defined}{ }{\triangleq}

\NewDocumentCommand{\range}{ O{1} m }{[#1, #2]}

% Order macros
\NewDocumentCommand{\upset}{ O{} m }{{\uparrow_{#1} #2}}
\NewDocumentCommand{\dwset}{ O{} m }{{\downarrow_{#1} #2}}


% Number theory
\NewDocumentCommand{\factorial}{ O{} m }{
    \if\relax\detokenize{#1}\relax
        #2!
    \else
        (#2)!
    \fi
}

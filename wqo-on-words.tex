%! TEX program = pdflatex
% WARNING: this is a generated file.
%
% Please do not edit this file directly. 
% - If you want to update the medatata of the paper (title, authors, abstract), please
%   edit the `paper-meta.yaml` file in the root of the repository.
% - If you want to update the content of the paper, please edit the latex files
%   in the `src` directory.
% - If you want to update the template itself (e.g., change the layout), please
%   edit the `templates/plain-article.tex` file instead.
\documentclass[11pt,a4paper,twosided]{article}

% we setup a custom geometry because the default one is too narrow
\usepackage{geometry}
\geometry{margin=3.5cm}

% utf-8 for old systems
\usepackage[utf8]{inputenc}
\usepackage[T1]{fontenc}

% babel for language settings
\usepackage[english]{babel}

% microtype for better typography
\usepackage{microtype}

\usepackage{todonotes}
\usepackage{lineno}
\linenumbers



% math packages
\usepackage{amsmath,amsthm,amssymb,stmaryrd,thmtools,upgreek}

% configure some theorems
\newtheorem{theorem}{Theorem}
\newtheorem{lemma}[theorem]{Lemma}
\newtheorem{corollary}[theorem]{Corollary}
\newtheorem{proposition}[theorem]{Proposition}
\newtheorem{conjecture}[theorem]{Conjecture}
\newtheorem{assumption}[theorem]{Assumption}
\theoremstyle{definition}
\newtheorem{definition}[theorem]{Definition}
\newtheorem{remark}[theorem]{Remark}
\newtheorem{example}[theorem]{Example}



% graphics packages
\usepackage{graphicx}
\usepackage{subcaption}
\usepackage[obeyclassoptions,mode=tex]{standalone}
\usepackage{tikz}
\usetikzlibrary{backgrounds}
\usetikzlibrary{shapes.geometric}
\usetikzlibrary{positioning}
\usetikzlibrary{automata}
\usetikzlibrary{tikzmark}
\usetikzlibrary{patterns}
\usetikzlibrary{arrows}
\tikzset{every state/.style={minimum size=1pt}}
\usepackage{tikz-cd}

% ornaments
\usepackage{pgfornament}


% links inside the document
\usepackage{hyperref}
\usepackage[capitalise,noabbrev,nameinlink]{cleveref}
\Crefname{assumption}{Assumption}{Assumptions}
\usepackage[composition,hyperref,xcolor,cleveref]{knowledge}
\knowledgeconfigure{notion}

% Tables 
\usepackage{booktabs}
\usepackage{varwidth}

% Algorithms
\usepackage{algorithm2e}
\Crefname{algocfline}{Algorithm}{Algorithms}
\crefname{algocfline}{Algorithm}{Algorithms}
\crefname{algocf}{Algorithm}{Algorithms}
\Crefname{algocf}{Algorithm}{Algorithms}

% Packages for macro definitions
\usepackage{xparse}
\usepackage{xpatch}
\usepackage{tokcycle}
\usepackage{ifthen}

% Proofs
\usepackage{bussproofs}

% Colors 
\usepackage{ensps-colorscheme}
% parametrize knowledge colors and style
% intro => A4 + emph 
% kl    => normal color, normal size not emph
\knowledgestyle{intro notion}{color={A2}, emphasize}
\knowledgestyle{notion}{color={B1}}
\hypersetup{
    colorlinks=true,
    breaklinks=true,
    linkcolor=A4,
    citecolor=A4,
    urlcolor=A4,
    filecolor=A4,
}


% we include whatever the user wants to include in the header

% we include libraries (tex files) usually written in the `lib` directory
\newcommand{\klogo}{%
\begin{tikzpicture}[scale=0.2,line/.style={draw, line width=0.2pt, line cap=round, line join=round}]
\coordinate (A00) at (0,0);
\coordinate (A01) at (0,1);
\coordinate (A10) at (1,0);
\coordinate (B10) at (1,0.2);
\coordinate (B01) at (0.2,1);

\coordinate (C01) at (0.4,0.7);
\coordinate (C10) at (0.7,0.4);
\coordinate (C12) at (0.4,1.2);
\coordinate (C21) at (1.2, 0.4);
\coordinate (C22) at (1.2, 1.2);

\coordinate (D00) at (C10);
\coordinate (D01) at (0.8,0.5);
\coordinate (D10) at (0.8,0.3);

\coordinate (E01) at (0.3,0.7);
\coordinate (E10) at (0.5,0.7);

\draw[line] (B01) -- (A01) -- (A00) -- (A10) -- (B10);
\draw[line] (C01) -- (C12) -- (C22) -- (C21) -- (C10);

\draw[line] (D01) -- (D00) -- (D10);
\draw[line] (E01) -- (E10);

\end{tikzpicture}%
}

% Upgreek letters
\makeatletter
\newcommand\mathgr[1]{\tokcycle
  {\addcytoks{##1}}
  {\processtoks{##1}}
  {\ifcsname up\expandafter\@gobble\string##1\endcsname
   \addcytoks[1]{\csname up\expandafter\@gobble\string##1\endcsname}%
    \else\addcytoks{##1}\fi}
  {\addcytoks{##1}}{#1}%
  \expandafter\mathrm\expandafter{\the\cytoks}%
}
\makeatother



% Helper to sanitize labels: remove : and - characters
% Uses expl3 regex features
\ExplSyntaxOn
\tl_new:N \l__proofof_label_tl
\cs_new_protected:Npn \__proofof_sanitize:n #1
  {
    \tl_set:Nn \l__proofof_label_tl { #1 }
    \regex_replace_all:nnN { \- } { } \l__proofof_label_tl
    \regex_replace_all:nnN { \: } { } \l__proofof_label_tl
  }
% Helper to check and call restatable command
\cs_new_protected:Npn \__proofof_maybe_restate:n #1
  {
    \__proofof_sanitize:n { #1 }
    \exp_args:NV \cs_if_exist:cT \l__proofof_label_tl
      {
        \cs_set_eq:NN \isInsideRestatedTheorem \c_one_int
        \exp_args:NV \use:c \l__proofof_label_tl *
      }
  }
% Document-level command that can be called from regular LaTeX
\cs_new_protected:Npn \ProofOfMaybeRestate #1
  {
    \__proofof_maybe_restate:n { #1 }
  }
\ExplSyntaxOff

% Create a new macro proofof
% taking as input a label of a theorem
% and creating a proof with a reference to that
% label
\NewDocumentEnvironment{proofof}{ m O{appendix} }{
    % Convert label to command name by removing : and -
    % For example: antichain-branches-prefix:lem -> antichainbranchesprefixlem
    \ProofOfMaybeRestate{#1}%
    \begin{proof}[Proof of {\cref{#1}} as stated on page {\pageref{#1}}]
        \phantomsection
        \label{#1:proof}
}{
        % if the optional argument is "appendix" 
        % then printout a "backlink"
        % and otherwise do nothing
        \ifthenelse{\equal{#2}{appendix}}{
        % Some link to go back to the theorem
        \marginpar{\vspace{-2em}\texttt{\tiny{\hyperref[#1]{$\triangleright$ Back to p.\pageref{#1}}}}}
        }{}
    \end{proof}
}

% Create a new macro proofref
% that takes as input a label of a theorem
% and creates a reference to its proof
\NewDocumentCommand{\proofref}{ m }{
    % checks if the label #1:proof exists, if yes
    % it creates a link to it, otherwise it writes nothing
    \IfRefUndefinedExpandable{#1:proof}{}{
        % Checks if we are inside a restated theorem
        % if yes, we do not print anything
        \ifdefined\isInsideRestatedTheorem
        \else
            \marginpar{\vspace{0.6em}\texttt{\tiny{\hyperref[#1:proof]{$\triangleright$ Proven p.\pageref{#1:proof}}}}}
        \fi
    }
}



% Automate the creation of new orderings
% based on a given symbol.
% For instance,
% \NewDocumentOrdering{\pref}{\preceq}{\prec}
% will create the following commands:
% \prefleq and \preflt
% that will respectively expand to
% \mathrel{\kl[\pref]{\preceq}} and \mathrel{\kl[\pref]{\prec}}
\NewDocumentCommand{\NewDocumentOrdering}{ m m m }{
    \expandafter\newcommand\csname #1leq\endcsname{
        \mathrel{\kl[#1]{#2}}
    }
    \expandafter\newcommand\csname #1lt\endcsname{
        \mathrel{\kl[#1]{#3}}
    }
    \knowledge{#1}{notion}
}

% Little math macros
\NewDocumentCommand{\set}{ m }{\{ #1 \}}
\NewDocumentCommand{\setof}{ m m }{\{ #1 \mid #2 \}}
\NewDocumentCommand{\card}{ m }{\left| #1 \right|}
\NewDocumentCommand{\Nat}{ }{\mathbb{N}}
\NewDocumentCommand{\Rel}{ }{\mathbb{Z}}
\NewDocumentCommand{\seqof}{ m O{n \in \Nat} }{\left( #1 \right)_{#2}}
\NewDocumentCommand{\defined}{ }{\triangleq}

\NewDocumentCommand{\range}{ O{1} m }{[#1, #2]}

% Order macros
\NewDocumentCommand{\upset}{ O{} m }{{\uparrow_{#1} #2}}
\NewDocumentCommand{\dwset}{ O{} m }{{\downarrow_{#1} #2}}


% Number theory
\NewDocumentCommand{\factorial}{ O{} m }{
    \if\relax\detokenize{#1}\relax
        #2!
    \else
        (#2)!
    \fi
}

\NewDocumentOrdering{pref}{\sqsubseteq_{\mathsf{pref}}}{\sqsubset_{\mathsf{pref}}}
\NewDocumentOrdering{suff}{\sqsubseteq_{\mathsf{suff}}}{\sqsubset_{\mathsf{suff}}}
\NewDocumentOrdering{inf}{\sqsubseteq_{\mathsf{infix}}}{\sqsubset_{\mathsf{infix}}}
\NewDocumentOrdering{hig}{\leq^*}{<^*}

\input{knowledges.kl}

% We include the title and author information based on the 
% `paper-meta.yaml` file.
 
\title{Well-quasi-orderings on word languages}

\author{
Nathan Lhote\thanks{Aix-Marseille University}
 \and
Aliaume Lopez\thanks{University of Warsaw}
 \and
Lia Schütze\thanks{Max Planck Institute for Software Systems}
}

% For the date, we first check if the user has provided a date,
% and otherwise use the git meta inforamtion (if available).
\date{2025-08-12 15:11:25 +0200\footnote{5ede8d5033f0419024d7857dd3aacd9d3c036068 -- branch main at git@github.com:AliaumeL/word-ordering-wqo.git}}

\newcommand{\repositoryUrl}{\url{https://github.com/AliaumeL/word-ordering-wqo}}

\knowledge{notion}
 | kl-usage

% Now, we create the document itself.
\begin{document}
% Generate the title page
\maketitle
% Print the abstract
\begin{abstract}
    The set of finite words over a well-quasi-ordered set is itself well-quasi-ordered. This seminal result by Higman is a cornerstone of the theory of well-quasi-orderings and has found numerous applications in computer science. However, this result is based on a specific choice of ordering on words, the (scattered) subword ordering. In this paper, we describe to what extent other natural orderings (prefix, suffix, and infix) on words can be used to derive Higman-like theorems. More specifically, we are interested in characterizing \emph{languages} of words that are well-quasi-ordered under these orderings. We show that a simple characterization is possible for the prefix and suffix orderings, and that under extra regularity assumptions, this also extends to the infix ordering. We furthermore provide decision procedures for a large class of languages, that contains regular and context-free languages.
    \paragraph{Keywords:}
    equivariant ideal, Hilbert basis, ideal membership problem, orbit finite, oligomorphic, well-quasi-ordering
\paragraph{Repository:} \repositoryUrl
\end{abstract}

\klogo\ This document uses \href{https://ctan.org/pkg/knowledge}{knowledge}:
\kl[kl-usage]{notion} points to its \intro[kl-usage]{definition}. 


% Include the content of the paper
% LTeX: language=en-GB 
% !TeX root=../wqo-on-words.tex
\section{Introduction}
\label{introduction:sec}

\AP A \intro{well-quasi-ordered} set is a set $X$ equipped with a quasi-order
$\preceq$ such that every infinite sequence $\seqof{x_n}$ of elements taken in
$X$ contains an increasing pair $x_i \preceq x_j$ with $i < j$. Well-quasi-orderings serve
as a core combinatorial tool powering many termination arguments, and was
successfully applied to the verification of infinite state transition
systems~\cite{ABDU96,ABDU98}. One of the appealing properties of
well-quasi-orderings is that they are closed under many operations, such as
taking products, finite unions, and finite powerset
constructions~\cite{SCSC12}. Perhaps more surprisingly, the class of
well-quasi-ordered sets is also stable under the operation of taking finite
words and finite trees labelled by elements of a well-quasi-ordered set
\cite{HIG52,KRU72}.

\AP Note that in the case of finite words and finite trees, the precise choice
of ordering is crucial to ensure that the resulting structure is
\kl{well-quasi-ordered}. The celebrated result of Higman states that the set of
finite words over an ordered alphabet $(X, \preceq)$ is \kl{well-quasi-ordered}
by the so-called \kl{subword embedding relation}~\cite{HIG52}. Let us recall
that the \kl{subword relation} for words over $(X, \preceq)$ is defined as
follows: a word $u$ is a \intro{subword} of a word $v$, written $u
\intro*\higleq v$, if there exists an increasing function $f \colon \{1,
\ldots, |u|\} \to \{1, \ldots, |v|\}$ such that $u_i \preceq v_{f(i)}$ for all
$i \in \{1, \ldots, |u|\}$.

\begin{figure}
    \centering
    \begin{subfigure}[t]{0.48\textwidth}
    	\centering
    	\includestandalone[width=\linewidth]{fig/prefix-embedding-standalone}
    	\caption{Prefix Relation}
   	\end{subfigure}%
   	\hfill%
   	\begin{subfigure}[t]{0.48\textwidth}
   		\centering
   		\includestandalone[width=\linewidth]{fig/infix-embedding-standalone}
   		\caption{Infix Relation}
   	\end{subfigure}
   	\begin{subfigure}[t]{0.48\textwidth}
   		\centering
   		\includestandalone[width=\linewidth]{fig/subword-embedding-standalone}
   		\caption{Subword Relation}
   	\end{subfigure}
   	
   	\caption{A simple representation of the \kl{subword relation},
        \kl{prefix relation},
        and \kl{infix relation},
        on the alphabet $\set{a,b}$ for words of
        length at most $3$. The figures are Hasse Diagrams,
        representing the successor relation of the order.
        Furthermore, we highlight in dashed red relations that are added
        when moving from the prefix relation to the infix one,
        and to the infix relation to the subword one.}
    \label{word-embeddings:fig}
\end{figure}

\AP However, there are many other natural orderings on words that could be
considered in the context of \kl{well-quasi-orderings}, even in the simplified
setting of a finite alphabet $\Sigma$ equipped with the equality relation. In
this setting, the three alternatives we consider are the \intro{prefix
relation} ($u \intro*\prefleq v$ if there exists $w$ with $uw = v$), the
\intro{suffix relation} ($u \intro*\suffleq v$ if there exists $w$ such that
$wu = v$), and the \intro{infix relation} ($u \intro*\infleq v$ if there exists
$w_1,w_2$ such that $w_1 u w_2 = v$). Note that these three relations
straightforwardly generalize to infinite quasi-ordered alphabets.
Unfortunately, it is easy to see that none of these constructions are 
well-quasi-ordered as soon as the alphabet contains two distinct letters:
for instance, the infinite sequence $a^n b^n a^n$ is \kl{well-quasi-ordered} by
the \kl{subword relation} but by neither the \kl{prefix relation}, nor the
\kl{suffix relation}, nor the \kl{infix relation}.

\AP While this dooms \kl[well-quasi-ordered]{well-quasi-orderedness} of these
relations in the general case, there may be \emph{subsets} of $\Sigma^*$ which
are \kl{well-quasi-ordered} by these relations. As a simple example, take the
case of finite sets of (finite) words which are all \kl{well-quasi-ordered}
regardless of the ordering considered. This raises the question of
characterizing exactly which subsets $L \subseteq \Sigma^*$ are
\kl{well-quasi-ordered} with respect to the \kl{prefix relation} (respectively,
the \kl{suffix relation} or the \kl{infix relation}), and designing
suitable decision procedures.

\AP Let us argue that these decision procedures fit a larger picture in the
research area of well-quasi-orderings.
Indeed, there has been recent breakthroughs in deciding whether a given order
is a \kl{well-quasi-order}, for instance in the context of the verification of
infinite state transition systems~\cite{DBLP:conf/fsttcs/FinkelG19} or in the
context of logic~\cite{DBLP:journals/pacmpl/BergstrasserGLZ24}.
Furthermore, a previous work by Kuske shows that any
\emph{reasonable}\footnote{ This will be made precise in
\cref{infix-embedding:thm}. } partially ordered set $(X, \leq)$ can
be embedded into $\set{a,b}^*$ with the \kl{infix relation}~\cite[Lemma
5.1]{DBLP:journals/ita/Kuske06}. Phrased differently, one can encode a large
class of partially ordered sets as subsets of $\set{a,b}^*$. As a consequence,
the following decision problem provides a reasonable abstract framework for
deciding whether a given partially ordered set is \kl{well-quasi-ordered}:
given a language $L \subseteq \Sigma^*$, decide whether $L$ is
\kl{well-quasi-ordered} by the \kl{infix relation}.

\AP When considering an algorithm based on a \kl{well-quasi-ordering}, the
runtime of the algorithm is deeply related to the ``complexity'' of the
underlying quasi-order~\cite{SCHMITZ17}. One way to measure this complexity is
to consider its so-called \kl{ordinal invariants}: for instance, the
\kl{maximal order type} (or \kl{m.o.t.}), originally defined by De Jongh and Parikh
\cite{dejongh77}, is the order type of the maximal linearization of a
well-quasi-ordered set. In the case of a finite set, the \kl{m.o.t.} is precisely
the size of the set. Better runtime bounds were obtained by considering two
other parameters~\cite{SCHMITZ19}: the \kl{ordinal height} introduced by
Schmidt \cite{schmidt81}, and the \kl{ordinal width} of Kříž and Thomas
\cite{kriz90b}. Therefore, when characterizing \kl{well-quasi-ordered}
languages, we will also be interested in deriving upper bounds on their
\kl{ordinal invariants}. We refer to \cref{ordinal-invariants:subsec} for a
more detailed introduction to these parameters and ordinal computations in
general.

\paragraph*{Contribution} In this paper, we focus on words over a finite
alphabet $\Sigma$, and characterize subsets $L \subseteq \Sigma^*$ that are
\kl{well-quasi-ordered} by the \kl{prefix relation}, the \kl{suffix relation},
and the \kl{infix relation}. Furthermore, we devise decision algorithms
whenever the languages are given by reasonable computational models. Finally,
we provide precise bounds on the possible \kl{ordinal invariants} of such
languages.

In the case of the \kl{prefix} and \kl{suffix} relations, we show that a
language $L \subseteq \Sigma^*$ is \kl{well-quasi-ordered} by the \kl{prefix
relation} (resp. \kl{suffix}) if and only if it is a finite union of
\kl{chains} of the \kl{prefix relation} (resp. \kl{prefix}), where a
\intro{chain} is a totally ordered set that is well-founded. Note that
\kl{chains} are the simplest possible \kl{well-quasi-ordered} sets (they are
\kl{totally ordered} and \kl{well-founded}) and it is known that finite unions
of \kl{well-quasi-ordered} sets are \kl{well-quasi-ordered}. As a consequence,
the above characterization states that only the simplest possible
\kl{well-quasi-ordered} sets are \kl{well-quasi-ordered} by the \kl{prefix
relation} (\cref{prefixes:thm}). Let us furthermore highlight that this
characterization holds without any restriction on the decidability of the
language $L$ itself, but heavily relies on the assumption that $\Sigma$ is
finite. This allows us to derive tight bounds on the \kl{ordinal invariants} of
such \kl{well-quasi-ordered} languages, which can be interpreted in two dual
ways: first, these languages are extremely simple, which means that one could
have hoped for using directly another well-quasi-ordered set without resorting
to finite words; second, any time one encounters a \kl{well-quasi-ordered}
language, it is going to provide relatively efficient algorithms. Indeed, we
prove that all languages $L \subseteq \Sigma^*$ are \kl{well-quasi-ordered} by
the \kl{prefix relation} (resp. \kl{suffix relation}) have an \kl{ordinal
height} of at most $\omegaOrd$, a finite \kl{ordinal width}, and a \kl{maximal
order type} strictly below $\omegaOrd^2$
(\cref{prefixes-ordinal-invariants:cor}).

\AP The straightforward generalization of the results for the \kl{prefix
relation} and the \kl{suffix relation} to the \kl{infix relation} is not
possible. Indeed, it follows from Kuske's result that $\Sigma^*$ equipped with
the \kl{subword relation} can be embedded into $\set{a,b}^*$ with the \kl{infix
relation}~\cite[Lemma 5.1]{DBLP:journals/ita/Kuske06}. This implies that there
are \kl{well-quasi-ordered} languages for the \kl{infix relation} that have
very large \kl{ordinal invariants}: for instance, the \kl{maximal order type}
of the \kl{subword relation} is $\omegaOrd^{\omegaOrd^{\card{\Sigma} - 1}}$,
which equals its \kl{ordinal width} \cite[Corollary 3.9, Theorem
4.21]{DZSCSC20}. We show that in two situations, this can be avoided: when the
language is \kl{downwards closed} (i.e., when it is closed under taking
\kl{infixes}), and when the language is \kl(language){bounded} (i.e., 
when it is included in some $w_1^* \cdots w_k^*$ for some finite choice of
words $w_1, \ldots, w_k$). In those cases, we are able to characterize
\kl{well-quasi-ordered} languages by the \kl{infix relation} and derive tight
bounds on their \kl{ordinal invariants}.

In the case of \kl{bounded languages}, we prove that a \kl{bounded language} $L
\subseteq \Sigma^*$ is \kl{well-quasi-ordered} by the \kl{infix relation} if
and only if it is (included in) a finite union of languages $S_i \cdot P_i$,
where each $S_i$ is a \kl{chain} for the \kl{suffix relation}, and where each
$P_i$ is a \kl{chain} for the \kl{prefix relation}
(\cref{bounded-language:thm}) This result directly translates into upper bounds
on the possible \kl{ordinal invariants} of such languages similarly as for the
\kl{prefix relation}. Notice that these upper bounds are significantly smaller
than \kl{ordinal invariants} of the \kl{subword relation}: they have an \kl{ordinal
height} of at most $\omegaOrd$, an \kl{ordinal width} strictly below
$\omegaOrd^2$, and a \kl{maximal order type} strictly below $\omegaOrd^3$
(\cref{ordinal-invariants-bounded:cor}).

In the case of \kl{downwards closed} languages, we prove that they are deeply
related to the notion of \kl{uniformly recurrent words}, borrowed from the
study of word combinatorics. As an intermediate result, we prove that an
infinite word $w$ is \kl{ultimately uniformly recurrent} if and only if the set
of all \kl{infixes} of $w$ is \kl{well-quasi-ordered} by the \kl{infix
relation} (\cref{ultimately-uniformly-recurrent:lem}). We show that every
language $L \subseteq \Sigma^*$ that is \kl{downwards closed} and
\kl{well-quasi-ordered} by the \kl{infix relation} is a finite union of the
sets of finite \kl{infixes} of some \kl{ultimately uniformly recurrent}
bi-infinite words. This also proves that such languages have an \kl{ordinal
height} of at most $\omegaOrd$, an \kl{ordinal width} strictly below
$\omegaOrd^2$, and a \kl{maximal order type} strictly below $\omegaOrd^3$
(\cref{small-ordinal-invariants:thm}).

Then, we turned our attention to decision procedures. To that end, we need to
choose a computational model representing languages. For \kl{downwards closed}
languages, because of their close connection with infinite words, we considered
a model based on \kl{automatic sequences}. Using this model, we can decide
whether a language is \kl{well-quasi-ordered} by the \kl{infix relation}
(\cref{automatic-wqo:thm}). We also studied another representation, where
languages are recognized by \kl{amalgamation systems} \cite{ASZZ24}. Such
systems will be formally introduced in \cref{amalgamation-systems:subsec}, but
for the moment let us just say that they include many classical computational
models such as finite automata, context-free grammars, and Petri nets
\cite{ASZZ24}. This provides us with a \emph{meta-algorithm} for deciding
whether a given language is \kl{well-quasi-ordered} by the \kl{prefix
relation}, the \kl{suffix relation}, or the \kl{infix relation} under mild
effective restrictions on the computational model that we call an \kl{effective
amalgamative class}, the formal definition of which we defer to
\cref{infixes-amalgamation-effective:subsec}. Given a class $\mathcal{C}$ that
is a \kl{strongly effective amalgamative class} of languages, we designed a
decision procedure that takes as input a language $L \in \mathcal{C}$, and
decides whether $L$ is \kl{well-quasi-ordered} by the \kl{prefix relation}, the
\kl{suffix relation}, or the \kl{infix relation}
(\cref{infix-wqo-is-emptiness:thm}). Quite surprisingly, we also showed that,
if a language recognized by an \kl{amalgamation system} is
\kl{well-quasi-ordered} for the \kl{infix relation}, then it is a \kl{bounded
language} (\cref{infix-amalgamation:thm}), which automatically bounds the
\kl{ordinal invariants} of the language. Let us point out that the above result
implies that the hypothesis of \kl{bounded languages} on the theoretical side
is not a restriction in practice. As a down-to-earth and more easily
understandable consequence, our generic decision procedure applies to the class
$\mathcal{C}_\text{aut}$ of all languages recognized by finite automata, and to
the class $\mathcal{C}_\text{cfg}$ of all languages recognized by context-free
grammars, which are both \kl{effective amalgamative classes}
(\cref{aut-cfg-infix:cor}).

Finally, we noticed that for \kl{downwards closed} languages that are
\kl{well-quasi-ordered} by the \kl{infix relation}, being
\kl(language){bounded} is the same as being \kl(language){regular}.
Furthermore, a \kl{bounded language} is \kl{well-quasi-ordered} by the
\kl{infix relation} if and only if its \kl{downwards closure} is
\kl{well-quasi-ordered} by the \kl{infix relation}
(\cref{bounded-wqo-dwclosed:cor}). This shows that, for \kl{bounded languages}
(and therefore, for all languages recognized by \kl{amalgamation systems}) that
are \kl{well-quasi-ordered} by the \kl{infix relation}, their \kl{downwards
closure} is a \kl{regular language}. This is a weak version of the usual result
that the \kl{downwards closure} for the \kl{scattered subword relation} is
always a \kl{regular language}.


\todo[inline]{
  Add a figure to recap our complexity results: 
  bounded / downwards closed / amalgamative / morphic | width.
  And only talk about WQO classes, collapse everything and say regular.
}

\paragraph*{Related work} The study of alternative \kl{well-quasi-ordered}
relations over finite words is far from new. For instance, orders obtained by
so-called \emph{derivation relations} where already analysed by Bucher,
Ehrenfeucht, and Haussler \cite{BUEUD85}, and were later extended by
D'Alessandro and Varricchio \cite{ALVA03,ALVA06}. However, in all those cases
the orderings are \emph{multiplicative}, that is, if $u_1 \preceq v_1$ and $u_2
\preceq v_2$ then $u_1u_2 \preceq v_1v_2$. This assumption does not hold for
the \kl{prefix}, \kl{suffix}, and \kl{infix} relations.

A similar question was studied by Atminas, Lozin, and Moshkov \cite{ALM17}, in
the hope of finding characterizations of classes of \emph{finite graphs} that
are \kl{well-quasi-ordered} by the \emph{induced subgraph relation}
\cite[Section 7]{ALM17}. In this setting, it is common to refer to classes of
graphs via a list of \emph{forbidden patterns}, which are finite graphs that
cannot be found as induced subgraphs in the class. Applying this reasoning to
finite words with the \kl{infix relation}, they provide an efficient decision
procedure for checking whether a language $L \subseteq \Sigma^*$ is
\kl{well-quasi-ordered} by the \kl{infix relation} whenever said language is
given as input via a list of \emph{forbidden factors} \cite[Theorem 1, Theorem
2]{ALM17}. The key construction of their paper is to study languages $L$ that
are \emph{regular} (recognized by some finite deterministic automata), for
which they can decide whether $L$ is \kl{well-quasi-ordered} by the \kl{infix
relation} \cite[Theorem 1]{ALM17}. Because it is easy to transform a list of
forbidden factors into a regular language \cite[Theorem 1]{ALM17}, this yields
the desired decision procedure. Our work extends this result in several ways:
first, we also consider the \kl{prefix relation} and the \kl{suffix relation},
then we consider non-regular languages, and finally, we provide very precise
descriptions of the \kl{well-quasi-ordered} languages, as well as tight bounds
on their \kl{ordinal invariants}. In particular, we highlight the role of
\kl{bounded languages} as the one that are well-behaved with respect to being
\kl{well-quasi-ordering} by the \kl{infix relation} \cref{infixes-bounded:sec}.
These non-trivial extensions continue to rely
on classical word combinatorial techniques that are already present in the work
of Atminas, Lozin, and Moshkov \cite[Section 3]{ALM17}.

\paragraph*{Outline} 
We introduce in \cref{prelims:sec} the
necessary background on \kl{well-quasi-orders} and \kl{ordinal invariants}.
In
\cref{prefixes:sec}, which is relatively
self-contained, we study the \kl{prefix relation} and prove in
\cref{prefixes:thm} the characterization of \kl{well-quasi-ordered}
languages by the \kl{prefix relation}. In
\cref{infixes-bounded:sec}, we
obtain the \kl[infix relation]{infix} analogue of \cref{prefixes:thm}
specifically for \kl{bounded languages}
(\cref{bounded-language:thm}). 
In \cref{infixes-dwclosed:sec}, we study the \kl{downwards closed}
languages, and compute bounds on their \kl{ordinal invariants} in \cref{small-ordinal-invariants:thm}.
Finally, 
we generalize these results to all
\kl{amalgamation systems} in \cref{infixes-amalgamation:sec}
in
(\cref{infix-amalgamation:thm}),
and provide a decision procedure for checking whether a language is
\kl{well-quasi-ordered} by the \kl{infix relation} (resp. \kl{prefix} and \kl{suffix}) in
this context (\cref{infix-wqo-is-emptiness:thm}).

\paragraph*{Acknowledgements} We would like to thank participants of the 2024
edition of \kl{Autobóz} for their helpful comments and discussions.
We would also like to thank Vincent Jugé for his pointers on word combinatorics.

\section{Preliminaries}
\label{prelims:sec}

\paragraph*{Finite words.} \AP In this paper, we use upper Greek letters
$\Sigma, \Gamma$ to denote finite alphabets, $\Sigma^*$ to denote the set of
finite words over $\Sigma$, and $\varepsilon$ for the empty word in $\Sigma^*$.
In order to give some intuition on the decision problems, we will sometimes use
the notion of \intro{finite automata}, \intro{regular languages}, and Monadic
Second Order logic ($\intro*\MSO$) over finite words, and assume the reader to
be familiar with them. We refer to the textbook of \cite{THOM97} for a detailed
introduction. However, we will require no prior knowledge on word
combinatorics.

\paragraph*{Orderings and Well-Quasi-Orderings.}
\AP
A \intro{quasi-order} is a
reflexive and transitive binary relation, it is a \intro{partial order} if it
is furthermore antisymmetric. A \intro{total order} is a \kl{partial order}
where any two elements are comparable. Let now us introduce some notations for
\kl{well-quasi-orders}. A sequence $\seqof{x_i}$ in a set $X$ is
\intro(sequence){good} if there exist $i < j$ such that $x_i \leq x_j$. It is
\intro(sequence){bad} otherwise. Therefore, a \kl{well-quasi-ordered} set is a
set where every infinite sequence is \kl(sequence){good}. A \intro{decreasing
sequence} is a sequence $\seqof{x_i}$ such that $x_{i+1} < x_i$ for all $i$,
and an \intro{antichain} is a set of pairwise incomparable elements. An
equivalent definition of a \kl{well-quasi-ordered} set is that it contains no
infinite \kl{decreasing sequences}, nor infinite \kl{antichains}. We refer to
\cite{SCSC12} for a detailed survey on well-quasi-orders.

The \kl{prefix relation} (resp. the \kl{suffix relation} and the \kl{infix
relation}) on $\Sigma^*$ are always \intro{well-founded}, i.e., there are no
infinite \kl{decreasing sequences} for this ordering. In particular, for a
language $L \subseteq \Sigma^*$ to be \kl{well-quasi-ordered}, it suffices to
prove that it contains no infinite \kl{antichain}. 

\paragraph*{Ordinal Invariants.} \label{ordinal-invariants:subsec}
\AP
An \intro{ordinal} is a \kl{well-founded} \kl{totally ordered}
set. We use $\alpha, \beta, \gamma$ to denote ordinals, and use $\intro*\omegaOrd$ to
denote the first infinite \kl{ordinal}, i.e., the set of natural numbers with the
usual ordering. We also use $\intro*\omegaOne$ to denote the first \emph{uncountable}
ordinal.
We only assume superficial familiarity with ordinal arithmetic, and
refer to the books of Kunen \cite{KUNEN80} and Krivine~\cite[Chapter
II]{KRIVINE71} for a detailed introduction to this domain.
Given a tree $T$
whose branches are all finite we can define an \kl{ordinal} $\alpha_T$ inductively
as follows: if $T$ is a leaf then $\alpha_T = 0$, if $T$ has children
$\seqof{T_i}$ then $\alpha_T = \sup \setof{\alpha_{T_i} + 1}{i \in \Nat}$. We
say that $\alpha_T$ is the \emph{rank} of $T$. 

\AP
Let $(X, \leq)$ be a \kl{well-quasi-ordered} set. One can define three
well-founded trees from $X$: the tree of \kl{bad sequences}, the tree of
decreasing sequences, and the tree of \kl{antichains}. The rank of these
respective trees are called respectively the \intro{maximal order type} of $X$
written $\intro*{\oType{X}}$ \cite{dejongh77}, the \intro{ordinal height} of
$X$ written $\intro*{\oHeight{X}}$ \cite{schmidt81}, and the \intro{ordinal
width} of $X$ written $\intro*{\oWidth{X}}$ \cite{kriz90b}. These three
parameters are called the \intro{ordinal invariants} of a
\kl{well-quasi-ordered} set $X$. 
As an example, for $(\Nat, \leq)$, all bad sequences are descending and there are no antichains. Therefore the maximal order type and the ordinal height of $\Nat$ are $\omega$ and the ordinal width is 1. \todo{Please check this!}
We refer to the survey of \cite{DZSCSC20} for
a detail discussion on these concepts and their computation on specific classes
of well-quasi-ordered sets.

\AP
We will use the following inequality between \kl{ordinal invariants}, due to
\cite{kriz90b}, and that was recalled in \cite[Theorem 3.8]{DZSCSC20}:
$\oType{X} \leq \oHeight{X} \oComProd \oWidth{X}$, where $\intro*\oComProd$ is
the \intro{commutative ordinal product}, also known as the \reintro{Hessenberg
product}. We will not recall the definition of this product here, and refer to
\cite[Section 3.5]{DZSCSC20} for a detailed introduction to this concept. The
only equalities we will use are $\omegaOrd \oComProd \omegaOrd = \omegaOrd^2$
and $\omegaOrd^2 \oComProd \omegaOrd = \omegaOrd^3$.


% LTeX: language=en-GB 
\section{Prefixes and Suffixes}
\label{prefixes-suffixes:sec}

In this section, we study the well-quasi-ordering of languages under the prefix
relation. A word $u$ is a \intro{prefix} of a word $w$ if there exists a word
$v$ such that $w = uv$. We denote this relation by $u \intro*\prefleq w$.
Similarly, a word $u$ is a \intro{suffix} of a word $w$ if there exists a word
$v$ such that $w = vu$. We denote this relation by $u \intro*\suffleq w$. Let
us immediately remark that the map $u \mapsto u^R$ that reverses a word is an
order-bijection between $(X^*, \prefleq)$ and $(X^*, \suffleq)$. Therefore, we
will focus on the prefix relation in the rest of this section.


Let us briefly restate the fact that some (even regular) languages 
are not well-quasi-ordered by the prefix relation.

\begin{example}
    The set $L = \setof{a^nb}{n \in \Nat}$ is an infinite antichain for the
    prefix relation.
\end{example}

In order to characterize the existence of infinite antichains for the prefix
relation, we will introduce the following tree construction that
will be useful in the rest of this section.

\begin{definition}
    The \intro{tree of prefixes} over a finite alphabet $\Sigma$
    is the infinite tree $T$ whose nodes are the words of $\Sigma^*$, and
    such that the children of a word $w$ are the words $wa$ for all $a \in
    \Sigma$. 
\end{definition}

Notice that the tree of prefixes is finitely branching. Let us now
observe how antichains in the prefix relation can be witnessed
by infinite branches in the tree of prefixes.

\begin{definition}
    An \intro{antichain branch} for a language $L$ is an infinite 
    branch $B$ of the tree $T$ such that from every point of the branch, 
    one can reach a word of $A$ that is not in the branch.
\end{definition}

It is clear that an \kl{antichain branch} for a language $L$ yields an infinite
antichain, and the converse is quite easy to prove. Because the notion of
antichain branch is \kl{$\MSO$-definable} whenever $L$ is regular, we
immediately obtain decidability as a corollary of this lemma.

\begin{lemma}
    Let $L \subseteq \Sigma^*$ be a language. Then, $L$ contains an infinite
    \kl{antichain} if and only if there exists an \kl{antichain branch} for $L$.
\end{lemma}

\begin{corollary}
    If $L$ is regular, then the existence of an infinite antichain is decidable.
\end{corollary}

Let us now go further and fully characterize languages $L$ such that the
prefix relation is well-quasi-ordered, even without any restriction on the
decidability of $L$ itself. Let us remark that finite unions of \kl{chains} are
always \kl{well-quasi-ordered} by the \kl{prefix relation} because they lack
infinite \kl{antichains} by definition. The following theorem states that this
is the only possible reason for a language $L$ to be \kl{well-quasi-ordered} by
the \kl{prefix relation}.

\begin{theorem}
    A language $L \subseteq \Sigma^*$ is \kl{well-quasi-ordered} by the
    \kl{prefix relation} if and only if $L$ is a union of \kl{chains}.
\end{theorem}

As an immediate consequence, we have a very fine-grained understanding of the
\kl{ordinal invariants} of such \kl{well-quasi-ordered} languages, which can be
leveraged in bounding the complexity of algorithms working on such languages.

\begin{corollary}
    \textbf{false: some chains can be finite}
    Let $L \subseteq \Sigma^*$ be a language that is \kl{well-quasi-ordered} by
    the \kl{prefix relation}. Then, there exists a finite $k \in \Nat$ such that
    the
    \kl{maximal order type} of $L$ is $k \cdot \omega$,
    the \kl{ordinal height} of $L$ is $\omega$, and its
    \kl{ordinal width} is $k$.
\end{corollary}


Let us conclude by noting that it is unsurprisingly not possible to decide
whether a decidable language is \kl{well-quasi-ordered} by the \kl{prefix
relation}.

\begin{lemma}
    todo.
\end{lemma}

% LTeX: language=en-GB 
\section{Infixes and Regular Languages}
\label{infixes-regular:sec}

\todo{Delete this section: the \cref{infix-embedding:thm} is already
    known, we should just restate it. The
    \cref{infix-finite-automata:thm} is a direct consequence of
    \cref{infix-amalgamation:thm} and \cref{bounded-language:thm}.
    I don't know if we should keep it for clarity, but it does not 
    seems to be necessary to me.
}

\AP
In this section, we study languages equipped with the \kl{infix relation}. As
opposed to the \kl{prefix} and \kl{suffix} relations, the \kl{infix relation}
can yield to very complicated \kl{well-quasi-ordered} languages. Formally, the
upcoming \cref{infix-embedding:thm} shows that \emph{any} countable
partial-ordering with finite initial segments can be embedded into the infix
relation of a language. To make the former statement precise,
Let us recall that an \intro{order embedding} from a quasi-ordered set $(X,
\preceq)$ into a quasi-ordered set $(Y, \preceq')$ is a function $f \colon X
\to Y$ such that for all $x, y \in X$, $x \preceq y$ if and only if $f(x)
\preceq' f(y)$. When such an embedding exists, we say that $X$ \reintro{embeds
into} $Y$. We say that a quasi-ordered set $(X, \preceq)$ is a \intro{partial
ordering} whenever the relation $\preceq$ is antisymmetric, that is $x \preceq
y$ and $y \preceq x$ implies $x = y$. 

The following result due to Kuske shows the structural complexity of the \kl{infix relation}

\begin{lemma}{\cite[Lemma 5.1]{DBLP:journals/ita/Kuske06}}
    \label{infix-embedding:thm}
    Let $(X, \preceq)$ be a \kl{partially ordered} set,
    and $\Sigma$ be an alphabet with at least two letters.
    Then the following are equivalent:
    \begin{enumerate}
        \item \label{infix-embedding-embeds:item} 
            $X$ \kl{embeds into} $(\Sigma^*, \infleq)$,
        \item \label{infix-embedding-count:item}
            $X$ is countable, and for every $x \in X$,
            its \kl{downwards closure}
            $\dwset[\preceq]{x}$ is finite.
    \end{enumerate}
\end{lemma}
\begin{figure}
    \centering
    \includestandalone[width=\linewidth]{fig/infix-encoding-standalone}
    \caption{Representation of the \kl{subword relation} for $\set{a,b}^*$
        inside the \kl{infix relation} for $\set{a,b,\#}^*$
        using the encoding of \cref{infix-embedding:thm}, restricted to words
        of length at most $3$. To obtain smaller words,
        we replaced $D_i$ by $\max D_i$ in this construction.
    }
    \label{infix-embedding:fig}
\end{figure}

As a consequence of \cref{infix-embedding:thm}, we cannot replay proofs of
\cref{prefixes:sec}, and will actually need to leverage some regularity of the
languages to obtain a characterization of \kl{well-quasi-ordered} languages
under the \kl{infix relation}. Let us first play this game for languages that
are recognized by finite automata. We assume that the reader is familiar with
the notions of deterministic finite automaton and regular languages, and refer
to the book of Thomas for a comprehensive introduction to the subject
\cite{THOM97}. The main goal of the remainder of this section is to prove the
following \cref{infix-finite-automata:thm}.

\begin{theorem}[restate=infix-finite-automata:thm,label=infix-finite-automata:thm]
    \proofref{infix-finite-automata:thm}
    Let $L \subseteq \Sigma^*$ be a language recognized by a finite automaton.
    Then $L$ is \kl{well-quasi-ordered} by the \kl{infix relation} if and only if $L$ is
    a finite union of \kl{chains} for the \kl{infix relation}.
\end{theorem}

\AP In order to prove \cref{infix-finite-automata:thm}, we will perform some
preliminary analysis on the structure of an automaton recognizing a
well-quasi-ordered language under the infix relation, which will be powered by
folklore results on \emph{periodic} words. Let us recall that a non-empty word
$w \in \Sigma^+$ is \intro(word){periodic} with period $x \in \Sigma^*$ if
there exists a $p \in \Nat$ such that $w \infleq x^p$. The \intro{periodic
length} of a word $u$ is the minimal length of a word $x$ such that $u$ is an
\kl{infix} of $x^p$ for some $p \in \Nat$ and $x \in \Sigma^+$. We will
essentially rely on the following result on periodic words.

\begin{lemma}
    \label{periodic-infixes:lem}
    Let $u,v \in \Sigma^*$ be two (non-empty) \kl{periodic words}
    having \kl{periodic lengths} $p$ and $q$ respectively.
    Then, if $u \infleq v$ and $\card{u} \geq \factorial[p]{p \times q}$,
    then $u$ and $v$ share the same \kl{periodic length}
    $p = q$.
\end{lemma}
\begin{proof}
    The fact that $u$ and $v$ are \kl{periodic length}
    respectively $p$ and $q$ translates into the fact that $u_{i+p} = u_i$ and
    $v_{i+q} = v_i$ for all indices $i \in \Nat$ such that those letters are
    well-defined.

    Now, assume that $u$ is an \kl{infix} of $v$, this provides the existence
    of a $k \in \Nat$ such that $u = v_{k} \cdots v_{k + \card{u} - 1}$. In
    particular, $v_{k+i+p} = v_{k+i}$ for all $i \in \Nat$ such that $k+i+p < k
    + \card{u}$. Since we also have $v_{k+i+q} = v_{k+i}$ for all $1 \leq i
    \leq \card{v} - k - q$. We conclude that both $u$ and $v$ are of
    \kl{periodic length} the greatest common divisor of $p$ and $q$, and by
    minimality of $q$ this must be equal to $q$ and to $p$.
\end{proof}

The upcoming \cref{powers-infixes:cor} is based on the observation that if $S$
is a \kl{chain} for the \kl{suffix relation} and $P$ is a \kl{chain} for the
\kl{prefix relation}, then $SP$ is a \kl{chain} for the \kl{infix relation}.

\begin{corollary}
    \label{powers-infixes:cor}
    Let $u,v \in \Sigma^*$ and $k, \ell \in \Nat$
    such that $k \geq \factorial[p]{\card{u} \times \card{v}}$,
    $\ell \geq \factorial[p]{\card{v} \times \card{u}}$,
    and $u^k \infleq v^\ell$.
    Then, there exists $w \in \Sigma^*$ of size at most
    $\min \set{\card{u}, \card{v}}$ and a $p \in \Nat$
    such that
    $u^k \infleq v^\ell \infleq w^p$.
\end{corollary}

The reason why \kl{periodic words} built using a given period $x \in \Sigma^+$
are interesting for the \kl{infix relation} is that they naturally create
\kl{chains}. Indeed, if $x \in \Sigma^+$ is a finite word, then $\setof{x^p}{p
\in \Nat}$ is a \kl{chain} for the \kl{infix relation}. Note that in general,
the \kl{downwards closure} of a chain is \emph{not} a chain. However, for the
chains generated using periodic words, the \kl{downwards closure}
$\dwset[\infleq]{\setof{x^p}{p \in \Nat}}$ is a \emph{finite union} of
\kl{chains}. Because this set will appear in bigger equations, we introduce the
shorter notation $\intro*\InfPeriodChain{x}$ for the set of \kl{infixes} of
words of the form $x^p$, where $p$ ranges over $\Nat$.

\begin{lemma}
    \label{inf-period-chain:lem}
    Let $x \in \Sigma^+$ be a word, and
    Then $\InfPeriodChain{x}$ is a finite union of \kl{chains}
    for the \kl{infix}, \kl{prefix} and \kl{suffix} relations 
    simultaneously.
\end{lemma}
\begin{proof}
    Let $x \in \Sigma^+$ be a word, and let $P_x$ be the (finite) set 
    of all \kl{prefixes} of $x$, and $S_x$ be the (finite)
    set of all \kl{suffixes} of $x$.
    Assume that $w \in \InfPeriodChain{x}$, then $w = u x^p v$ for some
    $u \in S_x$, $v \in P_x$, and $p \in \Nat$.
    We have proven that
    \begin{equation*}
        \InfPeriodChain{x} \subseteq \bigcup_{u \in P_x} \bigcup_{v \in S_x} u x^* v
        \quad .
    \end{equation*}

    Let us now demonstrate that for all $(u,v) \in S_x \times P_x$, the
    language $u x^* v$ is a \kl{chain} for the \kl{infix}, \kl{suffix} and \kl{prefix} relations.
    To that end,
    let $(u,v) \in S_x \times P_x$ and $\ell, k \in \Nat$ be such that $\ell <
    k$, let us prove that $u x^\ell v \infleq u x^k  v$. Because $v \prefleq
    x$, we know that there exists $w$ such that $vw = x$. In particular,
    $ux^\ell vw = u x^{\ell + 1}$, and because $\ell < k$, we conclude that $u
    x^{\ell + 1} \prefleq u x^k v$. By transitivity, $u x^\ell v \prefleq u x^k
    v$, and \emph{a fortiori}, $u x^\ell v \infleq u x^k v$. 
    Similarly, because $u \suffleq x$,  there exists $w$ such that $wu  = x$, 
    and we conclude that $u x^{\ell} v \suffleq w u x^\ell v = x^{\ell + 1} v \suffleq u x^k v$.
    \qedhere
\end{proof}


\begin{corollary}
    \label{inf-period-union-chains:lem}
    Let $x,u,y \in \Sigma^*$ be words.  The following 
    are finite unions of \kl{grids} for the infix relation:
    $\InfPeriodChain{x}u$, $u \InfPeriodChain{x}$,
    and $\InfPeriodChain{x} u \InfPeriodChain{y}$.
\end{corollary}
\begin{proof}
    Because $\InfPeriodChain{x}$ is a finite union of \kl{chains} for the \kl{suffix}
    relation (rsing \cref{inf-period-chain:lem}), we conclude that $\InfPeriodChain{x}u$ is one too for
    the \kl{infix} relation. This holds similarly for $u \InfPeriodChain{x}$.
    In the case of $\InfPeriodChain{x} u \InfPeriodChain{y}$,
    we remark that $\InfPeriodChain{x} u$ is a finite union of chains
    for the \kl{suffix relation},
    and that $\InfPeriodChain{y}$ is one for the \kl{prefix relation}
    (using again \cref{inf-period-chain:lem}).
    As a consequence,
    $\InfPeriodChain{x}u \InfPeriodChain{y}$ is a finite union 
    of \kl{chains} for the \kl{infix} relation.
\end{proof}

The following combinatorial lemma connects the property of being
\kl{well-quasi-ordered} to a property of the \kl{periodic lengths} of words in
a language, based on the assumption that some factors can be iterated. It is
the core result that powers the analysis done in the upcoming
\cref{infix-finite-automata:thm,infix-amalgamation:thm}.

\begin{lemma}
    \label{pumping-periods:lem}
    Let $L \subseteq \Sigma^*$ be a language
    that is \kl{well-quasi-ordered} by the \kl{infix relation}.
    Let $k \in \Nat$, $u_1, \cdots, u_{k+1} \in \Sigma^*$,
    and $v_1, \cdots, v_{k} \in \Sigma^+$
    be such that
    $w[\vec{n}] \defined (\prod_{i = 1}^k u_i v_i^{n_i}) u_{k+1}$
    belongs to $L$
    for arbitrarily large values of $\vec{n} \in \Nat^k$.
    Then, 
    there exists $x,y \in \Sigma^+$ of size 
    at most $\max \setof{\card{v_i}}{1 \leq i \leq k}$
    such that 
    one of the following holds for all
    $\vec{n} \in \Nat^{k}$:
    \begin{enumerate}
        \item $w[\vec{n}] \in u_1 \InfPeriodChain{x}$,
        \item $w[\vec{n}] \in \InfPeriodChain{x} u_{k+1}$,
        \item $w[\vec{n}] \in \InfPeriodChain{x} u_i \InfPeriodChain{y}$
            for some $1 \leq i \leq k + 1$.

    \end{enumerate}
\end{lemma}
\begin{proof}
    Note that the result is obvious if $k = 0$, and therefore
    we assume $k \geq 1$ in the following proof.

    Let us construct a sequence of words $\seqof{w_i}[i \in \Nat]$, where $w_i
    \defined w[\vec{n_i}]$ for some well-chosen indices $\vec{n_i} \in \Nat^k$. The goal
    being that 
    if $w[\vec{n_i}]$ is an \kl{infix} of $w[\vec{n_j}]$,
    then it can intersect at most \emph{two} iterated words,
    with an intersection that is long enough to successfully apply
    \cref{periodic-infixes:lem}.
    In order to achieve this,
    let us first define $s$ as the maximal size of a word $v_i$
    ($1 \leq i \leq k$) and $u_j$ ($1 \leq j \leq k+1$).
    Then,
    we consider $\vec{n_0} \in \Nat^k$ such that $\vec{n_0}$ has all 
    its components greater than $\factorial{s}$ and such that
    $w[\vec{n_0}]$ belongs to $L$. 
    Then, we inductively define 
    $\vec{n_{i+1}}$  as the smallest vector of numbers greater than $\vec{n_i}$,
    such that $w[\vec{n_{i+1}}]$ belongs to $L$, 
    and with $\vec{n_i}$ having all components greater than
    $2\card{w[\vec{n_i}]}$.


    Let us assume that $k \geq 2$ in the following proof for symmetry purposes,
    and argue later on that when $k = 1$ the same argument goes through.
    Because $L$ is \kl{well-quasi-ordered} by the \kl{infix relation}, there
    exists $i < j$ such that $w[\vec{n_i}]$ is an \kl{infix} of $w[\vec{n_j}]$.
    Now, because of the chosen values for $\vec{n_j}$, there exists $1 \leq \ell \leq
    k-1$ such that $w[\vec{n_i}]$ is actually an \kl{infix} of $u_{\ell}
    v_{\ell}^{n_{j,\ell}} u_{\ell+1} v_{\ell+1}^{n_{j,\ell+1}} u_{\ell+2}$.
    Even more,
    one of the three following equations holds:
    \begin{itemize}
        \item $w[\vec{n_i}] \infleq v_{\ell}^{n_{j,\ell}} u_{\ell+1} v_{\ell+1}^{n_{j,\ell+1}}$,
        \item $w[\vec{n_i}] \infleq u_{\ell}
            v_{\ell}^{n_{j,\ell}}$,
        \item $w[\vec{n_i}] \infleq
            v_{\ell+1}^{n_{j,\ell+1}} u_{\ell+2}$.
    \end{itemize}
    In all those cases, we conclude using \cref{powers-infixes:cor}
    that there exists $x,y \in \Sigma^+$ of size at most $s$, and 
    a number $1 \leq t \leq k$ such that
    $v_i^{n_i} \in \InfPeriodChain{x}$ for all $1 \leq i \leq t$,
    and
    $v_i^{n_i} \in \InfPeriodChain{y}$ for all $t < i \leq k$.
    In particular,
    $w[\vec{n_i}] \in \InfPeriodChain{x} u_{t} \InfPeriodChain{y}$.

    
    When $k = 1$, the situation is a bit more specific since we only have two
    cases: either $w_i \infleq u_1 v_1^{n_j}$ or $w_i \infleq v_1^{n_j} u_2$,
    and we conclude with an identical reasoning.
\end{proof}


We are now ready to restate and prove our main theorem, proven by combining
\cref{pumping-periods:lem} together with classical pumping arguments on finite
state automata.

\begin{proofof}{infix-finite-automata:thm}
    Let $w \in L$, because $Q$ is finite, there exists
    a factorization of $w$
    into words $(\prod_{i = 1}^k u_i v_i) u_{k+1}$
    such that 
    for all $1 \leq i \leq k+1$, $\card{u_i} \leq \card{Q}$,
    for all $1 \leq i \leq k$, $1 \leq \card{v_i} \leq \card{Q}$,
    and satisfying 
    that $w[\vec{X}] \defined 
    (\prod_{i = 1}^k u_i v_i^{X_i}) u_{k+1}$
    belongs to $L$ for all choices of values $\vec{X} \in \Nat^k$.

    Applying \cref{pumping-periods:lem}, we conclude that 
    there exists $x,y \in \Sigma^+$ of size at most $\card{Q}$
    such that 
    $w \in u_1 \InfPeriodChain{x} \cup \InfPeriodChain{y} u_{k+1}
    \cup \bigcup_{1 \leq i \leq k+1} \InfPeriodChain{x} u_i \InfPeriodChain{y}$.
    In particular, we conclude that
    \begin{equation}
        \label{infix-automata:eq}
        L
        \subseteq
        \bigcup_{x,y,u\in \Sigma^{\leq \card{Q}}}
        u \InfPeriodChain{x}
        \cup 
        \InfPeriodChain{x} u
        \cup
        \InfPeriodChain{x} u \InfPeriodChain{y}
        \quad .
    \end{equation}

    Now, thanks to \cref{inf-period-union-chains:lem},
    this happens to be a finite union of \kl{chains}
    for the \kl{infix relation}.
\end{proofof}



\begin{corollary}
    \label{reg-wqo-decidable:cor}
    Given a regular language $L$, it is decidable whether
    $L$ is \kl{well-quasi-ordered} for the \kl{infix relation}.
\end{corollary}
\begin{proof}
    It suffices to decide whether 
    \cref{infix-automata:eq} holds for the language $L$.
    This is an inclusion of regular languages, which is decidable.
\end{proof}


Let $(X,\preceq)$ be a quasi-ordered set, and $L \subseteq X$ be such that $(L,
\preceq)$ is \kl{well-quasi-ordered}. It is not true in general that
$(\dwset{L}, \preceq)$ is \kl{well-quasi-ordered}. In the case of $(\Sigma^*,
\infleq)$ a typical example is to start from an infinite \kl{antichain} $A$,
together with an enumeration $\seqof{w_i}$ of $A$, and build the language $L
\defined \setof{ \prod_{i = 0}^n w_i }{ i \in \Nat }$. By definition, $L$ is a
\kl{chain} for the \kl{infix} ordering, hence \kl{well-quasi-ordered}. However,
$\dwset[\infleq]{L}$ contains $A$, and is therefore not
\kl{well-quasi-ordered}. As a fun corollary of our analysis, we conclude that
this particular example cannot be realized by a regular language $A$.


\begin{corollary}
    \label{reg-wqo-iff-dwclosed-wqo:cor}
    Given a regular language $L$, it is \kl{well-quasi-ordered} 
    for the \kl{infix relation} if and only if 
    $\dwset[\infleq] L$ is.
\end{corollary}

\section{Bounded Languages and Amalgamation Systems}
\label{infixes-amalgamation:sec}

When proving \cref{infix-finite-automata:thm}, we have leveraged a powerful
combinatorial argument on regular langugages stated in
\cref{pumping-periods:lem}. Our first remark is that this idea can
be generalized to the larger class of \intro{bounded languages}, i.e.,
languages $L \subseteq \Sigma^*$ such that there exists words $w_1, \dots, w_n$
satisfying $L \subseteq w_1^* \cdots w_n^*$.

\begin{theorem}
    \label{bounded-language:thm}
    Let $L$ be a \kl{bounded language} of $\Sigma^*$. Then,
    $L$ is a \kl{well-quasi-order} when endowed with the 
    \kl{infix relation} if and only if it is a finite union of \kl{grids}.
\end{theorem}
\begin{proof}
    Let $w_1, \dots, w_n$ be such that
    $L \subseteq w_1^* \cdots w_n^*$.
    Let us define $m \defined \max \setof{\card{w_i}}{1 \leq i \leq n}$

    Let $w[\vec{k}] \defined w_1^{k_1} \cdots w_n^{k_n}$ be a map from $\Nat^k$
    to $\Sigma^*$. We are interested in the intersection of the image of $w$
    with $L$. Let us assume for instance that for all $\vec{k} \in \Nat^n$,
    there exists $\vec{\ell} \geq \vec{k}$ such that $w[\vec{\ell}] \in L$.
    Then, leveraging \cref{pumping-periods:lem}, we conclude that there exists
    $x,y$ of size at most $\max\setof{\card{w_i}}{1 \leq i \leq n}$ such that
    $w[\vec{k}] \in \InfPeriodChain{x} \cup \InfPeriodChain{x}
    \InfPeriodChain{y}$, and we conclude that $L \subseteq \InfPeriodChain{x}
    \cup \InfPeriodChain{x} \InfPeriodChain{y}$.

    Now, it may be the case that one cannot simultaneously assume that two
    component of the vector $\vec{k}$ are unbounded. In general, given a set $S
    \subseteq \set{1, \dots, n}$ of indices, we say that $S$ is admissible if
    there exists a bound $N_0$ such that for all $\vec{b} \in \Nat^S$, there
    exists a vector $\vec{k} \in \Nat^n$, such that $\vec{k}$ is greater than
    $\vec{b}$ on the $S$ components, and the other components are below the
    bound $N_0$. The language of an admissible set $S$ is the set of words
    obtained by repeating $w_i$ at most $N_0$ times if it is not in $S$
    ($w_i^{\leq N_0}$) and arbitrarily many times otherwise ($w_i^*$).
    Note that $L \subseteq \bigcup_{S \text{ admissible }} L(S)$.

    Now, admissible languages are ready to be pumped according to
    \cref{pumping-periods:lem}. For every admissible language,
    the size of a word that is not iterated is at most
    $N_0 \times m$ by definition, and we conclude that:
    \begin{equation*}
        L \subseteq 
        \bigcup_{x,y \in \Sigma^{\leq n}}
        \bigcup_{u \in \Sigma^{\leq m \times N_0}}
        \InfPeriodChain{x} u \InfPeriodChain{y}
        \cup
        \InfPeriodChain{x} u
        \cup
        u \InfPeriodChain{x}
        \quad .
    \end{equation*}
\end{proof}

Note that this immediately generalizes \cref{reg-wqo-iff-dwclosed-wqo:cor}
to all \kl{bounded languages}.

\begin{corollary}
    \label{bounded-wqo-dwclosed:cor}
    Let $L$ be a \kl{bounded language} of $\Sigma^*$. Then,
    $L$ is a \kl{well-quasi-order} when endowed with the
    \kl{infix relation} if and only if $\dwset[\infleq]{L}$ is.
\end{corollary}


In the light of \cref{reg-wqo-iff-dwclosed-wqo:cor,bounded-wqo-dwclosed:cor}
one may wonder whether the \kl{downwards closed}
languages for the \kl{infix relation} satisfy a similar characterization in
terms of union of \kl{grids} as stated in
\cref{infix-finite-automata:thm,bounded-language:thm}. Let us
first remark that from our previous results we obtain the following lemma that
illustrates how all the previous characterizations of \kl{well-quasi-ordered}
languages for the \kl{infix relation} are collapsing in the case of
\kl{downwards closed} languages.

\begin{lemma}
    \label{dwclosed-infixes-wqo:lem}
    Let $L \subseteq \Sigma^*$ be a \kl{downwards closed} language for the
    \kl{infix relation} that is \kl{well-quasi-ordered}. Then, the following
    are equivalent:
    \begin{enumerate}
        \item $L$ is \kl{bounded},
        \item $L$ is \kl{regular},
        \item $L$ is a finite union of \kl{chains} and \kl{grids}.
    \end{enumerate}
\end{lemma}
\begin{proof}
    TODO.
\end{proof}

One may think that all \kl{downwards closed} languages for the \kl{infix
relation} that are \kl{well-quasi-ordered} are \kl{regular}. Note that this is
what happens in the case of the \kl{subword embedding}, where any \kl{downwards
closed} language for this relation is characterized by finitely many excluded
patterns, hence \kl{regular}. However, this is not the case for the \kl{infix
relation}, as we will now illustrate with the following two examples.

\begin{example}
    \label{dwclosed-wqo-not-finite-excl:ex}
    Let $L \defined a^* b^* \cup b^* a^*$. This language is \kl{downwards
    closed} for the \kl{infix relation}, is \kl{well-quasi-ordered} for the
    \kl{infix relation}, but is characterized by an \emph{infinite} number 
    of excluded infixes, respectively of the form $ab^ka$ and $ba^kb$ where $k \geq 1$.
\end{example}

To strengthen \cref{dwclosed-wqo-not-finite-excl:ex}, we will
leverage the \intro{Thue-Morse sequence} $\intro*\ThueMorse \in
\set{0,1}^{\Nat}$, which we will use as a black-box for its two main
characteristics: it is \kl{cube-free} and \kl{uniformly recurrent}. Being
\intro{cube-free} means that no (finite) word of the form $uuu$ is an
\kl{infix} of $\ThueMorse$, and being \intro{uniformly recurrent} means that
for every word $u$ that is an \kl{infix} of $\ThueMorse$, there exists $k \geq
1$ such that $u$ is an \kl{infix} of every word $v$ of size at least $k$. We
refer the reader to a nice survey of Allouche and Shallit for more information
on this sequence and its properties \cite{ALSHA99}.

\begin{lemma}
    \label{thue-morse:lemma}
    There exists a language $L$ that is \kl{downwards closed} for the \kl{infix
    relation}, \kl{well-quasi-ordered} for the \kl{infix relation}, but is not
    \kl{regular}.
\end{lemma}
\begin{proof}
    Let $\ThueMorse$ be the \kl{Thue-Morse sequence}
    $L$ be the set of infixes of $w$. By construction $L$ is an \emph{infinite}
    \kl{downwards closed} for the \kl{infix relation}. Let us argue that $L$ is
    \kl{well-quasi-ordered} for the \kl{infix relation}, but is not \kl{regular}.

    Assume by contradiction that $L$ is \kl{bounded}. In this case, there exist
    words $w_1, \dots, w_k \in \Sigma^*$ such that $L \subseteq w_1^* \cdots
    w_k^*$. Since $L$ is infinite and \kl{downwards closed}, there exists a
    word $u \in L$ such that $u = w_i^3$ for some $1 \leq i \leq k$. This is absurd
    because $u \infleq \ThueMorse$, which is \kl{cube-free}.

    Furthermore, $L$ is \kl{well-quasi-ordered} for the \kl{infix relation}.
    Indeed, consider a sequence $\seqof{u_i}$ of words in $L$. Without loss of
    generality, we may consider a subsequence such that $\card{u_i} <
    \card{u_{i+1}}$ for all $i \in \Nat$. Because $\ThueMorse$ is \kl{uniformly
    recurrent}, there exists $k \geq 1$ such that $u_1$ is an \kl{infix} of
    every word $v$ of size at least $k$. In particular, $u_1$ is an \kl{infix}
    of $u_k$, hence the sequence $\seqof{u_i}$ is \kl(wqo){good}.

    We conclude thanks to
    \cref{dwclosed-infixes-wqo:lem}, that $L$ is not \kl{regular}.
\end{proof}



\subsection{Amalgamation Systems}
\label{infixes-amalgamation:subsec}

It turns out that there is a rather large family of systems for which pumping
arguments based on so-called \emph{minimal runs} exist: they are called
\emph{amalgamation systems} and were recently introduced by Anad, Schmitz,
Sch\"{u}tze, and Zetzsche \cite{ASZZ24}. Having done the heavy lifting on
\kl{bounded languages}, the rest of this section is mostly devoted to the
introduction of \kl{amalgamation systems} and collecting the necessary pumping
argument they enjoy. Using this meta-proof, we will generalize
\cref{infix-finite-automata:thm} to context-free grammars, languages recognized
by vector addition systems with state (VASS), and more. The goal of this
section is not to introduce all of these systems, or justify their usefulness,
but to state and prove the following theorem.

\begin{theorem}[label=infix-amalgamation:thm,restate=infix-amalgamation:thm]
    \proofref{infix-amalgamation:thm}
    Let $L \subseteq \Sigma^*$ be a language recognized by an 
    \kl{amalgamation system}.
    Then $L$ is well-quasi-ordered by the infix relation if and only if $L$ is
    a finite union of chains for the \kl{infix relation}.
\end{theorem}

Let us now formally introduce the notion of \kl{amalgamation systems}, and
recall some results from \cite{ASZZ24} that will be useful for the proof of
\cref{infix-amalgamation:thm}. The notion of \kl{amalgamation system} is
tailored to produce \emph{pumping arguments}, which is exactly what our
\cref{pumping-periods:lem} talks about. At the core of a pumping argument,
there is a notion of \emph{run}, which could for instance be a sequence of
transitions taken in a finite state automaton. Continuing on the analogy with
finite automata, there is a natural ordering between runs, i.e., a run is
smaller than another one if one can ``unroll loops" of the first run to obtain
the second one. Typical pumping arguments then rely on the fact that
\emph{minimal} runs are of finite size, and that all other runs are
obtained by ``gluing" minimal runs together. This is exactly what
\kl{amalgamation systems} are about.

\AP Let us recall that over an alphabet $(\Sigma, =)$ a \kl{subword embedding}
between two words $u \in \Sigma^*$ and $v \in \Sigma^*$ is a function $\rho
\colon \range{\card{u}} \to \range{\card{v}}$ such that $u_i = v_{\rho(i)}$ for
all $i \in \range{\card{u}}$. We write $\intro*\HigEmb(u,v)$ the set of all
\kl{subword embeddings} between $u$ and $v$. It may be useful to notice that
the set of finite words over $\Sigma$ forms a category when we consider
\kl{subword embeddings} as morphisms, which is a fancy way to state that
$\mathrm{id} \in \HigEmb(u,u)$ and that $f \circ g \in \HigEmb(u,w)$ whenever
$g \in \HigEmb(u,v)$ and $f \in \HigEmb(v,w)$, for any choice of words
$u,v,w \in \Sigma^*$.

\AP Given a \kl{subword embedding} $f \colon u \to v$ between two words $u$ and
$v$, there exists a unique decomposition $v = \GapWord{f}{0} u_1 \GapWord{f}{1}
\cdots \GapWord{f}{k-1} u_k \GapWord{f}{k}$ where $\GapWord{f}{i} =
v[f(i)+1:f(i+1)-1]$ for all $1 \leq i \leq k-1$, $\GapWord{f}{k} =
v[f(k)+1:\card{v}]$, and $\GapWord{f}{0}   = v[1: f(1)-1]$. We say that
$\intro*\GapWord{f}{i}$ is the $i$-th \intro{gap word} of $f$. We encourage the
reader to look at \cref{gap-word-embedding:fig} to see an example of the
\kl{gap words} resulting from a \kl{subword embedding} between two words. These
\kl{gap words} will be useful to describe how and where runs of a system
(described by words) can be combined.

\begin{figure}
    \centering
    \includestandalone[width=\linewidth]{fig/gap-word-embedding-standalone}
    \caption{The \kl{gap words} resulting from a \kl{subword embedding} between two 
    finite words.}
    \label{gap-word-embedding:fig}
\end{figure}

\begin{figure}
    \centering
    \includestandalone[width=\linewidth]{fig/run-amalgamation-standalone}
    \caption{We illustrate how 
        embeddings $f$ and $g$ between runs of an
        \kl{amalgamation system} can be glued
        together, seen on their canonical decomposition.
    }
    \label{amalgamation-runs:fig}
\end{figure}


\begin{definition}
    An \intro{amalgamation system}
    is a triple $(\Sigma, R, E, \canrun)$ where
    $\Sigma$ is a finite alphabet,
    $R$ is a set of so-called \emph{runs},
    and 
    $E$ is a set of so-called \emph{run embeddings},
    and $\canrun \colon R \to (\Sigma \uplus \set{\varepsilon})^*$ is a 
    function computing a \emph{canonical decomposition} of a run.
    Given a run $r \in R$, and $i \in \range[0]{\card{\canrun(r)}}$, 
    the \intro{gap language} of $r$ at position $i$ is $\GapLanguage{r}{i} \defined
    \setof{\GapWord{f}{i}}{\exists s \in R. \exists f \in E(r,s) }$.
    An \kl{amalgamation system} furthermore satisfies the following 
    properties:
    \begin{enumerate}
        \item \emph{Word Embeddings as Morphisms.} For all $r, s \in R$,
            $E(r,s) \subseteq \HigEmb(\canrun(r), \canrun(s))$.
        \item \emph{$(R, E)$ Forms a Category.}
            For all $r,s,t \in R$,
            $\mathrm{id} \in E(r,r)$,
            and whenever $f \in E(r,s)$ and $g \in E(s,t)$,
            then $g \circ f \in E(r,t)$.
        \item \emph{Well-Quasi-Ordered System.}
            $(R, \leq_E)$ is a well-quasi-ordered set,
            where $r \leq_E s$ if and only if $E(r,s) \neq \emptyset$.
        \item \emph{Controlled Amalgamation.}
            For all $r, s, t \in R$,
            for all $f \in E(r,s)$,
            for all $g \in E(r,t)$,
            and for all $0 \leq i_0 \leq \card{\canrun(r)}$,
            there exists a run $z \in R$ and morphisms
            $h \in E(s,z)$ and $k \in E(t,z)$
            satisfying
            $h \circ f = k \circ g$,
            $\GapWord{h \circ f}{i} = \GapWord{f}{i} \GapWord{g}{i}$
            or 
            $\GapWord{h \circ f}{i} = \GapWord{g}{i} \GapWord{f}{i}$
            for every $0 \leq j \leq n$,
            and
            $\GapWord{h \circ f}{i_0} = \GapWord{f}{i_0} \GapWord{g}{i_0}$.
            We refer to \cref{amalgamation-runs:fig} for an illustration 
            of this property.
    \end{enumerate}

    The \intro{amalgamation language} of such a system
    is the set of all words $w \in \Sigma^*$ such that
    there exists a run $r \in R$
    such that the concatenation of the letters of its
    canonical decomposition, written $\intro*\yieldrun(r)$,
    equals $w$.
\end{definition}

Intuitively, the definition of an amalgamation system allows the comparison of
runs, and the proper ``gluing" of runs together to obtain new runs. Let us
recall some examples of languages that can be recognized by \kl{amalgamation
systems} given by the authors of the original paper: regular languages
\cite[Theorem 5.3]{ASZZ24}, VASS as a consequence of \cite[Theorem
5.5]{ASZZ24}, context-free languages as a consequence of \cite[Theorem
5.10]{ASZZ24}. For this paper to be self-contained, we will also recall how
runs of a finite state automaton can be understood as an \kl{amalgamation
system}.

\begin{example}[{\cite[Section 3.2]{ASZZ24}}]
    Let $A = (Q, \delta, q_0, F)$ be a finite state automaton over a finite
    alphabet $\Sigma$. Let $\Delta$ be the set of transitions $(q_1, a, q_2)
    \in Q \times \Sigma \times Q$,
    and $R \subseteq \Delta^*$ be the set of 
    words over transitions that start with the initial state $q_0$,
    end in a final state $q_f \in F$, and such that the end state of a
    letter is the start state of the following one.
    The canonical decomposition $\canrun \colon R \to \Sigma^*$
    is defined as a morphism from $\Delta^*$ to $\Sigma^*$
    that maps $(q,a,p)$ to $a$.
    Finally, given two runs $r$ and $s$ of the automaton,
    we say that an embedding $f \in \HigEmb(\canrun(r), \canrun(s))$
    belongs to $E(r,s)$ when
    $f$ is also defining an embedding from $r$ to $s$ as words in $\Delta^*$,
    where $\Delta$ is equipped with the equality relation.

    The system $(\Sigma, R, E, \canrun)$ is an \kl{amalgamation system},
    whose language is precisely the language of words recognized
    by the automaton $A$.
\end{example}
\begin{proof}
    By definition, the embeddings inside $E(r,s)$ are defined as elements
    of $\HigEmb(\canrun(r), \canrun(s))$, and they compose properly.
    Because $\Delta = Q \times \Sigma \times Q$ is finite, it is 
    a \kl{well-quasi-ordering} when equipped with the equality relation, and 
    we conclude that $\Delta^*$ with $\higleq$ is a \kl{well-quasi-order}
    according to Higman’s Lemma \cite{HIG52}.
    
    Let us now move to proving that the system satisfies the amalgamation
    property. Given three runs $r,s,t \in R$, and two embeddings $f \in E(r,s)$
    and $g \in E(r,t)$, we want to construct an amalgamated run $s \vee t$.
    Because letters in the run $r$ respect the transitions of the automaton
    (i.e., if the letter $i$ ends in state $q$, then the letter $i+1$ starts in
    state $q$), then the \kl{gap word} at position $i$ starts in state $q$ and
    ends in state $q$ too. This means that for both embeddings
    $f$ and $g$, the \kl{gap words} are read by the automaton by looping
    on a state. In particular, these loops can be taken in any order
    and continue to represent a valid run. That is, we can even select
    the order of concatenation in the amalgamation for \emph{all} 
    $0 \leq i \leq \card{\canrun(r)}$ and not just for one separately.

    Let us now remark that 
    the language of this amalgamation system is
    the set of $\yieldrun(r)$ when $r$ ranges over $R$,
    and because $R$ is the set of valid runs of the automaton,
    and $\yieldrun(r)$ is the word read along this run,
    we immediately conclude.
\end{proof}


Let us now introduce a combinatorial lemma that explains how \kl{gap languages}
can be pumped. Note that a single \kl{gap language} is always stable under
concatenation, and our concern will be on a potential description of
\emph{simultaneously} pumping different \kl{gap languages} to produce valid
runs.

\begin{lemma}
    \label{pumping-gap-languages:lem}
    Let $(\Sigma, R, E, \canrun)$ be an \kl{amalgamation system}, let $r \in R$
    be a run of size $n \in \Nat$, and let $s, t \in R$ be two runs
    with embeddings $f \in E(r,s)$ and $g \in E(r,t)$.
    Let $\seqof{u_i}[0 \leq i \leq n]$ be the sequence of \kl{gap words}
    for $f$, that is, $u_i =
    \GapWord{f}{i}$ for all $0 \leq i \leq n$. Similarly, let
    $\seqof{v_i}[0 \leq i \leq n]$ be the sequence of \kl{gap words} for $g$.

    Then, for all $k \in \Nat$, and $0 < \ell < n$, the following words belong to 
    the language of the \kl{amalgamation system}.
    \begin{enumerate}
        \item $(\prod_{i = 0}^{\ell - 1} u_i^{x_i} v_i u_i^{y_i} v_i u_i^{z_i}) 
               v_\ell u_\ell^k v_\ell
               (\prod_{i = \ell+1}^{n} u_i^{x_i} v_i u_i^{y_i} v_i u_i^{z_i})$,
        \item $(\prod_{i = 0}^{n - 1} u_i^{x_i} v_i u_i^{y_i}) 
               v_n u_n^k$,
        \item $u_0^k v_0 
            (\prod_{i = 1}^{n} u_i^{x_i} v_i u_i^{y_i})$.
    \end{enumerate}
    Where, for all $0 \leq i \leq n$, $x_i + y_i + z_i = k$, with $z_i = 0$
    in the two last items.
\end{lemma}
\begin{proof}
    The result immediately follows from the embedding property
    applied inductively on runs obtained by gluing 
    $k$ times $s$ with one or two times $t$ (depending on the item number).
\end{proof}

In the upcoming proof of \cref{infix-amalgamation:thm}, we will use the notion
of \intro{bounded language} for languages $L \subseteq \Sigma^*$ such that
there exists words $w_1, \dots, w_n$ satisfying $L \subseteq w_1^* \cdots
w_n^*$. This will be useful for us to derive the inclusion of a language $L$
into a union of \kl{chains}. Note that we isolate this intermediate result in a
separate theorem that could be applied beyond \kl{amalgamation systems}.

\begin{theorem}
    \label{bounded-language:thm}
    Let $L$ be a \kl{bounded language} of $\Sigma^*$. Then,
    $L$ is a \kl{well-quasi-order} when endowed with the 
    \kl{infix relation} if and only if it is a finite union of \kl{chains}.
\end{theorem}
\begin{proof}
    Let $w_1, \dots, w_n$ be such that
    $L \subseteq w_1^* \cdots w_n^*$.
    Let us define $m \defined \max \setof{\card{w_i}}{1 \leq i \leq n}$

    Let $w[\vec{k}] \defined w_1^{k_1} \cdots w_n^{k_n}$ be a map from $\Nat^k$
    to $\Sigma^*$. We are interested in the intersection of the image of $w$
    with $L$. Let us assume for instance that for all $\vec{k} \in \Nat^n$,
    there exists $\vec{\ell} \geq \vec{k}$ such that $w[\vec{\ell}] \in L$.
    Then, leveraging \cref{pumping-periods:lem}, we conclude that there exists
    $x,y$ of size at most $\max\setof{\card{w_i}}{1 \leq i \leq n}$ such that
    $w[\vec{k}] \in \InfPeriodChain{x} \cup \InfPeriodChain{x}
    \InfPeriodChain{y}$, and we conclude that $L \subseteq \InfPeriodChain{x}
    \cup \InfPeriodChain{x} \InfPeriodChain{y}$.

    Now, it may be the case that one cannot simultaneously assume that two
    component of the vector $\vec{k}$ are unbounded. In general, given a set $S
    \subseteq \set{1, \dots, n}$ of indices, we say that $S$ is admissible if
    there exists a bound $N_0$ such that for all $\vec{b} \in \Nat^S$, there
    exists a vector $\vec{k} \in \Nat^n$, such that $\vec{k}$ is greater than
    $\vec{b}$ on the $S$ components, and the other components are below the
    bound $N_0$. The language of an admissible set $S$ is the set of words
    obtained by repeating $w_i$ at most $N_0$ times if it is not in $S$
    ($w_i^{\leq N_0}$) and arbitrarily many times otherwise ($w_i^*$).
    Note that $L \subseteq \bigcup_{S \text{ admissible }} L(S)$.

    Now, admissible languages are ready to be pumped according to
    \cref{pumping-periods:lem}. For every admissible language,
    the size of a word that is not iterated is at most
    $N_0 \times m$ by definition, and we conclude that:
    \begin{equation}
    	\label{amalgamation-infix-wqo-inclusion:eqn}
        L \subseteq 
        \bigcup_{x,y \in \Sigma^{\leq n}}
        \bigcup_{u \in \Sigma^{\leq m \times N_0}}
        \InfPeriodChain{x} u \InfPeriodChain{y}
        \cup
        \InfPeriodChain{x} u
        \cup
        u \InfPeriodChain{x}
        \quad .
    \end{equation}
\end{proof}

To conclude this section and prove our main \cref{infix-amalgamation:thm}, it
suffices to demonstrate that languages recognized by \kl{amalgamation systems}
that are \kl{well-quasi-ordered} for the \kl{infix relation} are \kl{bounded
languages}.

\begin{lemma}
    \label{infix-amalgamation-bounded:lem}
    Let $L \subseteq \Sigma^*$ be a language recognized
    by an \kl{amalgamation system} that is \kl{well-quasi-ordered}
    for the \kl{infix relation}. Then $L$ is a \kl{bounded language}.
\end{lemma}
\begin{proof}
    Assume that $L$ is \kl{well-quasi-ordered} by the \kl{infix relation},
    and obtained by an \kl{amalgamation system}
    $(\Sigma, R, E, \canrun)$.

    Let us consider the set $M$ of minimal runs for the relation $\leq_E$,
    which is finite because the latter is a \kl{well-quasi-ordering}. By
    definition, for every word $w \in L$, there exists $r \in M$, $s \in R$,
    and $f \in E(r,s)$ such that $w = \canrun(s)$.
    In particular, we conclude that
    \begin{equation*}
        L \subseteq \bigcup_{r \in M} 
        \left(
        \GapLanguage{r}{0}
        \prod_{i = 1}^{\card{\canrun(r)}-1} \canrun(r)_i \GapLanguage{r}{i} 
        \right)
        \quad .
    \end{equation*}

    Let $u$ and $v$ be two words in the \kl{gap language}
    $\GapLanguage{r}{\ell}$ for some $r \in M$ and $0 < \ell <
    \card{\canrun(r)} \defined n$. One can apply \cref{pumping-gap-languages:lem}
    to conclude that  for all $k \geq 1$,
    \begin{equation*}
       \left(\prod_{i = 0}^{\ell - 1} u_i^{x_i} v_i u_i^{y_i} v_i u_i^{z_i}\right) 
       v u^k v
       \left(\prod_{i = \ell+1}^{n} u_i^{x_i} v_i u_i^{y_i} v_i u_i^{z_i}\right)
       \in 
       L
    \end{equation*}
    Where $x_i + y_i + z_i = k$ for all $0 \leq i \leq n$
    Let us assume without loss of generality that the sequence
    $\seqof{(x_i, y_i, n_i)}[i \geq 1]$ is pointwise increasing,
    and even increasing coordinate wise. This is possible because
    at least one of the coordinates tends to $+\infty$ and in the case
    of a variable that is bounded, one can extract the sequence to
    simply not use this variable.

    Now, leveraging \cref{pumping-periods:lem}, we get that for infinitely many
    $k \geq 1$, $u v^k u$ is an infix of some $x^p$ where $x$ has size smaller
    or equal than $u$ and $v$. This proves  that $u$ and $v$ are comparable for
    the prefix, infix, and suffix relations. In particular, we have can
    conclude that $\GapLanguage{r}{i}$ is a \kl{bounded language}, because of
    \cite[Lemma 4.1]{ASZZ24}, when $0 < i < \card{\canrun(r)}$. The proof goes
    on similarly for proving that $\GapLanguage{r}{0}$ and $\GapLanguage{r}{n}$
    are \kl{bounded languages}, using the other items of
    \cref{pumping-gap-languages:lem}.

    Because \kl{bounded languages} are stable under concatenation and finite
    unions, we conclude that $L$ itself must be a \kl{bounded language}.
\end{proof}

\begin{proofof}{infix-amalgamation:thm}
    We first apply \cref{infix-amalgamation-bounded:lem},
    and conclude with \cref{bounded-language:thm}.
\end{proofof}

For the next result, we impose a mild restriction by requiring the language class to be closed under rational transductions. This property holds for most common language classes, including all known \kl{amalgamation systems}.

\begin{theorem}
	\label{infix-wqo-is-emptiness:thm}
	Let $\mathcal{C}$ be a class of languages recognized by amalgamation systems and closed under rational transductions. Then the following are equivalent:
	
	\begin{enumerate}
		\item\label{wqo-infix-decidable} \kl[well-quasi-order]{Well-quasi-orderedness} of the \kl{infix relation} is decidable.
		\item\label{emptiness-decidable} Emptiness is decidable.
	\end{enumerate}
\end{theorem}

\begin{proof}
	\cref{emptiness-decidable} $\Rightarrow$ \cref{wqo-infix-decidable}. We aim to make the inclusion test of \cref{amalgamation-infix-wqo-inclusion:eqn} effective. 
	
	Let $R = \bigcup_{x,y \in \Sigma^{\leq n}} \bigcup_{u \in \Sigma^{\leq m \times N_0}} \InfPeriodChain{x} u \InfPeriodChain{y} \cup \InfPeriodChain{x}u \cup u\InfPeriodChain{x}$. For any concrete values of the bounds $n$, $m$, and $N_0$, this language is regular. As $\mathcal{C}$ is closed under rational transductions, we can therefore reduce the inclusion to emptiness of $L \cap \overline{R}$. However, we need to find these bounds first.
	
	To determine values for $n$ and $m$, we first test if $L$ is bounded. As the algorithm in \cite[Section 4.2]{ASZZ24} is effective, this yields words $w_1, \ldots w_n$ such that $L \subseteq w_1^* \cdots w_n^*$ and therefore yields also the bounds in questions.
	
	To determine the value for $N_0$, we first computer maximal subsets $S \subseteq [n]$ such that $w_i$ for $i \in S$ are simultaneously unbounded. We do this by applying a transduction mapping $w_i$ to some fresh letter $a_i$ if $i \in S$ and $\varepsilon$ otherwise, and testing for simultaneous unboundedness \cite[Section 4.1]{ASZZ24}. Given a candidate bound $n_0$ for $N_0$, we construct the languages $L_{S,n_0} = w_1^{\circ_1}\cdots w_n^{\circ_n}$ where $\circ_i = *$ if $i \in S$ and $\circ_i = \leq n_0$ otherwise. We then test if $L \subseteq \bigcup_{S \text{ maximal}} L_{S,n_0}$. If yes, $n_0$ is our value for $N_0$, otherwise, we increase it and repeat the construction. As we chose maximal sets $S$, this procedure will eventually terminate.
	
	\cref{wqo-infix-decidable} $\Rightarrow$ \cref{emptiness-decidable}. We 
	consider the transduction $\mathcal T = \Sigma^* \times \{a, b\}$. Then 
	for any language $L \in \mathcal C$, $\mathcal TL$ is well-quasi-ordered 
	by infix if and only if $L$ is empty.
\end{proof}

Let us conclude this section by fully characterizing what are
actually the chains for the \kl{infix} relation, without any 
regularity assumption.

\begin{lemma}
    \label{chains-infix:lem}
    Let $L$ be a \kl{chain} for the \kl{infix relation}. Then, there exists
    a \kl{chain} $P$ for the \kl{prefix relation} and a \kl{chain}
    $S$ for the \kl{suffix relation} such that $L \subseteq SP$.
\end{lemma}
\begin{proof}
    Because $L$ is a chain,
    we have an enumeration $L = \seqof{w_n}$,
    where $s_n \defined \card{w_n}$ is increasing, 
    and functions
    $f_{i,j} \colon \set{1, \dots, s_i} \to \set{1, \dots s_{j}}$
    satisfying
    that $f_{j,k} \circ f_{i, j} = f_{i, k}$,
    $f_{i,i} = \mathrm{id}$, and
    $w_i = w_j[f_{i,j}(1):f_{i,j}(s_i)]$ for all $i \leq j$.
    Now, we can build the infinite word $w_\infty$ indexed by $\Rel$, and
    embeddings $f_{i, \infty} \colon \set{1, \dots, s_i} \to \Rel$ such that
    $w_i = w_\infty[f_{i, \infty}(1):f_{i, \infty}(s_i)]$, and $f_{i, \infty}
    \circ f_{j, i} = f_{j, \infty}$ for all $i \leq j$.

    Let us define $w_l$ as the (infinite) word $w_\infty[-\infty: f_{0,
    \infty}(s_0)]$ and $w_r$ as the (infinite) word $w_\infty[f_{0,
    \infty}(s_0): +\infty]$. It is now clear that every word $w_i$ can be
    written as $w_i = w_{i,l} w_{i,r}$, where $w_{i,l}$ is a prefix of $w_r$
    and $w_{i,r}$ is a suffix of $w_l$. Furthermore, the functions $f_{i,
    \infty}$ can be used to notice that $\seqof{w_{i,r}}$ is a \kl{chain} for
    the \kl{prefix relation}, and $\seqof{w_{i,l}}$ is a \kl{chain} for the
    \kl{suffix relation}, which allows us to conclude.
\end{proof}




\section{Decision Prodecure and Amalgamation Systems}
\label{infixes-amalgamation:sec}

\AP In this section, we are going to design an effective decision procedure for
\kl{well-quasi-ordering} by the \kl{infix relation}. To that end, the first
requirement is to fix a way to represent languages $L \subseteq \Sigma^*$.
Traditionally, one would use finite automata, context-free grammars, or for the
more adventurous, vector addition systems with states. However, our proof
technique will only require us to have a way to ``glue'' together runs of the
system to ``pump'' them and produce new runs: this is the usual pumping lemma
in automata theory, and Ogden's lemma for context-free grammars \cite{OGDEN68}.
It turns out that there is a rather large family of systems for which pumping
arguments based on so-called \emph{minimal runs} exist: they are called
\emph{amalgamation systems} and were recently introduced by Anad, Schmitz,
Schütze, and Zetzsche \cite{ASZZ24}.

\AP Our first result, of theoretical nature, is that \kl{amalgamation systems}
cannot define \kl{well-quasi-ordered} languages that are not
\kl(language){bounded}. This implies that all the results of
\cref{infixes-regular:sec}, and in
particular \cref{bounded-language:thm}, can safely be transferred to
\kl{amalgamation systems}.

\begin{theorem}[label=infix-amalgamation:thm,restate=infix-amalgamation:thm]
    \proofref{infix-amalgamation:thm}
    Let $L \subseteq \Sigma^*$ be a language recognized by an 
    \kl{amalgamation system}.
    If $L$ is \kl{well-quasi-ordered} by the \kl{infix relation} then $L$ is
    \kl(language){bounded}.
\end{theorem}

\AP Our second focus is of practical nature: we want to give a decision
procedure for being \kl{well-quasi-ordered}. This will require us to introduce
\emph{effectiveness assumptions} on the \kl{amalgamation systems}. While most
of them will be innocuous, an important consequence is that we have to consider
\emph{classes of languages} rather than individual ones, for instance: the
class of all regular language, or the class of all context-free languages. Such
classes will be called \kl{effective amalgamative classes} (\kcref{effective
amalgamative classes}). In the following theorem, we prove that under such
assumptions, testing \kl{well-quasi-ordering} is inter-reducible to testing
whether a language of the class is empty.

\begin{theorem}[restate=infix-wqo-is-emptiness:thm,label=infix-wqo-is-emptiness:thm]
% 	\label{infix-wqo-is-emptiness:thm}
	Let $\mathcal{C}$ be an \kl{effective amalgamative class} of languages.
    Then the following are equivalent:
	\begin{enumerate}
        \item\label{wqo-infix-decidable} \kl[wqo]{Well-quasi-orderedness} of the \kl{infix relation} is decidable for languages in $\mathcal{C}$.
        \item\label{wqo-prefix-decidable} \kl[wqo]{Well-quasi-orderedness} of the \kl{prefix relation} is decidable for languages in $\mathcal{C}$.
        \item\label{emptiness-decidable} Emptiness is decidable for languages in $\mathcal{C}$.
	\end{enumerate}
\end{theorem}

\AP We say that a \kl{strongly effective amalgamative class} is an
\kl{effective amalgamative class} for which the emptiness problem is decidable.
Let us immediately remark that the class of \kl{regular languages} is
\kl(amalg){strongly effective}, and so is the class of context-free languages.
Therefore, \cref{infix-wqo-is-emptiness:thm} provides a concrete decision
procedure for these classes.

\subsection{Amalgamation Systems}
\label{amalgamation-systems:subsec}

Let us now formally introduce the notion of \kl{amalgamation systems}, and
recall some results from \cite{ASZZ24} that will be useful for the proof of
\cref{infix-amalgamation:thm}. The notion of \kl{amalgamation system} is
tailored to produce \emph{pumping arguments}, which is exactly what our
\cref{pumping-periods:lem} talks about. At the core of a pumping argument,
there is a notion of \emph{run}, which could for instance be a sequence of
transitions taken in a finite state automaton. Continuing on the analogy with
finite automata, there is a natural ordering between runs, i.e., a run is
smaller than another one if one can ``unroll loops'' of the first run to obtain
the second one. Typical pumping arguments then rely on the fact that
\emph{minimal} runs are of finite size, and that all other runs are
obtained by ``gluing'' minimal runs together. This is exactly what
\kl{amalgamation systems} are about.

\AP Let us recall that over an alphabet $(\Sigma, =)$ a \kl{subword embedding}
between two words $u \in \Sigma^*$ and $v \in \Sigma^*$ is a function $\rho
\colon \range{\card{u}} \to \range{\card{v}}$ such that $u_i = v_{\rho(i)}$ for
all $i \in \range{\card{u}}$. We write $\intro*\HigEmb(u,v)$ the set of all
\kl{subword embeddings} between $u$ and $v$. It may be useful to notice that
the set of finite words over $\Sigma$ forms a category when we consider
\kl{subword embeddings} as morphisms, which is a fancy way to state that
$\mathrm{id} \in \HigEmb(u,u)$ and that $f \circ g \in \HigEmb(u,w)$ whenever
$g \in \HigEmb(u,v)$ and $f \in \HigEmb(v,w)$, for any choice of words
$u,v,w \in \Sigma^*$.

\AP Given a \kl{subword embedding} $f \colon u \to v$ between two words $u$ and
$v$, there exists a unique decomposition $v = \GapWord{f}{0} u_1 \GapWord{f}{1}
\cdots \GapWord{f}{k-1} u_k \GapWord{f}{k}$ where $\GapWord{f}{i} =
v[f(i)+1:f(i+1)-1]$ for all $1 \leq i \leq k-1$, $\GapWord{f}{k} =
v[f(k)+1:\card{v}]$, and $\GapWord{f}{0}   = v[1: f(1)-1]$. We say that
$\intro*\GapWord{f}{i}$ is the $i$-th \intro{gap word} of $f$. We encourage the
reader to look at \cref{gap-word-embedding:fig} to see an example of the
\kl{gap words} resulting from a \kl{subword embedding} between two words. These
\kl{gap words} will be useful to describe how and where runs of a system
(described by words) can be combined.

\begin{figure}
    \centering
    \includestandalone[width=\linewidth]{fig/gap-word-embedding-standalone}
    \caption{The \kl{gap words} resulting from a \kl{subword embedding} between two 
    finite words.}
    \label{gap-word-embedding:fig}
\end{figure}

\begin{figure}
    \centering
    \includestandalone[width=\linewidth]{fig/run-amalgamation-standalone}
    \caption{We illustrate how 
        embeddings $f$ and $g$ between runs of an
        \kl{amalgamation system} can be glued
        together, seen on their canonical decomposition.
    }
    \label{amalgamation-runs:fig}
\end{figure}


\begin{definition}
    An \intro{amalgamation system}
    is a triple $(\Sigma, R, E, \canrun)$ where
    $\Sigma$ is a finite alphabet,
    $R$ is a set of so-called \emph{runs},
    and 
    $E$ is a set of so-called \emph{run embeddings},
    and $\canrun \colon R \to (\Sigma \uplus \set{\varepsilon})^*$ is a 
    function computing a \emph{canonical decomposition} of a run.
    Given a run $r \in R$, and $i \in \range[0]{\card{\canrun(r)}}$, 
    the \intro{gap language} of $r$ at position $i$ is $\GapLanguage{r}{i} \defined
    \setof{\GapWord{f}{i}}{\exists s \in R. \exists f \in E(r,s) }$.
    An \kl{amalgamation system} furthermore satisfies the following 
    properties:
    \begin{enumerate}
        \item \emph{Word Embeddings as Morphisms.} For all $r, s \in R$,
            $E(r,s) \subseteq \HigEmb(\canrun(r), \canrun(s))$.
        \item \emph{$(R, E)$ Forms a Category.}
            For all $r,s,t \in R$,
            $\mathrm{id} \in E(r,r)$,
            and whenever $f \in E(r,s)$ and $g \in E(s,t)$,
            then $g \circ f \in E(r,t)$.
        \item \emph{Well-Quasi-Ordered System.}
            $(R, \leq_E)$ is a well-quasi-ordered set,
            where $r \leq_E s$ if and only if $E(r,s) \neq \emptyset$.
        \item \emph{Controlled Amalgamation.}
            For all $r, s, t \in R$,
            for all $f \in E(r,s)$,
            for all $g \in E(r,t)$,
            and for all $0 \leq i_0 \leq \card{\canrun(r)}$,
            there exists a run $z \in R$ and morphisms
            $h \in E(s,z)$ and $k \in E(t,z)$
            satisfying
            $h \circ f = k \circ g$,
            $\GapWord{h \circ f}{i} = \GapWord{f}{i} \GapWord{g}{i}$
            or 
            $\GapWord{h \circ f}{i} = \GapWord{g}{i} \GapWord{f}{i}$
            for every $0 \leq j \leq n$,
            and
            $\GapWord{h \circ f}{i_0} = \GapWord{f}{i_0} \GapWord{g}{i_0}$.
            We refer to \cref{amalgamation-runs:fig} for an illustration 
            of this property.
    \end{enumerate}

    The \intro{amalgamation language} of such a system
    is the set of all words $w \in \Sigma^*$ such that
    there exists a run $r \in R$
    such that the concatenation of the letters of its
    canonical decomposition, written $\intro*\yieldrun(r)$,
    equals $w$.
\end{definition}

Intuitively, the definition of an amalgamation system allows the comparison of
runs, and the proper ``gluing'' of runs together to obtain new runs. Let us
recall some examples of languages that can be recognized by \kl{amalgamation
systems}: regular languages \cite[Theorem 5.3]{ASZZ24}, VASS as a consequence
of \cite[Theorem 5.5]{ASZZ24}, context-free languages as a consequence of
\cite[Theorem 5.10]{ASZZ24}. For this paper to be self-contained, we will also
recall how runs of a finite state automaton can be understood as an
\kl{amalgamation system}.

\begin{example}[{\cite[Section 3.2]{ASZZ24}}]
    Let $A = (Q, \delta, q_0, F)$ be a finite state automaton over a finite
    alphabet $\Sigma$. Let $\Delta$ be the set of transitions $(q_1, a, q_2)
    \in Q \times \Sigma \times Q$,
    and $R \subseteq \Delta^*$ be the set of 
    words over transitions that start with the initial state $q_0$,
    end in a final state $q_f \in F$, and such that the end state of a
    letter is the start state of the following one.
    The canonical decomposition $\canrun \colon R \to \Sigma^*$
    is defined as a morphism from $\Delta^*$ to $\Sigma^*$
    that maps $(q,a,p)$ to $a$.
    Finally, given two runs $r$ and $s$ of the automaton,
    we say that an embedding $f \in \HigEmb(\canrun(r), \canrun(s))$
    belongs to $E(r,s)$ when
    $f$ is also defining an embedding from $r$ to $s$ as words in $\Delta^*$,
    where $\Delta$ is equipped with the equality relation.

    The system $(\Sigma, R, E, \canrun)$ is an \kl{amalgamation system},
    whose language is precisely the language of words recognized
    by the automaton $A$.
\end{example}
\begin{proof}
    By definition, the embeddings inside $E(r,s)$ are defined as elements
    of $\HigEmb(\canrun(r), \canrun(s))$, and they compose properly.
    Because $\Delta = Q \times \Sigma \times Q$ is finite, it is 
    a \kl{well-quasi-ordering} when equipped with the equality relation, and 
    we conclude that $\Delta^*$ with $\higleq$ is a \kl{well-quasi-order}
    according to Higman’s Lemma \cite{HIG52}.
    
    Let us now move to proving that the system satisfies the amalgamation
    property. Given three runs $r,s,t \in R$, and two embeddings $f \in E(r,s)$
    and $g \in E(r,t)$, we want to construct an amalgamated run $s \vee t$.
    Because letters in the run $r$ respect the transitions of the automaton
    (i.e., if the letter $i$ ends in state $q$, then the letter $i+1$ starts in
    state $q$), then the \kl{gap word} at position $i$ starts in state $q$ and
    ends in state $q$ too. This means that for both embeddings
    $f$ and $g$, the \kl{gap words} are read by the automaton by looping
    on a state. In particular, these loops can be taken in any order
    and continue to represent a valid run. That is, we can even select
    the order of concatenation in the amalgamation for \emph{all} 
    $0 \leq i \leq \card{\canrun(r)}$ and not just for one separately.

    Let us now remark that 
    the language of this amalgamation system is
    the set of $\yieldrun(r)$ when $r$ ranges over $R$,
    and because $R$ is the set of valid runs of the automaton,
    and $\yieldrun(r)$ is the word read along this run,
    we immediately conclude.
\end{proof}

\subsection{Amalgamation Systems, WQOs, and Bounded Languages}

Let us now introduce a combinatorial lemma that explains how \kl{gap languages}
can be pumped. Note that a single \kl{gap language} is always stable under
concatenation, and our concern will be on a potential description of
\emph{simultaneously} pumping different \kl{gap languages} to produce valid
runs.

\begin{lemma}
    \label{pumping-gap-languages:lem}
    Let $(\Sigma, R, E, \canrun)$ be an \kl{amalgamation system}, let $r \in R$
    be a run of size $n \in \Nat$, and let $s, t \in R$ be two runs
    with embeddings $f \in E(r,s)$ and $g \in E(r,t)$.
    Let $\seqof{u_i}[0 \leq i \leq n]$ be the sequence of \kl{gap words}
    for $f$, that is, $u_i =
    \GapWord{f}{i}$ for all $0 \leq i \leq n$. Similarly, let
    $\seqof{v_i}[0 \leq i \leq n]$ be the sequence of \kl{gap words} for $g$.

    Then, for all $k \in \Nat$, and $0 < \ell < n$, the following words belong to 
    the language of the \kl{amalgamation system}.
    \begin{enumerate}
        \item $(\prod_{i = 0}^{\ell - 1} u_i^{x_i} v_i u_i^{y_i} v_i u_i^{z_i}) 
               v_\ell u_\ell^k v_\ell
               (\prod_{i = \ell+1}^{n} u_i^{x_i} v_i u_i^{y_i} v_i u_i^{z_i})$,
        \item $(\prod_{i = 0}^{n - 1} u_i^{x_i} v_i u_i^{y_i}) 
               v_n u_n^k$,
        \item $u_0^k v_0 
            (\prod_{i = 1}^{n} u_i^{x_i} v_i u_i^{y_i})$.
    \end{enumerate}
    Where, for all $0 \leq i \leq n$, $x_i + y_i + z_i = k$, with $z_i = 0$
    in the two last items.
\end{lemma}
\begin{proof}
    The result immediately follows from the embedding property
    applied inductively on runs obtained by gluing 
    $k$ times $s$ with one or two times $t$ (depending on the item number).
\end{proof}

To conclude this section and prove our main \cref{infix-amalgamation:thm}, it
suffices to demonstrate that languages recognized by \kl{amalgamation systems}
that are \kl{well-quasi-ordered} for the \kl{infix relation} are \kl{bounded
languages}.

\begin{proofof}{infix-amalgamation:thm}
    Assume that $L$ is \kl{well-quasi-ordered} by the \kl{infix relation},
    and obtained by an \kl{amalgamation system}
    $(\Sigma, R, E, \canrun)$.

    Let us consider the set $M$ of minimal runs for the relation $\leq_E$,
    which is finite because the latter is a \kl{well-quasi-ordering}. By
    definition, for every word $w \in L$, there exists $r \in M$, $s \in R$,
    and $f \in E(r,s)$ such that $w = \canrun(s)$.
    In particular, we conclude that
    \begin{equation*}
        L \subseteq \bigcup_{r \in M} 
        \left(
        \GapLanguage{r}{0}
        \prod_{i = 1}^{\card{\canrun(r)}-1} \canrun(r)_i \GapLanguage{r}{i} 
        \right)
        \quad .
    \end{equation*}

    Let $u$ and $v$ be two words in the \kl{gap language}
    $\GapLanguage{r}{\ell}$ for some $r \in M$ and $0 < \ell <
    \card{\canrun(r)} \defined n$. One can apply \cref{pumping-gap-languages:lem}
    to conclude that  for all $k \geq 1$,
    \begin{equation*}
       \left(\prod_{i = 0}^{\ell - 1} u_i^{x_i} v_i u_i^{y_i} v_i u_i^{z_i}\right) 
       v u^k v
       \left(\prod_{i = \ell+1}^{n} u_i^{x_i} v_i u_i^{y_i} v_i u_i^{z_i}\right)
       \in 
       L
    \end{equation*}
    Where $x_i + y_i + z_i = k$ for all $0 \leq i \leq n$
    Let us assume without loss of generality that the sequence
    $\seqof{(x_i, y_i, n_i)}[i \geq 1]$ is pointwise increasing,
    and even increasing coordinate wise. This is possible because
    at least one of the coordinates tends to $+\infty$ and in the case
    of a variable that is bounded, one can extract the sequence to
    simply not use this variable.

    Now, leveraging \cref{pumping-periods:lem}, we get that for infinitely many
    $k \geq 1$, $u v^k u$ is an infix of some $x^p$ where $x$ has size smaller
    or equal than $u$ and $v$. This proves  that $u$ and $v$ are comparable for
    the prefix, infix, and suffix relations. In particular, we
    conclude that $\GapLanguage{r}{i}$ is a \kl{bounded language}, because of
    \cite[Lemma 4.1]{ASZZ24}, when $0 < i < \card{\canrun(r)}$. The proof goes
    on similarly for proving that $\GapLanguage{r}{0}$ and $\GapLanguage{r}{n}$
    are \kl{bounded languages}, using the other items of
    \cref{pumping-gap-languages:lem}.

    Because \kl{bounded languages} are stable under concatenation and finite
    unions, we conclude that $L$ itself must be a \kl{bounded language}.
\end{proofof}

Leveraging a similar reasoning, we conclude that being a \kl{bounded language},
a \kl{regular language}, or being included in a finite union of products of
\kl{chains} all collapse for \kl{well-quasi-ordered} and \kl{downwards closed}
languages for the \kl{infix relation}.

\begin{lemma}
    \label{dwclosed-infixes-wqo:lem}
    Let $L \subseteq \Sigma^*$ be a \kl{downwards closed} language for the
    \kl{infix relation} that is \kl{well-quasi-ordered}. Then, the following
    are equivalent:
    {\renewcommand{\theenumi}{\roman{enumi}}
     \renewcommand{\labelenumi}{(\theenumi)}
    \begin{enumerate}
        \item\label{dwci-reg:item} $L$ is a \kl{regular language},
        \item\label{dwci-aml:item} $L$ is recognized by \emph{some} \kl{amalgamation system},
        \item\label{dwci-bod:item} $L$ is a \kl{bounded language},
        \item\label{dwci-uoc:item} $L$ is included in a finite union of products of \kl{chains}
    \end{enumerate}
    }
\end{lemma}
\begin{proof}
    It is clear that \cref{dwci-reg:item} $\Rightarrow$ \cref{dwci-aml:item}
    because regular languages are recognized by finite automata, and finite
    automata are a particular case of \kl{amalgamation systems}.
    The implication \cref{dwci-aml:item} $\Rightarrow$ \cref{dwci-bod:item}
    is the content of \cref{infix-amalgamation:thm}.
    The implication \cref{dwci-bod:item} $\Rightarrow$ \cref{dwci-uoc:item}
    is \cref{bounded-language:thm}.
    Finally, the implication \cref{dwci-uoc:item} $\Rightarrow$ \cref{dwci-reg:item}
    is simply because a \kl{downwards closed} language 
    that is a finite union of products of \kl{chains} is a regular language.
\end{proof}

\AP Combining
\cref{thue-morse:lemma,dwclosed-infixes-wqo:lem}, we can
conclude that the collection of \kl{infixes} of the \kl{Thue-Morse sequence}
cannot be recognized by \emph{any} \kl{amalgamation system}. To put this result
in context, it was proven that the complement of the set $T$ of prefixes of the
\kl{Thue-Morse sequence} is context-free, and conjectured that the same holds
for the complement of $\LMorse$ \cite{BERST86}. To the knowledge of the
authors, this conjecture remains open. Our remark provides a negative answer to
this conjecture for the language $\LMorse$.


\subsection{Effective Decision Procedures}
\label{infixes-amalgamation-effective:subsec}

\AP Let us now introduce our effectiveness assumptions on \kl{amalgamation
systems}. We follow the approach of \cite{ASZZ24} and require that an
\kl{amalgamation system} $(\Sigma, R, E, \canrun)$ is effective when $R$ is
recursively enumerable, the function $\canrun(\cdot)$ is computable, and for
any two runs $r, s \in R$, the set $E(r,s)$ is computable.

\todo[inline]{In this definition, the alphabet $\Sigma$ can be changed 
(and we seem to actually use it in the proofs), it is strange.}

\AP We say that a class $\mathcal{C}$ of languages is an \intro{effective
amalgamative class} whenever for every $L \in \mathcal{C}$, there exists an
effective \kl{amalgamation system} recognizing $L$, and such that $\mathcal{C}$
is \kl{effectively closed under rational transductions}. Recall that a
\intro{rational transduction} is a \emph{realtion} $R \subseteq \Sigma^* \times
\Gamma^*$ that can be defined by a (nondeterministic) finite automaton with
output \cite[Chapter 5, page 64]{BERST79}. A class of languages $\mathcal{C}$
is \intro{effectively closed under rational transductions} when, for every
language $L \in \mathcal{C}$, and every rational transduction $R \subseteq
\Sigma^* \times \Gamma^*$, the image of $L$ through $R$ --- that is,
 $\setof{v \in \Gamma^*}{\exists u \in L. (u,v) \in R}$ --- is in $\mathcal{C}$
and effectively computable.

\AP A class $\mathcal{C}$ of languages is \intro(amalg){strongly effective}
whenever the emptiness problem for languages in $\mathcal{C}$ is decidable.
This notion is interesting because usual language problems such as boundedness
or simultaneous boundedness are decidable for \kl{strongly effective
amalgamative classes}~\cite{ASZZ24}. 

\begin{proofof}{infix-wqo-is-emptiness:thm}
	\cref{emptiness-decidable} $\Rightarrow$ \cref{wqo-infix-decidable}. We aim to make the inclusion test of \cref{bounded-language:eq} effective. 
	
    Let $R(n,m,N_0) \defined \bigcup_{x,y \in \Sigma^{\leq n}} \bigcup_{u \in
    \Sigma^{\leq m \times N_0}} \InfPeriodChain{x} u \InfPeriodChain{y} \cup
    \InfPeriodChain{x}u \cup u\InfPeriodChain{x}$. For any concrete values of
    the bounds $n$, $m$, and $N_0$, this language is regular. The map $L
    \mapsto L \cap \Sigma^* \setminus R(n,m,N_0)$  is a \kl{rational
    transduction} because $\Sigma^* \setminus R(n,m,N_0)$ is regular. Since
    $\mathcal{C}$ is \kl{closed under rational transductions}, we can therefore
    reduce the inclusion to emptiness of this language. However, we need to
    find these bounds first.
	
    To determine values for $n$ and $m$, we first test if $L$ is
    \kl(language){bounded}. Since emptiness is decidable, we can apply the
    algorithm in~\cite[Section 4.2]{ASZZ24} to decide if $L$ is
    \kl(language){bounded}. If $L$ is \kl(language){bounded}, this algorithm
    yields words $w_1, \ldots w_n$ such that $L \subseteq w_1^* \cdots w_n^*$
    and therefore yields also the bounds in questions: $n$ is the number of
    words, and $m$ is the maximal length of a word $w_i$ where $1 \leq i \leq
    n$. If $L$ is not bounded, then $L$ cannot be \kl{well-quasi-ordered} by
    the \kl{infix relation} because of \cref{infix-amalgamation:thm} and we
    immediately return false.
	
    To determine the value for $N_0$, we first computer maximal subsets $S
    \subseteq [n]$ such that $w_i$ for $i \in S$ are simultaneously
    unbounded.\todo{define this} We do this by applying a transduction mapping
    $w_i$ to some fresh letter $a_i$ if $i \in S$ and $\varepsilon$ otherwise,
    and testing for simultaneous unboundedness \cite[Section 4.1]{ASZZ24}.
    Given a candidate bound $n_0$ for $N_0$, we construct the languages
    $L_{S,n_0} = w_1^{\circ_1}\cdots w_n^{\circ_n}$ where $\circ_i = *$ if $i
    \in S$ and $\circ_i = \leq n_0$ otherwise. We then test if $L \subseteq
    \bigcup_{S \text{ maximal}} L_{S,n_0}$. If yes, $n_0$ is our value for
    $N_0$, otherwise, we increase it and repeat the construction. As we chose
    maximal sets $S$, this procedure will eventually terminate.

    \cref{wqo-infix-decidable} $\Rightarrow$
    \cref{wqo-prefix-decidable}. We just consider the transduction $f$
    that maps every word $w$ to $\# w$ where $\# $ is a fresh symbol. Then, for
    any language $L \in \mathcal C$, $L$ is \kl{well-quasi-ordered} by
    \kl{prefix} if and only if $f(L)$ is \kl{well-quasi-ordered} by \kl{infix}.
    \todo{fresh symbols? can we assume we have those??}
	
    \cref{wqo-prefix-decidable} $\Rightarrow$
    \cref{emptiness-decidable}. 
	We 
	consider the transduction $R \defined \Sigma^* \times \set{a, b}^*$. Then 
	for any language $L \in \mathcal C$,
    the image of $L$ through $R$ is \kl{well-quasi-ordered}
    by \kl{infix} if and only if $L$ is empty.
\end{proofof}

The class $\mathcal{C}_\text{aut}$ of \kl{regular languages} and the class
$\mathcal{C}_{\text{cfg}}$ of context-free languages are examples of
\kl{effective amalgamative classes}, hence the following corollary.

\begin{corollary}
    \label{aut-cfg-infix:cor}
    Let $\mathcal{C} \in \set{ \mathcal{C}_\text{aut}, \mathcal{C}_{\text{cfg}}}$.
    It is decidable whether a language in $\mathcal{C}$ is \kl{well-quasi-ordered}
    by the \kl{infix relation}.
    Furthermore, whenever it is \kl{well-quasi-ordered} by the \kl{infix relation},
    it is a \kl{bounded language}.
\end{corollary}

Let us conclude by noting that it is unsurprisingly not possible to decide
whether a decidable language is \kl{well-quasi-ordered} by the \kl{prefix
relation}. This is a very easy result whose sole purpose is to contrast with
the decidability result of \cref{aut-cfg-infix:cor}.

\begin{remark}
    The following problem is undecidable: given a language $L$
    decided by a Turing machine, answer whether 
    $L$ is \kl{well-quasi-ordered} for the \kl{prefix relation}.
\end{remark}
\begin{proof}
    We reduce the halting problem on the empty string $\varepsilon$.
    Let $M$ be a Turing Machine, we write the languages $L$ of finite runs
    of $M$ starting on the empty string,
    that we surround by special markers. This language is decidable,
    and 
    is
    \kl{well-quasi-ordered} if and only if it is finite
    if and only if $M$ terminates on $\varepsilon$.
\end{proof}

% LTeX: language=en-GB 
% !TeX root=../wqo-on-words.tex
\section{Conclusion}
\label{conclusion:sec}

We provided concretes statements that justify why the \kl{subword relation}
is used when defining \kl{well-quasi-orders} on finite words. Even if
\kl{prefix}, \kl{suffix} or \kl{infix} relations are meaningful, they are
\kl{well-quasi-ordered} if and only if they behave similarly to disjoint copies
of $\Nat$ or $\Nat^2$. However, our approach suffers some limitations 
and opens the road to natural continuation of this line of work.

\paragraph*{Towards infinite alphabets} In this paper, we restricted our
attention to \emph{finite} alphabets, having in mind the application to
\kl{regular languages}. However, the conclusions of
\cref{bounded-language:thm}, \cref{small-ordinal-invariants:thm}, and
\cref{prefixes:thm} could be conjecture to hold in the case of infinite
alphabets (themselves equipped with a \kl{well-quasi-ordering}). This would
require new techniques, as the finiteness of the alphabet is crucial to all of
our positive results.

\paragraph*{Monoid equations}  It could be interesting to understand which
monoids $M$ recognize languages that are \kl{well-quasi-ordered} by the
\kl{infix}, \kl{prefix} or \kl{suffix} relations. This research direction is
connected to finding which classes of graphs of \emph{bounded clique-width} are
\kl{well-quasi-ordered} with respect to the \emph{induced subgraph relation},
as shown in \cite{DRT10}, and recently revisited by one of the authors in
\cite{L24:arxiv:v2}.

\paragraph*{Complexity} We have chosen to disregard complexity considerations
when proving decidability results, as we do not believe that a fined grained
complexity analysis would be particularly enlightening. However, we conjecture
that for regular languages, the complexity should be polynomial in the size of
the minimal automaton recognizing the language.


\paragraph*{Lexicographic orderings} There is another natural ordering on
words, the \emph{lexicographic ordering}, which does not fit well in our
current framework because it is always of \kl{ordinal width} $1$. However, the
order-type of the lexicographic ordering over \kl{regular languages} has
already been investigated in the context of infinite words \cite{CACOPU18}, and
it would be interesting to see if one can extend these results to decide
whether such an ordering is \kl{well-founded} for languages recognized by
\kl{amalgamation systems}.


\subsubsection{Factor Complexity}

\AP Let us conclude this section with a few remarks on the notion of \kl{factor
complexity} of languages. Recall that the \intro{factor complexity} of a
language $L \subseteq \Sigma^*$ is the function $f_L: \Nat \to \Nat$ such that
$f_L(n)$ is the number of distinct words of size $n$ in $L$. A language $L$ has
\intro{exponential factor complexity} if there exists a constant $C > 1$ such
that $f_L(n) \geq C^n$ for infinitely many $n \in \Nat$. A language $L$ has
\intro{subaffine factor complexity} if $f_L(n) \leq C \cdot n + K$ for some
constants $C, K > 0$ and all $n \in \Nat$.
We extend the notion of \kl{factor complexity} to infinite and bi-infinite words
as the \kl{factor complexity} of their set of finite \kl{infixes}.

While there clearly are languages with \kl{subaffine factor complexity} that are
not \kl{well-quasi-ordered} for the \kl{infix relation}, such as 
the language $L \defined \dwset{ a b^* a }$;
one would expect that languages that are \kl{well-quasi-ordered} for the
\kl{infix relation} would have a low \kl{factor complexity}.

\AP It is the case that \kl{bounded languages}, studied in
\cref{infixes-bounded:sec}, have a polynomial factor complexity: such a
language is by definition included in $w_1^* \cdots w_n^*$ for some words $w_1,
\dots, w_n$, and an infix is a choice of the number of occurrences of each
$w_i$ in the infix, plus a choice of beginning and ending of the infix in the
leftmost and rightmost $w_i$. We showed in \cref{bounded-language:thm} and
particularly in \cref{bounded-language:lem}, that \kl{bounded languages} that
are \kl{well-quasi-ordered} for the \kl{infix relation} are in fact included in
some finite union of languages of the form $w_1^* w_2 w_3^*$, hence have
actually a quadratic factor complexity. However,
\cref{exponential-factor-complexity:lem} shows that there are
\kl{downwards-closed} languages that are \kl{well-quasi-ordered} for the
\kl{infix relation} but have an \kl{exponential factor complexity}.


\begin{lemma}
  \label{exponential-factor-complexity:lem}
  There exists a language $L$ with \kl{exponential factor complexity}
  that is \kl{well-quasi-ordered} for the \kl{infix relation}.
\end{lemma}
\begin{proof}
  We leverage results from \cite{CAKA97} on Toeplitz words.
  Namely, the 
  $(5, 3)$-Toeplitz word is \kl{uniformly recurrent} 
  \cite[p. 499]{CAKA97},
  but has an exponential factor complexity 
  \cite[Theorem 5]{CAKA97}.
\end{proof}


Our understanding of the situation is that whenever a computational model is
fixed to represent languages that are \kl{well-quasi-ordered} for the \kl{infix
relation}, the \kl{factor complexity} of the represented languages drops to a
quadratic one. This is the case for languages recognized by \kl{amalgamation
systems} \cref{amalgamation-systems:subsec}, that are \kl{bounded languages}
when they are \kl{well-quasi-ordered} (\cref{infix-amalgamation:thm}). 
This is
also the case for languages described as the infixes of a finite set of pairs
of \kl{morphic sequences}. Indeed, the factor complexity of a \kl{morphic
sequence} that is \kl{uniformly recurrent} is linear \cite[Theorem 24]{NIPR09}.
and therefore the factor complexity of the set of infixes of a bi-infinite word
obtained as $\bar{w_1} u w_2$ where $w_1, w_2$ are \kl{morphic sequences} and
$u$ is a finite word is quadratic.




% Include the bibliography
\bibliographystyle{plainurl}
\bibliography{papers.bib}

% If there are any appendices, we include them here.


\end{document}

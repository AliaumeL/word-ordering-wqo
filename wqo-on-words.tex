%! TEX program = pdflatex
% WARNING: this is a generated file.
%
% Please do not edit this file directly. 
% - If you want to update the medatata of the paper (title, authors, abstract), please
%   edit the `paper-meta.yaml` file in the root of the repository.
% - If you want to update the content of the paper, please edit the latex files
%   in the `src` directory.
% - If you want to update the template itself (e.g., change the layout), please
%   edit the `templates/plain-article.tex` file instead.
\documentclass[11pt,a4paper,twosided]{article}

% we setup a custom geometry because the default one is too narrow
\usepackage{geometry}
\geometry{margin=3.5cm}

% utf-8 for old systems
\usepackage[utf8]{inputenc}
\usepackage[T1]{fontenc}

% babel for language settings
\usepackage[english]{babel}

% microtype for better typography
\usepackage{microtype}

\usepackage{todonotes}
\usepackage{lineno}
\linenumbers



% math packages
\usepackage{amsmath,amsthm,amssymb,stmaryrd,thmtools,upgreek}

% configure some theorems
\newtheorem{theorem}{Theorem}
\newtheorem{lemma}[theorem]{Lemma}
\newtheorem{corollary}[theorem]{Corollary}
\newtheorem{proposition}[theorem]{Proposition}
\newtheorem{conjecture}[theorem]{Conjecture}
\newtheorem{assumption}[theorem]{Assumption}
\theoremstyle{definition}
\newtheorem{definition}[theorem]{Definition}
\newtheorem{remark}[theorem]{Remark}
\newtheorem{example}[theorem]{Example}



% graphics packages
\usepackage{graphicx}
\usepackage{subcaption}
\usepackage[obeyclassoptions,mode=tex]{standalone}
\usepackage{tikz}
\usetikzlibrary{backgrounds}
\usetikzlibrary{shapes.geometric}
\usetikzlibrary{positioning}
\usetikzlibrary{automata}
\usetikzlibrary{tikzmark}
\usetikzlibrary{patterns}
\usetikzlibrary{arrows}
\tikzset{every state/.style={minimum size=1pt}}
\usepackage{tikz-cd}

% ornaments
\usepackage{pgfornament}


% links inside the document
\usepackage{hyperref}
\usepackage[capitalise,noabbrev,nameinlink]{cleveref}
\Crefname{assumption}{Assumption}{Assumptions}
\usepackage[composition,hyperref,xcolor,cleveref]{knowledge}
\knowledgeconfigure{notion}

% Tables 
\usepackage{booktabs}
\usepackage{varwidth}

% Algorithms
\usepackage{algorithm2e}
\Crefname{algocfline}{Algorithm}{Algorithms}
\crefname{algocfline}{Algorithm}{Algorithms}
\crefname{algocf}{Algorithm}{Algorithms}
\Crefname{algocf}{Algorithm}{Algorithms}

% Packages for macro definitions
\usepackage{xparse}
\usepackage{xpatch}
\usepackage{tokcycle}
\usepackage{ifthen}

% Proofs
\usepackage{bussproofs}

% Colors 
\usepackage{ensps-colorscheme}
% parametrize knowledge colors and style
% intro => A4 + emph 
% kl    => normal color, normal size not emph
\knowledgestyle{intro notion}{color={A2}, emphasize}
\knowledgestyle{notion}{color={B1}}
\hypersetup{
    colorlinks=true,
    breaklinks=true,
    linkcolor=A4,
    citecolor=A4,
    urlcolor=A4,
    filecolor=A4,
}


% we include whatever the user wants to include in the header

% we include libraries (tex files) usually written in the `lib` directory


% Upgreek letters
\makeatletter
\newcommand\mathgr[1]{\tokcycle
  {\addcytoks{##1}}
  {\processtoks{##1}}
  {\ifcsname up\expandafter\@gobble\string##1\endcsname
   \addcytoks[1]{\csname up\expandafter\@gobble\string##1\endcsname}%
    \else\addcytoks{##1}\fi}
  {\addcytoks{##1}}{#1}%
  \expandafter\mathrm\expandafter{\the\cytoks}%
}
\makeatother



% Create a new macro proofof
% taking as input a label of a theorem
% and creating a proof with a reference to that
% label
\NewDocumentEnvironment{proofof}{ m }{
    % if the command \#1 exists, then 
    % call \#1* to restate the theorem
    \ifcsname #1\endcsname
        \def\isInsideRestatedTheorem{1}
        \csname #1\endcsname*
    \fi
    \begin{proof}[Proof of {\cref{#1}} page {\pageref{#1}}]
        \phantomsection
        \label{#1:proof}
}{
        
        % Some link to go back to the theorem
        \noindent\hyperref[#1]{$\triangleright$ Go back to \cref{#1} on page \pageref{#1}}
    \end{proof}
}

% Create a new macro proofref
% that takes as input a label of a theorem
% and creates a reference to its proof
\NewDocumentCommand{\proofref}{ m }{
    % checks if the label #1:proof exists, if yes
    % it creates a link to it, otherwise it writes nothing
    \IfRefUndefinedExpandable{#1:proof}{}{
        % Checks if we are inside a restated theorem
        % if yes, we do not print anything
        \ifdefined\isInsideRestatedTheorem
        \else
        \hfill\hyperref[#1:proof]{\textsf{(Go to proof p.\pageref{#1:proof})}}
        \fi
    }
}



% Automate the creation of new orderings
% based on a given symbol.
% For instance,
% \NewDocumentOrdering{\pref}{\preceq}{\prec}
% will create the following commands:
% \prefleq and \preflt
% that will respectively expand to
% \mathrel{\kl[\pref]{\preceq}} and \mathrel{\kl[\pref]{\prec}}
\NewDocumentCommand{\NewDocumentOrdering}{ m m m }{
    \expandafter\newcommand\csname #1leq\endcsname{
        \mathrel{\kl[#1]{#2}}
    }
    \expandafter\newcommand\csname #1lt\endcsname{
        \mathrel{\kl[#1]{#3}}
    }
    \knowledge{#1}{notion}
}

% Little math macros
\NewDocumentCommand{\set}{ m }{\{ #1 \}}
\NewDocumentCommand{\setof}{ m m }{\{ #1 \mid #2 \}}
\NewDocumentCommand{\card}{ m }{\left| #1 \right|}
\NewDocumentCommand{\Nat}{ }{\mathbb{N}}

\NewDocumentOrdering{pref}{\sqsubseteq_{\mathsf{pref}}}{\sqsubset_{\mathsf{pref}}}
\NewDocumentOrdering{suff}{\sqsubseteq_{\mathsf{suff}}}{\sqsubset_{\mathsf{suff}}}
\NewDocumentOrdering{inf}{\sqsubseteq_{\mathsf{infix}}}{\sqsubset_{\mathsf{infix}}}
\NewDocumentOrdering{hig}{\leq^*}{<^*}

\NewDocumentCommand{\InfPeriodChain}{ m }{
    \mathop{\kl[\InfPeriodChain]{\mathsf{PI}\!\!\!\!\downarrow}}
    ( #1 )
}
\knowledge{\InfPeriodChain}{notion}

\NewDocumentCommand{\canrun}{O{can}}{
    \mathop{\mathsf{#1}}
}
\NewDocumentCommand{\yieldrun}{}{
    \mathop{\mathsf{yield}}
}

\NewDocumentCommand{\HigEmb}{}{
    \mathop{\kl[\HigEmb]{\mathsf{Hom}^*}}
}
\knowledge{\HigEmb}{notion}

\NewDocumentCommand{\GapWord}{ m m }{
    \mathop{\kl[\GapWord]{\mathsf{G}}}_{#2}^{#1}
}
\knowledge{\GapWord}{notion}

\NewDocumentCommand{\GapLanguage}{ m m }{
    \mathop{\kl[\GapLanguage]{\mathsf{L}}}_{#2}^{#1}
}
\knowledge{\GapLanguage}{notion}

\input{knowledges.kl}

% We include the title and author information based on the 
% `paper-meta.yaml` file.
 
\title{Well-quasi-orderings on word languages}

\author{
Nathan Lhote\thanks{Aix-Marseille University}
 \and
Aliaume Lopez\thanks{University of Warsaw}
 \and
Lia Schütze\thanks{Max Planck Institute for Software Systems}
}

% For the date, we first check if the user has provided a date,
% and otherwise use the git meta inforamtion (if available).
\date{2025-07-01 13:26:43 +0200\footnote{99064007a226e55157ecfac20bbb40804687f3e2 -- branch main at git@github.com:AliaumeL/word-ordering-wqo.git}}

\newcommand{\repositoryUrl}{\url{https://github.com/AliaumeL/word-ordering-wqo}}

\knowledge{notion}
 | kl-usage

% Now, we create the document itself.
\begin{document}
% Generate the title page
\maketitle
% Print the abstract
\begin{abstract}
    The set of finite words over a well-quasi-ordered set is itself well-quasi-ordered. This seminal result by Higman is a cornerstone of the theory of well-quasi-orderings and has found numerous applications in computer science. However, this result is based on a specific choice of ordering on words, the (scattered) subword ordering. In this paper, we describe to what extent other natural orderings (prefix, suffix, and infix) on words can be used to derive Higman-like theorems. More specifically, we are interested in characterizing \emph{languages} of words that are well-quasi-ordered under these orderings. We show that a simple characterization is possible for the prefix and suffix orderings, and that under extra regularity assumptions, this also extends to the infix ordering. We furthermore provide decision procedures for a large class of languages, that contains regular and context-free languages.
    \paragraph{Keywords:}
    equivariant ideal, Hilbert basis, ideal membership problem, orbit finite, oligomorphic, well-quasi-ordering
\paragraph{Repository:} \repositoryUrl
\end{abstract}

\klogo\ This document uses \href{https://ctan.org/pkg/knowledge}{knowledge}:
\kl[kl-usage]{notion} points to its \intro[kl-usage]{definition}. 


% Include the content of the paper
\section{Introduction}
\label{introduction:sec}

A \intro{well-quasi-ordered} set is a set $X$ equipped with a quasi-order
$\preceq$ such that every infinite sequence $x_0 \preceq x_1 \preceq x_2
\preceq \ldots$ contains a pair $x_i \preceq x_j$ with $i < j$.
Well-quasi-orderings serve as a core combinatorial tool powering many
termination arguments, and in particular applies to the verification of
infinite state transition systems \cite{ABDU96,ABDU98}.

One of the appealing properties of well-quasi-orderings is that they are closed
under many operations, such as taking products, finite unions, and finite
powersets \cite{SCSC12}. More surprisingly, the class of well-quasi-ordered
sets is also stable under the operation of taking finite words and finite trees
labelled by elements of a well-quasi-ordered set \cite{HIG52,KRU72}. 

Note that in these latter cases, the precise choice of ordering on words and
trees is crucial to ensure that the resulting structure is well-quasi-ordered.
A celebrated result of Higman implies that the set of finite words over a
finite alphabet is well-quasi-ordered by the so-called \kl{subword relation}
\cite{HIG52}. The subword relation for words over $(X, \preceq)$ is defined as
follows: a word $u$ is a \intro{subword} of a word $v$ if there exists an
increasing function $f \colon \{1, \ldots, |u|\} \to \{1, \ldots, |v|\}$ such
that $u_i \preceq v_{f(i)}$ for all $i \in \{1, \ldots, |u|\}$.

There are many other natural orderings on words that could be considered in the
context of well-quasi-orderings. For instance, one could consider the
\kl{prefix relation}, \kl{suffix relation} or the \kl{infix relation}, all of
which are not well-quasi-ordered as soon as the alphabet contains two distinct
letters. Let us remark that these other word orderings are however
well-founded, and therefore the (only) obstruction to being well-quasi-ordered
is not due to the existence of infinite sets of incomparable elements.


\subparagraph{Contribution} In this paper, we characterize for which languages
$L \subseteq \Sigma^*$ the other natural orderins on word are
well-quasi-ordered. When languages are decided by simple enough machines (e.g.,
finite automata or pushdown automata), we will also provide \emph{decision
procedures} to check whether the language is well-quasi-ordered by the
considered relation. 

\subparagraph{Outline}

\section{Preliminaries}
\label{prelims:sec}

\paragraph*{Finite words.} In this paper, we use letters $\Sigma, \Gamma$ to
denote finite alphabets, $\Sigma^*$ to denote the set of finite words over
$\Sigma$, and $\varepsilon$ for the empty word in $\Sigma^*$. In order to give
some intuition on the decision problems, we will sometimes use the notion of
finite automata, regular languages, and Monadic Second Order ($\intro*\MSO$)
logic on finite words, and assume is familiar with them. We refer to the
textbook of \cite{THOM97} for a detailed introduction to this topic.

\paragraph*{Well-quasi-orders.} Let us introduce some notations for
\kl{well-quasi-orders}. A sequence $\seqof{x_i}$ in a set $X$ is
\intro(sequence){good} if there exist $i < j$ such that $x_i \leq x_j$. It is
\intro(sequence){bad} otherwise. Therefore, a \kl{well-quasi-ordered} set is a
set where every infinite sequence is \kl(sequence){good}. A \intro{decreasing
sequence} is a sequence $\seqof{x_i}$ such that $x_{i+1} < x_i$ for all $i$,
and an \intro{antichain} is a set of pairwise incomparable elements. An
equivalent definition of a \kl{well-quasi-ordered} set is that it contains no
infinite \kl{decreasing sequences}, nor infinite \kl{antichains}. We refer to
\cite{SCSC12} for a detailed survey on well-quasi-orders.

Let us point out that the \kl{prefix relation} (resp. the \kl{suffix relation}
and the \kl{infix relation}) on $\Sigma^*$ are always \kl{well-founded}, i.e.,
there are no infinite \kl{decreasing sequences} for this ordering. In
particular, for a language $L \subseteq \Sigma^*$ to be
\kl{well-quasi-ordered}, it suffices to prove that it contains no infinite
\kl{antichain}. 

\paragraph*{Ordinal Invariants.} An ordinal is a well-founded ordered set. We
use $\alpha, \beta, \gamma$ to denote ordinals, and use $\omega$ to denote the
first infinite ordinal, i.e., the set of natural numbers with the usual
ordering. We assume superficial familiarity with ordinal arithmetic, and refer
to \cite{KUNEN80} for a detailed introduction to this domain. Given a tree $T$
whose branches are all finite we can define an ordinal $\alpha_T$ inductively
as follows: if $T$ is a leaf then $\alpha_T = 0$, if $T$ has children
$\seqof{T_i}$ then $\alpha_T = \sup \setof{\alpha_{T_i} + 1}{i \in \Nat}$. We
say that $\alpha_T$ is the \emph{rank} of $T$. 

Let $(X, \leq)$ be a \kl{well-quasi-ordered} set. One can define three
well-founded trees from $X$: the tree of \kl{bad sequences}, the tree of
decreasing sequences, and the tree of \kl{antichains}. The rank of these
respective trees are called respectively the \kl{maximal order type}
\cite{dejongh77}, the \kl{ordinal height} \cite{schmidt81}, and the \kl{ordinal
width} of $X$ \cite{kriz90b}. We refer to the survey of \cite{DZSCSC20} for a
detail discussion on these concepts and their computation on specific classes
of well-quasi-ordered sets.




% LTeX: language=en-GB 
\section{Prefixes and Suffixes}
\label{prefixes:sec}

In this section, we study the well-quasi-ordering of languages under the
\kl{prefix relation}. Let us immediately remark that the map $u \mapsto u^R$
that reverses a word is an order-bijection between $(X^*, \prefleq)$ and $(X^*,
\suffleq)$, that is, $u \prefleq v$ if and only if $u^R \suffleq v^R$.
Therefore, we will focus on the prefix relation in the rest of this section, as
$(L, \prefleq)$ is \kl{well-quasi-ordered} if and only if $(L^R, \suffleq)$ is.

The next remark that we make is that $\Sigma^*$ is not \kl{well-quasi-ordered}
by the \kl{prefix relation} as soon as $\Sigma$ contains two distinct letters
$a$ and $b$. As an example of infinite \kl{antichain}, we can consider the set
of words $a^n b$ for $n \in \Nat$. As mentioned in the introduction, there are
some languages however that are \kl{well-quasi-ordered} by the \kl{prefix
relation}. A simple example being the (regular) language $a^* \subseteq
\set{a,b}^*$, which is order-isomorphic to natural numbers with their usual
orderings $(\Nat, \leq)$.

In order to characterize the existence of infinite \kl{antichains} for the
\kl{prefix relation}, we will introduce the following tree construction that
will be useful in the rest of this section.

\begin{definition}
    The \intro{tree of prefixes} over a finite alphabet $\Sigma$
    is the infinite tree $T$ whose nodes are the words of $\Sigma^*$, and
    such that the children of a word $w$ are the words $wa$ for all $a \in
    \Sigma$. 
\end{definition}

We will use this \kl{tree of prefixes} to find simple witnesses
of the existence of infinite \kl{antichains} in the \kl{prefix relation}
for a given language $L$, namely by introducing \kl{antichain branches}.


\begin{definition}
    An \intro{antichain branch} for a language $L$ is an infinite 
    branch $B$ of the \kl{tree of prefixes} such that from every point of the branch, 
    one can reach a word of $L$ that is not in the branch.
\end{definition}

Let us illustrate the notion of \kl{antichain branch} over the alphabet $\Sigma
= \set{a,b}$, and the language $L = a^* b$. In this case, the set $a^*$ (which
is a branch of the \kl{tree of prefixes}) is an \kl{antichain branch} for $L$.
This holds because for any $a^k$ in the branch, $a \prefleq a^kb$ which is in
$L$ and not in the branch. In general, the existence of an \kl{antichain
branch} for a language $L$ implies that $L$ contains an infinite
\kl{antichain}, and because the alphabet $\Sigma$ is assumed to be finite, one
can leverage the fact that the \kl{tree of prefixes} is finitely branching to
prove that the converse holds as well.

\begin{lemma}
    \label{antichain-branches-prefix:lem}
    Let $L \subseteq \Sigma^*$ be a language. Then, $L$ contains an infinite
    \kl{antichain} if and only if there exists an \kl{antichain branch} for $L$.
\end{lemma}
\begin{proof}
    Assume that $L$ contains an \kl{antichain branch}. Let us construct an
    infinite \kl{antichain} as follows. We start with a set $A_0 \defined
    \emptyset$ and a node $v_0$ at the root of the tree. At step $i$, we
    consider a word $w_i$ such that $v_i$ is a \kl{prefix} of $w_i$, and $w_i
    \in L \setminus B$, which exists by definition of \kl{antichain branches}.
    We then set $A_{i+1} \defined A_i \cup \set{w_i}$. To compute $v_{i+1}$, we
    consider the largest prefix of $w_i$ that belongs to $B$, and set $v_{i+1}$
    to be the successor of this prefix in $B$. By an immediate induction, we
    conclude that for all $i \in \Nat$, $A_i$ is an \kl{antichain}, and that
    $v_i$ is a node in the \kl{antichain branch} $B$ such that $v_i$ is not a
    prefix of any word in $A_i$. 

    Conversely, assume that $L$ contains an infinite \kl{antichain} $A$. Let us
    construct an \kl{antichain branch}. Let us consider the subtree of the
    \kl{tree of prefixes} that consists in words that are \kl{prefixes} of
    words in $A$. This subtree is infinite, and by König's lemma, it contains
    an infinite branch. By definition this is an \kl{antichain branch}.
\end{proof}

One immediate application of \cref{antichain-branches-prefix:lem} is that
\kl{antichain branches} can be described inside the \kl{tree of prefixes} by a
monadic second order formula (\kl{$\MSO$-formula}), allowing us to leverage the
decidability of $\MSO$ over infinite binary trees \cite[Theorem 1.1]{RAB69}.

\begin{corollary}
    \label{prefix-wqo-reg-decidable:cor}
    If $L$ is regular, then the existence of an infinite antichain is decidable.
\end{corollary}
\begin{proof}
    If $L$ is regular, then it is \kl{$\MSO$-definable}, and there 
    exists a formula $\varphi(x)$ in \kl{$\MSO$} that selects nodes 
    of the \kl{tree of prefixes} that belong to $L$. Now, to decide whether there
    exists an \kl{antichain branch} for $L$, we can simply check whether
    the following formula is satisfied:
    \begin{equation*}
        \exists B. 
        B \text{ is a branch } \land
        \forall x \in B, \exists y. y \text{ is a child of } x \land \varphi(y) \land y \not\in B
        \quad .
    \end{equation*}
    Because the above formula is an $\MSO$-formula over the infinite
    $\Sigma$-branching tree, whether it is satisfied is decidable
    as an easy consequence of the decidability of $\MSO$ over infinite binary
    trees
    \cite[Theorem 1.1]{RAB69}.
\end{proof}

Let us now go further and fully characterize languages $L$ such that the
prefix relation is well-quasi-ordered, even without any restriction on the
decidability of $L$ itself. Let us remark that finite unions of \kl{chains} are
always \kl{well-quasi-ordered} by the \kl{prefix relation} because they lack
infinite \kl{antichains} by definition. The following theorem states that this
is the only possible reason for a language $L$ to be \kl{well-quasi-ordered} by
the \kl{prefix relation}.

For the proof of the following theorem, we will use special notations to
describe the \intro{upwards closure} of a set $S$ for a relation $\preceq$,
which is defined as $\upset[\preceq]{S} \defined \setof{y \in \Sigma^*}{
\exists x \in S. x \preceq y}$. Anticipating the use of the symmetric notion of
\intro{downwards closure}, let us introduce its notation 
straight away as follows: $\dwset[\preceq]{S} \defined \setof{y
\in \Sigma^*}{ \exists x \in S. y \preceq x}$. Abusing notations, we will
write $\upset{w}$ and $\dwset{w}$ for the upwards and downwards closure of a
single element $w$, and omit the ordering relation when it is clear from the
context.

\begin{theorem}
    \label{prefixes:thm}
    A language $L \subseteq \Sigma^*$ is \kl{well-quasi-ordered} by the
    \kl{prefix relation} if and only if $L$ is a union of \kl{chains}.
\end{theorem}
\begin{proof}
    Assume that $L$ is a finite union of \kl{chains}.
    Because the \kl{prefix relation} is \kl{well-founded},
    and that finite unions of \kl{chains} have finite \kl{antichains},
    we conclude that $L$ is \kl{well-quasi-ordered}.

    Conversely, assume that $L$ is \kl{well-quasi-ordered} by the \kl{prefix
    relation}. Let us define $S$ the set of words $w$ such that there exists
    two words $wu$ and $wv$ both in $L$ that are not comparable for the
    \kl{prefix relation}. Assume by contradiction that $S$ is infinite.
    Then, $S$ equipped with the \kl{prefix relation} is an infinite
    tree with finite branching, and therefore contains an infinite
    branch, which is by definition an \kl{antichain branch} for $L$.
    This contradicts the assumption that $L$ is \kl{well-quasi-ordered}.
    Now, consider $w$ be a maximal element for the \kl{prefix ordering}
    in $S$. By definition, all words in $L$ that contain $w$ as a prefix
    must be comparable for the \kl{prefix relation}. Therefore,
    $(\upset[\prefleq]{w}) \cap L$ is a \kl{chain} for the \kl{prefix relation}.
    In particular, letting $S_{\max}$ be the set of all maximal elements
    of $S$,
    we conclude that 
    \begin{equation*}
        L \subseteq S \cup \bigcup_{w \in S_{\max}} (\upset[\prefleq]{w}) \cap L
        \quad .
    \end{equation*}
    Hence, that $L$ is finite union of \kl{chains}.
\end{proof}

As an immediate consequence, we have a very fine-grained understanding of the
\kl{ordinal invariants} of such \kl{well-quasi-ordered} languages, which can be
leveraged in bounding the complexity of algorithms working on such languages.

\begin{corollary}
    Let $L \subseteq \Sigma^*$ be an infinite language that is \kl{well-quasi-ordered} by
    the \kl{prefix relation}. Then, there exists finite numbers $k,\ell \in \Nat$ such that
    the
    \kl{maximal order type} of $L$ is $k \cdot \omega$,
    the \kl{ordinal height} of $L$ is $\omega$, and its
    \kl{ordinal width} is $\ell$.
\end{corollary}

Let us conclude by noting that it is unsurprisingly not possible to decide
whether a decidable language is \kl{well-quasi-ordered} by the \kl{prefix
relation}. This is a very easy result whose sole purpose is to contrast with
the decidability result of \cref{prefix-wqo-reg-decidable:cor}.

\begin{lemma}
    The following problem is undecidable: given a language $L$
    decided by a Turing machine, answer whether 
    $L$ is \kl{well-quasi-ordered} for the \kl{prefix relation}.
\end{lemma}
\begin{proof}
    We reduce the halting problem on the empty string $\varepsilon$.
    Let $M$ be a Turing Machine, we write the languages $L$ of finite runs
    of $M$ starting on the empty string,
    that we surround by special markers. This language is decidable,
    and 
    is
    \kl{well-quasi-ordered} if and only if it is finite
    if and only if $M$ terminates on $\varepsilon$.
\end{proof}

% LTeX: language=en-GB 
% !TeX root=../wqo-on-words.tex
\section{Infixes and Bounded Languages}
\label{infixes-regular:sec}

\AP In this section, we study languages equipped with the \kl{infix relation}.
As opposed to the \kl{prefix} and \kl{suffix} relations, the \kl{infix
relation} can lead to very complicated \kl{well-quasi-ordered} languages.
Formally, the upcoming \cref{infix-embedding:thm} due to Kuske shows that
\emph{any} countable partial-ordering with finite initial segments can be
embedded into the infix relation of a language. To make the former statement
precise, let us recall that an \intro{order embedding} from a quasi-ordered set
$(X, \preceq)$ into a quasi-ordered set $(Y, \preceq')$ is a function $f \colon
X \to Y$ such that for all $x, y \in X$, $x \preceq y$ if and only if $f(x)
\preceq' f(y)$. When such an embedding exists, we say that $X$ \reintro{embeds
into} $Y$. Recall that a quasi-ordered set $(X, \preceq)$ is a \kl{partial
ordering} whenever the relation $\preceq$ is antisymmetric, that is $x \preceq
y$ and $y \preceq x$ implies $x = y$. 
A simplified version of the embedding defined in \cref{infix-embedding:thm} is illustrated
for the \kl{subword relation} in \cref{infix-embedding:fig}.
\begin{lemma}{\cite[Lemma 5.1]{DBLP:journals/ita/Kuske06}}
    \label{infix-embedding:thm}
    Let $(X, \preceq)$ be a \kl{partially ordered} set,
    and $\Sigma$ be an alphabet with at least two letters.
    Then the following are equivalent:
    \begin{enumerate}
        \item 
            $X$ \kl{embeds into} $(\Sigma^*, \infleq)$,
        \item 
            $X$ is countable, and for every $x \in X$,
            its \kl{downwards closure}
            $\dwset[\preceq]{x}$ is finite.
    \end{enumerate}
\end{lemma}
\begin{figure}
    \centering
    \includestandalone[width=\linewidth]{fig/infix-encoding-standalone}
    \caption{Representation of the \kl{subword relation} for $\set{a,b}^*$
        inside the \kl{infix relation} for $\set{a,b,\#}^*$
        using a simplified version of \cref{infix-embedding:thm}, restricted to words
        of length at most $3$. 
    }
    \label{infix-embedding:fig}
\end{figure}

\AP As a consequence of \cref{infix-embedding:thm}, we cannot replay
proofs of \cref{prefixes:sec}, and will
actually need to leverage some regularity of the languages to obtain a
characterization of \kl{well-quasi-ordered} languages under the \kl{infix
relation}. This regularity will be imposed through the notion of \intro{bounded
languages}, i.e., languages $L \subseteq \Sigma^*$ such that there exists words
$w_1, \dots, w_n$ satisfying $L \subseteq w_1^* \cdots w_n^*$.

\begin{theorem}[restate=bounded-language:thm,label=bounded-language:thm]
    %\label{bounded-language:thm}
    Let $L$ be a \kl{bounded language} of $\Sigma^*$. Then,
    $L$ is a \kl{well-quasi-order} when endowed with the 
    \kl{infix relation} if and only if it included in a finite union of 
    products $S_i \cdot P_i$ where 
    $S_i$ is a \kl{chain} for the \kl{suffix relation}, and 
    $P_i$ is a \kl{chain} for the \kl{prefix relation},
    for all $1 \leq i \leq n$.
\end{theorem}

Let us first remark that if $S$ is a \kl{chain} for the \kl{suffix relation}
and $P$ is a \kl{chain} for the \kl{prefix relation}, then $SP$ is
\kl{well-quasi-ordered} for the \kl{infix relation}. This proves the (easy)
right-to-left implication of \cref{bounded-language:thm}. Furthermore, any
language $L$ included in a finite union of products $S_i \cdot P_i$ where $S_i$
is in fact a \kl{bounded language}.

\AP In order to prove the (difficult) left-to-right implication of
\cref{bounded-language:thm}, we will rely heavily on the
combinatorics of periodic words. Let us recall that a non-empty word $w \in
\Sigma^+$ is \intro(word){periodic} with period $x \in \Sigma^*$ if there
exists a $p \in \Nat$ such that $w \infleq x^p$. The \intro{periodic length} of
a word $u$ is the minimal length of a word $x$ such that $u$ is an \kl{infix}
of $x^p$ for some $p \in \Nat$ and $x \in \Sigma^+$.

\begin{lemma}
    \label{periodic-infixes:lem}
    Let $u,v \in \Sigma^*$ be two (non-empty) \kl{periodic words}
    having \kl{periodic lengths} $p$ and $q$ respectively.
    Then, if $u \infleq v$ and $\card{u} \geq \factorial[p]{p \times q}$,
    then $u$ and $v$ share the same \kl{periodic length}
    $p = q$.
\end{lemma}
\begin{proof}
    The fact that $u$ and $v$ are \kl{periodic length}
    respectively $p$ and $q$ translates into the fact that $u_{i+p} = u_i$ and
    $v_{i+q} = v_i$ for all indices $i \in \Nat$ such that those letters are
    well-defined.

    Now, assume that $u$ is an \kl{infix} of $v$, this provides the existence
    of a $k \in \Nat$ such that $u = v_{k} \cdots v_{k + \card{u} - 1}$. In
    particular, $v_{k+i+p} = v_{k+i}$ for all $i \in \Nat$ such that $k+i+p < k
    + \card{u}$. Since we also have $v_{k+i+q} = v_{k+i}$ for all $1 \leq i
    \leq \card{v} - k - q$. We conclude that both $u$ and $v$ are of
    \kl{periodic length} the greatest common divisor of $p$ and $q$, and by
    minimality of $q$ this must be equal to $q$ and to $p$.
\end{proof}

\begin{corollary}
    \label{powers-infixes:cor}
    Let $u,v \in \Sigma^*$ and $k, \ell \in \Nat$
    such that $k \geq \factorial[p]{\card{u} \times \card{v}}$,
    $\ell \geq \factorial[p]{\card{v} \times \card{u}}$,
    and $u^k \infleq v^\ell$.
    Then, there exists $w \in \Sigma^*$ of size at most
    $\min \set{\card{u}, \card{v}}$ and a $p \in \Nat$
    such that
    $u^k \infleq v^\ell \infleq w^p$.
\end{corollary}

The reason why \kl{periodic words} built using a given period $x \in \Sigma^+$
are interesting for the \kl{infix relation} is that they naturally create
\kl{chains} for the \kl{prefix} and \kl{suffix} relations. Indeed, if $x \in
\Sigma^+$ is a finite word, then $\setof{x^p}{p \in \Nat}$ is a \kl{chain} for
the \kl{infix relation}. Note that in general, the \kl{downwards closure} of a
chain is \emph{not} a chain (see \cref{dw-closure-not-wqo:rem}). However, for the chains generated using periodic
words, the \kl{downwards closure} $\dwset[\infleq]{\setof{x^p}{p \in \Nat}}$ is
a \emph{finite union} of \kl{chains}. Because this set will appear in bigger
equations, we introduce the shorter notation $\intro*\InfPeriodChain{x}$ for
the set of \kl{infixes} of words of the form $x^p$, where $p$ ranges over
$\Nat$.


\begin{remark}
    \label{dw-closure-not-wqo:rem}
    Let $(X,\preceq)$ be a quasi-ordered set, and $L \subseteq X$ be such that $(L,
    \preceq)$ is \kl{well-quasi-ordered}. It is not true in general that
    $(\dwset{L}, \preceq)$ is \kl{well-quasi-ordered}. In the case of $(\Sigma^*,
    \infleq)$ a typical example is to start from an infinite \kl{antichain} $A$,
    together with an enumeration $\seqof{w_i}$ of $A$, and build the language $L
    \defined \setof{ \prod_{i = 0}^n w_i }{ i \in \Nat }$. By definition, $L$ is a
    \kl{chain} for the \kl{infix} ordering, hence \kl{well-quasi-ordered}. However,
    $\dwset[\infleq]{L}$ contains $A$, and is therefore not
    \kl{well-quasi-ordered}. 
\end{remark}

\begin{lemma}
    \label{inf-period-chain:lem}
    Let $x \in \Sigma^+$ be a word, and
    Then $\InfPeriodChain{x}$ is a finite union of \kl{chains}
    for the \kl{infix}, \kl{prefix} and \kl{suffix} relations 
    simultaneously.
\end{lemma}
\begin{proof}
    Let $x \in \Sigma^+$ be a word, and let $P_x$ be the (finite) set 
    of all \kl{prefixes} of $x$, and $S_x$ be the (finite)
    set of all \kl{suffixes} of $x$.
    Assume that $w \in \InfPeriodChain{x}$, then $w = u x^p v$ for some
    $u \in S_x$, $v \in P_x$, and $p \in \Nat$.
    We have proven that
    \begin{equation*}
        \InfPeriodChain{x} \subseteq \bigcup_{u \in P_x} \bigcup_{v \in S_x} u x^* v
        \quad .
    \end{equation*}

    Let us now demonstrate that for all $(u,v) \in S_x \times P_x$, the
    language $u x^* v$ is a \kl{chain} for the \kl{infix}, \kl{suffix} and \kl{prefix} relations.
    To that end,
    let $(u,v) \in S_x \times P_x$ and $\ell, k \in \Nat$ be such that $\ell <
    k$, let us prove that $u x^\ell v \infleq u x^k  v$. Because $v \prefleq
    x$, we know that there exists $w$ such that $vw = x$. In particular,
    $ux^\ell vw = u x^{\ell + 1}$, and because $\ell < k$, we conclude that $u
    x^{\ell + 1} \prefleq u x^k v$. By transitivity, $u x^\ell v \prefleq u x^k
    v$, and \emph{a fortiori}, $u x^\ell v \infleq u x^k v$. 
    Similarly, because $u \suffleq x$,  there exists $w$ such that $wu  = x$, 
    and we conclude that $u x^{\ell} v \suffleq w u x^\ell v = x^{\ell + 1} v \suffleq u x^k v$.
    \qedhere
\end{proof}



The following combinatorial lemma connects the property of being
\kl{well-quasi-ordered} to a property of the \kl{periodic lengths} of words in
a language, based on the assumption that some factors can be iterated. It is
the core result that powers the analysis done in the upcoming
\cref{bounded-language:thm,infix-amalgamation:thm}.

\begin{lemma}
    \label{pumping-periods:lem}
    Let $L \subseteq \Sigma^*$ be a language
    that is \kl{well-quasi-ordered} by the \kl{infix relation}.
    Let $k \in \Nat$, $u_1, \cdots, u_{k+1} \in \Sigma^*$,
    and $v_1, \cdots, v_{k} \in \Sigma^+$
    be such that
    $w[\vec{n}] \defined (\prod_{i = 1}^k u_i v_i^{n_i}) u_{k+1}$
    belongs to $L$
    for arbitrarily large values of $\vec{n} \in \Nat^k$.
    Then, 
    there exists $x,y \in \Sigma^+$ of size 
    at most $\max \setof{\card{v_i}}{1 \leq i \leq k}$
    such that 
    one of the following holds for all
    $\vec{n} \in \Nat^{k}$:
    \begin{enumerate}
        \item $w[\vec{n}] \in u_1 \InfPeriodChain{x}$,
        \item $w[\vec{n}] \in \InfPeriodChain{x} u_{k+1}$,
        \item $w[\vec{n}] \in \InfPeriodChain{x} u_i \InfPeriodChain{y}$
            for some $1 \leq i \leq k + 1$.

    \end{enumerate}
\end{lemma}
\begin{proof}
    Note that the result is obvious if $k = 0$, and therefore
    we assume $k \geq 1$ in the following proof.

    Let us construct a sequence of words $\seqof{w_i}[i \in \Nat]$, where $w_i
    \defined w[\vec{n_i}]$ for some well-chosen indices $\vec{n_i} \in \Nat^k$. The goal
    being that 
    if $w[\vec{n_i}]$ is an \kl{infix} of $w[\vec{n_j}]$,
    then it can intersect at most \emph{two} iterated words,
    with an intersection that is long enough to successfully apply
    \cref{periodic-infixes:lem}.
    In order to achieve this,
    let us first define $s$ as the maximal size of a word $v_i$
    ($1 \leq i \leq k$) and $u_j$ ($1 \leq j \leq k+1$).
    Then,
    we consider $\vec{n_0} \in \Nat^k$ such that $\vec{n_0}$ has all 
    its components greater than $\factorial{s}$ and such that
    $w[\vec{n_0}]$ belongs to $L$. 
    Then, we inductively define 
    $\vec{n_{i+1}}$  as the smallest vector of numbers greater than $\vec{n_i}$,
    such that $w[\vec{n_{i+1}}]$ belongs to $L$, 
    and with $\vec{n_i}$ having all components greater than
    $2\card{w[\vec{n_i}]}$.


    Let us assume that $k \geq 2$ in the following proof for symmetry purposes,
    and argue later on that when $k = 1$ the same argument goes through.
    Because $L$ is \kl{well-quasi-ordered} by the \kl{infix relation}, there
    exists $i < j$ such that $w[\vec{n_i}]$ is an \kl{infix} of $w[\vec{n_j}]$.
    Now, because of the chosen values for $\vec{n_j}$, there exists $1 \leq \ell \leq
    k-1$ such that $w[\vec{n_i}]$ is actually an \kl{infix} of $u_{\ell}
    v_{\ell}^{n_{j,\ell}} u_{\ell+1} v_{\ell+1}^{n_{j,\ell+1}} u_{\ell+2}$.
    Even more,
    one of the three following equations holds:
    \begin{itemize}
        \item $w[\vec{n_i}] \infleq v_{\ell}^{n_{j,\ell}} u_{\ell+1} v_{\ell+1}^{n_{j,\ell+1}}$,
        \item $w[\vec{n_i}] \infleq u_{\ell}
            v_{\ell}^{n_{j,\ell}}$,
        \item $w[\vec{n_i}] \infleq
            v_{\ell+1}^{n_{j,\ell+1}} u_{\ell+2}$.
    \end{itemize}
    In all those cases, we conclude using \cref{powers-infixes:cor}
    that there exists $x,y \in \Sigma^+$ of size at most $s$, and 
    a number $1 \leq t \leq k$ such that
    $v_i^{n_i} \in \InfPeriodChain{x}$ for all $1 \leq i \leq t$,
    and
    $v_i^{n_i} \in \InfPeriodChain{y}$ for all $t < i \leq k$.
    In particular,
    $w[\vec{n_i}] \in \InfPeriodChain{x} u_{t} \InfPeriodChain{y}$.

    
    When $k = 1$, the situation is a bit more specific since we only have two
    cases: either $w_i \infleq u_1 v_1^{n_j}$ or $w_i \infleq v_1^{n_j} u_2$,
    and we conclude with an identical reasoning.
\end{proof}

\begin{lemma}
    \label{bounded-language:lem}
    Let $L \subseteq \Sigma^*$ be a \kl{bounded language}
    that is \kl{well-quasi-ordered} by the \kl{infix relation}.
    Then, there exists a finite subset $E \subseteq (\Sigma^*)^3$,
    such that:
    \begin{equation*}
        L \subseteq \bigcup_{(x,u,y) \in E} \InfPeriodChain{x} u \InfPeriodChain{y}
        \quad .
    \end{equation*}
\end{lemma}
\begin{proof}
    Let $w_1, \dots, w_n$ be such that
    $L \subseteq w_1^* \cdots w_n^*$.
    Let us define $m \defined \max \setof{\card{w_i}}{1 \leq i \leq n}$

    Let $w[\vec{k}] \defined w_1^{k_1} \cdots w_n^{k_n}$ be a map from $\Nat^k$
    to $\Sigma^*$. We are interested in the intersection of the image of $w$
    with $L$. Let us assume for instance that for all $\vec{k} \in \Nat^n$,
    there exists $\vec{\ell} \geq \vec{k}$ such that $w[\vec{\ell}] \in L$.
    Then, leveraging \cref{pumping-periods:lem}, we conclude that there exists
    $x,y$ of size at most $\max\setof{\card{w_i}}{1 \leq i \leq n}$ such that
    $w[\vec{k}] \in \InfPeriodChain{x} \cup \InfPeriodChain{x}
    \InfPeriodChain{y}$, and we conclude that $L \subseteq \InfPeriodChain{x}
    \cup \InfPeriodChain{x} \InfPeriodChain{y}$.

    Now, it may be the case that one cannot simultaneously assume that two
    component of the vector $\vec{k}$ are unbounded. In general, given a set $S
    \subseteq \set{1, \dots, n}$ of indices, we say that $S$ is admissible if
    there exists a bound $N_0$ such that for all $\vec{b} \in \Nat^S$, there
    exists a vector $\vec{k} \in \Nat^n$, such that $\vec{k}$ is greater than
    $\vec{b}$ on the $S$ components, and the other components are below the
    bound $N_0$. The language of an admissible set $S$ is the set of words
    obtained by repeating $w_i$ at most $N_0$ times if it is not in $S$
    ($w_i^{\leq N_0}$) and arbitrarily many times otherwise ($w_i^*$).
    Note that $L \subseteq \bigcup_{S \text{ admissible }} L(S)$.

    Now, admissible languages are ready to be pumped according to
    \cref{pumping-periods:lem}. For every admissible language,
    the size of a word that is not iterated is at most
    $N_0 \times m$ by definition, and we conclude that:
    \begin{equation}
        \label{bounded-language:eq}
        L \subseteq 
        \bigcup_{x,y \in \Sigma^{\leq n}}
        \bigcup_{u \in \Sigma^{\leq m \times N_0}}
        \InfPeriodChain{x} u \InfPeriodChain{y}
        \cup
        \InfPeriodChain{x} u
        \cup
        u \InfPeriodChain{x}
        \quad .
    \end{equation}
\end{proof}

\begin{proofof}{bounded-language:thm}
    We apply \cref{bounded-language:lem}, and conclude
    because $\InfPeriodChain{x}$ is a finite union of \kl{chains}
    for the \kl{prefix}, \kl{suffix} and \kl{infix} relations
    (\cref{inf-period-chain:lem}).
\end{proofof}


Let us now discuss the implications of this characterization in terms of
\kl{downwards closures}: if $L$ is a \kl{bounded language}, then considering
$L$ or its \kl{downwards closure} is equivalent with respect to being
\kl{well-quasi-ordered} by the \kl{infix relation}.

\begin{corollary}
    \label{bounded-wqo-dwclosed:cor}
    Let $L$ be a \kl{bounded language} of $\Sigma^*$. Then,
    $L$ is a \kl{well-quasi-order} when endowed with the
    \kl{infix relation} if and only if $\dwset[\infleq]{L}$ is.
\end{corollary}
\begin{proof}
    Because $L \subseteq \dwset[\infleq]{L}$, the right-to-left implication
    is trivial.
    For the left-to-right implication, let us assume that $L$ is a
    \kl{well-quasi-ordered} language for the \kl{infix relation}.
    Then $L$ is included in a finite union 
    of products of \kl{chains}:
    \begin{equation*}
        L \subseteq \bigcup_{i = 1}^n S_i \cdot P_i \quad .
    \end{equation*}
    Remark that the \kl{downwards closure} of a product of two \kl{chains}
    is a finite union of products of two chains.
    As a consequence, we conclude that $\dwset[\infleq]{L}$ is itself included
    in a finite union of products of \kl{chains}.
    Note that this also proves that $\dwset[\infleq]{L}$ is a \kl{bounded language},
    hence that it is \kl{well-quasi-ordered} by the \kl{infix relation} 
    via
    \cref{bounded-language:thm}.
\end{proof}

\section{Infixes and Downwards Closed Languages}

One may think that all \kl{downwards closed} languages for the \kl{infix
relation} that are \kl{well-quasi-ordered} are \kl(language){bounded}. Note
that this is what happens in the case of the \kl{subword embedding}, where any
\kl{downwards closed} language for this relation is characterized by finitely
many excluded \kl{subwords}, hence provides a \kl{bounded language}. However, this
is not the case for the \kl{infix relation}, as we will now illustrate with the
following two examples.

\begin{example}
    \label{dwclosed-wqo-not-finite-excl:ex}
    Let $L \defined a^* b^* \cup b^* a^*$. This language is \kl{downwards
    closed} for the \kl{infix relation}, is \kl{well-quasi-ordered} for the
    \kl{infix relation}, but is characterized by an \emph{infinite} number 
    of excluded infixes, respectively of the form $ab^ka$ and $ba^kb$ where $k \geq 1$.
\end{example}

To strengthen \cref{dwclosed-wqo-not-finite-excl:ex}, we will
leverage the \intro{Thue-Morse sequence} $\intro*\ThueMorse \in
\set{0,1}^{\Nat}$, which we will use as a black-box for its two main
characteristics: it is \kl{cube-free} and \kl{uniformly recurrent}. Being
\intro{cube-free} means that no (finite) word of the form $uuu$ is an
\kl{infix} of $\ThueMorse$, and being \intro{uniformly recurrent} means that
for every word $u$ that is an \kl{infix} of $\ThueMorse$, there exists $k \geq
1$ such that $u$ is an \kl{infix} of every word $v$ of size at least $k$. We
refer the reader to a nice survey of Allouche and Shallit for more information
on this sequence and its properties \cite{ALSHA99}.

\begin{lemma}
    \label{thue-morse:lemma}
    The language $\intro*\LMorse$ of \kl{infixes} of the \kl{Thue-Morse sequence}
    is \kl{downwards closed} for the \kl{infix
    relation}, \kl{well-quasi-ordered} for the \kl{infix relation}, but is not
    \kl(language){bounded}.
\end{lemma}
\begin{proof}
    By construction $\LMorse$ is an \emph{infinite}
    \kl{downwards closed} for the \kl{infix relation}. Let us argue that $\LMorse$ is
    \kl{well-quasi-ordered} for the \kl{infix relation}, but is not \kl(language){bounded}.

    Assume by contradiction that $\LMorse$ is \kl(language){bounded}. In this case, there exist
    words $w_1, \dots, w_k \in \Sigma^*$ such that $L \subseteq w_1^* \cdots
    w_k^*$. Since $L$ is infinite and \kl{downwards closed}, there exists a
    word $u \in L$ such that $u = w_i^3$ for some $1 \leq i \leq k$. This is absurd
    because $u \infleq \ThueMorse$, which is \kl{cube-free}.

    Furthermore, $L$ is \kl{well-quasi-ordered} for the \kl{infix relation}.
    Indeed, consider a sequence $\seqof{u_i}$ of words in $L$. Without loss of
    generality, we may consider a subsequence such that $\card{u_i} <
    \card{u_{i+1}}$ for all $i \in \Nat$. Because $\ThueMorse$ is \kl{uniformly
    recurrent}, there exists $k \geq 1$ such that $u_1$ is an \kl{infix} of
    every word $v$ of size at least $k$. In particular, $u_1$ is an \kl{infix}
    of $u_k$, hence the sequence $\seqof{u_i}$ is \kl(sequence){good}.
\end{proof}

One may refine our analysis of the \kl{Thue-Morse sequence} to obtain 
precise bounds on the \kl{ordinal invariants} of its language of \kl{infixes}.

\begin{lemma}
    \label{thue-morse-ordinal:lemma}
    The \kl{maximal order type} of $\LMorse$ is $\omegaOrd$,
    the \kl{ordinal height} of $\LMorse$ is $\omegaOrd$,
    the \kl{ordinal width} of $\LMorse$ is $\omegaOrd$.
\end{lemma}
\begin{proof}
    Let us prove that these are upper bounds for the \kl{ordinal invariants} of
    $L$. For the \kl{ordinal height}, it is true for any language $\LMorse$.
    For the \kl{maximal order type}, we remark that
    the maximal length of a \kl{bad sequence} is determined by its first element,
    hence that it is at most $\omegaOrd$.
    Finally, because the \kl{ordinal width} is at most the \kl{maximal order type},
    we conclude that the \kl{ordinal width} is also at most $\omegaOrd$.

    Now, let us prove that these bounds are tight. It is clear that
    $\oHeight{\LMorse} = \omegaOrd$: given any number $n \in \Nat$, one can construct a
    \kl{decreasing sequence} of words in $L$ of length $n$, for instance by
    considering the first $n$ prefixes of the \kl{Thue-Morse sequence} by
    decreasing size.
    Let us now prove that $\oWidth{\LMorse} = \omegaOrd$. Assume by contradiction that
    $\oWidth{\LMorse}$ is finite. Then, $L$ can be written as a finite union of
    \kl{chains} for the \kl{infix relation}, and in particular, $L$ is
    \kl(language){bounded}, which is absurd by \cref{thue-morse:lemma}.
    Finally, because the \kl{ordinal width} is at most the \kl{maximal order
    type}, we conclude that the \kl{maximal order type} of $\LMorse$ is also $\omegaOrd$.
\end{proof}

This suggests the following conjecture which states that for \kl{downwards
closed} languages, the \kl{ordinal invariants} are relatively small.

\begin{conjecture}
    \label{small-ordinal-invariants:conj}
    Let $L$ be a \kl{well-quasi-ordered} language for the \kl{infix relation}
    that is \kl{downwards closed}.
    Then, the \kl{maximal order type} of $L$ is at most $\omegaOrd^3$,
    its \kl{ordinal height} is at most $\omegaOrd$,
    and its \kl{ordinal width} is at most $\omegaOrd^2$.
\end{conjecture}

\todo[inline]{
    Write this in a nicer way.
    Maybe talk a bit more about automatic sequences?
    This could be a section ``Infixes and downwards closed languages''. Before the
    ``infixes and amalgamation systems''.
}

\begin{theorem}
    \label{small-ordinal-invariants:thm}
    Let $L$ be a \kl{well-quasi-ordered} language for the \kl{infix relation}
    that is \kl{downwards closed}.
    Then, the \kl{maximal order type} of $L$ is strictly less than $\omegaOrd^3$,
    its \kl{ordinal height} is at most $\omegaOrd$,
    and its \kl{ordinal width} is at most $\omegaOrd^2$.
\end{theorem}

The \cref{small-ordinal-invariants:thm} is a direct consequence of the following
remarks.

\begin{itemize}
    \item If $L$ is \kl{wqo} and \kl{downwards closed}, then $L = \bigcup_{i = 1}^n L_i$ where $L_i$
        is an \kl{order ideal}.
    \item If $L$ is an \kl{order ideal} that is \kl{well-quasi-ordered}, then
        $L$ is the set of infixes of some \emph{bi-infinite} word $w \in \Sigma^{\Rel}$.
    \item The set of infixes of a \emph{bi-infinte} word can be understood 
        as the set of \kl{infixes} of two infinite words $u,v \in \Sigma^{\Nat}$.
    \item The set of infixes of an infinite word $w \in \Sigma^{\Nat}$ is 
        \kl{well-quasi-ordered} if and only if $w$ is \kl{ultimately uniformly recurrent},
        that is, if there exists an infinite suffix of $w$ that is \kl{uniformly recurrent}.
    \item The ordinal invariants of a \kl{uniformly recurrent} word are small.
\end{itemize}

Note that being \kl{uniformly recurrent} still leaves open a lot of potential behaviours,
for instance, Sturmian words are \kl{uniformly recurrent}, but are not \kl{automatic}.
\todo{Find citations for this.}

\begin{theorem}
    Given an \kl{automatic sequence} $w \in \Sigma^{\Nat}$, one can decide
    whether the set of \kl{infixes} of $w$ is \kl{well-quasi-ordered} for the
    \kl{infix relation}, and if so, compute the associated \kl{ordinal
    invariants}.
\end{theorem}
\begin{proof}
    One can check if $w$ is \kl{ultimately uniformly recurrent} by checking if
    $\exists N_0 \in \Nat, \dots$. Because $w$ is \kl{automatic}, this is decidable.
    Now, to decide whether $w$ has width $\omegaOrd$, then one can check if $w$ is 
    \kl{ultimately periodic}. If not, then $w$ has width $\omegaOrd$.
    If it is \kl{ultimately periodic}, one can compute the smallest size of
    the period, which gives us the width.
    Using the width, and because we know the height is at most $\omegaOrd$, we can
    compute the maximal order type.
\end{proof}

% LTeX: language=en-GB 
% !TeX root=../wqo-on-words.lncs.tex
\section{Infixes and Amalgamation Systems}
\label{infixes-amalgamation:sec}

\AP In the previous section, we have represented languages that are
\kl{downwards closed} by the \kl{infix relation} as \kl{infixes} of infinite
words. However, there are many other natural ways to represent languages, such
as finite automata or context-free grammars. In this section, we are going to
show that our results on \kl{bounded languages} can be applied to a large class
of systems, called \kl{amalgamation systems}, that includes as particular
examples finite automata and context-free grammars. 

\AP Our first result, of theoretical nature, is that \kl{amalgamation systems}
cannot define \kl{well-quasi-ordered} languages that are not
\kl(language){bounded}. This implies that all the results of
\cref{infixes-bounded:sec}, and in particular \cref{bounded-language:thm}, can
safely be applied to \kl{amalgamation systems}.

\begin{theorem}
    \label{infix-amalgamation:thm}
    \proofref{infix-amalgamation:thm}
    Let $L \subseteq \Sigma^*$ be a language recognized by an 
    \kl{amalgamation system}.
    If $L$ is \kl{well-quasi-ordered} by the \kl{infix relation} then $L$ is
    \kl(language){bounded}.
\end{theorem}

\AP Our second focus is of practical nature: we want to give a decision
procedure for being \kl{well-quasi-ordered}. This will require us to introduce
\emph{effectiveness assumptions} on the \kl{amalgamation systems}. While most
of them will be innocuous, an important consequence is that we have to consider
\emph{classes of languages} rather than individual ones, for instance: the
class of all regular language, or the class of all context-free languages. Such
classes will be called \kl(amalgamation)[effectiveness assumptions]{effective
amalgamative classes} (\kcref{effective amalgamation system}). In the
following theorem, we prove that under such assumptions, testing
\kl{well-quasi-ordering} is inter-reducible to testing whether a language of
the class is empty, which is usually the simplest problem for a computational
model.

\begin{theorem}
	\label{infix-wqo-is-emptiness:thm}
  \proofref{infix-wqo-is-emptiness:thm}
	Let $\mathcal{C}$ be an \kl{effective amalgamative class} of languages.
    Then the following are equivalent:
	\begin{enumerate}
        \item\label{wqo-infix-decidable} \kl[wqo]{Well-quasi-orderedness} of the \kl{infix relation} is decidable for languages in $\mathcal{C}$.
        \item\label{wqo-prefix-decidable} \kl[wqo]{Well-quasi-orderedness} of the \kl{prefix relation} is decidable for languages in $\mathcal{C}$.
        \item\label{emptiness-decidable} Emptiness is decidable for languages in $\mathcal{C}$.
	\end{enumerate}
\end{theorem}

\subsection{Amalgamation Systems}
\label{amalgamation-systems:subsec}

Let us now formally introduce the notion of \kl{amalgamation systems}, and
recall some results from \cite{ASZZ24} that will be useful for the proof of
\cref{infix-amalgamation:thm}. The notion of \kl{amalgamation system} is
tailored to produce \emph{pumping arguments}, which is exactly what our
\cref{pumping-periods:lem} talks about. At the core of a pumping argument,
there is a notion of a \emph{run}, which could for instance be a sequence of
transitions taken in a finite state automaton. Continuing on the analogy with
finite automata, there is a natural ordering between runs, i.e., a run is
smaller than another one if one can ``delete'' loops of the larger run to obtain
the other. Typical pumping arguments then rely on the fact that
\emph{minimal} runs are of finite size, and that all other runs are
obtained by ``gluing'' loops to minimal runs. Generalizing this notion yields the 
notion of \kl{amalgamation systems}.

\AP Let us recall that over an alphabet $(\Sigma, =)$ a \kl{subword embedding}
between two words $u \in \Sigma^*$ and $v \in \Sigma^*$ is a function $\rho
\colon \range{\card{u}} \to \range{\card{v}}$ such that $u_i = v_{\rho(i)}$ for
all $i \in \range{\card{u}}$. We write $\intro*\HigEmb(u,v)$ the set of all
\kl{subword embeddings} between $u$ and $v$. It may be useful to notice that
the set of finite words over $\Sigma$ forms a category when we consider
\kl{subword embeddings} as morphisms, which is a fancy way to state that
$\mathrm{id} \in \HigEmb(u,u)$ and that $f \circ g \in \HigEmb(u,w)$ whenever
$g \in \HigEmb(u,v)$ and $f \in \HigEmb(v,w)$, for any choice of words
$u,v,w \in \Sigma^*$.

\AP Given a \kl{subword embedding} $f \colon u \to v$ between two words $u$ and
$v$, there exists a unique decomposition $v = \GapWord{f}{0} u_1 \GapWord{f}{1}
\cdots \GapWord{f}{k-1} u_k \GapWord{f}{k}$ where $\GapWord{f}{i} =
v_{f(i)+1} \cdots v_{f(i+1)-1}$ for all $1 \leq i \leq k-1$, $\GapWord{f}{k} =
v_{f(k)+1} \cdots v_{\card v}$, and $\GapWord{f}{0}   = v_1 \cdots v_{f(1)-1}$. We say that
$\intro*\GapWord{f}{i}$ is the $i$-th \intro{gap word} of $f$. We encourage the
reader to look at \cref{gap-word-embedding:fig} to see an example of the
\kl{gap words} resulting from a \kl{subword embedding} between two words. These
\kl{gap words} will be useful to describe how and where runs of a system
(described by words) can be combined.


\begin{definition}
    An \intro{amalgamation system}
    is a tuple $(\Sigma, R, \canrun, E)$ where
    $\Sigma$ is a finite alphabet,
    $R$ is a set of so-called \emph{runs},
    $\canrun \colon R \to (\Sigma \uplus \set{\cansep})^*$ is a 
    function computing a \intro{canonical decomposition} of a run,
    and $E$ describes the so-called \intro{admissible embeddings} between runs: If $\rho$ and $\sigma$ are runs from $R$, then $E(\rho, \sigma)$ is a subset of the subword embeddings between $\canrun(\rho)$ and $\canrun(\sigma)$. We write $\rho \intro*\runleq \sigma$ if $E(\rho, \sigma)$ is non-empty. If we want to refer to a specific embedding $f \in E(\rho, \sigma)$, we also write $\rho \runleq_f \sigma$.
    Given a run $r \in R$, and $i \in \range[0]{\card{\canrun(r)}}$, 
    the \intro{gap language} of $r$ at position $i$ is $\intro*\GapLanguage{r}{i} \defined
    \setof{\GapWord{f}{i}}{\exists s \in R. \exists f \in E(r,s) }$.
    An \kl{amalgamation system} furthermore satisfies the following 
    properties:
    \begin{enumerate}
        \item \emph{$(R, E)$ Forms a Category.}
            For all $\rho, \sigma, \tau \in R$,
            $\mathrm{id} \in E(\rho,\rho)$,
            and whenever $f \in E(\rho,\sigma)$ and $g \in E(\sigma,\tau)$,
            then $g \circ f \in E(\rho,\tau)$.
        \item \emph{Well-Quasi-Ordered System.}
            $(R, \runleq)$ is a well-quasi-ordered set.
        \item \emph{Concatenative Amalgamation.}
            Let $\rho_0, \rho_1, \rho_2$ be runs 
            with $\rho_0 \runleq_f \rho_1$ 
            and $\rho_0 \runleq_g \rho_2$.
            Then for all $0 \leq i \leq \card{\canrun(\rho_0)}$,
            there exists a run $\rho_3 \in R$ 
            and embeddings $\rho_1 \runleq_{g'} \rho_3$
            and $\rho_2 \runleq_{f'} \rho_3$ 
            satisfying two conditions:
            (a) $g' \circ f = f' \circ g$ (we write $h$ for this composition) and
            (b) for every $0 \leq j \leq \card{\canrun[\rho_0]}$, 
            the gap word $\GapWord{h}{j}$
            is either $\GapWord{f}{j} \GapWord{g}{j}$
            or $\GapWord{h}{j} = \GapWord{g}{j} \GapWord{f}{j}$. 
            Specifically, for $i$ we may fix $\GapWord{h}{i} = \GapWord{f}{i} \GapWord{g}{i}$.
            We refer to \cref{amalgamation-runs:fig} for an illustration 
            of this property.
    \end{enumerate}

	The \emph{yield} of a run is obtained by projecting away the separator symbol \cansep~from the canonical decomposition, i.e. $\intro*\yieldrun(\rho) = \project_\Sigma( \canrun[\rho])$. The language recognized by an \kl{amalgamation system} is $\yieldrun(R)$.
    
    We say a language $L$ is an \intro{amalgamation language} 
    if there exists an \kl{amalgamation system} recognizing it. 
\end{definition}

Intuitively, the definition of an amalgamation system allows the comparison of
runs, and the proper ``gluing'' of runs together to obtain new runs. A number
of well-known language classes can be seen to be recognized by \kl{amalgamation
systems}, e.g., regular languages \cite[Theorem 5.3]{ASZZ24}, reachability and
coverability languages of VASS \cite[Theorem 5.5]{ASZZ24}, and context-free
languages \cite[Theorem 5.10]{ASZZ24}. 

%\subsection{$\infleq$-Well-Quasi-Ordered Amalgamation Systems}

We can now show a simple lemma that illuminates much of the structure of
amalgamation systems whose language is well-quasi-ordered by $\infleq$. Note
that \cref{gap-words-prefix-ordered:lem} uses \cref{pumping-periods:lem} in its
proof, and our \cref{infix-amalgamation:thm} follows from it.

\begin{lemma}
	\label{gap-words-prefix-ordered:lem}
  \proofref{gap-words-prefix-ordered:lem}
	Let $L$ by an \kl{amalgamation language} recognized by $(\Sigma, R, E, \canrun)$ that is well-quasi-ordered by $\infleq$. Let $\rho$ be a run with $\canrun[\rho] = a_1 \cdots a_n$, and let $\sigma, \tau$ be runs with $\rho \runleq_f \sigma$ and $\rho \runleq_g \sigma$. 
	
	For any $0 \leq \ell \leq n$, we have $\GapWord{f}{\ell} \infleq \GapWord{g}{\ell}$ or vice versa.
\end{lemma}

If we additionally assume that such a language is closed under taking infixes,
we obtain an even stronger structure: All such languages are regular!

\begin{lemma}
    \label{dwclosed-infixes-wqo:lem}
    \proofref{dwclosed-infixes-wqo:lem}
    Let $L \subseteq \Sigma^*$ be a \kl{downwards closed} language for the
    \kl{infix relation} that is \kl{well-quasi-ordered}. Then, the following
    are equivalent:
    {\renewcommand{\theenumi}{\roman{enumi}}
     \renewcommand{\labelenumi}{(\theenumi)}
    \begin{enumerate}
        \item\label{dwci-reg:item} $L$ is a \kl{regular language},
        \item\label{dwci-aml:item} $L$ is recognized by \emph{some} \kl{amalgamation system},
        \item\label{dwci-bod:item} $L$ is a \kl{bounded language},
        \item\label{dwci-uoc:item} There exists 
            a finite set $E \subseteq (\Sigma^*)^3$
            such that $L = \bigcup_{(x,u,y) \in E} \InfPeriodChain{x} u \InfPeriodChain{y}$.
    \end{enumerate}
    }
\end{lemma}

\AP Combining \cref{thue-morse:lemma,dwclosed-infixes-wqo:lem}, we can conclude
that the collection of \kl{infixes} of the \kl{Thue-Morse sequence} cannot be
recognized by \emph{any} \kl{amalgamation system}. 


%\subsection{Effective Decision Procedures}
%\label{infixes-amalgamation-effective:subsec}

\AP To construct a decision procedure for well-quasi-orderedness under
$\infleq$, we need our \kl{amalgamation systems} to satisfy certain
\intro(amalgamation){effectiveness assumptions}. We require that for an
\kl{amalgamation system} $(\Sigma, R, E, \canrun)$, $R$ is recursively
enumerable, the function $\canrun(\cdot)$ is computable, and for any two runs
$\rho, \sigma \in R$, the set $E(\rho,\sigma)$ is computable. Additionally, we
require the class to be effectively closed under \kl{rational transductions}
\cite[Chapter 5, page 64]{BERST79}. 

Under these assumptions, one can transform the inclusion test of
\cref{bounded-language:eq} of \cref{bounded-language:thm} into an effective
procedure, using pumping arguments from \cite[Section 4.2]{ASZZ24}, which, in
turn, allows us to prove \cref{infix-wqo-is-emptiness:thm}. Since the class
$\mathcal{C}_\text{aut}$ of \kl{regular languages} and the class
$\mathcal{C}_{\text{cfg}}$ of context-free languages are examples of
\kl{effective amalgamative classes}, the following corollary is immediate.

\begin{corollary}
    \label{aut-cfg-infix:cor}
    Let $\mathcal{C} \in \set{ \mathcal{C}_\text{aut}, \mathcal{C}_{\text{cfg}}}$.
    It is decidable whether a language in $\mathcal{C}$ is \kl{well-quasi-ordered}
    by the \kl{infix relation}.
    Furthermore, whenever it is \kl{well-quasi-ordered} by the \kl{infix relation},
    it is a \kl{bounded language}.
\end{corollary}

% LTeX: language=en-GB 
% !TeX root=../wqo-on-words.tex
\section{Conclusion}
\label{conclusion:sec}

We provided concretes statements that justify why the \kl{subword relation}
is used when defining \kl{well-quasi-orders} on finite words. Even if
\kl{prefix}, \kl{suffix} or \kl{infix} relations are meaningful, they are
\kl{well-quasi-ordered} if and only if they behave similarly to disjoint copies
of $\Nat$ or $\Nat^2$. However, our approach suffers some limitations 
and opens the road to natural continuation of this line of work.

\paragraph*{Towards infinite alphabets} In this paper, we restricted our
attention to \emph{finite} alphabets, having in mind the application to
\kl{regular languages}. However, the conclusions of
\cref{bounded-language:thm}, \cref{small-ordinal-invariants:thm}, and
\cref{prefixes:thm} could be conjecture to hold in the case of infinite
alphabets (themselves equipped with a \kl{well-quasi-ordering}). This would
require new techniques, as the finiteness of the alphabet is crucial to all of
our positive results.

\paragraph*{Monoid equations}  It could be interesting to understand which
monoids $M$ recognize languages that are \kl{well-quasi-ordered} by the
\kl{infix}, \kl{prefix} or \kl{suffix} relations. This research direction is
connected to finding which classes of graphs of \emph{bounded clique-width} are
\kl{well-quasi-ordered} with respect to the \emph{induced subgraph relation},
as shown in \cite{DRT10}, and recently revisited by one of the authors in
\cite{L24:arxiv:v2}.

\paragraph*{Complexity} We have chosen to disregard complexity considerations
when proving decidability results, as we do not believe that a fined grained
complexity analysis would be particularly enlightening. However, we conjecture
that for regular languages, the complexity should be polynomial in the size of
the minimal automaton recognizing the language.


\paragraph*{Lexicographic orderings} There is another natural ordering on
words, the \emph{lexicographic ordering}, which does not fit well in our
current framework because it is always of \kl{ordinal width} $1$. However, the
order-type of the lexicographic ordering over \kl{regular languages} has
already been investigated in the context of infinite words \cite{CACOPU18}, and
it would be interesting to see if one can extend these results to decide
whether such an ordering is \kl{well-founded} for languages recognized by
\kl{amalgamation systems}.


\subsubsection{Factor Complexity}

\AP Let us conclude this section with a few remarks on the notion of \kl{factor
complexity} of languages. Recall that the \intro{factor complexity} of a
language $L \subseteq \Sigma^*$ is the function $f_L: \Nat \to \Nat$ such that
$f_L(n)$ is the number of distinct words of size $n$ in $L$. A language $L$ has
\intro{exponential factor complexity} if there exists a constant $C > 1$ such
that $f_L(n) \geq C^n$ for infinitely many $n \in \Nat$. A language $L$ has
\intro{subaffine factor complexity} if $f_L(n) \leq C \cdot n + K$ for some
constants $C, K > 0$ and all $n \in \Nat$.
We extend the notion of \kl{factor complexity} to infinite and bi-infinite words
as the \kl{factor complexity} of their set of finite \kl{infixes}.

While there clearly are languages with \kl{subaffine factor complexity} that are
not \kl{well-quasi-ordered} for the \kl{infix relation}, such as 
the language $L \defined \dwset{ a b^* a }$;
one would expect that languages that are \kl{well-quasi-ordered} for the
\kl{infix relation} would have a low \kl{factor complexity}.

\AP It is the case that \kl{bounded languages}, studied in
\cref{infixes-bounded:sec}, have a polynomial factor complexity: such a
language is by definition included in $w_1^* \cdots w_n^*$ for some words $w_1,
\dots, w_n$, and an infix is a choice of the number of occurrences of each
$w_i$ in the infix, plus a choice of beginning and ending of the infix in the
leftmost and rightmost $w_i$. We showed in \cref{bounded-language:thm} and
particularly in \cref{bounded-language:lem}, that \kl{bounded languages} that
are \kl{well-quasi-ordered} for the \kl{infix relation} are in fact included in
some finite union of languages of the form $w_1^* w_2 w_3^*$, hence have
actually a quadratic factor complexity. However,
\cref{exponential-factor-complexity:lem} shows that there are
\kl{downwards closed} languages that are \kl{well-quasi-ordered} for the
\kl{infix relation} but have an \kl{exponential factor complexity}.


\begin{lemma}
  \label{exponential-factor-complexity:lem}
  There exists a language $L$ with \kl{exponential factor complexity}
  that is \kl{well-quasi-ordered} for the \kl{infix relation}.
\end{lemma}
\begin{proof}
  We leverage results from \cite{CAKA97} on Toeplitz words.
  Namely, the 
  $(5, 3)$-Toeplitz word is \kl{uniformly recurrent} 
  \cite[p. 499]{CAKA97},
  but has an exponential factor complexity 
  \cite[Theorem 5]{CAKA97}.
\end{proof}


Our understanding of the situation is that whenever a computational model is
fixed to represent languages that are \kl{well-quasi-ordered} for the \kl{infix
relation}, the \kl{factor complexity} of the represented languages drops to a
quadratic one. This is the case for languages recognized by \kl{amalgamation
systems} \cref{amalgamation-systems:subsec}, that are \kl{bounded languages}
when they are \kl{well-quasi-ordered} (\cref{infix-amalgamation:thm}). This is
also the case for languages described as the infixes of a finite set of pairs
of \kl{morphic sequences}. Indeed, the factor complexity of a \kl{morphic
sequence} that is \kl{uniformly recurrent} is linear \cite[Theorem 24]{NIPR09},
therefore the \kl{factor complexity} of a language given by \kl{sequence
representation} using \kl{morphic sequences} is at most quadratic.




% Include the bibliography
\bibliographystyle{plainurl}
\bibliography{papers.bib}

% If there are any appendices, we include them here.


\end{document}

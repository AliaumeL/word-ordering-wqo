\section{Trash}

Let us first play this game for languages that are recognized by
finite automata. We assume that the reader is familiar with the notions of
deterministic finite automaton and regular languages, and refer to the book of
Thomas for a comprehensive introduction to the subject \cite{THOM97}. The main
goal of the remainder of this section is to prove the following
\cref{infix-finite-automata:thm}.

\begin{theorem}[restate=infix-finite-automata:thm,label=infix-finite-automata:thm]
    \proofref{infix-finite-automata:thm}
    Let $L \subseteq \Sigma^*$ be a language recognized by a finite automaton.
    Then $L$ is \kl{well-quasi-ordered} by the \kl{infix relation} if and only if $L$ is
    a finite union of \kl{chains} for the \kl{infix relation}.
\end{theorem}
\begin{proofof}{infix-finite-automata:thm}
    Let $w \in L$, because $Q$ is finite, there exists
    a factorization of $w$
    into words $(\prod_{i = 1}^k u_i v_i) u_{k+1}$
    such that 
    for all $1 \leq i \leq k+1$, $\card{u_i} \leq \card{Q}$,
    for all $1 \leq i \leq k$, $1 \leq \card{v_i} \leq \card{Q}$,
    and satisfying 
    that $w[\vec{X}] \defined 
    (\prod_{i = 1}^k u_i v_i^{X_i}) u_{k+1}$
    belongs to $L$ for all choices of values $\vec{X} \in \Nat^k$.

    Applying \cref{pumping-periods:lem}, we conclude that 
    there exists $x,y \in \Sigma^+$ of size at most $\card{Q}$
    such that 
    $w \in u_1 \InfPeriodChain{x} \cup \InfPeriodChain{y} u_{k+1}
    \cup \bigcup_{1 \leq i \leq k+1} \InfPeriodChain{x} u_i \InfPeriodChain{y}$.
    In particular, we conclude that
    \begin{equation}
        \label{infix-automata:eq}
        L
        \subseteq
        \bigcup_{x,y,u\in \Sigma^{\leq \card{Q}}}
        u \InfPeriodChain{x}
        \cup 
        \InfPeriodChain{x} u
        \cup
        \InfPeriodChain{x} u \InfPeriodChain{y}
        \quad .
    \end{equation}

    Now, thanks to \cref{inf-period-union-chains:lem},
    this happens to be a finite union of \kl{chains}
    for the \kl{infix relation}.
\end{proofof}

\begin{corollary}
    \label{reg-wqo-decidable:cor}
    Given a regular language $L$, it is decidable whether
    $L$ is \kl{well-quasi-ordered} for the \kl{infix relation}.
\end{corollary}
\begin{proof}
    It suffices to decide whether 
    \cref{infix-automata:eq} holds for the language $L$.
    This is an inclusion of regular languages, which is decidable.
\end{proof}


Let $(X,\preceq)$ be a quasi-ordered set, and $L \subseteq X$ be such that $(L,
\preceq)$ is \kl{well-quasi-ordered}. It is not true in general that
$(\dwset{L}, \preceq)$ is \kl{well-quasi-ordered}. In the case of $(\Sigma^*,
\infleq)$ a typical example is to start from an infinite \kl{antichain} $A$,
together with an enumeration $\seqof{w_i}$ of $A$, and build the language $L
\defined \setof{ \prod_{i = 0}^n w_i }{ i \in \Nat }$. By definition, $L$ is a
\kl{chain} for the \kl{infix} ordering, hence \kl{well-quasi-ordered}. However,
$\dwset[\infleq]{L}$ contains $A$, and is therefore not
\kl{well-quasi-ordered}. As a fun corollary of our analysis, we conclude that
this particular example cannot be realized by a regular language $A$.

\begin{corollary}
    \label{reg-wqo-iff-dwclosed-wqo:cor}
    Given a regular language $L$, it is \kl{well-quasi-ordered} 
    for the \kl{infix relation} if and only if 
    $\dwset[\infleq] L$ is.
\end{corollary}

% LTeX: language=en-GB 
\section{Prefixes and Suffixes}
\label{prefixes:sec}

In this section, we study the well-quasi-ordering of languages under the
\kl{prefix relation}. Let us immediately remark that the map $u \mapsto u^R$
that reverses a word is an order-bijection between $(X^*, \prefleq)$ and $(X^*,
\suffleq)$, that is, $u \prefleq v$ if and only if $u^R \suffleq v^R$.
Therefore, we will focus on the prefix relation in the rest of this section, as
$(L, \prefleq)$ is \kl{well-quasi-ordered} if and only if $(L^R, \suffleq)$ is.

The next remark we make is that $\Sigma^*$ is not \kl{well-quasi-ordered} by
the \kl{prefix relation} as soon as $\Sigma$ contains two distinct letters $a$
and $b$. As an example of infinite \kl{antichain}, we can consider the set of
words $a^n b$ for $n \in \Nat$. As mentioned in the introduction, there are
however some languages  that are \kl{well-quasi-ordered} by the \kl{prefix
relation}. A simple example being the (regular) language $a^* \subseteq
\set{a,b}^*$, which is order-isomorphic to natural numbers with their usual
orderings $(\Nat, \leq)$.

In order to characterize the existence of infinite \kl{antichains} for the
\kl{prefix relation}, we will introduce the following tree construction that
will be useful in the rest of this section.

\begin{figure}
    \centering
    \includestandalone{src/antichain-branch-standalone}
    \caption{An \kl{antichain branch} for the language $a^* b$,
        represented in the \kl{tree of prefixes} over the alphabet $\set{a,b}$.
        The branch is represented with dashed lines in turquoise, and the
        \kl{antichain} is represented in dotted lines in blood-red.
    }
    \label{antichain-branch:fig}
\end{figure}

\begin{definition}
    The \intro{tree of prefixes} over a finite alphabet $\Sigma$
    is the infinite tree $T$ whose nodes are the words of $\Sigma^*$, and
    such that the children of a word $w$ are the words $wa$ for all $a \in
    \Sigma$. 
\end{definition}

We will use this \kl{tree of prefixes} to find simple witnesses
of the existence of infinite \kl{antichains} in the \kl{prefix relation}
for a given language $L$, namely by introducing \kl{antichain branches}.

\begin{definition}
    An \intro{antichain branch} for a language $L$ is an infinite 
    branch $B$ of the \kl{tree of prefixes} such that from every point of the branch, 
    one can reach a word in $L \setminus B$.
\end{definition}

Let us illustrate the notion of \kl{antichain branch} over the alphabet $\Sigma
= \set{a,b}$, and the language $L = a^* b$. In this case, the set $a^*$ (which
is a branch of the \kl{tree of prefixes}) is an \kl{antichain branch} for $L$.
This holds because for any $a^k$, the word $a \prefleq a^kb$ belongs to $L
\setminus a^*$. In general, the existence of an \kl{antichain branch} for a
language $L$ implies that $L$ contains an infinite \kl{antichain}, and because
the alphabet $\Sigma$ is assumed to be finite, one can leverage the fact that
the \kl{tree of prefixes} is finitely branching to prove that the converse
holds as well.

\begin{lemma}
    \label{antichain-branches-prefix:lem}
    Let $L \subseteq \Sigma^*$ be a language. Then, $L$ contains an infinite
    \kl{antichain} if and only if there exists an \kl{antichain branch} for $L$.
\end{lemma}
\begin{proof}
    Assume that $L$ contains an \kl{antichain branch}. Let us construct an
    infinite \kl{antichain} as follows. We start with a set $A_0 \defined
    \emptyset$ and a node $v_0$ at the root of the tree. At step $i$, we
    consider a word $w_i$ such that $v_i$ is a \kl{prefix} of $w_i$, and $w_i
    \in L \setminus B$, which exists by definition of \kl{antichain branches}.
    We then set $A_{i+1} \defined A_i \cup \set{w_i}$. To compute $v_{i+1}$, we
    consider the largest prefix of $w_i$ that belongs to $B$, and set $v_{i+1}$
    to be the successor of this prefix in $B$. By an immediate induction, we
    conclude that for all $i \in \Nat$, $A_i$ is an \kl{antichain}, and that
    $v_i$ is a node in the \kl{antichain branch} $B$ such that $v_i$ is not a
    prefix of any word in $A_i$. 

    Conversely, assume that $L$ contains an infinite \kl{antichain} $A$. Let us
    construct an \kl{antichain branch}. Let us consider the subtree of the
    \kl{tree of prefixes} that consists in words that are \kl{prefixes} of
    words in $A$. This subtree is infinite, and by König's lemma, it contains
    an infinite branch. By definition this is an \kl{antichain branch}.
\end{proof}

One immediate application of \cref{antichain-branches-prefix:lem} is that
\kl{antichain branches} can be described inside the \kl{tree of prefixes} by a
monadic second order formula (\kl{$\MSO$-formula}), allowing us to leverage the
decidability of $\MSO$ over infinite binary trees \cite[Theorem 1.1]{RAB69}.

\begin{corollary}
    \label{prefix-wqo-reg-decidable:cor}
    If $L$ is regular, then the existence of an infinite antichain is decidable.
\end{corollary}
\begin{proof}
    If $L$ is regular, then it is \kl{$\MSO$-definable}, and there 
    exists a formula $\varphi(x)$ in \kl{$\MSO$} that selects nodes 
    of the \kl{tree of prefixes} that belong to $L$. Now, to decide whether there
    exists an \kl{antichain branch} for $L$, we can simply check whether
    the following formula is satisfied:
    \begin{equation*}
        \exists B. 
        B \text{ is a branch } \land
        \forall x \in B, \exists y. y \text{ is a child of } x \land \varphi(y) \land y \not\in B
        \quad .
    \end{equation*}
    Because the above formula is an $\MSO$-formula over the infinite
    $\Sigma$-branching tree, whether it is satisfied is decidable
    as an easy consequence of the decidability of $\MSO$ over infinite binary
    trees
    \cite[Theorem 1.1]{RAB69}.
\end{proof}

Let us now go further and fully characterize languages $L$ such that the
prefix relation is well-quasi-ordered, even without any restriction on the
decidability of $L$ itself. Let us remark that finite unions of \kl{chains} are
always \kl{well-quasi-ordered} by the \kl{prefix relation} because they lack
infinite \kl{antichains} by definition. The following theorem states that this
is the only possible reason for a language $L$ to be \kl{well-quasi-ordered} by
the \kl{prefix relation}.

For the proof of the following theorem, we will use special notations to
describe the \intro{upwards closure} of a set $S$ for a relation $\preceq$,
which is defined as $\upset[\preceq]{S} \defined \setof{y \in \Sigma^*}{
\exists x \in S. x \preceq y}$. Anticipating the use of the symmetric notion of
\intro{downwards closure}, let us introduce it as follows: $\dwset[\preceq]{S} \defined \setof{y
\in \Sigma^*}{ \exists x \in S. y \preceq x}$. Abusing notations, we will
write $\upset{w}$ and $\dwset{w}$ for the upwards and downwards closure of a
single element $w$, omitting the ordering relation when it is clear from the
context.

\begin{figure}
    \centering
    TODO.
    \caption{Illustrating the construction of 
        \cref{prefixes:thm} consisting in taking the words
        that lead to prefix-incomparable elements in $S$,
        and then the upwards closure of the maximal such elements.
    }
    \label{prefixes-theorem:fig}
\end{figure}

\begin{theorem}
    \label{prefixes:thm}
    A language $L \subseteq \Sigma^*$ is \kl{well-quasi-ordered} by the
    \kl{prefix relation} if and only if $L$ is a union of \kl{chains}.
\end{theorem}
\begin{proof}
    Assume that $L$ is a finite union of \kl{chains}.
    Because the \kl{prefix relation} is \kl{well-founded},
    and that finite unions of \kl{chains} have finite \kl{antichains},
    we conclude that $L$ is \kl{well-quasi-ordered}.

    Conversely, assume that $L$ is \kl{well-quasi-ordered} by the \kl{prefix
    relation}. Let us define $S$ the set of words $w$ such that there exists
    two words $wu$ and $wv$ both in $L$ that are not comparable for the
    \kl{prefix relation}. Assume by contradiction that $S$ is infinite.
    Then, $S$ equipped with the \kl{prefix relation} is an infinite
    tree with finite branching, and therefore contains an infinite
    branch, which is by definition an \kl{antichain branch} for $L$.
    This contradicts the assumption that $L$ is \kl{well-quasi-ordered}.
    Now, consider $w$ be a maximal element for the \kl{prefix ordering}
    in $S$. By definition, all words in $L$ that contain $w$ as a prefix
    must be comparable for the \kl{prefix relation}. Therefore,
    $(\upset[\prefleq]{w}) \cap L$ is a \kl{chain} for the \kl{prefix relation}.
    In particular, letting $S_{\max}$ be the set of all maximal elements
    of $S$,
    we conclude that 
    \begin{equation*}
        L \subseteq S \cup \bigcup_{w \in S_{\max}} (\upset[\prefleq]{w}) \cap L
        \quad .
    \end{equation*}
    Hence, that $L$ is finite union of \kl{chains}.
\end{proof}

As an immediate consequence, we have a very fine-grained understanding of the
\kl{ordinal invariants} of such \kl{well-quasi-ordered} languages, which could be
leveraged in bounding the complexity of algorithms working on such languages.

\begin{corollary}
    Let $L \subseteq \Sigma^*$ be an infinite language that is \kl{well-quasi-ordered} by
    the \kl{prefix relation}. Then, there exists finite numbers $k,\ell \in \Nat$ such that
    the
    \kl{maximal order type} of $L$ is $k \cdot \omega$,
    the \kl{ordinal height} of $L$ is $\omega$, and its
    \kl{ordinal width} is $\ell$.
\end{corollary}

Let us conclude by noting that it is unsurprisingly not possible to decide
whether a decidable language is \kl{well-quasi-ordered} by the \kl{prefix
relation}. This is a very easy result whose sole purpose is to contrast with
the decidability result of \cref{prefix-wqo-reg-decidable:cor}.

\begin{lemma}
    The following problem is undecidable: given a language $L$
    decided by a Turing machine, answer whether 
    $L$ is \kl{well-quasi-ordered} for the \kl{prefix relation}.
\end{lemma}
\begin{proof}
    We reduce the halting problem on the empty string $\varepsilon$.
    Let $M$ be a Turing Machine, we write the languages $L$ of finite runs
    of $M$ starting on the empty string,
    that we surround by special markers. This language is decidable,
    and 
    is
    \kl{well-quasi-ordered} if and only if it is finite
    if and only if $M$ terminates on $\varepsilon$.
\end{proof}

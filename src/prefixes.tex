% LTeX: language=en-GB 
\section{Prefixes and Suffixes}
\label{prefixes-suffixes:sec}

In this section, we study the well-quasi-ordering of languages under the prefix
relation. A word $u$ is a \intro{prefix} of a word $w$ if there exists a word
$v$ such that $w = uv$. We denote this relation by $u \intro*\prefleq w$.
Similarly, a word $u$ is a \intro{suffix} of a word $w$ if there exists a word
$v$ such that $w = vu$. We denote this relation by $u \intro*\suffleq w$. Let
us immediately remark that the map $u \mapsto u^R$ that reverses a word is an
order-bijection between $(X^*, \prefleq)$ and $(X^*, \suffleq)$. Therefore, we
will focus on the prefix relation in the rest of this section.


Let us briefly restate the fact that some (even regular) languages 
are not well-quasi-ordered by the prefix relation.

\begin{example}
    The set $L = \setof{a^nb}{n \in \Nat} \subseteq \set{a,b}^*$ is an infinite \kl{antichain} for the
    \kl{prefix relation}.
\end{example}

In order to characterize the existence of infinite antichains for the prefix
relation, we will introduce the following tree construction that
will be useful in the rest of this section.

\begin{definition}
    The \intro{tree of prefixes} over a finite alphabet $\Sigma$
    is the infinite tree $T$ whose nodes are the words of $\Sigma^*$, and
    such that the children of a word $w$ are the words $wa$ for all $a \in
    \Sigma$. 
\end{definition}

\begin{figure}
    \centering
    \caption{The tree of prefixes over the alphabet $\set{a,b}$,
        with an \kl{antichain} in red, and
        an \kl{antichain branch} in blue.}
\end{figure}

Notice that the tree of prefixes is finitely branching. Let us now
observe how antichains in the prefix relation can be witnessed
by infinite branches in the tree of prefixes.

\begin{definition}
    An \intro{antichain branch} for a language $L$ is an infinite 
    branch $B$ of the \kl{tree of prefixes} such that from every point of the branch, 
    one can reach a word of $L$ that is not in the branch.
\end{definition}

It is clear that an \kl{antichain branch} for a language $L$ yields an infinite
antichain, and the converse is quite easy to prove. Because the notion of
antichain branch is \kl{$\MSO$-definable} whenever $L$ is regular, we
immediately obtain decidability as a corollary of this lemma.

\begin{lemma}
    Let $L \subseteq \Sigma^*$ be a language. Then, $L$ contains an infinite
    \kl{antichain} if and only if there exists an \kl{antichain branch} for $L$.
\end{lemma}
\begin{proof}
    Assume that $L$ contains an \kl{antichain branch}. Let us construct an
    infinite \kl{antichain} as follows. We start with a set $A_0 \defined
    \emptyset$ and a node $v_0$ at the root of the tree. At step $i$, we
    consider a word $w_i$ such that $v_i$ is a \kl{prefix} of $w_i$, and $w_i
    \in L \setminus B$, which exists by definition of \kl{antichain branches}.
    We then set $A_{i+1} \defined A_i \cup \set{w_i}$. To compute $v_{i+1}$, we
    consider the largest prefix of $w_i$ that belongs to $B$, and set $v_{i+1}$
    to be the successor of this prefix in $B$. By an immediate induction, we
    conclude that for all $i \in \Nat$, $A_i$ is an \kl{antichain}, and that
    $v_i$ is a node in the \kl{antichain branch} $B$ such that $v_i$ is not a
    prefix of any word in $A_i$. 

    Conversely, assume that $L$ contains an infinite \kl{antichain} $A$. Let us
    construct an \kl{antichain branch}. Let us consider the subtree of the
    \kl{tree of prefixes} that consists in words that are \kl{prefixes} of
    words in $A$. This subtree is infinite, and by König's lemma, it contains
    an infinite branch. By definition this is an \kl{antichain branch}.
\end{proof}

\begin{corollary}
    If $L$ is regular, then the existence of an infinite antichain is decidable.
\end{corollary}
\begin{proof}
    If $L$ is regular, then it is \kl{$\MSO$-definable}, and there 
    exists a formula $\varphi(x)$ in \kl{$\MSO$} that selects nodes 
    of the \kl{tree of prefixes} that belong to $L$. Now, to decide whether there
    exists an \kl{antichain branch} for $L$, we can simply check whether
    the following formula is satisfied:
    \begin{equation*}
        \exists B. 
        B \text{ is a branch } \land
        \forall x \in B, \exists y. y \text{ is a child of } x \land \varphi(y) \land y \not\in B
        \quad .
    \end{equation*}
    Because the above formula is an $\MSO$-formula over the infinite
    binary tree, whether it is satisfied is decidable.
    \textbf{TODO: cite}.
\end{proof}

Let us now go further and fully characterize languages $L$ such that the
prefix relation is well-quasi-ordered, even without any restriction on the
decidability of $L$ itself. Let us remark that finite unions of \kl{chains} are
always \kl{well-quasi-ordered} by the \kl{prefix relation} because they lack
infinite \kl{antichains} by definition. The following theorem states that this
is the only possible reason for a language $L$ to be \kl{well-quasi-ordered} by
the \kl{prefix relation}.

\begin{theorem}
    A language $L \subseteq \Sigma^*$ is \kl{well-quasi-ordered} by the
    \kl{prefix relation} if and only if $L$ is a union of \kl{chains}.
\end{theorem}
\begin{proof}
    Assume that $L$ is a finite union of \kl{chains}.
    Because the \kl{prefix relation} is \kl{well-founded},
    and that finite unions of \kl{chains} have finite \kl{antichains},
    we conclude that $L$ is \kl{well-quasi-ordered}.

    Conversely, assume that $L$ is not a finite union of \kl{chains}. Then
    there exists an infinite \kl{antichain branch} $B$ for $L$. Indeed
    \textbf{todo}. In particular, $L$ is not \kl{well-quasi-ordered}.
\end{proof}

As an immediate consequence, we have a very fine-grained understanding of the
\kl{ordinal invariants} of such \kl{well-quasi-ordered} languages, which can be
leveraged in bounding the complexity of algorithms working on such languages.

\begin{corollary}
    \textbf{false: some chains can be finite}
    Let $L \subseteq \Sigma^*$ be a language that is \kl{well-quasi-ordered} by
    the \kl{prefix relation}. Then, there exists a finite $k \in \Nat$ such that
    the
    \kl{maximal order type} of $L$ is $k \cdot \omega$,
    the \kl{ordinal height} of $L$ is $\omega$, and its
    \kl{ordinal width} is $k$.
\end{corollary}


Let us conclude by noting that it is unsurprisingly not possible to decide
whether a decidable language is \kl{well-quasi-ordered} by the \kl{prefix
relation}.

\begin{lemma}
    todo.
\end{lemma}

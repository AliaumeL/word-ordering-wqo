% LTeX: language=en-GB 
% !TeX root=../wqo-on-words.tex
\section{Prefixes and Suffixes}
\label{prefixes:sec}

In this section, we study the well-quasi-ordering of languages under the
\kl{prefix relation}. Let us immediately remark that the map $u \mapsto u^R$
that reverses a word is an order-bijection between $(X^*, \prefleq)$ and $(X^*,
\suffleq)$, that is, $u \prefleq v$ if and only if $u^R \suffleq v^R$.
Therefore, we will focus on the prefix relation in the rest of this section, as
$(L, \prefleq)$ is \kl{well-quasi-ordered} if and only if $(L^R, \suffleq)$ is.

The next remark we make is that $\Sigma^*$ is not \kl{well-quasi-ordered} by
the \kl{prefix relation} as soon as $\Sigma$ contains two distinct letters $a$
and $b$. As an example of infinite \kl{antichain}, we can consider the set of
words $a^n b$ for $n \in \Nat$. As mentioned in the introduction, there are
however some languages  that are \kl{well-quasi-ordered} by the \kl{prefix
relation}. A simple example being the (regular) language $a^* \subseteq
\set{a,b}^*$, which is order-isomorphic to natural numbers with their usual
orderings $(\Nat, \leq)$.

In order to characterize the existence of infinite \kl{antichains} for the
\kl{prefix relation}, we will introduce the following tree.

\begin{definition}
    The \intro{tree of prefixes} over a finite alphabet $\Sigma$
    is the infinite tree $T$ whose nodes are the words of $\Sigma^*$, and
    such that the children of a word $w$ are the words $wa$ for all $a \in
    \Sigma$. 
\end{definition}

We will use this \kl{tree of prefixes} to find simple witnesses
of the existence of infinite \kl{antichains} in the \kl{prefix relation}
for a given language $L$, namely by introducing \kl{antichain branches}.

\begin{definition}
    An \intro{antichain branch} for a language $L$ is an infinite 
    branch $B$ of the \kl{tree of prefixes} such that from every point of the branch, 
    one can reach a word in $L \setminus B$. Formally:
    $\forall u \in B, \exists v \in \Sigma^*, uv \in L \setminus B$.
\end{definition}

Let us illustrate the notion of \kl{antichain branch} over the alphabet $\Sigma
= \set{a,b}$, and the language $L = a^* b$. In this case, the set $a^*$ (which
is a branch of the \kl{tree of prefixes}) is an \kl{antichain branch} for $L$.
This holds because for any $a^k$, the word $a^k \prefleq a^kb$ belongs to $L
\setminus a^*$. In general, the existence of an \kl{antichain branch} for a
language $L$ implies that $L$ contains an infinite \kl{antichain}, and because
the alphabet $\Sigma$ is assumed to be finite, one can leverage the fact that
the \kl{tree of prefixes} is finitely branching to prove that the converse
holds as well.

\begin{lemma}
    \label{antichain-branches-prefix:lem}
    \proofref{antichain-branches-prefix:lem}
    Let $L \subseteq \Sigma^*$ be a language. Then, $L$ contains an infinite
    \kl{antichain} if and only if there exists an \kl{antichain branch} for $L$.
\end{lemma}

One immediate application of \cref{antichain-branches-prefix:lem} is
that \kl{antichain branches} can be described inside the \kl{tree of prefixes}
by a monadic second order formula ($\MSO$-formula), allowing us to
leverage the decidability of $\MSO$ over infinite binary trees \cite[Theorem
1.1]{RAB69}. This result will follow from our general decidability result
(\cref{infix-wqo-is-emptiness:thm}) but is worth stating on its own for its
simplicity.

\begin{corollary}
    \label{prefix-wqo-reg-decidable:cor}
    \proofref{prefix-wqo-reg-decidable:cor}
    If $L$ is regular, then the existence of an infinite antichain is decidable.
\end{corollary}

Let us now go further and fully characterize languages $L$ such that the
prefix relation is well-quasi-ordered, without any restriction on the
decidability of $L$ itself.
For the proof of the following theorem, we will use special notations to
describe the \intro{upwards closure} of a set $S$ for a relation $\preceq$,
which is defined as $\upset[\preceq]{S} \defined \setof{y \in \Sigma^*}{
\exists x \in S. x \preceq y}$. Anticipating the use of the symmetric notion of
\intro{downwards closure}, let us introduce it as follows: $\dwset[\preceq]{S} \defined \setof{y
\in \Sigma^*}{ \exists x \in S. y \preceq x}$. Abusing notations, we will
write $\upset{w}$ and $\dwset{w}$ for the upwards and downwards closure of a
single element $w$, omitting the ordering relation when it is clear from the
context. A set $S$ is called \intro{downwards closed} if $\dwset{S} = S$.

% \begin{figure}
%     \centering
%     \includestandalone[width=\linewidth]{fig/prefix-core-and-branches-standalone}
%     \caption{Illustrating the construction of 
%         \cref{prefixes:thm}. We represent the Hasse Diagram of a
%         \kl{partial order}, with elements in $S$ being bigger and 
%         in desert sand, elements of $S_{\max}$ being in
%         blood-red, and \kl{chains} of elements in sky blue.
%         The upwards closure of an element in $S_{\max}$ 
%         is depicted in pale green, and cannot intersect two distinct
%         branches, as it does not contain two incomparable elements.
%     }
%     \label{prefixes-theorem:fig}
% \end{figure}

\begin{theorem}
    \label{prefixes:thm}
    \proofref{prefixes:thm}
    A language $L \subseteq \Sigma^*$ is \kl{well-quasi-ordered} by the
    \kl{prefix relation} if and only if $L$ is a union of \kl{chains}.
\end{theorem}

As an immediate consequence, we have a very fine-grained understanding of the
\kl{ordinal invariants} of such \kl{well-quasi-ordered} languages, which can be
leveraged in bounding the complexity of algorithms working on such languages.


\begin{corollary}
    \label{prefixes-ordinal-invariants:cor}
    Let $L \subseteq \Sigma^*$ be a language that is
    \kl{well-quasi-ordered} by the \kl{prefix relation}. Then,
    \kl{maximal order type} of $L$ strictly smaller than $\omega^2$,
    the \kl{ordinal height} of $L$ is at most $\omega$, and
    its \kl{ordinal width} is finite. Furthermore, these bounds 
    are tight.
\end{corollary}
\begin{proof}
    The upper bounds follow from the fact that $L$ is a finite union of
    \kl{chains}. The tightness can be obtained by considering the languages
    $L_k \defined \bigcup_{i=0}^{k-1} a^i b^*$ for $k \in \Nat$, which 
    are \kl{well-quasi-ordered} by the \kl{prefix relation} (as they are
    finite unions of \kl{chains}), and satisfy
    that $\oWidth{L_k} = k$, $\oHeight{L_k} = \omega$,
    and therefore $\oType{L_k} = k \cdot \omega$.
\end{proof}

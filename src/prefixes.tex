% LTeX: language=en-GB 
\section{Prefixes and Suffixes}
\label{prefixes-suffixes:sec}

In this section, we study the well-quasi-ordering of languages under the prefix
relation. A word $u$ is a \intro{prefix} of a word $w$ if there exists a word
$v$ such that $w = uv$. We denote this relation by $u \intro*\prefleq w$.
Similarly, a word $u$ is a \intro{suffix} of a word $w$ if there exists a word
$v$ such that $w = vu$. We denote this relation by $u \intro*\suffleq w$.

Let us immediately remark that the map $u \mapsto u^R$ that reverses a word is
an order-bijection between $(X^*, \prefleq)$ and $(X^*, \suffleq)$. Therefore,
we will focus on the prefix relation in the following.

As mentioned in the introduction, the prefix relation is not a well-quasi-order
on words over a two-letter alphabet, as witnessed by the infinite
\kl{antichain} $\setof{a^nb}{n \in \Nat}$. 

Let us order the words in $\Sigma^*$ by the \kl{prefix relation} to build an
infinite tree $T$. The root of the tree is the empty word $\varepsilon$, and
the children of a word $w$ are the words $wa$ for all $a \in \Sigma$. This tree
is finitely branching. Let $L \subseteq \Sigma^*$ be a language. Assume that
$L$ contains an infinite \kl{antichain} $A$ for the \kl{prefix relation}. In
particular, no two word of $A$ belong to the same branch of the tree $T$. Let
us prove the following: there exists an infinite branch $B$ of $T$ such that from
every point of the branch, one can reach a word of $A$ that is not in the
branch. 

\begin{definition}
    An \intro{antichain branch} for a language $L$ is an infinite 
    branch $B$ of the tree $T$ such that from every point of the branch, 
    one can reach a word of $A$ that is not in the branch.
\end{definition}

\begin{lemma}
    Let $L \subseteq \Sigma^*$ be a language. Then, $L$ contains an infinite
    \kl{antichain} if and only if there exists an \kl{antichain branch} for $L$.
\end{lemma}

\begin{corollary}
    If $L$ is regular, then the existence of an infinite antichain is decidable.
\end{corollary}

\begin{theorem}
    A language $L \subseteq \Sigma^*$ is \kl{well-quasi-ordered} by the
    \kl{prefix relation} if and only if $L$ is a union of \kl{chains}.
\end{theorem}

\begin{corollary}
    Let $L \subseteq \Sigma^*$ be a language that is \kl{well-quasi-ordered} by
    the \kl{prefix relation}. Then, there exists a finite $k \in \Nat$ such that
    the
    \kl{maximal order type} of $L$ is $k \cdot \omega$,
    the \kl{ordinal height} of $L$ is $\omega$, and its
    \kl{ordinal width} is $k$.
\end{corollary}


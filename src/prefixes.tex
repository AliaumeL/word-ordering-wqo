% LTeX: language=en-GB 
\section{Prefixes and Suffixes}
\label{prefixes-suffixes:sec}

In this section, we study the well-quasi-ordering of languages under the prefix
relation. A word $u$ is a \intro{prefix} of a word $w$ if there exists a word
$v$ such that $w = uv$. We denote this relation by $u \intro*\prefleq w$.
Similarly, a word $u$ is a \intro{suffix} of a word $w$ if there exists a word
$v$ such that $w = vu$. We denote this relation by $u \intro*\suffleq w$. Let
us immediately remark that the map $u \mapsto u^R$ that reverses a word is an
order-bijection between $(X^*, \prefleq)$ and $(X^*, \suffleq)$. Therefore, we
will focus on the prefix relation in the rest of this section.


Let us briefly restate the fact that some (even regular) languages 
are not well-quasi-ordered by the prefix relation.

\begin{example}
    The set $L = \setof{a^nb}{n \in \Nat}$ is an infinite antichain for the
    prefix relation.
\end{example}

In order to characterize the existence of infinite antichains for the prefix
relation, we will introduce the following tree construction that
will be useful in the rest of this section.

\begin{definition}
    The \intro{tree of prefixes} over a finite alphabet $\Sigma$
    is the infinite tree $T$ whose nodes are the words of $\Sigma^*$, and
    such that the children of a word $w$ are the words $wa$ for all $a \in
    \Sigma$. 
\end{definition}

Notice that the tree of prefixes is finitely branching. Let us now
observe how antichains in the prefix relation can be witnessed
by infinite branches in the tree of prefixes.

\begin{definition}
    An \intro{antichain branch} for a language $L$ is an infinite 
    branch $B$ of the tree $T$ such that from every point of the branch, 
    one can reach a word of $A$ that is not in the branch.
\end{definition}

It is clear that an \kl{antichain branch} for a language $L$ yields an infinite
antichain, and the converse is quite easy to prove. Because the notion of
antichain branch is \kl{$\MSO$-definable} whenever $L$ is regular, we
immediately obtain decidability as a corollary of this lemma.

\begin{lemma}
    Let $L \subseteq \Sigma^*$ be a language. Then, $L$ contains an infinite
    \kl{antichain} if and only if there exists an \kl{antichain branch} for $L$.
\end{lemma}

\begin{corollary}
    If $L$ is regular, then the existence of an infinite antichain is decidable.
\end{corollary}

Let us now go further and fully characterize languages $L$ such that the
prefix relation is well-quasi-ordered, even without any restriction on the
decidability of $L$ itself. Let us remark that finite unions of \kl{chains} are
always \kl{well-quasi-ordered} by the \kl{prefix relation} because they lack
infinite \kl{antichains} by definition. The following theorem states that this
is the only possible reason for a language $L$ to be \kl{well-quasi-ordered} by
the \kl{prefix relation}.

\begin{theorem}
    A language $L \subseteq \Sigma^*$ is \kl{well-quasi-ordered} by the
    \kl{prefix relation} if and only if $L$ is a union of \kl{chains}.
\end{theorem}

As an immediate consequence, we have a very fine-grained understanding of the
\kl{ordinal invariants} of such \kl{well-quasi-ordered} languages, which can be
leveraged in bounding the complexity of algorithms working on such languages.

\begin{corollary}
    \textbf{false: some chains can be finite}
    Let $L \subseteq \Sigma^*$ be a language that is \kl{well-quasi-ordered} by
    the \kl{prefix relation}. Then, there exists a finite $k \in \Nat$ such that
    the
    \kl{maximal order type} of $L$ is $k \cdot \omega$,
    the \kl{ordinal height} of $L$ is $\omega$, and its
    \kl{ordinal width} is $k$.
\end{corollary}


Let us conclude by noting that it is unsurprisingly not possible to decide
whether a decidable language is \kl{well-quasi-ordered} by the \kl{prefix
relation}.

\begin{lemma}
    todo.
\end{lemma}

% LTeX: language=en-GB 
% !TeX root=../wqo-on-words.lncs.tex
\section{Proofs for Section~\ref{prefixes:sec}}
\begin{figure}
    \centering
    \includestandalone{fig/antichain-branch-standalone}
    \caption{An \kl{antichain branch} for the language $a^* b$,
        represented in the \kl{tree of prefixes} over the alphabet $\set{a,b}$.
        The branch is represented with dashed lines in turquoise, and the
        \kl{antichain} is represented in dotted lines in blood-red.
    }
    \label{antichain-branch:fig}
\end{figure}


\begin{proofof}{antichain-branches-prefix:lem}
    Assume that $L$ contains an \kl{antichain branch}. Let us construct an
    infinite \kl{antichain} as follows. We start with a set $A_0 \defined
    \emptyset$ and a node $v_0$ at the root of the tree. At step $i$, we
    consider a word $w_i$ such that $v_i$ is a \kl{prefix} of $w_i$, and $w_i
    \in L \setminus B$, which exists by definition of \kl{antichain branches}.
    We then set $A_{i+1} \defined A_i \cup \set{w_i}$. To compute $v_{i+1}$, we
    consider the largest prefix of $w_i$ that belongs to $B$, and set $v_{i+1}$
    to be the successor of this prefix in $B$. By an immediate induction, we
    conclude that for all $i \in \Nat$, $A_i$ is an \kl{antichain}, and that
    $v_i$ is a node in the \kl{antichain branch} $B$ such that $v_i$ is not a
    prefix of any word in $A_i$. 

    Conversely, assume that $L$ contains an infinite \kl{antichain} $A$. Let us
    construct an \kl{antichain branch}. Let us consider the subtree of the
    \kl{tree of prefixes} that consists in words that are \kl{prefixes} of
    words in $A$. This subtree is infinite, and by König's lemma, it contains
    an infinite branch. By definition this is an \kl{antichain branch}.
\end{proofof}


\begin{proofof}{prefix-wqo-reg-decidable:cor}
	If $L$ is regular, then it is $\MSO$-definable, and there 
	exists a formula $\varphi(x)$ in $\MSO$ that selects nodes 
	of the \kl{tree of prefixes} that belong to $L$. Now, to decide whether there
	exists an \kl{antichain branch} for $L$, we can simply check whether
	the following formula is satisfied:
	\begin{equation*}
		\exists B. 
		B \text{ is a branch } \land
		\forall x \in B, \exists y. y \text{ is a child of } x \land \varphi(y) \land y \not\in B
		\quad .
	\end{equation*}
	Because the above formula is an $\MSO$-formula over the infinite
	$\Sigma$-branching tree, whether it is satisfied is decidable
	as an easy consequence of the decidability of $\MSO$ over infinite binary
	trees
	\cite[Theorem 1.1]{RAB69}.
\end{proofof}


\begin{proofof}{prefixes:thm}
    Assume that $L$ is a finite union of \kl{chains}.
    Because the \kl{prefix relation} is \kl{well-founded},
    and that finite unions of \kl{chains} have finite \kl{antichains},
    we conclude that $L$ is \kl{well-quasi-ordered}.
    
    Conversely, assume that $L$ is \kl{well-quasi-ordered} by the \kl{prefix
    relation}. Let us define $S_{\mathrm{split}}$ the set of words $w \in \Sigma^*$ 
    such that there exists
    two words $wu$ and $wv$ both in $L$ that are not comparable for the
    \kl{prefix relation}. Let $S = S_{\mathrm{split}}~\cup~\min_{\prefleq} L$\todo{kinda ugly, but i can't find a nicer way of phrasing it}. 
    Assume by contradiction that $S$ is infinite.
    Then, $S$ equipped with the \kl{prefix relation} is an infinite
    tree with finite branching, and therefore contains an infinite
    branch, which is by definition an \kl{antichain branch} for $L$.
    This contradicts the assumption that $L$ is \kl{well-quasi-ordered}.
    
    Now, let $w$ be a maximal element for the \kl{prefix ordering}
    in $S$. 
    The upward closure of $w$ in $L$, $(\upset[\prefleq]{w}) \cap L$, must be a 
    finite union of \kl{chains}. Otherwise at least two of the chains would share a common 
    prefix in $w\Sigma$, contradicting the maximality of $w$.
    
    In particular, letting $S_{\max}$ be the set of all maximal elements
    of $S$,
    we conclude that 
    \begin{equation*}
        L \subseteq S \cup \bigcup_{w \in S_{\max}} (\upset[\prefleq]{w}) \cap L
        \quad .
    \end{equation*}
    Hence, that $L$ is finite union of \kl{chains}.
\end{proofof}


\section{Proofs for Section~\ref{infixes-bounded:sec}}

\begin{figure}
    \centering
    \includestandalone[width=\linewidth]{fig/infix-encoding-standalone}
    \caption{Representation of the \kl{subword relation} for $\set{a,b}^*$
        inside the \kl{infix relation} for $\set{a,b,\#}^*$
        using a simplified version of \cref{infix-embedding:thm}, restricted to words
        of length at most $3$. 
    }
    \label{infix-embedding:fig}
\end{figure}


\begin{proofof}{inf-period-chain:lem}
    Let $x \in \Sigma^+$ be a word, and let $P_x$ be the (finite) set 
    of all \kl{prefixes} of $x$, and $S_x$ be the (finite)
    set of all \kl{suffixes} of $x$.
    Assume that $w \in \InfPeriodChain{x}$, then $w = u x^p v$ for some
    $u \in S_x$, $v \in P_x$, and $p \in \Nat$.
    We have proven that
    \begin{equation*}
        \InfPeriodChain{x} \subseteq \bigcup_{u \in P_x} \bigcup_{v \in S_x} u x^* v
        \quad .
    \end{equation*}

    Let us now demonstrate that for all $(u,v) \in S_x \times P_x$, the
    language $u x^* v$ is a \kl{chain} for the \kl{infix}, \kl{suffix} and \kl{prefix} relations.
    To that end,
    let $(u,v) \in S_x \times P_x$ and $\ell, k \in \Nat$ be such that $\ell <
    k$, let us prove that $u x^\ell v \infleq u x^k  v$. Because $v \prefleq
    x$, we know that there exists $w$ such that $vw = x$. In particular,
    $ux^\ell vw = u x^{\ell + 1}$, and because $\ell < k$, we conclude that $u
    x^{\ell + 1} \prefleq u x^k v$. By transitivity, $u x^\ell v \prefleq u x^k
    v$, and \emph{a fortiori}, $u x^\ell v \infleq u x^k v$. 
    Similarly, because $u \suffleq x$,  there exists $w$ such that $wu  = x$, 
    and we conclude that $u x^{\ell} v \suffleq w u x^\ell v = x^{\ell + 1} v \suffleq u x^k v$.
\end{proofof}

\begin{proofof}{bounded-language:lem}
	\begin{proof}
		Let $w_1, \dots, w_n$ be such that
		$L \subseteq w_1^* \cdots w_n^*$.
		Let us define $m \defined \max \setof{\card{w_i}}{1 \leq i \leq n}$
		
		Let $w[\vec{k}] \defined w_1^{k_1} \cdots w_n^{k_n}$ be a map from $\Nat^k$
		to $\Sigma^*$. We are interested in the intersection of the image of $w$
		with $L$. Let us assume for instance that for all $\vec{k} \in \Nat^n$,
		there exists $\vec{\ell} \geq \vec{k}$ such that $w[\vec{\ell}] \in L$.
		Then, leveraging \cref{pumping-periods:lem}, we conclude that there exists
		$x,y$ of size at most $\max\setof{\card{w_i}}{1 \leq i \leq n}$ such that
		$w[\vec{k}] \in \InfPeriodChain{x} \cup \InfPeriodChain{x}
		\InfPeriodChain{y}$, and we conclude that $L \subseteq \InfPeriodChain{x}
		\cup \InfPeriodChain{x} \InfPeriodChain{y}$.
		
		Now, it may be the case that one cannot simultaneously assume that two
		component of the vector $\vec{k}$ are unbounded. In general, given a set $S
		\subseteq \set{1, \dots, n}$ of indices, we say that $S$ is admissible if
		there exists a bound $N_0$ such that for all $\vec{b} \in \Nat^S$, there
		exists a vector $\vec{k} \in \Nat^n$, such that $\vec{k}$ is greater than
		$\vec{b}$ on the $S$ components, and the other components are below the
		bound $N_0$. The language of an admissible set $S$ is the set of words
		obtained by repeating $w_i$ at most $N_0$ times if it is not in $S$
		($w_i^{\leq N_0}$) and arbitrarily many times otherwise ($w_i^*$).
		Note that $L \subseteq \bigcup_{S \text{ admissible }} L(S)$.
		
		Now, admissible languages are ready to be pumped according to
		\cref{pumping-periods:lem}. For every admissible language,
		the size of a word that is not iterated is at most
		$N_0 \times m$ by definition, and we conclude that:
		\begin{equation}
			\label{bounded-language:eq}
			L \subseteq 
			\bigcup_{x,y \in \Sigma^{\leq n}}
			\bigcup_{u \in \Sigma^{\leq m \times N_0}}
			\InfPeriodChain{x} u \InfPeriodChain{y}
			\cup
			\InfPeriodChain{x} u
			\cup
			u \InfPeriodChain{x}
			\quad .
		\end{equation}
	\end{proof}
\end{proofof}


\begin{lemma}
	\label{automatic-uur:lem}
	\proofref{automatic-uur:lem}
	Given an \kl{automatic sequence} $w \in \Sigma^{\Nat}$, one can decide
	whether it is \kl{ultimately uniformly recurrent}.
\end{lemma}

\begin{proofof}{automatic-uur:lem}
	We can rewrite this as a question on the \kl{automatic sequence} $w$
	as follows:
	\begin{align*}
		&\exists N_0,                   &   \text{ultimately} \\
		&\forall i_s \geq N_0,          &   \text{for every infix (start) } u \\
		&\forall i_e > i_s,             &   \text{for every infix (end) }   u \\
		&\exists k \geq 1,              &   \text{there exists a bound} \\
		&\forall j_s \geq N_0,          &   \text{for every other infix (start) } v \\
		&\forall j_e \geq j_s + k,      &   \text{of size at least $k$} \\
		&\exists l \geq 0,              &   \text{there exists a position in } v \\
		&\forall 0 \leq m < i_e - i_s,  &   \text{where } u \text{ can be found} \\
		&j_s + m + l < j_e \land
		w(i_s + m) = w(j_s + m + l) \quad .
	\end{align*}
	Because $w$ is computable by a finite automaton, one can reduce the above
	formula to a regular language, for which it suffices to check emptiness, which
	is decidable.
\end{proofof}


\begin{proofof}{small-ordinal-invariants:lem}
	Let $N_0$ be a bound such that $w_{\geq N_0}$ is \kl{uniformly recurrent}.
	Let us write $\infset{w}$ the set of finite infixes of $w$.
	We prove that $\oWidth{\infset{w}} \leq \omegaOrd + N_0$.
	Let $u_1 \infleq w$ be a finite word. 
	
	If $u_1$ is an \kl{infix} of $w_{\geq N_0}$, then there exists $k \geq 1$
	such that $u_1$ is an \kl{infix} of every word of size at least $k$. In
	particular, there is finite bound on the length of every sequence of
	incomparable elements starting with $u_1$. We conclude in particular that
	$\infset{w} \setminus \upset{u_1}$ has a finite \kl{ordinal width}.
	
	Otherwise, $u_1$ can only be found \emph{before} $N_0$. In this case, we
	consider a second element of a \kl{bad sequence} $u_2 \infleq w$, which is
	incomparable with $u_1$ for the \kl{infix relation}. If $u_2$ is an
	\kl{infix} of $w_{\geq N_0}$, then we can conclude as before. Otherwise,
	notice that $u_1$ and $u_2$ cannot start at the same position in $w$
	(because they are incomparable). Continuing this argument, we conclude that
	there are at most $N_0$ elements starting before $N_0$
	at the start of any sequence of
	incomparable elements in $\infset{w}$. We conclude that
	$\oWidth{\infset{w}} \leq \omegaOrd + N_0$.
	
	Let us now justify that this bound is tight.
	The \kl{Thue-Morse sequence} over a binary
	alphabet $\set{a,b}$ has \kl{ordinal width} $\omegaOrd$
	from \cref{thue-morse-ordinal:lemma}.
	Given a number $N_0
	\in \Nat$, one can construct an arbitrarily long \kl{antichain} of words for
	the \kl{infix relation} by using a new letter $c$. When concatenating this
	(finite) antichain as a prefix of the \kl{Thue-Morse sequence}, one obtains a
	new (infinite) word $w$. It is clear that the \kl{ordinal width} of
	$\infset{w}$ is now at least $\omegaOrd + N_0$.
\end{proofof}



\begin{proofof}{from-bi-to-single:lem}
	Given a bi-infinite word $w \in \Sigma^{\Rel}$, recall that we can consider $w_+ \in
	\Sigma^\Nat$ and $w_- \in \Sigma^\Nat$ the two infinite words obtained as
	follows: for all $i \in \Nat$, $(w_+)_i = w(i)$ and $(w_-)_i = w(-i)$. Note
	that the two share the letter at position $0$.
	
	To obtain the upper bound of $3 \cdot \omegaOrd$, we can consider the same
	argument as for \cref{small-ordinal-invariants:lem}. We let $N_0$ be such
	that $w_{\geq N_0}$ and $(w_-)_{\geq N_0}$ are \kl{uniformly recurrent}
	words. In any sequence of incomparable elements of $\infset{w}$, there are
	less than $N_0^2$ elements that are found in $(w_{< N_0})_{> -N_0}$. Then,
	one has to pick a finite \kl{infix} in either $w_{\geq N_0}$ or $w_{\leq
		-N_0}$. Because of \cref{small-ordinal-invariants:lem}, any sequence of
	incomparable elements of these two infinite words has length bounded based
	on the choice of the first element of that sequence. This means that the
	\kl{ordinal width} of $\infset{w}$ is at most $\omegaOrd + \omegaOrd +
	N_0^2$. We conclude that $\oWidth{\infset{w}} < 3 \cdot \omegaOrd$.
	
	Let us briefly argue that the bound is tight. Indeed, one can
	construct a bi-infinite word $w$ by concatenating a reversed \kl{Thue-Morse
		sequence} on a binary alphabet $\set{a,b}$, a finite antichain of arbitrarily
	large size over a distinct alphabet $\set{c,d}$, and then a \kl{Thue-Morse
		sequence} on a binary alphabet $\set{e,f}$. The \kl{ordinal width} of the set
	of \kl{infixes} of $w$ is then at least $2 \cdot \omegaOrd + K$, where $K$ is the
	size of the chosen antichain, following the same argument as in the proof of
	\cref{small-ordinal-invariants:lem}, using \cref{thue-morse-ordinal:lemma}.
\end{proofof}

\section{Proofs for Section~\ref{infixes-amalgamation:sec}}

\begin{figure}
    \centering
    \includestandalone[width=\linewidth]{fig/gap-word-embedding-standalone}
    \caption{The \kl{gap words} resulting from a \kl{subword embedding} between two 
    finite words.}
    \label{gap-word-embedding:fig}
\end{figure}

\begin{figure}
    \centering
    \includestandalone[width=\linewidth]{fig/run-amalgamation-standalone}
    \caption{We illustrate how 
        embeddings $f$ and $g$ between runs of an
        \kl{amalgamation system} can be glued
        together, seen on their canonical decomposition.
    }
    \label{amalgamation-runs:fig}
\end{figure}



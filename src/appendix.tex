% LTeX: language=en-GB 
% !TeX root=../wqo-on-words.lncs.tex

\clearpage

\section{Proofs for Section~\ref{prefixes:sec}}
\begin{figure}
    \centering
    \includestandalone{fig/antichain-branch-standalone}
    \caption{An \kl{antichain branch} for the language $a^* b$,
        represented in the \kl{tree of prefixes} over the alphabet $\set{a,b}$.
        The branch is represented with dashed lines in turquoise, and the
        \kl{antichain} is represented in dotted lines in blood-red.
    }
    \label{antichain-branch:fig}
\end{figure}


\begin{proofof}{antichain-branches-prefix:lem}
    Assume that $L$ contains an \kl{antichain branch}. Let us construct an
    infinite \kl{antichain} as follows. We start with a set $A_0 \defined
    \emptyset$ and a node $v_0$ at the root of the tree. At step $i$, we
    consider a word $w_i$ such that $v_i$ is a \kl{prefix} of $w_i$, and $w_i
    \in L \setminus B$, which exists by definition of \kl{antichain branches}.
    We then set $A_{i+1} \defined A_i \cup \set{w_i}$. To compute $v_{i+1}$, we
    consider the largest prefix of $w_i$ that belongs to $B$, and set $v_{i+1}$
    to be the successor of this prefix in $B$. By an immediate induction, we
    conclude that for all $i \in \Nat$, $A_i$ is an \kl{antichain}, and that
    $v_i$ is a node in the \kl{antichain branch} $B$ such that $v_i$ is not a
    prefix of any word in $A_i$. 

    Conversely, assume that $L$ contains an infinite \kl{antichain} $A$. Let us
    construct an \kl{antichain branch}. Let us consider the subtree of the
    \kl{tree of prefixes} that consists in words that are \kl{prefixes} of
    words in $A$. This subtree is infinite, and by König's lemma, it contains
    an infinite branch. By definition this is an \kl{antichain branch}.
\end{proofof}


\begin{proofof}{prefix-wqo-reg-decidable:cor}
	If $L$ is regular, then it is $\MSO$-definable, and there 
	exists a formula $\varphi(x)$ in $\MSO$ that selects nodes 
	of the \kl{tree of prefixes} that belong to $L$. Now, to decide whether there
	exists an \kl{antichain branch} for $L$, we can simply check whether
	the following formula is satisfied:
	\begin{equation*}
		\exists B. 
		B \text{ is a branch } \land
		\forall x \in B, \exists y. y \text{ is a child of } x \land \varphi(y) \land y \not\in B
		\quad .
	\end{equation*}
	Because the above formula is an $\MSO$-formula over the infinite
	$\Sigma$-branching tree, whether it is satisfied is decidable
	as an easy consequence of the decidability of $\MSO$ over infinite binary
	trees
	\cite[Theorem 1.1]{RAB69}.
\end{proofof}


\begin{proofof}{prefixes:thm}
    Assume that $L$ is a finite union of \kl{chains}.
    Because the \kl{prefix relation} is \kl{well-founded},
    and that finite unions of \kl{chains} have finite \kl{antichains},
    we conclude that $L$ is \kl{well-quasi-ordered}.
    
    Conversely, assume that $L$ is \kl{well-quasi-ordered} by the \kl{prefix
    relation}. Let us define $S_{\mathrm{split}}$ the set of words $w \in \Sigma^*$ 
    such that there exists
    two words $wu$ and $wv$ both in $L$ that are not comparable for the
    \kl{prefix relation}. Let $S = S_{\mathrm{split}}~\cup~\min_{\prefleq} L$\todo{kinda ugly, but i can't find a nicer way of phrasing it}. 
    Assume by contradiction that $S$ is infinite.
    Then, $S$ equipped with the \kl{prefix relation} is an infinite
    tree with finite branching, and therefore contains an infinite
    branch, which is by definition an \kl{antichain branch} for $L$.
    This contradicts the assumption that $L$ is \kl{well-quasi-ordered}.
    
    Now, let $w$ be a maximal element for the \kl{prefix ordering}
    in $S$. 
    The upward closure of $w$ in $L$, $(\upset[\prefleq]{w}) \cap L$, must be a 
    finite union of \kl{chains}. Otherwise at least two of the chains would share a common 
    prefix in $w\Sigma$, contradicting the maximality of $w$.
    
    In particular, letting $S_{\max}$ be the set of all maximal elements
    of $S$,
    we conclude that 
    \begin{equation*}
        L \subseteq S \cup \bigcup_{w \in S_{\max}} (\upset[\prefleq]{w}) \cap L
        \quad .
    \end{equation*}
    Hence, that $L$ is finite union of \kl{chains}.
\end{proofof}

\begin{proofof}{prefix-wqo-reg-decidable:cor}
    If $L$ is regular, then it is $\MSO$-definable, and there 
    exists a formula $\varphi(x)$ in $\MSO$ that selects nodes 
    of the \kl{tree of prefixes} that belong to $L$. Now, to decide whether there
    exists an \kl{antichain branch} for $L$, we can simply check whether
    the following formula is satisfied:
    \begin{equation*}
        \exists B. 
        B \text{ is a branch } \land
        \forall x \in B, \exists y. y \text{ is a child of } x \land \varphi(y) \land y \not\in B
        \quad .
    \end{equation*}
    Because the above formula is an $\MSO$-formula over the infinite
    $\Sigma$-branching tree, whether it is satisfied is decidable
    as an easy consequence of the decidability of $\MSO$ over infinite binary
    trees
    \cite[Theorem 1.1]{RAB69}.
\end{proofof}

\section{Proofs for Section~\ref{infixes-bounded:sec}}

\begin{figure}
    \centering
    \includestandalone[width=\linewidth]{fig/infix-encoding-standalone}
    \caption{Representation of the \kl{subword relation} for $\set{a,b}^*$
        inside the \kl{infix relation} for $\set{a,b,\#}^*$
        using a simplified version of \cref{infix-embedding:thm}, restricted to words
        of length at most $3$. 
    }
    \label{infix-embedding:fig}
\end{figure}


\begin{proofof}{inf-period-chain:lem}
    Let $x \in \Sigma^+$ be a word, and let $P_x$ be the (finite) set 
    of all \kl{prefixes} of $x$, and $S_x$ be the (finite)
    set of all \kl{suffixes} of $x$.
    Assume that $w \in \InfPeriodChain{x}$, then $w = u x^p v$ for some
    $u \in S_x$, $v \in P_x$, and $p \in \Nat$.
    We have proven that
    \begin{equation*}
        \InfPeriodChain{x} \subseteq \bigcup_{u \in P_x} \bigcup_{v \in S_x} u x^* v
        \quad .
    \end{equation*}

    Let us now demonstrate that for all $(u,v) \in S_x \times P_x$, the
    language $u x^* v$ is a \kl{chain} for the \kl{infix}, \kl{suffix} and \kl{prefix} relations.
    To that end,
    let $(u,v) \in S_x \times P_x$ and $\ell, k \in \Nat$ be such that $\ell <
    k$, let us prove that $u x^\ell v \infleq u x^k  v$. Because $v \prefleq
    x$, we know that there exists $w$ such that $vw = x$. In particular,
    $ux^\ell vw = u x^{\ell + 1}$, and because $\ell < k$, we conclude that $u
    x^{\ell + 1} \prefleq u x^k v$. By transitivity, $u x^\ell v \prefleq u x^k
    v$, and \emph{a fortiori}, $u x^\ell v \infleq u x^k v$. 
    Similarly, because $u \suffleq x$,  there exists $w$ such that $wu  = x$, 
    and we conclude that $u x^{\ell} v \suffleq w u x^\ell v = x^{\ell + 1} v \suffleq u x^k v$.
\end{proofof}



\begin{proofof}{bounded-language:lem}
    Let $w_1, \dots, w_n$ be such that
    $L \subseteq w_1^* \cdots w_n^*$.
    Let us define $m \defined \max \setof{\card{w_i}}{1 \leq i \leq n}$

      Let $w[\vec{k}] \defined w_1^{k_1} \cdots w_n^{k_n}$ be a map from $\Nat^k$
      to $\Sigma^*$. We are interested in the intersection of the image of $w$
      with $L$. Let us assume for instance that for all $\vec{k} \in \Nat^n$,
      there exists $\vec{\ell} \geq \vec{k}$ such that $w[\vec{\ell}] \in L$.
      Then, leveraging \cref{pumping-periods:lem}, we conclude that there exists
      $x,y$ of size at most $\max\setof{\card{w_i}}{1 \leq i \leq n}$ such that
      $w[\vec{k}] \in \InfPeriodChain{x} \cup \InfPeriodChain{x}
      \InfPeriodChain{y}$, and we conclude that $L \subseteq \InfPeriodChain{x}
      \cup \InfPeriodChain{x} \InfPeriodChain{y}$.

      Now, it may be the case that one cannot simultaneously assume that two
      component of the vector $\vec{k}$ are unbounded. In general, given a set $S
      \subseteq \set{1, \dots, n}$ of indices, we say that $S$ is admissible if
      there exists a bound $N_0$ such that for all $\vec{b} \in \Nat^S$, there
      exists a vector $\vec{k} \in \Nat^n$, such that $\vec{k}$ is greater than
      $\vec{b}$ on the $S$ components, and the other components are below the
      bound $N_0$. The language of an admissible set $S$ is the set of words
      obtained by repeating $w_i$ at most $N_0$ times if it is not in $S$
      ($w_i^{\leq N_0}$) and arbitrarily many times otherwise ($w_i^*$).
      Note that $L \subseteq \bigcup_{S \text{ admissible }} L(S)$.

      Now, admissible languages are ready to be pumped according to
      \cref{pumping-periods:lem}. For every admissible language,
      the size of a word that is not iterated is at most
      $N_0 \times m$ by definition, and we conclude that:
      \begin{equation}
          \label{bounded-language:eq}
          L \subseteq 
          \bigcup_{x,y \in \Sigma^{\leq n}}
          \bigcup_{u \in \Sigma^{\leq m \times N_0}}
          \InfPeriodChain{x} u \InfPeriodChain{y}
          \cup
          \InfPeriodChain{x} u
          \cup
          u \InfPeriodChain{x}
          \quad .
      \end{equation}
  \end{proofof}

\begin{proofof}{bounded-wqo-dwclosed:cor}
    Because $L \subseteq \dwset[\infleq]{L}$, the right-to-left implication
    is trivial.
    For the left-to-right implication, let us assume that $L$ is a
    \kl{well-quasi-ordered} language for the \kl{infix relation}.
    Then $L$ is included in a finite union 
    of products of \kl{chains} for the \kl{prefix} and \kl{suffix} relations
    thanks to
    \cref{bounded-language:thm}:
    \begin{equation*}
        L \subseteq \bigcup_{i = 1}^n S_i \cdot P_i \quad .
    \end{equation*}
    Remark that if $S_i$ is a \kl{chain} for the \kl{suffix relation}
    and $P_i$ is a \kl{chain} for the \kl{prefix relation},
    then 
    \begin{equation*}
      \dwset[\infleq]{(S_i \cdot P_i)} = (\dwset[\suffleq]{S_i}) \cdot (\dwset[\prefleq]{P_i})
        \quad .
    \end{equation*}
    Indeed, any \kl{infix} of a word in $S_i P_i$ can be split into
    a \kl{suffix} of a word in $S_i$ and a \kl{prefix} of a word in $P_i$.
    Conversely, any such concatenations are \kl{infixes} of a word in $S_i P_i$.


    As a consequence, we conclude that $\dwset[\infleq]{L}$ is itself included
    in a finite union of products of \kl{chains}.
    Furthermore, by definition of \kl{bounded languages},
    $\dwset[\infleq]{L}$ is also a \kl{bounded language}.
    Hence, it is \kl{well-quasi-ordered} by the \kl{infix relation} 
    via
    \cref{bounded-language:thm}.
\end{proofof}


\section{Proofs for Section~\ref{infixes-dwclosed:sec}}


\begin{proofof}{bi-infinite:lem}
    Let us assume that $L$ is infinite. The case when it is finite 
    is similar, but will result in a finite word.
    \todo{aliaume: do we need wqo here? the proof
      should go through without it: the sequence 
      $u_i$ is already increasing for infix}

    Because the alphabet $\Sigma$ is finite, we can enumerate the words of $L$
    as $\seqof{w_i}[i \in \Nat]$. From $\seqof{w_i}[i \in \Nat]$, we construct a sequence $\seqof{u_i}[i \in \Nat]$ by induction
    as follows: $u_0 = w_0$, and $u_{i+1}$ is a word that contains $u_i$ and
    $w_i$, which exists in $L$ because $L$ is \kl(subset){directed}. Since $L$ is
    \kl{well-quasi-ordered}, one can extract an infinite set of indices $I
    \subseteq \Nat$ such that $u_i \infleq u_{j}$ for all $i \leq j \in I$.

    We can build a word $w$ as the limit of the sequence $\seqof{u_i}[i \in
    I]$. This word is infinite or bi-infinite, and contains as infixes all the
    words $u_i$ for $i \in I$. Because every word of $L$ is an infix of every
    $u_i$ for a large enough $I$, one concludes that $L$ is contained in the
    set of finite infixes of $w$. Conversely, every finite infix of $w$ is 
    an infix of some $u_i$ by definition of the limit construction, hence
    belongs to $L$ since $u_i \in L$ and $L$ is \kl{downwards closed}.
\end{proofof}

\begin{proofof}{ultimately-uniformly-recurrent:lem}

    Assume that $w$ is \kl{ultimately uniformly recurrent}. Consider a sequence
    of words $\seqof{w_i}[i \in \Nat]$ that are finite \kl{infixes} of $w$. Because $w$ is
    \kl{ultimately uniformly recurrent}, there exists a bound $N_0$ such that
    $w_{\geq N_0}$ is \kl{uniformly recurrent}. Let $i < N_0$, we claim that,
    without loss of generality, only finitely many words in the sequence
    $\seqof{w_i}[i \in \Nat]$ can be found starting at the position $i$ in $w$. Indeed, if
    it is not the case, then we have an infinite subsequence of words that are
    all comparable for the \kl{infix relation}, and therefore a \kl{good
    sequence}, because the \kl{infix relation} is \kl{well-founded}. We can
    therefore assume that all words in the sequence $\seqof{w_i}[i \in \Nat]$ are such that
    they start at a position $i \geq N_0$. But then they are all finite
    \kl{infixes} of $w_{\geq N_0}$, which is a \kl{uniformly recurrent} word,
    whose set of finite \kl{infixes} is \kl{well-quasi-ordered}
    (\cref{uniformly-recurrent:lem}).

    Conversely, assume that the set of finite infixes of $w$ is
    \kl{well-quasi-ordered}. Let us write $\mathsf{Rec}(w)$ the set of finite
    \kl{infixes} of $w$ that appear infinitely often. We can similarly define
    $\mathsf{Rec}(w_{\geq i})$ for any (infinite) suffix of $w$. The sequence
    $R_i \defined \mathsf{Rec}(w_{\geq i})$ is a descending sequence of \kl{downwards
    closed} sets of finite words, included in the set of finite infixes of $w$
    by definition. Because the latter is \kl{well-quasi-ordered}, there exists
    an $N_0 \in \Nat$, such that $\bigcap_{i \in \Nat} R_i = R_{N_0}$.
    Now, consider $v \defined w_{\geq N_0}$. By construction, 
    every finite infix of $v$ 
    appears infinitely often in $v$. 
    Given
    some finite infix $u \infleq v$, we 
    there exists a bound $N_u$
    on the distance
    between two consecutive occurrences of $u$ in $v$.
    Indeed, if it is not the case, then there exists an infinite sequence
    $\seqof{u x_i u}[i \in \Nat]$ of infixes of $v$, such that $x_i$ is a word
    of size $\geq i$
    and no shorter word $u y u$ is an infix of $u x_i u$.
    Because the finite infixes of $w$ (hence, of $v$) are \kl{well-quasi-ordered},
    one can extract an infinite set of indices $I \subseteq \Nat$
    such that $u x_i u \infleq u x_{j} u$ for all $i \leq j \in I$.
    In particular, $u x_i u \infleq u x_{j} u$ for some $j > |x_i|$, 
    which contradicts the fact that $u x_j u$ coded two consecutive
    occurrences of $u$ in $v$.

    We have shown that for every finite infix $u$ of $v$, there exists a bound
    $N_u$ such that every two occurrences of $u$ in $v$ start at distance at
    most $N_u$. In particular, there exists a bound $M_u$ such that every infix
    of $v$ of size at least $M_u$ contains $u$. We have proven that
    $v$ is \kl{uniformly recurrent}, hence that $w$ is \kl{ultimately uniformly
    recurrent}.
\end{proofof}


\begin{proofof}{bi-infinite-uur:lem}
  Given a bi-infinite word $w \in \Sigma^{\Rel}$, we can consider $w_+ \in
\Sigma^\Nat$ and $w_- \in \Sigma^\Nat$ the two infinite words obtained as
follows: for all $i \in \Nat$, $(w_+)_i = w(i)$ and $(w_-)_i = w(-i)$. Note
that the two share the letter at position $0$.

    Assume that $w_+$ and $w_-$ are \kl{ultimately uniformly recurrent}. Let us
    write $\infset{w}$ the set of finite infixes of $w$. Consider an infinite
    sequence of words $\seqof{u_i}[i \in \Nat]$ in $\infset{w}$. If there is an
    infinite subsequence of words that are all in $\infset{w_+}$, then there
    exists an increasing pair of indices $i < j$ such that $u_i \infleq
    u_j$ because \cref{uniformly-recurrent:lem} applies to $w_+$. Similarly, if
    there is an infinite subsequence of words that are all in $\infset{w_-}$,
    then there exists an increasing pair of indices $i < j$ such that $u_i
    \infleq u_j$ because \cref{uniformly-recurrent:lem} applies to $w_-$ (and
    the \kl{infix relation} is compatible with mirroring). Otherwise, one can
    assume without loss of generality that all words in the sequence have a
    starting position in $w_-$ and an ending position in $w_+$. In this case,
    let us write $(k_i,l_i) \in \Nat^2$ the pair of indices such that $u_i$ is the infix
    of $w$ that starts at position $-k_i$ of $w$ (i.e., $k_i$ of $w_-$) and
    ends at position $l_i$ of $w$ (i.e., $l_i$ of $w_+$).
    Because $\Nat^2$ is a \kl{well-quasi-ordering} with the product ordering,
    there exists $i < j$ such that $k_i \leq k_j$ and $l_i \leq l_j$, 
    in particular, $u_i \infleq u_j$. We have proven that every infinite
    sequence of words in $\infset{w}$ is \kl(wqo){good}, hence $\infset{w}$ is
    \kl{well-quasi-ordered}.

    Conversely, assume that $\infset{w}$ is \kl{well-quasi-ordered}. In
    particular, the subset $\infset{w_+} \subseteq \infset{w}$ is \kl{well-quasi-ordered}.
    Similarly, $\infset{w_-}$ is \kl{well-quasi-ordered} because the \kl{infix
    relation} is compatible with mirroring. Applying
    \cref{ultimately-uniformly-recurrent:lem}, we conclude that both are
    \kl{ultimately uniformly recurrent} words.

\end{proofof}

\begin{proofof}{from-bi-to-single:lem}
  Given a bi-infinite word $w \in \Sigma^{\Rel}$, recall that we can consider $w_+ \in
\Sigma^\Nat$ and $w_- \in \Sigma^\Nat$ the two infinite words obtained as
follows: for all $i \in \Nat$, $(w_+)_i = w(i)$ and $(w_-)_i = w(-i)$. Note
that the two share the letter at position $0$.

    To obtain the upper bound of $3 \cdot \omegaOrd$, we can consider the same
    argument as for \cref{small-ordinal-invariants:lem}. We let $N_0$ be such
    that $w_{\geq N_0}$ and $(w_-)_{\geq N_0}$ are \kl{uniformly recurrent}
    words. In any sequence of incomparable elements of $\infset{w}$, there are
    less than $N_0^2$ elements that are found in $(w_{< N_0})_{> -N_0}$. Then,
    one has to pick a finite \kl{infix} in either $w_{\geq N_0}$ or $w_{\leq
    -N_0}$. Because of \cref{small-ordinal-invariants:lem}, any sequence of
    incomparable elements of these two infinite words has length bounded based
    on the choice of the first element of that sequence. This means that the
    \kl{ordinal width} of $\infset{w}$ is at most $\omegaOrd + \omegaOrd +
    N_0^2$. We conclude that $\oWidth{\infset{w}} < 3 \cdot \omegaOrd$.

  Let us briefly argue that the bound is tight. Indeed, one can
  construct a bi-infinite word $w$ by concatenating a reversed \kl{Thue-Morse
  sequence} on a binary alphabet $\set{a,b}$, a finite antichain of arbitrarily
  large size over a distinct alphabet $\set{c,d}$, and then a \kl{Thue-Morse
  sequence} on a binary alphabet $\set{e,f}$. The \kl{ordinal width} of the set
  of \kl{infixes} of $w$ is then at least $2 \cdot \omegaOrd + K$, where $K$ is the
  size of the chosen antichain, following the same argument as in the proof of
  \cref{small-ordinal-invariants:lem}, using \cref{thue-morse-ordinal:lemma}.
\end{proofof}

\section{Proofs for Section~\ref{infixes-amalgamation:sec}}

\begin{figure}
    \centering
    \includestandalone[width=\linewidth]{fig/gap-word-embedding-standalone}
    \caption{The \kl{gap words} resulting from a \kl{subword embedding} between two 
    finite words.}
    \label{gap-word-embedding:fig}
\end{figure}

\begin{figure}
    \centering
    \includestandalone[width=\linewidth]{fig/run-amalgamation-standalone}
    \caption{We illustrate how 
        embeddings $f$ and $g$ between runs of an
        \kl{amalgamation system} can be glued
        together, seen on their canonical decomposition.
    }
    \label{amalgamation-runs:fig}
\end{figure}


\begin{proofof}{gap-words-prefix-ordered:lem}
	Write $u$ for $\GapWord{f}{\ell}$ and $v$ for $\GapWord{g}{\ell}$. 
	We may assume that both $u$ and $v$ are non-empty, as otherwise the lemma holds trivially.
	Then, for all $k \in \Nat$, there exists a run with canonical decomposition
	$$
	w_k = L_0 a_1 \cdots a_n L_n,
	$$
	where $L_i \in \set{vv u^k, vu^kv, u^k vv}$ and specifically $L_\ell = vu^kv$.
	
	From \cref{pumping-periods:lem}, we may conclude that there are a finite number of words $x, y,$ and $w$ 
	such that each $w_k$ is contained in a language 
	$\InfPeriodChain{x}w\InfPeriodChain{y}$.
	
	As there is an infinite number of words $w_k$, 
	we may fix $x, y,$ and $w$ and an infinite subset $I \subseteq \Nat$ 
	such that $\set{w_i \mid i \in I} \subseteq \InfPeriodChain{x}w\InfPeriodChain{y}$. 
	This implies that either for infinitely many $m \in \Nat$, $u^m v \in \InfPeriodChain{y}$ 
	or for infinitely many $m$, $v u^m \in \InfPeriodChain{x}$. 
	
	In either case, we may conclude that either $u \infleq v$ or $v \infleq u$: Let $m, n \in \Nat$
	such that $m < n$ and $u^m v, u^n v \in \InfPeriodChain{y}$ (the case for $v u^m$ and $v u^n$ 
	proceeding analogously). Without loss of generality, assume that $\card{u^m}$ and $\card{u^n}$ are
	multiples of $\card{y}$. We therefore find $p \prefleq y, s \suffleq y$ such that $u^m, u^n \in sy^*p$, 
	ergo $ps = y$.
	In other words, we can write $u^m = (sp)^{m'}, u^n = (sp)^{n'}$. As $u^mv \in \InfPeriodChain{y}$, it 
	follows that $v$ is a prefix of some word in $(sp)^*$. Hence either $v$ is a prefix of $u$ or $u$ vice versa.
\end{proofof}



\begin{proofof}{dwclosed-infixes-wqo:lem}
    It is clear that \cref{dwci-reg:item} $\Rightarrow$ \cref{dwci-aml:item}
    because regular languages are recognized by finite automata, and finite
    automata are a particular case of \kl{amalgamation systems}.
    The implication \cref{dwci-aml:item} $\Rightarrow$ \cref{dwci-bod:item}
    is the content of \cref{infix-amalgamation:thm}.
    The implication \cref{dwci-bod:item} $\Rightarrow$ \cref{dwci-uoc:item}
    is \cref{bounded-language:lem}.
    Finally, the implication \cref{dwci-uoc:item} $\Rightarrow$ \cref{dwci-reg:item}
    is simply because a \kl{downwards closed} language 
    that is a finite union of products of \kl{chains} is a regular language.

    Indeed, assume that
    $L$ is \kl{downwards closed} and included in a finite union of sets of the form
    $\InfPeriodChain{x} u \InfPeriodChain{y}$ where $x,y,u$ are possibly empty words.
    We can assume without loss of generality that
    for every $n$, $x^n u y^n$ is in $L$, otherwise, we have a bound on the maximal $n$ such that
    $x^n u y^n$ is in $L$, and we can increase the number of languages in the union, replacing $x$ or $y$
    with the empty word as necessary.
    Let us write $L' \defined \bigcup_{i = 1}^k x_i^* u_i y_i^*$. Then, $L'
    \subseteq L$ by construction. Furthermore, $L \subseteq \dwset{L'}$, also
    by construction. Finally, we conclude that $L = \dwset{L'}$ because $L$ is
    \kl{downwards closed}. Now, because $L'$ is a \kl{regular language}, and 
    \kl{regular languages} are closed under \kl{downwards closure}, we conclude
    that $L$ is a \kl{regular language}.
\end{proofof}

\section{Decision Prodecure and Amalgamation Systems}
\label{infixes-amalgamation:sec}

\AP
In this section, we are going to design an effective decision procedure for
\kl{well-quasi-ordering} by the \kl{infix relation}. To that end, the first
requirement is to fix a way to represent languages $L \subseteq \Sigma^*$.
Traditionally, one would use finite automata, context-free grammars, or for the
more adventurous, vector addition systems with states. However, our proof
technique will only require us to have a way to ``glue'' together runs of the
system to ``pump'' them and produce new runs: this is the usual pumping lemma
in automata theory, and Ogden's lemma for context-free grammars.
\todo{Cite the pumping lemma for VASS. Cite Ogden's lemma.}

\AP It turns out that there is a rather large family of systems for which
pumping arguments based on so-called \emph{minimal runs} exist: they are called
\emph{amalgamation systems} and were recently introduced by Anad, Schmitz,
Sch\"{u}tze, and Zetzsche \cite{ASZZ24}. Having done the heavy lifting on
\kl{bounded languages}, the rest of this section is mostly devoted to the
introduction of \kl{amalgamation systems} and collecting the necessary pumping
argument they enjoy. Using this meta-proof, we will generalize
\cref{infix-finite-automata:thm} to context-free grammars, languages recognized
by vector addition systems with state (VASS), and more.
\todo{unclear formulation}
The goal of this
section is not to introduce all of these systems, or justify their usefulness,
but to state and prove the following two theorems.

\begin{theorem}[label=infix-amalgamation:thm,restate=infix-amalgamation:thm]
    \proofref{infix-amalgamation:thm}
    Let $L \subseteq \Sigma^*$ be a language recognized by an 
    \kl{amalgamation system}.
    Then $L$ is well-quasi-ordered by the infix relation if and only if $L$ is
    \kl(language){bounded}.
\end{theorem}

\begin{theorem}[restated=infix-wqo-is-emptiness:thm,label=infix-wqo-is-emptiness:thm]
% 	\label{infix-wqo-is-emptiness:thm}
	Let $\mathcal{C}$ be a class of languages recognized by 
    \kl{amalgamation systems}
    and closed under
    \kl{rational transductions}. 
    Then the following are equivalent:
	\begin{enumerate}
        \item\label{wqo-infix-decidable} \kl[wqo]{Well-quasi-orderedness} of the \kl{infix relation} is decidable for languages in $\mathcal{C}$.
        \item\label{wqo-prefix-decidable} \kl[wqo]{Well-quasi-orderedness} of the \kl{prefix relation} is decidable for languages in $\mathcal{C}$.
        \item\label{emptiness-decidable} Emptiness is decidable for languages in $\mathcal{C}$.
	\end{enumerate}
\end{theorem}

\subsection{Amalgamation Systems}
\label{infixes-amalgamation:subsec}

Let us now formally introduce the notion of \kl{amalgamation systems}, and
recall some results from \cite{ASZZ24} that will be useful for the proof of
\cref{infix-amalgamation:thm}. The notion of \kl{amalgamation system} is
tailored to produce \emph{pumping arguments}, which is exactly what our
\cref{pumping-periods:lem} talks about. At the core of a pumping argument,
there is a notion of \emph{run}, which could for instance be a sequence of
transitions taken in a finite state automaton. Continuing on the analogy with
finite automata, there is a natural ordering between runs, i.e., a run is
smaller than another one if one can ``unroll loops" of the first run to obtain
the second one. Typical pumping arguments then rely on the fact that
\emph{minimal} runs are of finite size, and that all other runs are
obtained by ``gluing" minimal runs together. This is exactly what
\kl{amalgamation systems} are about.

\AP Let us recall that over an alphabet $(\Sigma, =)$ a \kl{subword embedding}
between two words $u \in \Sigma^*$ and $v \in \Sigma^*$ is a function $\rho
\colon \range{\card{u}} \to \range{\card{v}}$ such that $u_i = v_{\rho(i)}$ for
all $i \in \range{\card{u}}$. We write $\intro*\HigEmb(u,v)$ the set of all
\kl{subword embeddings} between $u$ and $v$. It may be useful to notice that
the set of finite words over $\Sigma$ forms a category when we consider
\kl{subword embeddings} as morphisms, which is a fancy way to state that
$\mathrm{id} \in \HigEmb(u,u)$ and that $f \circ g \in \HigEmb(u,w)$ whenever
$g \in \HigEmb(u,v)$ and $f \in \HigEmb(v,w)$, for any choice of words
$u,v,w \in \Sigma^*$.

\AP Given a \kl{subword embedding} $f \colon u \to v$ between two words $u$ and
$v$, there exists a unique decomposition $v = \GapWord{f}{0} u_1 \GapWord{f}{1}
\cdots \GapWord{f}{k-1} u_k \GapWord{f}{k}$ where $\GapWord{f}{i} =
v[f(i)+1:f(i+1)-1]$ for all $1 \leq i \leq k-1$, $\GapWord{f}{k} =
v[f(k)+1:\card{v}]$, and $\GapWord{f}{0}   = v[1: f(1)-1]$. We say that
$\intro*\GapWord{f}{i}$ is the $i$-th \intro{gap word} of $f$. We encourage the
reader to look at \cref{gap-word-embedding:fig} to see an example of the
\kl{gap words} resulting from a \kl{subword embedding} between two words. These
\kl{gap words} will be useful to describe how and where runs of a system
(described by words) can be combined.

\begin{figure}
    \centering
    \includestandalone[width=\linewidth]{fig/gap-word-embedding-standalone}
    \caption{The \kl{gap words} resulting from a \kl{subword embedding} between two 
    finite words.}
    \label{gap-word-embedding:fig}
\end{figure}

\begin{figure}
    \centering
    \includestandalone[width=\linewidth]{fig/run-amalgamation-standalone}
    \caption{We illustrate how 
        embeddings $f$ and $g$ between runs of an
        \kl{amalgamation system} can be glued
        together, seen on their canonical decomposition.
    }
    \label{amalgamation-runs:fig}
\end{figure}


\begin{definition}
    An \intro{amalgamation system}
    is a triple $(\Sigma, R, E, \canrun)$ where
    $\Sigma$ is a finite alphabet,
    $R$ is a set of so-called \emph{runs},
    and 
    $E$ is a set of so-called \emph{run embeddings},
    and $\canrun \colon R \to (\Sigma \uplus \set{\varepsilon})^*$ is a 
    function computing a \emph{canonical decomposition} of a run.
    Given a run $r \in R$, and $i \in \range[0]{\card{\canrun(r)}}$, 
    the \intro{gap language} of $r$ at position $i$ is $\GapLanguage{r}{i} \defined
    \setof{\GapWord{f}{i}}{\exists s \in R. \exists f \in E(r,s) }$.
    An \kl{amalgamation system} furthermore satisfies the following 
    properties:
    \begin{enumerate}
        \item \emph{Word Embeddings as Morphisms.} For all $r, s \in R$,
            $E(r,s) \subseteq \HigEmb(\canrun(r), \canrun(s))$.
        \item \emph{$(R, E)$ Forms a Category.}
            For all $r,s,t \in R$,
            $\mathrm{id} \in E(r,r)$,
            and whenever $f \in E(r,s)$ and $g \in E(s,t)$,
            then $g \circ f \in E(r,t)$.
        \item \emph{Well-Quasi-Ordered System.}
            $(R, \leq_E)$ is a well-quasi-ordered set,
            where $r \leq_E s$ if and only if $E(r,s) \neq \emptyset$.
        \item \emph{Controlled Amalgamation.}
            For all $r, s, t \in R$,
            for all $f \in E(r,s)$,
            for all $g \in E(r,t)$,
            and for all $0 \leq i_0 \leq \card{\canrun(r)}$,
            there exists a run $z \in R$ and morphisms
            $h \in E(s,z)$ and $k \in E(t,z)$
            satisfying
            $h \circ f = k \circ g$,
            $\GapWord{h \circ f}{i} = \GapWord{f}{i} \GapWord{g}{i}$
            or 
            $\GapWord{h \circ f}{i} = \GapWord{g}{i} \GapWord{f}{i}$
            for every $0 \leq j \leq n$,
            and
            $\GapWord{h \circ f}{i_0} = \GapWord{f}{i_0} \GapWord{g}{i_0}$.
            We refer to \cref{amalgamation-runs:fig} for an illustration 
            of this property.
    \end{enumerate}

    The \intro{amalgamation language} of such a system
    is the set of all words $w \in \Sigma^*$ such that
    there exists a run $r \in R$
    such that the concatenation of the letters of its
    canonical decomposition, written $\intro*\yieldrun(r)$,
    equals $w$.
\end{definition}

Intuitively, the definition of an amalgamation system allows the comparison of
runs, and the proper ``gluing" of runs together to obtain new runs. Let us
recall some examples of languages that can be recognized by \kl{amalgamation
systems} given by the authors of the original paper: regular languages
\cite[Theorem 5.3]{ASZZ24}, VASS as a consequence of \cite[Theorem
5.5]{ASZZ24}, context-free languages as a consequence of \cite[Theorem
5.10]{ASZZ24}. For this paper to be self-contained, we will also recall how
runs of a finite state automaton can be understood as an \kl{amalgamation
system}.

\begin{example}[{\cite[Section 3.2]{ASZZ24}}]
    Let $A = (Q, \delta, q_0, F)$ be a finite state automaton over a finite
    alphabet $\Sigma$. Let $\Delta$ be the set of transitions $(q_1, a, q_2)
    \in Q \times \Sigma \times Q$,
    and $R \subseteq \Delta^*$ be the set of 
    words over transitions that start with the initial state $q_0$,
    end in a final state $q_f \in F$, and such that the end state of a
    letter is the start state of the following one.
    The canonical decomposition $\canrun \colon R \to \Sigma^*$
    is defined as a morphism from $\Delta^*$ to $\Sigma^*$
    that maps $(q,a,p)$ to $a$.
    Finally, given two runs $r$ and $s$ of the automaton,
    we say that an embedding $f \in \HigEmb(\canrun(r), \canrun(s))$
    belongs to $E(r,s)$ when
    $f$ is also defining an embedding from $r$ to $s$ as words in $\Delta^*$,
    where $\Delta$ is equipped with the equality relation.

    The system $(\Sigma, R, E, \canrun)$ is an \kl{amalgamation system},
    whose language is precisely the language of words recognized
    by the automaton $A$.
\end{example}
\begin{proof}
    By definition, the embeddings inside $E(r,s)$ are defined as elements
    of $\HigEmb(\canrun(r), \canrun(s))$, and they compose properly.
    Because $\Delta = Q \times \Sigma \times Q$ is finite, it is 
    a \kl{well-quasi-ordering} when equipped with the equality relation, and 
    we conclude that $\Delta^*$ with $\higleq$ is a \kl{well-quasi-order}
    according to Higman’s Lemma \cite{HIG52}.
    
    Let us now move to proving that the system satisfies the amalgamation
    property. Given three runs $r,s,t \in R$, and two embeddings $f \in E(r,s)$
    and $g \in E(r,t)$, we want to construct an amalgamated run $s \vee t$.
    Because letters in the run $r$ respect the transitions of the automaton
    (i.e., if the letter $i$ ends in state $q$, then the letter $i+1$ starts in
    state $q$), then the \kl{gap word} at position $i$ starts in state $q$ and
    ends in state $q$ too. This means that for both embeddings
    $f$ and $g$, the \kl{gap words} are read by the automaton by looping
    on a state. In particular, these loops can be taken in any order
    and continue to represent a valid run. That is, we can even select
    the order of concatenation in the amalgamation for \emph{all} 
    $0 \leq i \leq \card{\canrun(r)}$ and not just for one separately.

    Let us now remark that 
    the language of this amalgamation system is
    the set of $\yieldrun(r)$ when $r$ ranges over $R$,
    and because $R$ is the set of valid runs of the automaton,
    and $\yieldrun(r)$ is the word read along this run,
    we immediately conclude.
\end{proof}

\subsection{Amalgamation Systems, WQOs, and Bounded Languages}

Let us now introduce a combinatorial lemma that explains how \kl{gap languages}
can be pumped. Note that a single \kl{gap language} is always stable under
concatenation, and our concern will be on a potential description of
\emph{simultaneously} pumping different \kl{gap languages} to produce valid
runs.

\begin{lemma}
    \label{pumping-gap-languages:lem}
    Let $(\Sigma, R, E, \canrun)$ be an \kl{amalgamation system}, let $r \in R$
    be a run of size $n \in \Nat$, and let $s, t \in R$ be two runs
    with embeddings $f \in E(r,s)$ and $g \in E(r,t)$.
    Let $\seqof{u_i}[0 \leq i \leq n]$ be the sequence of \kl{gap words}
    for $f$, that is, $u_i =
    \GapWord{f}{i}$ for all $0 \leq i \leq n$. Similarly, let
    $\seqof{v_i}[0 \leq i \leq n]$ be the sequence of \kl{gap words} for $g$.

    Then, for all $k \in \Nat$, and $0 < \ell < n$, the following words belong to 
    the language of the \kl{amalgamation system}.
    \begin{enumerate}
        \item $(\prod_{i = 0}^{\ell - 1} u_i^{x_i} v_i u_i^{y_i} v_i u_i^{z_i}) 
               v_\ell u_\ell^k v_\ell
               (\prod_{i = \ell+1}^{n} u_i^{x_i} v_i u_i^{y_i} v_i u_i^{z_i})$,
        \item $(\prod_{i = 0}^{n - 1} u_i^{x_i} v_i u_i^{y_i}) 
               v_n u_n^k$,
        \item $u_0^k v_0 
            (\prod_{i = 1}^{n} u_i^{x_i} v_i u_i^{y_i})$.
    \end{enumerate}
    Where, for all $0 \leq i \leq n$, $x_i + y_i + z_i = k$, with $z_i = 0$
    in the two last items.
\end{lemma}
\begin{proof}
    The result immediately follows from the embedding property
    applied inductively on runs obtained by gluing 
    $k$ times $s$ with one or two times $t$ (depending on the item number).
\end{proof}

To conclude this section and prove our main \cref{infix-amalgamation:thm}, it
suffices to demonstrate that languages recognized by \kl{amalgamation systems}
that are \kl{well-quasi-ordered} for the \kl{infix relation} are \kl{bounded
languages}.

\begin{proofof}{infix-amalgamation:thm}
    Assume that $L$ is \kl{well-quasi-ordered} by the \kl{infix relation},
    and obtained by an \kl{amalgamation system}
    $(\Sigma, R, E, \canrun)$.

    Let us consider the set $M$ of minimal runs for the relation $\leq_E$,
    which is finite because the latter is a \kl{well-quasi-ordering}. By
    definition, for every word $w \in L$, there exists $r \in M$, $s \in R$,
    and $f \in E(r,s)$ such that $w = \canrun(s)$.
    In particular, we conclude that
    \begin{equation*}
        L \subseteq \bigcup_{r \in M} 
        \left(
        \GapLanguage{r}{0}
        \prod_{i = 1}^{\card{\canrun(r)}-1} \canrun(r)_i \GapLanguage{r}{i} 
        \right)
        \quad .
    \end{equation*}

    Let $u$ and $v$ be two words in the \kl{gap language}
    $\GapLanguage{r}{\ell}$ for some $r \in M$ and $0 < \ell <
    \card{\canrun(r)} \defined n$. One can apply \cref{pumping-gap-languages:lem}
    to conclude that  for all $k \geq 1$,
    \begin{equation*}
       \left(\prod_{i = 0}^{\ell - 1} u_i^{x_i} v_i u_i^{y_i} v_i u_i^{z_i}\right) 
       v u^k v
       \left(\prod_{i = \ell+1}^{n} u_i^{x_i} v_i u_i^{y_i} v_i u_i^{z_i}\right)
       \in 
       L
    \end{equation*}
    Where $x_i + y_i + z_i = k$ for all $0 \leq i \leq n$
    Let us assume without loss of generality that the sequence
    $\seqof{(x_i, y_i, n_i)}[i \geq 1]$ is pointwise increasing,
    and even increasing coordinate wise. This is possible because
    at least one of the coordinates tends to $+\infty$ and in the case
    of a variable that is bounded, one can extract the sequence to
    simply not use this variable.

    Now, leveraging \cref{pumping-periods:lem}, we get that for infinitely many
    $k \geq 1$, $u v^k u$ is an infix of some $x^p$ where $x$ has size smaller
    or equal than $u$ and $v$. This proves  that $u$ and $v$ are comparable for
    the prefix, infix, and suffix relations. In particular, we have can
    conclude that $\GapLanguage{r}{i}$ is a \kl{bounded language}, because of
    \cite[Lemma 4.1]{ASZZ24}, when $0 < i < \card{\canrun(r)}$. The proof goes
    on similarly for proving that $\GapLanguage{r}{0}$ and $\GapLanguage{r}{n}$
    are \kl{bounded languages}, using the other items of
    \cref{pumping-gap-languages:lem}.

    Because \kl{bounded languages} are stable under concatenation and finite
    unions, we conclude that $L$ itself must be a \kl{bounded language}.
\end{proofof}

One can leverage the same kind of reasoning to take the opposite direction, and
notice that the notions of being a \kl{bounded language}, a \kl{regular
language}, or being included in a finite union of products of \kl{chains} all
collapse for \kl{well-quasi-ordered} and \kl{downwards closed} languages for
the \kl{infix relation}.

\begin{lemma}
    \label{dwclosed-infixes-wqo:lem}
    Let $L \subseteq \Sigma^*$ be a \kl{downwards closed} language for the
    \kl{infix relation} that is \kl{well-quasi-ordered}. Then, the following
    are equivalent:
    {\renewcommand{\theenumi}{\roman{enumi}}
     \renewcommand{\labelenumi}{(\theenumi)}
    \begin{enumerate}
        \item\label{dwci-reg:item} $L$ is a \kl{regular language},
        \item\label{dwci-aml:item} $L$ is recognized by \emph{some} \kl{amalgamation system},
        \item\label{dwci-bod:item} $L$ is a \kl{bounded language},
        \item\label{dwci-uoc:item} $L$ is included in a finite union of products of \kl{chains}
    \end{enumerate}
    }
\end{lemma}
\begin{proof}
    It is clear that \cref{dwci-reg:item} $\Rightarrow$ \cref{dwci-aml:item}
    because regular languages are recognized by finite automata, and finite
    automata are a particular case of \kl{amalgamation systems}.
    The implication \cref{dwci-aml:item} $\Rightarrow$ \cref{dwci-bod:item}
    is the content of \cref{infix-amalgamation:thm}.
    The implication \cref{dwci-bod:item} $\Rightarrow$ \cref{dwci-uoc:item}
    is \cref{bounded-language:thm}.
    Finally, the implication \cref{dwci-uoc:item} $\Rightarrow$ \cref{dwci-reg:item}
    is simply because a \kl{downwards closed} language 
    that is a finite union of products of \kl{chains} is a regular language.
\end{proof}

\subsection{Effective Decision Procedures}
\label{infixes-amalgamation-effective:subsec}

For the next result, we impose a mild restriction by requiring the language
class to be closed under \kl{rational transductions}. This property holds for
most common language classes, including all known \kl{amalgamation systems}.

\AP A class of languages $\mathcal{C}$ is \intro{recognized by amalgamation
systems} when, for every language $L \in \mathcal{C}$, there exists an
\kl{amalgamation system} $(\Sigma, R, E, \canrun)$ such that $L$ is the
language of this system.

\AP A \intro{rational transduction} is a function $f \colon \Sigma^* \to
\Gamma^*$ that can be defined by a finite automaton with output, or
equivalently by a finite set of rules of the form $u \to v$ where $u \in
\Sigma^*$ and $v \in \Gamma^*$. We write $f(L) = \setof{f(w)}{w \in L}$ for the
image of a language $L$ by a rational transduction $f$. A class of languages
$\mathcal{C}$ is \intro{closed under rational transductions} when, for every
language $L \in \mathcal{C}$, and every rational transduction $f \colon
\Sigma^* \to \Gamma^*$, the language $f(L)$ is in $\mathcal{C}$.\todo{check these definitions ...}


\begin{proofof}{infix-wqo-is-emptiness:thm}
	\cref{emptiness-decidable} $\Rightarrow$ \cref{wqo-infix-decidable}. We aim to make the inclusion test of \cref{bounded-language:eq} effective. 
	
    Let $R = \bigcup_{x,y \in \Sigma^{\leq n}} \bigcup_{u \in \Sigma^{\leq m
    \times N_0}} \InfPeriodChain{x} u \InfPeriodChain{y} \cup
    \InfPeriodChain{x}u \cup u\InfPeriodChain{x}$. For any concrete values of
    the bounds $n$, $m$, and $N_0$, this language is regular. As $\mathcal{C}$
    is \kl{closed under rational transductions}, we can therefore reduce the
    inclusion to emptiness of $L \cap \overline{R}$.\todo{explain that $L
    \mapsto L \cap R$ is a rational  transduction?} However, we need to find
    these bounds first.
	
    To determine values for $n$ and $m$, we first test if $L$ is bounded. As
    the algorithm in \cite[Section 4.2]{ASZZ24} is effective, this yields words
    $w_1, \ldots w_n$ such that $L \subseteq w_1^* \cdots w_n^*$ and therefore
    yields also the bounds in questions.
	
    To determine the value for $N_0$, we first computer maximal subsets $S
    \subseteq [n]$ such that $w_i$ for $i \in S$ are simultaneously unbounded.\todo{define this}
    We do this by applying a transduction mapping $w_i$ to some fresh letter
    $a_i$ if $i \in S$ and $\varepsilon$ otherwise, and testing for
    simultaneous unboundedness \cite[Section 4.1]{ASZZ24}. Given a candidate
    bound $n_0$ for $N_0$, we construct the languages $L_{S,n_0} =
    w_1^{\circ_1}\cdots w_n^{\circ_n}$ where $\circ_i = *$ if $i \in S$ and
    $\circ_i = \leq n_0$ otherwise. We then test if $L \subseteq \bigcup_{S
    \text{ maximal}} L_{S,n_0}$. If yes, $n_0$ is our value for $N_0$,
    otherwise, we increase it and repeat the construction. As we chose maximal
    sets $S$, this procedure will eventually terminate.

    \cref{wqo-infix-decidable} $\Rightarrow$ \cref{wqo-prefix-decidable}. We
    just consider the transduction 
	
    \cref{wqo-prefix-decidable} $\Rightarrow$
    \cref{emptiness-decidable}. We consider the transduction $\mathcal T
    = \Sigma^* \times \{a, b\}$.\todo{Define this transduction explicitly} Then
    for any language $L \in \mathcal C$, $\mathcal TL$ is well-quasi-ordered by
    infix if and only if $L$ is empty.
\end{proofof}

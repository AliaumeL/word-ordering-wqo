\section{Preliminaries}
\label{prelims:sec}

\paragraph*{Finite words.} In this paper, we use letters $\Sigma, \Gamma$ to
denote finite alphabets, $\Sigma^*$ to denote the set of finite words over
$\Sigma$, and $\varepsilon$ for the empty word in $\Sigma^*$. In order to give
some intuition on the decision problems, we will sometimes use the notion of
\intro{finite automata}, \intro{regular languages}, and Monadic Second Order
logic ($\intro*\MSO$) over finite words, and assume is familiar with them. We
refer to the textbook of \cite{THOM97} for a detailed introduction to this
topic.

\paragraph*{Well-quasi-orders.} Let us introduce some notations for
\kl{well-quasi-orders}. A sequence $\seqof{x_i}$ in a set $X$ is
\intro(sequence){good} if there exist $i < j$ such that $x_i \leq x_j$. It is
\intro(sequence){bad} otherwise. Therefore, a \kl{well-quasi-ordered} set is a
set where every infinite sequence is \kl(sequence){good}. A \intro{decreasing
sequence} is a sequence $\seqof{x_i}$ such that $x_{i+1} < x_i$ for all $i$,
and an \intro{antichain} is a set of pairwise incomparable elements. An
equivalent definition of a \kl{well-quasi-ordered} set is that it contains no
infinite \kl{decreasing sequences}, nor infinite \kl{antichains}. We refer to
\cite{SCSC12} for a detailed survey on well-quasi-orders.

Let us point out that the \kl{prefix relation} (resp. the \kl{suffix relation}
and the \kl{infix relation}) on $\Sigma^*$ are always \kl{well-founded}, i.e.,
there are no infinite \kl{decreasing sequences} for this ordering. In
particular, for a language $L \subseteq \Sigma^*$ to be
\kl{well-quasi-ordered}, it suffices to prove that it contains no infinite
\kl{antichain}. 

\paragraph*{Ordinal Invariants.} 
\label{ordinal-invariants:subsec} An ordinal is a well-founded ordered set. We
use $\alpha, \beta, \gamma$ to denote ordinals, and use $\omega$ to denote the
first infinite ordinal, i.e., the set of natural numbers with the usual
ordering. We assume superficial familiarity with ordinal arithmetic, and refer
to \cite{KUNEN80} for a detailed introduction to this domain.
Given a tree $T$ whose branches are all finite we can define
an ordinal $\alpha_T$ inductively as follows: if $T$ is a leaf then $\alpha_T =
0$, if $T$ has children $\seqof{T_i}$ then $\alpha_T = \sup \setof{\alpha_{T_i}
+ 1}{i \in \Nat}$. We say that $\alpha_T$ is the \emph{rank} of $T$. 

Let $(X, \leq)$ be a \kl{well-quasi-ordered} set. One can define three
well-founded trees from $X$: the tree of \kl{bad sequences}, the tree of
decreasing sequences, and the tree of \kl{antichains}. The rank of these
respective trees are called respectively the \intro{maximal order type} of $X$
written $\intro*{\oType{X}}$ \cite{dejongh77}, the \intro{ordinal height} of
$X$ written $\intro*{\oHeight{X}}$ \cite{schmidt81}, and the \intro{ordinal
width} of $X$ written $\intro*{\oWidth{X}}$ \cite{kriz90b}. We refer to the
survey of \cite{DZSCSC20} for a detail discussion on these concepts and their
computation on specific classes of well-quasi-ordered sets.




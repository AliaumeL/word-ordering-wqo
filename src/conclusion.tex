% LTeX: language=en-GB 
% !TeX root=../wqo-on-words.tex
\section{Conclusion}
\label{conclusion:sec}

We have described the landscapes of \kl{well-quasi-ordered} languages for the
natural orderings on finite words: \kl{prefix}, \kl{suffix}, and \kl{infix}
relations. While the prefix and suffix relation exhibit very simple behaviours,
the infix relation can encode many complex quasi-orders (and even simulate the
\kl{subword ordering}). In the case of languages that are described by simple
computational models, or languages that are ``structurally simple''
(\kl{bounded languages}, \kl{downwards closed} languages), we showed that only
very simple \kl{well-quasi-orders} can be obtained: they are essentially
isomorphic to disjoint unions of copies of finite sets, $(\Nat, \leq)$, and
$(\Nat^2, \leq)$. Finally, under effectiveness assumptions on the language
(such as being recognized by an \kl{amalgamation system}, or being the set of
infixes of an \kl{automatic sequence}), we proved the decidability of being
\kl{well-quasi-ordered} for the \kl{infix relation}. We believe that these very
encouraging results pave the way for further research on deciding which sets
are \kl{well-quasi-ordered} for other orderings. Let us now discuss
some possible research directions and remarks.

\paragraph*{Towards infinite alphabets} In this paper, we restricted our
attention to \emph{finite} alphabets, having in mind the application to
\kl{regular languages}. However, the conclusions of
\cref{bounded-language:thm}, \cref{small-ordinal-invariants:thm}, and
\cref{prefixes:thm} could be conjectured to hold in the case of infinite
alphabets (themselves equipped with a \kl{well-quasi-ordering}). This would
require new techniques, as the finiteness of the alphabet is crucial to all of
our positive results.

\paragraph*{Monoid equations}  It could be interesting to understand which
monoids $M$ recognize languages that are \kl{well-quasi-ordered} by the
\kl{infix}, \kl{prefix} or \kl{suffix} relations. This research direction is
connected to finding which classes of graphs of \emph{bounded clique-width} are
\kl{well-quasi-ordered} with respect to the \emph{induced subgraph relation},
as shown in \cite{DRT10}, and recently revisited in \cite{lopez25}. 

\paragraph*{Lexicographic orderings} There is another natural ordering on
words, the \emph{lexicographic ordering}, which does not fit well in our
current framework because it is always of \kl{ordinal width} $1$. However, the
order-type of the lexicographic ordering over \kl{regular languages} has
already been investigated in the context of infinite words \cite{CACOPU18}, and
it would be interesting to see if one can extend these results to decide
whether such an ordering is \kl{well-founded} for languages recognized by
\kl{amalgamation systems}.

\paragraph*{Factor Complexity} \AP Let us conclude this section with a few
remarks on the notion of \kl{factor complexity} of languages. Recall that the
\intro{factor complexity} of a language $L \subseteq \Sigma^*$ is the function
$f_L: \Nat \to \Nat$ such that $f_L(n)$ is the number of distinct words of size
$n$ in $L$. We extend the notion of \kl{factor complexity} to finite, infinite,
and bi-infinite words as the \kl{factor complexity} of their set of finite
\kl{infixes}. For the \kl{prefix relation} and the \kl{suffix relation}, all
\kl{well-quasi-ordered} languages have a bounded \kl{factor complexity}, since
they are finite unions of \kl{chains}.

While there clearly are languages with low \kl{factor complexity} that are not
\kl{well-quasi-ordered} for the \kl{infix relation}, such as the language $L
\defined \dwset{ a b^* a }$; one would expect that languages that are
\kl{well-quasi-ordered} for the \kl{infix relation} would have a low \kl{factor
complexity}.

In some sense, our results confirm this intuition in the case of languages
described by a simple computational model. For languages recognized by
\kl{amalgamation systems}, being \kl{well-quasi-ordered} implies being a
\kl{bounded language}, and therefore being  included in some finite union of
languages of the form $w_1^* w_2 w_3^*$. Hence, these languages have at most a
quadratic \kl{factor complexity}. This is also the case for languages described
as the infixes of a finite set of pairs of \kl{morphic sequences}. Indeed, the
factor complexity of a \kl{morphic sequence} that is \kl{uniformly recurrent}
is linear \cite[Theorem 24]{NIPR09}, therefore the \kl{factor complexity} of a
language given by \kl{sequence representation} using \kl{morphic sequences} is
at most quadratic.

However, there are \kl{downwards closed} languages that are
\kl{well-quasi-ordered} for the \kl{infix relation} but have an \kl{exponential
factor complexity}: the $(5, 3)$-Toeplitz word is \kl{uniformly recurrent}
\cite[p. 499]{CAKA97}, but has an exponential factor complexity \cite[Theorem
5]{CAKA97}. This shows that our computational models somehow fail to capture
vast classes of \kl{well-quasi-ordered} languages with a high \kl{factor
complexity}. It would be interesting to understand which new proof techniques
would be required to obtain decidability for these languages.

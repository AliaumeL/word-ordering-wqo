% LTeX: language=en-GB 
\section{Conclusion}
\label{conclusion:sec}

We provided concretes statements that justify why the \kl{subword relation}
is used when defining \kl{well-quasi-orders} on finite words. Even if
\kl{prefix}, \kl{suffix} or \kl{infix} relations are meaningful, they are
\kl{well-quasi-ordered} if and only if they behave similarly to disjoint copies
of $\Nat$ or $\Nat^2$. However, our approach suffers some limitations 
and opens the road to natural continuation of this line of work.

\subparagraph{Towards infinite alphabets} In this paper, we restricted our
attention to \emph{finite} alphabets, having in mind the application to regular
languages. However, the conclusions of \cref{infix-finite-automata:thm} and
\cref{prefixes:thm} could be conjecture to hold in the case of infinite
alphabets (themselves equipped with a \kl{well-quasi-ordering}). This would
require new techniques, as the finiteness of the alphabet is crucial to all of
our positive results.

\subparagraph{Monoid equations}  It could be interesting to understand which
monoids $M$ recognize languages that are \kl{well-quasi-ordered} by the
\kl{infix}, \kl{prefix} or \kl{suffix} relations. This is connected to finding
which classes of graphs of \emph{bounded clique-width} are
\kl{well-quasi-ordered} with respect to the \emph{induced subgraph relation}, a
subject that is recently gaining traction.


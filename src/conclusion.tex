% LTeX: language=en-GB 
% !TeX root=../wqo-on-words.tex
\section{Conclusion}
\label{conclusion:sec}

We provided concretes statements that justify why the \kl{subword relation}
is used when defining \kl{well-quasi-orders} on finite words. Even if
\kl{prefix}, \kl{suffix} or \kl{infix} relations are meaningful, they are
\kl{well-quasi-ordered} if and only if they behave similarly to disjoint copies
of $\Nat$ or $\Nat^2$. However, our approach suffers some limitations 
and opens the road to natural continuation of this line of work.

\paragraph*{Towards infinite alphabets} In this paper, we restricted our
attention to \emph{finite} alphabets, having in mind the application to
\kl{regular languages}. However, the conclusions of
\cref{bounded-language:thm}, \cref{small-ordinal-invariants:thm}, and
\cref{prefixes:thm} could be conjecture to hold in the case of infinite
alphabets (themselves equipped with a \kl{well-quasi-ordering}). This would
require new techniques, as the finiteness of the alphabet is crucial to all of
our positive results.

\paragraph*{Monoid equations}  It could be interesting to understand which
monoids $M$ recognize languages that are \kl{well-quasi-ordered} by the
\kl{infix}, \kl{prefix} or \kl{suffix} relations. This research direction is
connected to finding which classes of graphs of \emph{bounded clique-width} are
\kl{well-quasi-ordered} with respect to the \emph{induced subgraph relation},
as shown in \cite{DRT10}, and recently revisited by one of the authors in
\cite{L24:arxiv:v2}.

\paragraph*{Complexity} We have chosen to disregard complexity considerations
when proving decidability results, as we do not believe that a fined grained
complexity analysis would be particularly enlightening. However, we conjecture
that for regular languages, the complexity should be polynomial in the size of
the minimal automaton recognizing the language.


\paragraph*{Lexicographic orderings} There is another natural ordering on
words, the \emph{lexicographic ordering}, which does not fit well in our
current framework because it is always of \kl{ordinal width} $1$. However, the
order-type of the lexicographic ordering over \kl{regular languages} has
already been investigated in the context of infinite words \cite{CACOPU18}, and
it would be interesting to see if one can extend these results to decide
whether such an ordering is \kl{well-founded} for languages recognized by
\kl{amalgamation systems}.


\subsubsection{Factor Complexity}

\AP Let us conclude this section with a few remarks on the notion of \kl{factor
complexity} of languages. Recall that the \intro{factor complexity} of a
language $L \subseteq \Sigma^*$ is the function $f_L: \Nat \to \Nat$ such that
$f_L(n)$ is the number of distinct words of size $n$ in $L$. A language $L$ has
\intro{exponential factor complexity} if there exists a constant $C > 1$ such
that $f_L(n) \geq C^n$ for infinitely many $n \in \Nat$. A language $L$ has
\intro{subaffine factor complexity} if $f_L(n) \leq C \cdot n + K$ for some
constants $C, K > 0$ and all $n \in \Nat$.
We extend the notion of \kl{factor complexity} to infinite and bi-infinite words
as the \kl{factor complexity} of their set of finite \kl{infixes}.

While there clearly are languages with \kl{subaffine factor complexity} that are
not \kl{well-quasi-ordered} for the \kl{infix relation}, such as 
the language $L \defined \dwset{ a b^* a }$;
one would expect that languages that are \kl{well-quasi-ordered} for the
\kl{infix relation} would have a low \kl{factor complexity}.

\AP It is the case that \kl{bounded languages}, studied in
\cref{infixes-bounded:sec}, have a polynomial factor complexity: such a
language is by definition included in $w_1^* \cdots w_n^*$ for some words $w_1,
\dots, w_n$, and an infix is a choice of the number of occurrences of each
$w_i$ in the infix, plus a choice of beginning and ending of the infix in the
leftmost and rightmost $w_i$. We showed in \cref{bounded-language:thm} and
particularly in \cref{bounded-language:lem}, that \kl{bounded languages} that
are \kl{well-quasi-ordered} for the \kl{infix relation} are in fact included in
some finite union of languages of the form $w_1^* w_2 w_3^*$, hence have
actually a quadratic factor complexity. However,
\cref{exponential-factor-complexity:lem} shows that there are
\kl{downwards closed} languages that are \kl{well-quasi-ordered} for the
\kl{infix relation} but have an \kl{exponential factor complexity}.


\begin{lemma}
  \label{exponential-factor-complexity:lem}
  There exists a language $L$ with \kl{exponential factor complexity}
  that is \kl{well-quasi-ordered} for the \kl{infix relation}.
\end{lemma}
\begin{proof}
  We leverage results from \cite{CAKA97} on Toeplitz words.
  Namely, the 
  $(5, 3)$-Toeplitz word is \kl{uniformly recurrent} 
  \cite[p. 499]{CAKA97},
  but has an exponential factor complexity 
  \cite[Theorem 5]{CAKA97}.
\end{proof}


Our understanding of the situation is that whenever a computational model is
fixed to represent languages that are \kl{well-quasi-ordered} for the \kl{infix
relation}, the \kl{factor complexity} of the represented languages drops to a
quadratic one. This is the case for languages recognized by \kl{amalgamation
systems} \cref{amalgamation-systems:subsec}, that are \kl{bounded languages}
when they are \kl{well-quasi-ordered} (\cref{infix-amalgamation:thm}). This is
also the case for languages described as the infixes of a finite set of pairs
of \kl{morphic sequences}. Indeed, the factor complexity of a \kl{morphic
sequence} that is \kl{uniformly recurrent} is linear \cite[Theorem 24]{NIPR09},
therefore the \kl{factor complexity} of a language given by \kl{sequence
representation} using \kl{morphic sequences} is at most quadratic.


\todo{nat: might be worth noting that factor complexity is O(1)}
\todo{aliaume: I would put it in the section "factor complexity" in the concluding
sections}

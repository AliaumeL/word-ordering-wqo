% LTeX: language=en-GB 
% !TeX root=../wqo-on-words.tex
\section{Conclusion}
\label{conclusion:sec}

We provided concretes statements that justify why the \kl{subword relation}
is used when defining \kl{well-quasi-orders} on finite words. Even if
\kl{prefix}, \kl{suffix} or \kl{infix} relations are meaningful, they are
\kl{well-quasi-ordered} if and only if they behave similarly to disjoint copies
of $\Nat$ or $\Nat^2$. However, our approach suffers some limitations 
and opens the road to natural continuation of this line of work.

\paragraph*{Towards infinite alphabets} In this paper, we restricted our
attention to \emph{finite} alphabets, having in mind the application to
\kl{regular languages}. However, the conclusions of
\cref{bounded-language:thm}, \cref{small-ordinal-invariants:thm}, and
\cref{prefixes:thm} could be conjecture to hold in the case of infinite
alphabets (themselves equipped with a \kl{well-quasi-ordering}). This would
require new techniques, as the finiteness of the alphabet is crucial to all of
our positive results.

\paragraph*{Monoid equations}  It could be interesting to understand which
monoids $M$ recognize languages that are \kl{well-quasi-ordered} by the
\kl{infix}, \kl{prefix} or \kl{suffix} relations. This research direction is
connected to finding which classes of graphs of \emph{bounded clique-width} are
\kl{well-quasi-ordered} with respect to the \emph{induced subgraph relation},
as shown in \cite{DRT10}, and recently revisited by one of the authors in
\cite{L24:arxiv:v2}.

\paragraph*{Complexity} We have chosen to disregard complexity considerations
when proving decidability results, as we do not believe that a fined grained
complexity analysis would be particularly enlightening. However, we conjecture
that for regular languages, the complexity should be polynomial in the size of
the minimal automaton recognizing the language.


\paragraph*{Lexicographic orderings} There is another natural ordering on
words, the \emph{lexicographic ordering}, which does not fit well in our
current framework because it is always of \kl{ordinal width} $1$. However, the
order-type of the lexicographic ordering over \kl{regular languages} has
already been investigated in the context of infinite words \cite{CACOPU18}, and
it would be interesting to see if one can extend these results to decide
whether such an ordering is \kl{well-founded} for languages recognized by
\kl{amalgamation systems}.

\paragraph*{Graph Classes}
\todo[inline]{aliaume: retalk about \cite{ALM17} and \cite{L24:arxiv:v2}}


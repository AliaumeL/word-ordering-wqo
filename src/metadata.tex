% Description: Metadata for the plain template
%
% Warning: this file is generated automatically
% from the paper-meta.yaml file. If you really want to edit
% this file, please mirror the changes to paper-meta.yaml.

\title{Well-quasi-orderings on word languages}
\author{%
        Roland Guttenberg
    \thanks{Technical University of Munich, Germany}
     \and
        Nathan Lhote
    \thanks{Aix-Marseille University}
     \and
        Aliaume Lopez
    \thanks{University of Warsaw}
     \and
        Corto Mascle
    \thanks{LaBRI, University of Bordeaux}
     \and
        Vincent Michielini
    \thanks{University of Warsaw}
     \and
        Lia Schütze
    \thanks{MPI-SWS, Germany}
     \and
        Omid Yaghoubi
    \thanks{University of Warsaw}
    }


\newcommand{\makeabstract}{
\begin{abstract}
    The set of finite words over a well-quasi-ordered set is itself
    well-quasi-ordered. This seminal result by Higman is a cornerstone
    of the theory of well-quasi-orderings and has found numerous
    applications in computer science. However, this result is based on a
    specific choice of ordering on words, the (scattered) subword
    ordering. In this paper, we describe to what extent other natural
    orderings (prefix, suffix, infix) on words can be used to derive
    Higman-like theorems. More specifically, we are interested in
    characterizing \emph{languages} of words that are well-quasi-ordered
    under these orderings. We show that a simple characterization is
    possible for the prefix and suffix orderings, and that under extra
    regularity assumptions, this also lifts to the infix ordering.
\end{abstract}
}

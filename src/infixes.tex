% lang: en-US

\section{Infixes}
\label{infixes:sec}


In this section, we study the well-quasi-ordering of languages under the
\kl{infix relation}. As opposed to the \kl{prefix} and \kl{suffix} relations,
the \kl{infix relation} can yield to very complicated \kl{well-quasi-ordered}
languages. Formally, the following theorem shows that \emph{any} countable
quasi-ordering with finite initial segments can be embedded into the infix
relation of a language.

\begin{theorem}
    \label{thm:infix-embedding}
    Let $(X, \preceq)$ be a quasi-ordered set.
    Then the following are equivalent
    \begin{itemize}
        \item $X$ embeds into $(\Sigma^*, \preceq_{\text{infix}})$
        \item $X$ is countable, and for every $x \in X$,
            $\{y \in X \mid y \preceq x\}$ is finite.
    \end{itemize}

    Furthermore, if $X$ is recursively enumerable and $\preceq$ decidable,
    then
    the embedding is effective.
\end{theorem}

As a consequence of \cref{thm:infix-embedding}, we cannot replay proofs of
\cref{prefixes-suffixes:sec}, and will actually need to leverage some
regularity of the languages to obtain a characterization of well-quasi-ordered
languages under the infix relation. Let us first play this game for languages
that are recognized by finite automata.

\begin{theorem}
    \label{thm:infix-finite-automata}
    Let $L \subseteq \Sigma^*$ be a language recognized by a finite automaton.
    Then $L$ is well-quasi-ordered by the infix relation if and only if $L$ is
    a finite union of chains for the \kl{infix relation}.
\end{theorem}

\begin{corollary}
    Decidability?
\end{corollary}

\subsection{Amalgamation systems}

This section will use so-called \emph{amalgamation systems}.

\begin{itemize}
    \item Define amalgamation systems
    \item Pumping argument
    \item Characterization
    \item Decision procedure
\end{itemize}

\subsection{Monoid equations?}

\begin{itemize}
    \item What are the finite monoids $(M, \cdot)$ such that
    every language recognized by $M$ is well-quasi-ordered by the infix relation?
    \item What are the equations on $M$ and $P \subseteq M$ characterizing 
    the well-quasi-ordering of $L(M, P)$?
\end{itemize}


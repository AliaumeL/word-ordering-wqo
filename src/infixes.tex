% LTeX: language=en-GB 
\section{Infixes and Regular Languages}
\label{infixes-regular:sec}


In this section, we study the well-quasi-ordering of languages under the
\kl{infix relation}. As opposed to the \kl{prefix} and \kl{suffix} relations,
the \kl{infix relation} can yield to very complicated \kl{well-quasi-ordered}
languages. Formally, the following theorem shows that \emph{any} countable
quasi-ordering with finite initial segments can be embedded into the infix
relation of a language.

Let us recall that an \intro{order embedding} from a quasi-ordered set $(X,
\preceq)$ into a quasi-ordered set $(Y, \preceq')$ is a function $f \colon X
\to Y$ such that for all $x, y \in X$, $x \preceq y$ if and only if $f(x)
\preceq' f(y)$. When such an embedding exists, we say that $X$ \reintro{embeds
into} $Y$.

We say that a quasi-ordered set $(X, \preceq)$ is a \intro{partial ordering}
whenever the relation $\preceq$ is antisymmetric, that is $x \preceq y$ and $y
\preceq x$ implies $x = y$. 

\begin{theorem}
    \label{infix-embedding:thm}
    Let $(X, \preceq)$ be a \kl{partially ordered} set,
    and $\Sigma \defined \set{a,b,c}$.
    Then the following are equivalent:
    \begin{enumerate}
        \item \label{infix-embedding-embeds:item} 
            $X$ \kl{embeds into} $(\Sigma^*, \infleq)$,
        \item \label{infix-embedding-count:item}
            $X$ is countable, and for every $x \in X$,
        $\setof{y \in X}{y \preceq x}$ is finite.
    \end{enumerate}
    Furthermore, if $X$ is recursively enumerable and $\preceq$ decidable,
    then
    the \kl{embedding} is computable.
\end{theorem}
\begin{proof}
    Let us first prove that \cref{infix-embedding-embeds:item} implies
    \cref{infix-embedding-count:item}. Let $f \colon X \to \Sigma^*$ be an
    \kl{embedding}. Note that $f$ is injective, because $X$ and $\Sigma^*$
    are \kl{partially
    ordered}. In particular, because $\dwset[\infleq]{w}$ is finite
    for every $w \in \Sigma^*$, then so is $\setof{y \in X}{y \preceq x}$
    for every $x \in X$. Similarly, $X$ must be countable
    because $\Sigma^*$ is.

    Conversely, let us prove that \cref{infix-embedding-count:item} implies
    \cref{infix-embedding-embeds:item}. To that end, let us first define
    inductively a sequence $\seqof{X_n}$ of subsets of $X$ as follows: $X_0$ is
    the set of minimal elements of $X$, and $X_{n+1}$ is the set of minimal
    elements of $X \setminus \bigcup_{i \leq n} X_i$. Because initial segments
    of $X$ are finite, it is clear that $X = \bigcup_{n \in \Nat} X_n$. To
    simplify notations, we will also use $Y_n \defined \bigcup_{i \leq n} X_i$.
    We also let $L_0 \defined a b^* a$, and 
    $L_{n+1} \defined L_n \cup c^{n+1} a^{n+1} b^* a^{n+1} c^{n+1} (L_n c^{n+1})^*$.
    Now, let us inductively define \kl{embeddings} $f_n \colon Y_n \to
    L_n$.

    For the base case, let $\seqof{x_i}[i \in \Nat]$ be an enumeration of
    $X_0$, which is possible because $X$ is countable. We define $f_0 \colon
    X_0 \to \Sigma^*$ as follows: $f_0(x_i) \defined a b^i a$. It is immediate
    that $f_0$ is an \kl{embedding}. For the inductive step, let $\seqof{x_i}[i
    \in \Nat]$ be an enumeration of $X_{n+1}$, and assume that $f_n$ is an
    \kl{embedding} from $Y_n$ to $\Sigma^*$.
    Let us define $D_i \defined \setof{ y \in Y_n }{ y \preceq x_i }$
    for $i \in \Nat$, all of which are finite by assumption.
    We then let for all $i \in \Nat$:
    \begin{equation*}
        f_{n+1}(x_i) \defined 
        \underbrace{c^{n+1} a^{n+1} b^i a^{n+1} c^{n+1} \prod_{y \in D_i} \overbrace{f_n(y)}^{\in L_n} c^{n+1}}_{\in L_{n+1}}
        \quad .
    \end{equation*}
    And let $f_{n+1}(y) \defined f_n(y)$ for all $y \in Y_n$.

    To check that $f_{n+1}$ is an \kl{embedding}, let us first notice that
    words $f_{n+1}(x_i)$ and $f_{n+1}(x_j)$ are incomparable for the \kl{infix
    relation} when $i \neq j$, because the only factor of the form $c^{n+1}
    a^{n+1} b^t a^{n+1} c^{n+1}$ appears at the beginning of these words, and
    precisely encode $i$ (resp. $j$) in the value of $t$. Now, let us consider
    $y \in Y_n$ and $x_i \in X_{n+1}$ such that $y \preceq x_i$. Because $y \in
    D_i$, we have $f_n(y) \infleq f_{n+1}(x_i)$ by construction, and since
    $f_n(y) = f_{n+1}(y)$, we conclude that $f_{n+1}(y) \infleq f_{n+1}(x_i)$. 
    Conversely, assume that
    $f_{n+1}(y) \infleq f_{n+1}(x_i)$. Again, this means
    that $f_n(y) \infleq f_{n+1}(x_i)$. Now, the encoding
    is robust enough so that we can conclude
    $f_{n}(y) \infleq f_{n}(z)$ for some $z \in D_i$.
    By induction hypothesis, $y \preceq z$, hence, $y \preceq x_i$.

    We conclude by noticing that $\bigcup_{n \in \Nat} f_n$ is an \kl{embedding}
    of $X$ into $\Sigma^*$.
\end{proof}

As a consequence of \cref{infix-embedding:thm}, we cannot replay proofs of
\cref{prefixes:sec}, and will actually need to leverage some
regularity of the languages to obtain a characterization of well-quasi-ordered
languages under the infix relation. Let us first play this game for languages
that are recognized by finite automata.

\textbf{TODO: introduce finite automata and refer to the book.}.


\begin{theorem}
    \label{infix-finite-automata:thm}
    Let $L \subseteq \Sigma^*$ be a language recognized by a finite automaton.
    Then $L$ is well-quasi-ordered by the infix relation if and only if $L$ is
    a finite union of chains for the \kl{infix relation}.
\end{theorem}

In order to prove \cref{infix-finite-automata:thm}, we will perform
some preliminary analysis on the structure of an automaton recognizing a
well-quasi-ordered language under the infix relation.

\begin{lemma}
    Let $A = (Q, \Sigma, \delta, q_0, F)$ be a finite automaton recognizing a
    language $L \subseteq \Sigma^*$. Assume furthermore that
    $L$ is \kl{well-quasi-ordered} by the \kl{infix relation}.
    Then, for all words $u,w,v \in \Sigma^*$, and states
    $q, q_f \in Q \times F$ such that
    $q_0 \xrightarrow{u} q \xrightarrow{w} q \xrightarrow{v} q_f$,
    one of the following holds:
    \begin{itemize}
        \item There exists $k \in \Nat$
            such that $u$ is a \kl{suffix} of $w^k$,
        \item There exists $k \in \Nat$
            such that $v$ is a \kl{prefix} of $w^k$.
    \end{itemize}
\end{lemma}

\begin{lemma}
    Let $A = (Q, \Sigma, \delta, q_0, F)$ be a finite automaton recognizing a
    language $L \subseteq \Sigma^*$. Assume furthermore that
    $L$ is \kl{well-quasi-ordered} by the \kl{infix relation}.
    Then, for all words $w_1, w_2 \in \Sigma^*$, and states
    $q, q_f \in Q \times F$ such that
    $q_0 \rightarrow^* q \xrightarrow{w} q \rightarrow^* q_f$
    and 
    $q_0 \rightarrow^* q \xrightarrow{w_2} q \rightarrow^* q_f$,
    there exists a word $w \in \Sigma^*$
    such that $w_1$ and $w_2$ are powers of $w$.
\end{lemma}

\begin{lemma}
    Let $A = (Q, \Sigma, \delta, q_0, F)$ be a finite automaton recognizing a
    language $L \subseteq \Sigma^*$. Assume furthermore that
    $L$ is \kl{well-quasi-ordered} by the \kl{infix relation}.
    Then, for all words $u, w_1, w_2 \in \Sigma^*$, and states
    $q_1, q_2, q_f \in Q^2 \times F$,
    such that
    $q_0 \rightarrow^* q_1 \xrightarrow{w_1} q_1 \xrightarrow{u} q_2 \xrightarrow{w_2} q_2
    \rightarrow^* q_f$,
    there exists a $w \in \Sigma^*$ such that $w_1$, $w_2$, and $u$ are powers of $w$.
\end{lemma}

We are now ready to prove the main result.

\begin{proofof}{infix-finite-automata:thm}
    Let $w \in L$. Consider a run of the automaton on $w$. If $w$ is too long,
    then there exists a loop, and we can prove that $w = u x^p v$ for some
    short period $x$. By repeating the process for $u$ and $v$, we conclude
    that $w$ has the form $w = u_0 x_0^{p_0} u_1 x_1^{p_1} \cdots u_k x_k^{p_k}
    v_k$. Now, applying the loop merging lemma, we conclude that $x_0 = x_1 =
    \cdots = x_k$. Hence, $w = u_0 x^{p} v_k$. Finally, we know that either
    $u_0$ is a prefix of $x^{p}$ or that $v_k$ is a suffix of $x^{p}$. We
    conclude in particular that $\dwset[\infleq]{ \setof{ u x^p v }{ p \in \Nat
    }}$ is a finite union of \kl{chains}. Because $u,v,x$ are of bounded
    length, we conclude that $L$ itself included in a finite union of
    \kl{chains}, hence is a finite union of \kl{chains}.
\end{proofof}

\begin{corollary}
    Decidability?
\end{corollary}

\section{Infixes and Amalgamation systems}
\label{infixes-amalgamation:sec}

This section will use so-called \emph{amalgamation systems}.

\begin{theorem}
    \label{infix-amalgamation:thm}
    Let $L \subseteq \Sigma^*$ be a language recognized by an 
    \kl{amalgamation system}.
    Then $L$ is well-quasi-ordered by the infix relation if and only if $L$ is
    a finite union of chains for the \kl{infix relation}.
\end{theorem}

\begin{itemize}
    \item Define amalgamation systems
    \item Pumping argument
    \item Characterization
    \item Decision procedure
\end{itemize}

% LTeX: language=en-GB 
\section{Infixes and Regular Languages}
\label{infixes-regular:sec}


In this section, we study the well-quasi-ordering of languages under the
\kl{infix relation}. As opposed to the \kl{prefix} and \kl{suffix} relations,
the \kl{infix relation} can yield to very complicated \kl{well-quasi-ordered}
languages. Formally, the upcoming \cref{infix-embedding:thm} shows that \emph{any} countable
quasi-ordering with finite initial segments can be embedded into the infix
relation of a language.

\AP Let us recall that an \intro{order embedding} from a quasi-ordered set $(X,
\preceq)$ into a quasi-ordered set $(Y, \preceq')$ is a function $f \colon X
\to Y$ such that for all $x, y \in X$, $x \preceq y$ if and only if $f(x)
\preceq' f(y)$. When such an embedding exists, we say that $X$ \reintro{embeds
into} $Y$. We say that a quasi-ordered set $(X, \preceq)$ is a \intro{partial
ordering} whenever the relation $\preceq$ is antisymmetric, that is $x \preceq
y$ and $y \preceq x$ implies $x = y$. 

\begin{theorem}
    \label{infix-embedding:thm}
    Let $(X, \preceq)$ be a \kl{partially ordered} set,
    and $\Sigma \defined \set{a,b,c}$.
    Then the following are equivalent:
    \begin{enumerate}
        \item \label{infix-embedding-embeds:item} 
            $X$ \kl{embeds into} $(\Sigma^*, \infleq)$,
        \item \label{infix-embedding-count:item}
            $X$ is countable, and for every $x \in X$,
            its \kl{downwards closure}
            $\dwset[\preceq]{x}$ is finite.
    \end{enumerate}
\end{theorem}
\begin{proof}
    Let us first prove that \cref{infix-embedding-embeds:item} implies
    \cref{infix-embedding-count:item}. Let $f \colon X \to \Sigma^*$ be an
    \kl{embedding}. Note that $f$ is injective, because $X$ and $\Sigma^*$
    are \kl{partially
    ordered}. In particular, because $\dwset[\infleq]{w}$ is finite
    for every $w \in \Sigma^*$, then so is $\dwset[\preceq]{x}$
    for every $x \in X$. Similarly, $X$ must be countable
    because $\Sigma^*$ is.

    Conversely, let us prove that \cref{infix-embedding-count:item} implies
    \cref{infix-embedding-embeds:item}. To that end, let us first define
    inductively a sequence $\seqof{X_n}$ of subsets of $X$ as follows: $X_0$ is
    the set of minimal elements of $X$, and $X_{n+1}$ is the set of minimal
    elements of $X \setminus \bigcup_{i \leq n} X_i$. Because initial segments
    of $X$ are finite, it is clear that $X = \bigcup_{n \in \Nat} X_n$. To
    simplify notations, we will also use $Y_n \defined \bigcup_{i \leq n} X_i$.
    We also let $L_0 \defined a b^* a$, and 
    $L_{n+1} \defined L_n \cup c^{n+1} b^* c^{n+1} (L_n c^{n+1})^*$.
    Now, let us inductively define \kl{embeddings} $f_n \colon Y_n \to
    L_n$.

    For the base case, let $\seqof{x_i}[i \in \Nat]$ be an enumeration of
    $X_0$, which is possible because $X$ is countable. We define $f_0 \colon
    X_0 \to \Sigma^*$ as follows: $f_0(x_i) \defined a b^i a$. It is immediate
    that $f_0$ is an \kl{embedding}. For the inductive step, let $\seqof{x_i}[i
    \in \Nat]$ be an enumeration of $X_{n+1}$, and assume that $f_n$ is an
    \kl{embedding} from $Y_n$ to $\Sigma^*$.
    Let us define $D_i \defined \setof{ y \in Y_n }{ y \preceq x_i }$
    for $i \in \Nat$, all of which are finite by assumption.
    We then let for all $i \in \Nat$:
    \begin{equation*}
        f_{n+1}(x_i) \defined 
        \underbrace{c^{n+1} b^i c^{n+1} \prod_{y \in D_i} \overbrace{f_n(y)}^{\in L_n} c^{n+1}}_{\in L_{n+1}}
        \quad .
    \end{equation*}
    And let $f_{n+1}(y) \defined f_n(y)$ for all $y \in Y_n$.

    To check that $f_{n+1}$ is an \kl{embedding}, let us first notice that
    words $f_{n+1}(x_i)$ and $f_{n+1}(x_j)$ are incomparable for the \kl{infix
    relation} when $i \neq j$, because the only factor of the form $c^{n+1}
    a^{n+1} b^t a^{n+1} c^{n+1}$ appears at the beginning of these words, and
    precisely encode $i$ (resp. $j$) in the value of $t$. Now, let us consider
    $y \in Y_n$ and $x_i \in X_{n+1}$ such that $y \preceq x_i$. Because $y \in
    D_i$, we have $f_n(y) \infleq f_{n+1}(x_i)$ by construction, and since
    $f_n(y) = f_{n+1}(y)$, we conclude that $f_{n+1}(y) \infleq f_{n+1}(x_i)$.
    Conversely, assume that $f_{n+1}(y) \infleq f_{n+1}(x_i)$. Again, this
    means that $f_n(y) \infleq f_{n+1}(x_i)$. Now, the encoding is robust
    enough so that we can conclude $f_{n}(y) \infleq f_{n}(z)$ for some $z \in
    D_i$ By induction hypothesis, $y \preceq z$, hence, $y \preceq x_i$.

    We conclude by noticing that $\bigcup_{n \in \Nat} f_n$ is an \kl{embedding}
    of $X$ into $\Sigma^*$.
\end{proof}

As a consequence of \cref{infix-embedding:thm}, we cannot replay proofs of
\cref{prefixes:sec}, and will actually need to leverage some regularity of the
languages to obtain a characterization of \kl{well-quasi-ordered} languages
under the \kl{infix relation}. Let us first play this game for languages that
are recognized by finite automata. We assume that the reader is familiar with
the notions of deterministic finite automaton and regular languages, and refer
to the book of Thomas for a comprehensive introduction to the subject
\cite{THOM97}. The main goal of the remainder of this section is to prove the
following \cref{infix-finite-automata:thm}.

\begin{theorem}[restate=infix-finite-automata:thm,label=infix-finite-automata:thm]
    Let $L \subseteq \Sigma^*$ be a language recognized by a finite automaton.
    Then $L$ is well-quasi-ordered by the infix relation if and only if $L$ is
    a finite union of chains for the \kl{infix relation}.
\end{theorem}

\AP In order to prove \cref{infix-finite-automata:thm}, we will perform some
preliminary analysis on the structure of an automaton recognizing a
well-quasi-ordered language under the infix relation, which will be powered by
folklore results on \emph{periodic} words. Let us recall that a non-empty word
$w \in \Sigma^+$ is \intro(word){periodic} with period $x \in \Sigma^*$ if
there exists a $p \in \Nat$ such that $w \infleq x^p$. The \intro{periodic
length} of a word $u$ is the minimal length of a word $x$ such that $u$ is an
\kl{infix} of $x^p$ for some $p \in \Nat$ and $x \in \Sigma^+$. We will
essentially rely on the following result on periodic words.

\begin{lemma}
    \label{periodic-infixes:lem}
    Let $u,v \in \Sigma^*$ be two (non-empty) \kl{periodic words}
    having \kl{periodic lengths} $p$ and $q$ respectively.
    Then, if $u \infleq v$ and $\card{u} \geq \factorial[p]{p \times q}$,
    then $u$ and $v$ share the same \kl{periodic length}
    $p = q$.
\end{lemma}
\begin{proof}
    The fact that $u$ and $v$ are \kl{periodic length}
    respectively $p$ and $q$ translates into the fact that $u_{i+p} = u_i$ and
    $v_{i+q} = v_i$ for all indices $i \in \Nat$ such that those letters are
    well-defined.

    Now, assume that $u$ is an \kl{infix} of $v$, this provides the existence
    of a $k \in \Nat$ such that $u = v_{k} \cdots v_{k + \card{u} - 1}$. In
    particular, $v_{k+i+p} = v_{k+i}$ for all $i \in \Nat$ such that $k+i+p < k
    + \card{u}$. Since we also have $v_{k+i+q} = v_{k+i}$ for all $1 \leq i
    \leq \card{v} - k - q$. We conclude that both $u$ and $v$ are of
    \kl{periodic length} the greatest common divisor of $p$ and $q$, and by
    minimality of $q$ this must be equal to $q$ and to $p$.
\end{proof}

The reason why \kl{periodic words} built using a given period $x \in \Sigma^+$
are interesting for the \kl{infix relation} is that they naturally create
\kl{chains}. Indeed, if $x \in \Sigma^+$ is a finite word, then $\setof{x^p}{p
\in \Nat}$ is a \kl{chain} for the \kl{infix relation}. Note that in general,
the \kl{downwards closure} of a chain is \emph{not} a chain. However, for the chains
generated using periodic words, the \kl{downwards closure}
$\dwset[\infleq]{\setof{x^p}{p \in \Nat}}$ is a \emph{finite union} of
\kl{chains}. Because this set will appear in bigger equations,
we introduce the shorter notation 
$\intro*\InfPeriodChain{x}$ for the set of 
\kl{infixes} of words of the form $x^p$, where $p$ ranges over $\Nat$.

\begin{lemma}
    \label{inf-period-chain:lem}
    Let $x \in \Sigma^+$ be a word, and
    Then $\InfPeriodChain{x}$ is a finite union of \kl{chains}
    for the \kl{infix}, \kl{prefix} and \kl{suffix} relations 
    simultaneously.
\end{lemma}
\begin{proof}
    Let $x \in \Sigma^+$ be a word, and let $P_x$ be the (finite) set 
    of all \kl{prefixes} of $x$, and $S_x$ be the (finite)
    set of all \kl{suffixes} of $x$.
    Assume that $w \in \InfPeriodChain{x}$, then $w = u x^p v$ for some
    $u \in S_x$, $v \in P_x$, and $p \in \Nat$.
    We have proven that
    \begin{equation*}
        \InfPeriodChain{x} \subseteq \bigcup_{u \in P_x} \bigcup_{v \in S_x} u x^* v
        \quad .
    \end{equation*}

    Let us now demonstrate that for all $(u,v) \in S_x \times P_x$, the
    language $u x^* v$ is a \kl{chain} for the \kl{infix}, \kl{suffix} and \kl{prefix} relations.
    To that end,
    let $(u,v) \in S_x \times P_x$ and $\ell, k \in \Nat$ be such that $\ell <
    k$, let us prove that $u x^\ell v \infleq u x^k  v$. Because $v \prefleq
    x$, we know that there exists $w$ such that $vw = x$. In particular,
    $ux^\ell vw = u x^{\ell + 1}$, and because $\ell < k$, we conclude that $u
    x^{\ell + 1} \prefleq u x^k v$. By transitivity, $u x^\ell v \prefleq u x^k
    v$, and \emph{a fortiori}, $u x^\ell v \infleq u x^k v$. 
    Similarly, because $u \suffleq x$,  there exists $w$ such that $wu  = x$, 
    and we conclude that $u x^{\ell} v \suffleq w u x^\ell v = x^{\ell + 1} v \suffleq u x^k v$.
    \qedhere
\end{proof}


\begin{corollary}
    \label{inf-period-union-chains:lem}
    Let $x,u,y \in \Sigma^*$ be words.  The following 
    are finite unions of \kl{chains} for the infix relation:
    $\InfPeriodChain{x}u$, $u \InfPeriodChain{x}$,
    and $\InfPeriodChain{x} u \InfPeriodChain{y}$.
\end{corollary}
\begin{proof}
    Because $\InfPeriodChain{x}$ is a finite union of \kl{chains} for the \kl{suffix}
    relation (using \cref{inf-period-chain:lem}), we conclude that $\InfPeriodChain{x}u$ is one too for
    the \kl{infix} relation. This holds similarly for $u \InfPeriodChain{x}$.
    In the case of $\InfPeriodChain{x} u \InfPeriodChain{y}$,
    we remark that $\InfPeriodChain{x} u$ is a finite union of chains
    for the \kl{suffix relation},
    and that $\InfPeriodChain{y}$ is one for the \kl{prefix relation}
    (using again \cref{inf-period-chain:lem}).
    As a consequence,
    $\InfPeriodChain{x}u \InfPeriodChain{y}$ is a finite union 
    of \kl{chains} for the \kl{infix} relation.
\end{proof}

The following combinatorial lemma connects the property of being
\kl{well-quasi-ordered} to a property of the \kl{periodic lengths} of words in
a language, based on the assumption that some factors can be iterated.

\begin{lemma}
    \label{pumping-periods:lem}
    Let $L \subseteq \Sigma^*$ be a language
    that is \kl{well-quasi-ordered} by the \kl{infix relation}.
    Let $k \in \Nat$, $u_1, \cdots, u_{k+1} \in \Sigma^*$,
    and $v_1, \cdots, v_{k} \in \Sigma^+$
    be such that
    $w[\vec{n}] \defined (\prod_{i = 1}^k u_i v_i^{n_i}) u_{k+1}$
    belongs to $L$
    for arbitrarily large values of $\vec{n} \in \Nat^k$.
    Then, 
    there exists $x,y \in \Sigma^+$ of size 
    at most $\max \setof{\card{v_i}}{1 \leq i \leq k}$
    such that 
    one of the following holds for all
    $\vec{n} \in \Nat^{k}$:
    \begin{enumerate}
        \item $w[\vec{n}] \in u_1 \InfPeriodChain{x}$,
        \item $w[\vec{n}] \in \InfPeriodChain{x} u_{k+1}$,
        \item $w[\vec{n}] \in \InfPeriodChain{x} u_i \InfPeriodChain{y}$
            for some $1 \leq i \leq k + 1$.

    \end{enumerate}
\end{lemma}
\begin{proof}
    Note that the result is obvious if $k = 0$, and therefore
    we assume $k \geq 1$ in the following proof.

    Let us construct a sequence of words $\seqof{w_i}[i \in \Nat]$, where $w_i
    \defined w[\vec{n_i}]$ for some well-chosen indices $\vec{n_i} \in \Nat^k$. The goal
    being that 
    if $w[\vec{n_i}]$ is an \kl{infix} of $w[\vec{n_j}]$,
    then it can intersect at most \emph{two} iterated words,
    with an intersection that is long enough to successfully apply
    \cref{periodic-infixes:lem}.
    In order to achieve this,
    let us first define $s$ as the maximal size of a word $v_i$
    ($1 \leq i \leq k$) and $u_j$ ($1 \leq j \leq k+1$).
    Then,
    we consider $\vec{n_0} \in \Nat^k$ such that $\vec{n_0}$ has all 
    its components greater than $\factorial{s}$ and such that
    $w[\vec{n_0}]$ belongs to $L$. 
    Then, we inductively define 
    $\vec{n_{i+1}}$  as the smallest vector of numbers greater than $\vec{n_i}$,
    such that $w[\vec{n_{i+1}}]$ belongs to $L$, 
    and with $\vec{n_i}$ having all components greater than
    $2\card{w[\vec{n_i}]}$.


    Let us assume that $k \geq 2$ in the following proof for symmetry purposes,
    and argue later on that when $k = 1$ the same argument goes through.
    Because $L$ is \kl{well-quasi-ordered} by the \kl{infix relation}, there
    exists $i < j$ such that $w[\vec{n_i}]$ is an \kl{infix} of $w[\vec{n_j}]$.
    Now, because of the values of $\vec{n_j}$, there exists $1 \leq \ell \leq
    k-1$ such that $w[\vec{n_i}]$ is actually an \kl{infix} of $u_{\ell}
    v_{\ell}^{n_{j,\ell}} u_{\ell+1} v_{\ell+1}^{n_{j,\ell+1}} u_{\ell+2}$.
    Even more,
    one of the three following equations holds:
    \begin{itemize}
        \item $w[\vec{n_i}] \infleq v_{\ell}^{n_{j,\ell}} u_{\ell+1} v_{\ell+1}^{n_{j,\ell+1}}$,
        \item $w[\vec{n_i}] \infleq u_{\ell}
            v_{\ell}^{n_{j,\ell}}$,
        \item $w[\vec{n_i}] \infleq
            v_{\ell+1}^{n_{j,\ell+1}} u_{\ell+2}$.
    \end{itemize}
    In all those cases, we conclude using \cref{periodic-infixes:lem}.
    
    When $k = 1$, the situation is a bit more specific since we only have two
    cases: either $w_i \infleq u_1 v_1^{n_j}$ or $w_i \infleq v_1^{n_j} u_2$,
    and we conclude with an identical reasoning.
\end{proof}


We are now ready to restate and proof our main theorem.

\begin{proofof}{infix-finite-automata:thm}[main]
    Let $w \in L$, because $Q$ is finite, there exists
    a factorization of $w$
    into words $(\prod_{i = 1}^k u_i v_i) u_{k+1}$
    such that 
    for all $1 \leq i \leq k+1$, $\card{u_i} \leq \card{Q}$,
    for all $1 \leq i \leq k$, $1 \leq \card{v_i} \leq \card{Q}$,
    and satisfying 
    that $w[\vec{X}] \defined 
    (\prod_{i = 1}^k u_i v_i^{X_i}) u_{k+1}$
    belongs to $L$ for all choices of values $\vec{X} \in \Nat^k$.

    Applying \cref{pumping-periods:lem}, we conclude that 
    there exists $x,y \in \Sigma^+$ of size at most $\card{Q}$
    such that 
    $w \in u_1 \InfPeriodChain{x} \cup \InfPeriodChain{y} u_{k+1}
    \cup \bigcup_{1 \leq i \leq k+1} \InfPeriodChain{x} u_i \InfPeriodChain{y}$.
    In particular, we conclude that
    \begin{equation}
        \label{infix-automata:eq}
        L
        \subseteq
        \bigcup_{x,y,u\in \Sigma^{\leq \card{Q}}}
        u \InfPeriodChain{x}
        \cup 
        \InfPeriodChain{x} u
        \cup
        \InfPeriodChain{x} u \InfPeriodChain{y}
        \quad .
    \end{equation}

    Now, thanks to \cref{inf-period-union-chains:lem},
    this happens to be a finite union of \kl{chains}
    for the \kl{infix relation}.
\end{proofof}



\begin{corollary}
    Given a regular language $L$, it is decidable whether
    $L$ is \kl{well-quasi-ordered} for the \kl{infix relation}.
\end{corollary}
\begin{proof}
    It suffices to decide whether the equation 
    \cref{infix-automata:eq} holds for the language $L$.
    This is an inclusion of regular languages, which is decidable.
\end{proof}

\section{Infixes and Amalgamation Systems}
\label{infixes-amalgamation:sec}

When proving \cref{infix-finite-automata:thm}, we have leveraged a powerful
combinatorial argument from \cref{pumping-periods:lem}. It turns out that there
is a rather large family of systems for which pumping arguments based on
so-called \emph{minimal runs} exist: they are called \emph{amalgamation
systems} and were recently introduced by Anad, Schmitz, Sch\"{u}tze, and
Zetzsche \cite{ASZZ24}. Having done the heavy lifting on finite automata, the
rest of this section is mostly devoted to the introduction of \kl{amalgamation
systems} and collecting the necessary pumping argument they enjoy. Using this
meta-proof, we will generalize \cref{infix-finite-automata:thm} to context-free
grammars, languages recognized by vector addition systems with state (VASS),
and more. The goal of this section is not to introduce all of these systems, or
justify their usefulness, but to state and prove the following theorem.

\begin{theorem}[label=infix-amalgamation:thm,restate=infix-amalgamation:thm]
    Let $L \subseteq \Sigma^*$ be a language recognized by an 
    \kl{amalgamation system}.
    Then $L$ is well-quasi-ordered by the infix relation if and only if $L$ is
    a finite union of chains for the \kl{infix relation}.
\end{theorem}

\AP Let us now formally introduce the notion of \kl{amalgamation systems}, and
recall some results from \cite{ASZZ24} that will be useful for the proof of
\cref{infix-amalgamation:thm}. Let us recall that over an alphabet $(\Sigma,
=)$ a \kl{subword embedding} between two words $u \in \Sigma^*$ and $v \in
\Sigma^*$ is a function $\rho \colon \range{\card{u}} \to \range{\card{v}}$
such that $u_i = v_{\rho(i)}$ for all $i \in \range{\card{u}}$. We write
$\intro*\HigEmb(u,v)$ the set of all \kl{subword embeddings} between $u$ and
$v$. It may be useful to notice that the set of finite words over $\Sigma$
forms a category when we consider \kl{subword embeddings} as morphisms.

\AP Given a \kl{subword embedding} $f \colon u \to v$ between two words $u$ and
$v$, there exists a unique decomposition $v = \GapWord{f}{0} u_1 \GapWord{f}{1}
\cdots \GapWord{f}{k-1} u_k \GapWord{f}{k}$ where $\GapWord{f}{i} =
v[f(i)+1:f(i+1)-1]$ for all $1 \leq i \leq k-1$, $\GapWord{f}{k} =
v[f(k)+1:\card{v}]$, and $\GapWord{f}{0}   = v[1: f(1)-1]$.  We say that
$\intro*\GapWord{f}{i}$ is the $i$-th \intro{gap word} of $f$. Given a run $r
\in R$, and $i \in \range[0]{\card{\canrun(r)}}$, we define the \intro{gap
language} of $r$ at position $i$ $\GapLanguage{r}{i} \defined
\setof{\GapWord{f}{i}}{\exists s \in R. \exists f \in E(r,s) }$.


\begin{definition}
    An \intro{amalgamation system}
    is a triple $(\Sigma, R, E, \canrun)$ where
    $\Sigma$ is a finite alphabet,
    $R$ is a set of so-called \emph{runs},
    and 
    $E$ is a set of so-called \emph{run embeddings}.
    \begin{itemize}
        \item For all $r, s \in R$,
            $E(r,s) \subseteq \HigEmb(\canrun(r), \canrun(s))$.
        \item $(R, E)$ forms a category.
        \item $(R, \leq_E)$ is a well-quasi-ordered set.
        \item concatenative amalagamation
    \end{itemize}

    The \intro{amalgamation language} of such a system
    is the set of all words $w \in \Sigma^*$ such that
    there exists a run $r \in R$
    such that the concatenation of the letters (and possibly the empty word) in
    $\canrun(r)$ equals $w$.
\end{definition}

The notion of \kl{amalgamation system} is tailored to produce \emph{pumping
arguments}, which is exactly what our \cref{pumping-periods:lem} talks about.
Because the notion is quite abstract, it may be useful to replicate some
intuitions developed by the authors of \cite{ASZZ24}.


Let us recall some examples of languages that can be recognized by
\kl{amalgamation systems} given by the authors of the original paper: regular
languages \cite[Theorem 5.3]{ASZZ24}, VASS as a consequence of \cite[Theorem
5.5]{ASZZ24}, context-free languages as a consequence of \cite[Theorem
5.10]{ASZZ24}.

\begin{proofof}{infix-amalgamation:thm}
    Assume that $L$ is \kl{well-quasi-ordered} by the \kl{infix relation},
    and obtained by an \kl{amalgamation system}
    $(\Sigma, R, E, \canrun)$.

    Let us consider the set $M$ of minimal runs for the relation $\leq_E$,
    which is finite because the latter is a \kl{well-quasi-ordering}. By
    definition, for every word $w \in L$, there exists $r \in M$, $s \in R$,
    and $f \in E(r,s)$ such that $w = \canrun(s)$.
    In particular, we conclude that
    \begin{equation*}
        L \subseteq \bigcup_{r \in M} 
        \left(
        \GapLanguage{r}{0}
        \prod_{i = 1}^{\card{\canrun(r)}-1} \canrun(r)_i \GapLanguage{r}{i} 
        \right)
        \quad .
    \end{equation*}

    Let us first argue that languages $\GapLanguage{r}{i}$ are all \kl{chains}
    for the \kl{prefix ordering}. According to \cite[Lemma 4.1]{ASZZ24}, either
    $\GapLanguage{r}{i}$ is a \kl{chain} for the \kl{prefix relation}, or it is
    an \kl{unbounded language}. Note that \kl{unbounded languages}
    are \kl{antichains} for the \kl{infix relation}, and therefore
    \textbf{unclear}.

    Then, we can leverage \cref{pumping-periods:lem}
    to conclude that $L$ is a finite union of \kl{chains}
    for the \kl{infix relation}.
\end{proofof}

\begin{conjecture}
    Let $L$ be given by an \kl{effective amalgamation system},
    then it is decidable whether $L$ 
    is \kl{well-quasi-ordered} by the \kl{infix relation}.
\end{conjecture}

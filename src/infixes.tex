% LTeX: language=en-GB 
\section{Infixes and Bounded Languages}
\label{infixes-regular:sec}

\AP In this section, we study languages equipped with the \kl{infix relation}.
As opposed to the \kl{prefix} and \kl{suffix} relations, the \kl{infix
relation} can lead to very complicated \kl{well-quasi-ordered} languages.
Formally, the upcoming \cref{infix-embedding:thm} shows that
\emph{any} countable partial-ordering with finite initial segments can be
embedded into the infix relation of a language. To make the former statement
precise, let us recall that an \intro{order embedding} from a quasi-ordered set
$(X, \preceq)$ into a quasi-ordered set $(Y, \preceq')$ is a function $f \colon
X \to Y$ such that for all $x, y \in X$, $x \preceq y$ if and only if $f(x)
\preceq' f(y)$. When such an embedding exists, we say that $X$ \reintro{embeds
into} $Y$. We say that a quasi-ordered set $(X, \preceq)$ is a \intro{partial
ordering} whenever the relation $\preceq$ is antisymmetric, that is $x \preceq
y$ and $y \preceq x$ implies $x = y$. 
The embedding defined in \cref{infix-embedding:thm} is illustrated
for the \kl{subword relation} in \cref{infix-embedding:fig}.
\begin{lemma}{\cite[Lemma 5.1]{DBLP:journals/ita/Kuske06}}
    \label{infix-embedding:thm}
    Let $(X, \preceq)$ be a \kl{partially ordered} set,
    and $\Sigma$ be an alphabet with at least two letters.
    Then the following are equivalent:
    \begin{enumerate}
        \item 
            $X$ \kl{embeds into} $(\Sigma^*, \infleq)$,
        \item 
            $X$ is countable, and for every $x \in X$,
            its \kl{downwards closure}
            $\dwset[\preceq]{x}$ is finite.
    \end{enumerate}
\end{lemma}
\begin{figure}
    \centering
    \includestandalone[width=\linewidth]{fig/infix-encoding-standalone}
    \caption{Representation of the \kl{subword relation} for $\set{a,b}^*$
        inside the \kl{infix relation} for $\set{a,b,\#}^*$
        using the encoding of \cref{infix-embedding:thm}, restricted to words
        of length at most $3$. To obtain smaller words,
        we replaced $D_i$ by $\max D_i$ in this construction.
    }
    \label{infix-embedding:fig}
\end{figure}

\AP As a consequence of \cref{infix-embedding:thm}, we cannot replay
proofs of \cref{prefixes:sec}, and will
actually need to leverage some regularity of the languages to obtain a
characterization of \kl{well-quasi-ordered} languages under the \kl{infix
relation}. This regularity will be imposed through the notion of \intro{bounded
languages}, i.e., languages $L \subseteq \Sigma^*$ such that there exists words
$w_1, \dots, w_n$ satisfying $L \subseteq w_1^* \cdots w_n^*$.

\begin{theorem}[restate=bounded-language:thm,label=bounded-language:thm]
    %\label{bounded-language:thm}
    Let $L$ be a \kl{bounded language} of $\Sigma^*$. Then,
    $L$ is a \kl{well-quasi-order} when endowed with the 
    \kl{infix relation} if and only if it included in a finite union of 
    products $S_i \cdot P_i$ where 
    $S_i$ is a \kl{chain} for the \kl{suffix relation}, and 
    $P_i$ is a \kl{chain} for the \kl{prefix relation},
    for all $1 \leq i \leq n$.
\end{theorem}

Let us first remark that if $S$ is a \kl{chain} for the \kl{suffix relation}
and $P$ is a \kl{chain} for the \kl{prefix relation}, then $SP$ is
\kl{well-quasi-ordered} for the \kl{infix relation}. This proves the (easy)
right-to-left implication of \cref{bounded-language:thm}. Furthermore, any
language $L$ included in a finite union of products $S_i \cdot P_i$ where $S_i$
is in fact a \kl{bounded language}.

\AP In order to prove the (difficult) left-to-right implication of
\cref{bounded-language:thm}, we will rely heavily on the
combinatorics of periodic words. Let us recall that a non-empty word $w \in
\Sigma^+$ is \intro(word){periodic} with period $x \in \Sigma^*$ if there
exists a $p \in \Nat$ such that $w \infleq x^p$. The \intro{periodic length} of
a word $u$ is the minimal length of a word $x$ such that $u$ is an \kl{infix}
of $x^p$ for some $p \in \Nat$ and $x \in \Sigma^+$.

\begin{lemma}
    \label{periodic-infixes:lem}
    Let $u,v \in \Sigma^*$ be two (non-empty) \kl{periodic words}
    having \kl{periodic lengths} $p$ and $q$ respectively.
    Then, if $u \infleq v$ and $\card{u} \geq \factorial[p]{p \times q}$,
    then $u$ and $v$ share the same \kl{periodic length}
    $p = q$.
\end{lemma}
\begin{proof}
    The fact that $u$ and $v$ are \kl{periodic length}
    respectively $p$ and $q$ translates into the fact that $u_{i+p} = u_i$ and
    $v_{i+q} = v_i$ for all indices $i \in \Nat$ such that those letters are
    well-defined.

    Now, assume that $u$ is an \kl{infix} of $v$, this provides the existence
    of a $k \in \Nat$ such that $u = v_{k} \cdots v_{k + \card{u} - 1}$. In
    particular, $v_{k+i+p} = v_{k+i}$ for all $i \in \Nat$ such that $k+i+p < k
    + \card{u}$. Since we also have $v_{k+i+q} = v_{k+i}$ for all $1 \leq i
    \leq \card{v} - k - q$. We conclude that both $u$ and $v$ are of
    \kl{periodic length} the greatest common divisor of $p$ and $q$, and by
    minimality of $q$ this must be equal to $q$ and to $p$.
\end{proof}

\begin{corollary}
    \label{powers-infixes:cor}
    Let $u,v \in \Sigma^*$ and $k, \ell \in \Nat$
    such that $k \geq \factorial[p]{\card{u} \times \card{v}}$,
    $\ell \geq \factorial[p]{\card{v} \times \card{u}}$,
    and $u^k \infleq v^\ell$.
    Then, there exists $w \in \Sigma^*$ of size at most
    $\min \set{\card{u}, \card{v}}$ and a $p \in \Nat$
    such that
    $u^k \infleq v^\ell \infleq w^p$.
\end{corollary}

The reason why \kl{periodic words} built using a given period $x \in \Sigma^+$
are interesting for the \kl{infix relation} is that they naturally create
\kl{chains} for the \kl{prefix} and \kl{suffix} relations. Indeed, if $x \in
\Sigma^+$ is a finite word, then $\setof{x^p}{p \in \Nat}$ is a \kl{chain} for
the \kl{infix relation}. Note that in general, the \kl{downwards closure} of a
chain is \emph{not} a chain. However, for the chains generated using periodic
words, the \kl{downwards closure} $\dwset[\infleq]{\setof{x^p}{p \in \Nat}}$ is
a \emph{finite union} of \kl{chains}. Because this set will appear in bigger
equations, we introduce the shorter notation $\intro*\InfPeriodChain{x}$ for
the set of \kl{infixes} of words of the form $x^p$, where $p$ ranges over
$\Nat$.

\begin{lemma}
    \label{inf-period-chain:lem}
    Let $x \in \Sigma^+$ be a word, and
    Then $\InfPeriodChain{x}$ is a finite union of \kl{chains}
    for the \kl{infix}, \kl{prefix} and \kl{suffix} relations 
    simultaneously.
\end{lemma}
\begin{proof}
    Let $x \in \Sigma^+$ be a word, and let $P_x$ be the (finite) set 
    of all \kl{prefixes} of $x$, and $S_x$ be the (finite)
    set of all \kl{suffixes} of $x$.
    Assume that $w \in \InfPeriodChain{x}$, then $w = u x^p v$ for some
    $u \in S_x$, $v \in P_x$, and $p \in \Nat$.
    We have proven that
    \begin{equation*}
        \InfPeriodChain{x} \subseteq \bigcup_{u \in P_x} \bigcup_{v \in S_x} u x^* v
        \quad .
    \end{equation*}

    Let us now demonstrate that for all $(u,v) \in S_x \times P_x$, the
    language $u x^* v$ is a \kl{chain} for the \kl{infix}, \kl{suffix} and \kl{prefix} relations.
    To that end,
    let $(u,v) \in S_x \times P_x$ and $\ell, k \in \Nat$ be such that $\ell <
    k$, let us prove that $u x^\ell v \infleq u x^k  v$. Because $v \prefleq
    x$, we know that there exists $w$ such that $vw = x$. In particular,
    $ux^\ell vw = u x^{\ell + 1}$, and because $\ell < k$, we conclude that $u
    x^{\ell + 1} \prefleq u x^k v$. By transitivity, $u x^\ell v \prefleq u x^k
    v$, and \emph{a fortiori}, $u x^\ell v \infleq u x^k v$. 
    Similarly, because $u \suffleq x$,  there exists $w$ such that $wu  = x$, 
    and we conclude that $u x^{\ell} v \suffleq w u x^\ell v = x^{\ell + 1} v \suffleq u x^k v$.
    \qedhere
\end{proof}



The following combinatorial lemma connects the property of being
\kl{well-quasi-ordered} to a property of the \kl{periodic lengths} of words in
a language, based on the assumption that some factors can be iterated. It is
the core result that powers the analysis done in the upcoming
\cref{infix-finite-automata:thm,infix-amalgamation:thm}.

\begin{lemma}
    \label{pumping-periods:lem}
    Let $L \subseteq \Sigma^*$ be a language
    that is \kl{well-quasi-ordered} by the \kl{infix relation}.
    Let $k \in \Nat$, $u_1, \cdots, u_{k+1} \in \Sigma^*$,
    and $v_1, \cdots, v_{k} \in \Sigma^+$
    be such that
    $w[\vec{n}] \defined (\prod_{i = 1}^k u_i v_i^{n_i}) u_{k+1}$
    belongs to $L$
    for arbitrarily large values of $\vec{n} \in \Nat^k$.
    Then, 
    there exists $x,y \in \Sigma^+$ of size 
    at most $\max \setof{\card{v_i}}{1 \leq i \leq k}$
    such that 
    one of the following holds for all
    $\vec{n} \in \Nat^{k}$:
    \begin{enumerate}
        \item $w[\vec{n}] \in u_1 \InfPeriodChain{x}$,
        \item $w[\vec{n}] \in \InfPeriodChain{x} u_{k+1}$,
        \item $w[\vec{n}] \in \InfPeriodChain{x} u_i \InfPeriodChain{y}$
            for some $1 \leq i \leq k + 1$.

    \end{enumerate}
\end{lemma}
\begin{proof}
    Note that the result is obvious if $k = 0$, and therefore
    we assume $k \geq 1$ in the following proof.

    Let us construct a sequence of words $\seqof{w_i}[i \in \Nat]$, where $w_i
    \defined w[\vec{n_i}]$ for some well-chosen indices $\vec{n_i} \in \Nat^k$. The goal
    being that 
    if $w[\vec{n_i}]$ is an \kl{infix} of $w[\vec{n_j}]$,
    then it can intersect at most \emph{two} iterated words,
    with an intersection that is long enough to successfully apply
    \cref{periodic-infixes:lem}.
    In order to achieve this,
    let us first define $s$ as the maximal size of a word $v_i$
    ($1 \leq i \leq k$) and $u_j$ ($1 \leq j \leq k+1$).
    Then,
    we consider $\vec{n_0} \in \Nat^k$ such that $\vec{n_0}$ has all 
    its components greater than $\factorial{s}$ and such that
    $w[\vec{n_0}]$ belongs to $L$. 
    Then, we inductively define 
    $\vec{n_{i+1}}$  as the smallest vector of numbers greater than $\vec{n_i}$,
    such that $w[\vec{n_{i+1}}]$ belongs to $L$, 
    and with $\vec{n_i}$ having all components greater than
    $2\card{w[\vec{n_i}]}$.


    Let us assume that $k \geq 2$ in the following proof for symmetry purposes,
    and argue later on that when $k = 1$ the same argument goes through.
    Because $L$ is \kl{well-quasi-ordered} by the \kl{infix relation}, there
    exists $i < j$ such that $w[\vec{n_i}]$ is an \kl{infix} of $w[\vec{n_j}]$.
    Now, because of the chosen values for $\vec{n_j}$, there exists $1 \leq \ell \leq
    k-1$ such that $w[\vec{n_i}]$ is actually an \kl{infix} of $u_{\ell}
    v_{\ell}^{n_{j,\ell}} u_{\ell+1} v_{\ell+1}^{n_{j,\ell+1}} u_{\ell+2}$.
    Even more,
    one of the three following equations holds:
    \begin{itemize}
        \item $w[\vec{n_i}] \infleq v_{\ell}^{n_{j,\ell}} u_{\ell+1} v_{\ell+1}^{n_{j,\ell+1}}$,
        \item $w[\vec{n_i}] \infleq u_{\ell}
            v_{\ell}^{n_{j,\ell}}$,
        \item $w[\vec{n_i}] \infleq
            v_{\ell+1}^{n_{j,\ell+1}} u_{\ell+2}$.
    \end{itemize}
    In all those cases, we conclude using \cref{powers-infixes:cor}
    that there exists $x,y \in \Sigma^+$ of size at most $s$, and 
    a number $1 \leq t \leq k$ such that
    $v_i^{n_i} \in \InfPeriodChain{x}$ for all $1 \leq i \leq t$,
    and
    $v_i^{n_i} \in \InfPeriodChain{y}$ for all $t < i \leq k$.
    In particular,
    $w[\vec{n_i}] \in \InfPeriodChain{x} u_{t} \InfPeriodChain{y}$.

    
    When $k = 1$, the situation is a bit more specific since we only have two
    cases: either $w_i \infleq u_1 v_1^{n_j}$ or $w_i \infleq v_1^{n_j} u_2$,
    and we conclude with an identical reasoning.
\end{proof}

\begin{proofof}{bounded-language:thm}
    Let $w_1, \dots, w_n$ be such that
    $L \subseteq w_1^* \cdots w_n^*$.
    Let us define $m \defined \max \setof{\card{w_i}}{1 \leq i \leq n}$

    Let $w[\vec{k}] \defined w_1^{k_1} \cdots w_n^{k_n}$ be a map from $\Nat^k$
    to $\Sigma^*$. We are interested in the intersection of the image of $w$
    with $L$. Let us assume for instance that for all $\vec{k} \in \Nat^n$,
    there exists $\vec{\ell} \geq \vec{k}$ such that $w[\vec{\ell}] \in L$.
    Then, leveraging \cref{pumping-periods:lem}, we conclude that there exists
    $x,y$ of size at most $\max\setof{\card{w_i}}{1 \leq i \leq n}$ such that
    $w[\vec{k}] \in \InfPeriodChain{x} \cup \InfPeriodChain{x}
    \InfPeriodChain{y}$, and we conclude that $L \subseteq \InfPeriodChain{x}
    \cup \InfPeriodChain{x} \InfPeriodChain{y}$.

    Now, it may be the case that one cannot simultaneously assume that two
    component of the vector $\vec{k}$ are unbounded. In general, given a set $S
    \subseteq \set{1, \dots, n}$ of indices, we say that $S$ is admissible if
    there exists a bound $N_0$ such that for all $\vec{b} \in \Nat^S$, there
    exists a vector $\vec{k} \in \Nat^n$, such that $\vec{k}$ is greater than
    $\vec{b}$ on the $S$ components, and the other components are below the
    bound $N_0$. The language of an admissible set $S$ is the set of words
    obtained by repeating $w_i$ at most $N_0$ times if it is not in $S$
    ($w_i^{\leq N_0}$) and arbitrarily many times otherwise ($w_i^*$).
    Note that $L \subseteq \bigcup_{S \text{ admissible }} L(S)$.

    Now, admissible languages are ready to be pumped according to
    \cref{pumping-periods:lem}. For every admissible language,
    the size of a word that is not iterated is at most
    $N_0 \times m$ by definition, and we conclude that:
    \begin{equation}
        \label{bounded-language:eq}
        L \subseteq 
        \bigcup_{x,y \in \Sigma^{\leq n}}
        \bigcup_{u \in \Sigma^{\leq m \times N_0}}
        \InfPeriodChain{x} u \InfPeriodChain{y}
        \cup
        \InfPeriodChain{x} u
        \cup
        u \InfPeriodChain{x}
        \quad .
    \end{equation}
\end{proofof}


Let us now discuss the implications of this characterization in terms of
\kl{downwards closures}: if $L$ is a \kl{bounded language}, then considering
$L$ or its \kl{downwards closure} is equivalent with respect to being
\kl{well-quasi-ordered} by the \kl{infix relation}.

\begin{corollary}
    \label{bounded-wqo-dwclosed:cor}
    Let $L$ be a \kl{bounded language} of $\Sigma^*$. Then,
    $L$ is a \kl{well-quasi-order} when endowed with the
    \kl{infix relation} if and only if $\dwset[\infleq]{L}$ is.
\end{corollary}
\begin{proof}
    Because $L \subseteq \dwset[\infleq]{L}$, the right-to-left implication
    is trivial.
    For the left-to-right implication, let us assume that $L$ is a
    \kl{well-quasi-ordered} language for the \kl{infix relation}.
    Then $L$ is included in a finite union 
    of products of \kl{chains}:
    \begin{equation*}
        L \subseteq \bigcup_{i = 1}^n S_i \cdot P_i \quad .
    \end{equation*}
    Remark that the \kl{downwards closure} of a product of two \kl{chains}
    is a product of two chains.
    As a consequence, we conclude that $\dwset[\infleq]{L}$ is itself included
    in a finite union of products of \kl{chains}.
    Note that this also proves that $\dwset[\infleq]{L}$ is a \kl{bounded language},
    hence that it is \kl{well-quasi-ordered} by the \kl{infix relation} 
    via
    \cref{bounded-language:thm}.
\end{proof}


One may think that all \kl{downwards closed} languages for the \kl{infix
relation} that are \kl{well-quasi-ordered} are \kl(language){bounded}. Note
that this is what happens in the case of the \kl{subword embedding}, where any
\kl{downwards closed} language for this relation is characterized by finitely
many excluded \kl{subwords}, hence provides a \kl{bounded language}. However, this
is not the case for the \kl{infix relation}, as we will now illustrate with the
following two examples.

\begin{example}
    \label{dwclosed-wqo-not-finite-excl:ex}
    Let $L \defined a^* b^* \cup b^* a^*$. This language is \kl{downwards
    closed} for the \kl{infix relation}, is \kl{well-quasi-ordered} for the
    \kl{infix relation}, but is characterized by an \emph{infinite} number 
    of excluded infixes, respectively of the form $ab^ka$ and $ba^kb$ where $k \geq 1$.
\end{example}

To strengthen \cref{dwclosed-wqo-not-finite-excl:ex}, we will
leverage the \intro{Thue-Morse sequence} $\intro*\ThueMorse \in
\set{0,1}^{\Nat}$, which we will use as a black-box for its two main
characteristics: it is \kl{cube-free} and \kl{uniformly recurrent}. Being
\intro{cube-free} means that no (finite) word of the form $uuu$ is an
\kl{infix} of $\ThueMorse$, and being \intro{uniformly recurrent} means that
for every word $u$ that is an \kl{infix} of $\ThueMorse$, there exists $k \geq
1$ such that $u$ is an \kl{infix} of every word $v$ of size at least $k$. We
refer the reader to a nice survey of Allouche and Shallit for more information
on this sequence and its properties \cite{ALSHA99}.

\begin{lemma}
    \label{thue-morse:lemma}
    There exists a language $L$ that is \kl{downwards closed} for the \kl{infix
    relation}, \kl{well-quasi-ordered} for the \kl{infix relation}, but is not
    \kl(language){bounded}.
\end{lemma}
\begin{proof}
    Let $\ThueMorse$ be the \kl{Thue-Morse sequence}
    $L$ be the set of infixes of $w$. By construction $L$ is an \emph{infinite}
    \kl{downwards closed} for the \kl{infix relation}. Let us argue that $L$ is
    \kl{well-quasi-ordered} for the \kl{infix relation}, but is not \kl(language){bounded}.

    Assume by contradiction that $L$ is \kl(language){bounded}. In this case, there exist
    words $w_1, \dots, w_k \in \Sigma^*$ such that $L \subseteq w_1^* \cdots
    w_k^*$. Since $L$ is infinite and \kl{downwards closed}, there exists a
    word $u \in L$ such that $u = w_i^3$ for some $1 \leq i \leq k$. This is absurd
    because $u \infleq \ThueMorse$, which is \kl{cube-free}.

    Furthermore, $L$ is \kl{well-quasi-ordered} for the \kl{infix relation}.
    Indeed, consider a sequence $\seqof{u_i}$ of words in $L$. Without loss of
    generality, we may consider a subsequence such that $\card{u_i} <
    \card{u_{i+1}}$ for all $i \in \Nat$. Because $\ThueMorse$ is \kl{uniformly
    recurrent}, there exists $k \geq 1$ such that $u_1$ is an \kl{infix} of
    every word $v$ of size at least $k$. In particular, $u_1$ is an \kl{infix}
    of $u_k$, hence the sequence $\seqof{u_i}$ is \kl(sequence){good}.
\end{proof}

One may refine our analysis of the \kl{Thue-Morse sequence} to obtain 
precise bounds on the \kl{ordinal invariants} of its language of \kl{infixes}.

\begin{lemma}
    \label{thue-morse-ordinal:lemma}
    Let $L$ be the language of \kl{infixes} of the \kl{Thue-Morse sequence}.
    Then, the \kl{ordinal invariants} of $L$ are as follows:
    the \kl{maximal order type} of $L$ is $\omega$,
    the \kl{ordinal height} of $L$ is $\omega$,
    the \kl{ordinal width} of $L$ is $\omega$.
\end{lemma}
\begin{proof}
    Let us prove that these are upper bounds for the \kl{ordinal invariants} of
    $L$. For the \kl{ordinal height}, it is true for any language $L$.
    For the \kl{maximal order type}, we remark that
    the maximal length of a \kl{bad sequence} is determined by its first element,
    hence that it is at most $\omega$.
    Finally, because the \kl{ordinal width} is at most the \kl{maximal order type},
    we conclude that the \kl{ordinal width} is also at most $\omega$.

    Now, let us prove that these bounds are tight. It is clear that
    $\oHeight{L} = \omega$: given any number $n \in \Nat$, one can construct a
    \kl{decreasing sequence} of words in $L$ of length $n$, for instance by
    considering the first $n$ prefixes of the \kl{Thue-Morse sequence} by
    decreasing size.
    Let us now prove that $\oWidth{L} = \omega$. Assume by contradiction that
    $\oWidth{L}$ is finite. Then, $L$ can be written as a finite union of
    \kl{chains} for the \kl{infix relation}, and in particular, $L$ is
    \kl(language){bounded}, which is absurd by \cref{thue-morse:lemma}.
    Finally, because the \kl{ordinal width} is at most the \kl{maximal order
    type}, we conclude that the \kl{maximal order type} of $L$ is also $\omega$.
\end{proof}

This suggests the following conjecture which states that for \kl{downwards
closed} languages, the \kl{ordinal invariants} are relatively small.

\begin{conjecture}
    \label{small-ordinal-invariants:conj}
    Let $L$ be a \kl{well-quasi-ordered} language for the \kl{infix relation}
    that is \kl{downwards closed}.
    Then, the \kl{maximal order type} of $L$ is at most $\omega^3$,
    its \kl{ordinal height} is at most $\omega$,
    and its \kl{ordinal width} is at most $\omega^2$.
\end{conjecture}

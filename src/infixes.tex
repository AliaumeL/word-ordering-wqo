% LTeX: language=en-GB 
\section{Infixes and Regular Languages}
\label{infixes-regular:sec}

\todo{Delete this section: the \cref{infix-embedding:thm} is already
    known, we should just restate it. The
    \cref{infix-finite-automata:thm} is a direct consequence of
    \cref{infix-amalgamation:thm} and \cref{bounded-language:thm}.
    I don't know if we should keep it for clarity, but it does not 
    seems to be necessary to me.
}

\AP
In this section, we study languages equipped with the \kl{infix relation}. As
opposed to the \kl{prefix} and \kl{suffix} relations, the \kl{infix relation}
can yield to very complicated \kl{well-quasi-ordered} languages. Formally, the
upcoming \cref{infix-embedding:thm} shows that \emph{any} countable
partial-ordering with finite initial segments can be embedded into the infix
relation of a language. To make the former statement precise,
Let us recall that an \intro{order embedding} from a quasi-ordered set $(X,
\preceq)$ into a quasi-ordered set $(Y, \preceq')$ is a function $f \colon X
\to Y$ such that for all $x, y \in X$, $x \preceq y$ if and only if $f(x)
\preceq' f(y)$. When such an embedding exists, we say that $X$ \reintro{embeds
into} $Y$. We say that a quasi-ordered set $(X, \preceq)$ is a \intro{partial
ordering} whenever the relation $\preceq$ is antisymmetric, that is $x \preceq
y$ and $y \preceq x$ implies $x = y$. 

The following result due to Kuske shows the structural complexity of the \kl{infix relation}

\begin{lemma}{\cite[Lemma 5.1]{DBLP:journals/ita/Kuske06}}
    \label{infix-embedding:thm}
    Let $(X, \preceq)$ be a \kl{partially ordered} set,
    and $\Sigma$ be an alphabet with at least two letters.
    Then the following are equivalent:
    \begin{enumerate}
        \item \label{infix-embedding-embeds:item} 
            $X$ \kl{embeds into} $(\Sigma^*, \infleq)$,
        \item \label{infix-embedding-count:item}
            $X$ is countable, and for every $x \in X$,
            its \kl{downwards closure}
            $\dwset[\preceq]{x}$ is finite.
    \end{enumerate}
\end{lemma}
\begin{figure}
    \centering
    \includestandalone[width=\linewidth]{fig/infix-encoding-standalone}
    \caption{Representation of the \kl{subword relation} for $\set{a,b}^*$
        inside the \kl{infix relation} for $\set{a,b,\#}^*$
        using the encoding of \cref{infix-embedding:thm}, restricted to words
        of length at most $3$. To obtain smaller words,
        we replaced $D_i$ by $\max D_i$ in this construction.
    }
    \label{infix-embedding:fig}
\end{figure}

As a consequence of \cref{infix-embedding:thm}, we cannot replay proofs of
\cref{prefixes:sec}, and will actually need to leverage some regularity of the
languages to obtain a characterization of \kl{well-quasi-ordered} languages
under the \kl{infix relation}. Let us first play this game for languages that
are recognized by finite automata. We assume that the reader is familiar with
the notions of deterministic finite automaton and regular languages, and refer
to the book of Thomas for a comprehensive introduction to the subject
\cite{THOM97}. The main goal of the remainder of this section is to prove the
following \cref{infix-finite-automata:thm}.

\begin{theorem}[restate=infix-finite-automata:thm,label=infix-finite-automata:thm]
    \proofref{infix-finite-automata:thm}
    Let $L \subseteq \Sigma^*$ be a language recognized by a finite automaton.
    Then $L$ is \kl{well-quasi-ordered} by the \kl{infix relation} if and only if $L$ is
    a finite union of \kl{chains} for the \kl{infix relation}.
\end{theorem}

\AP In order to prove \cref{infix-finite-automata:thm}, we will perform some
preliminary analysis on the structure of an automaton recognizing a
well-quasi-ordered language under the infix relation, which will be powered by
folklore results on \emph{periodic} words. Let us recall that a non-empty word
$w \in \Sigma^+$ is \intro(word){periodic} with period $x \in \Sigma^*$ if
there exists a $p \in \Nat$ such that $w \infleq x^p$. The \intro{periodic
length} of a word $u$ is the minimal length of a word $x$ such that $u$ is an
\kl{infix} of $x^p$ for some $p \in \Nat$ and $x \in \Sigma^+$. We will
essentially rely on the following result on periodic words.

\begin{lemma}
    \label{periodic-infixes:lem}
    Let $u,v \in \Sigma^*$ be two (non-empty) \kl{periodic words}
    having \kl{periodic lengths} $p$ and $q$ respectively.
    Then, if $u \infleq v$ and $\card{u} \geq \factorial[p]{p \times q}$,
    then $u$ and $v$ share the same \kl{periodic length}
    $p = q$.
\end{lemma}
\begin{proof}
    The fact that $u$ and $v$ are \kl{periodic length}
    respectively $p$ and $q$ translates into the fact that $u_{i+p} = u_i$ and
    $v_{i+q} = v_i$ for all indices $i \in \Nat$ such that those letters are
    well-defined.

    Now, assume that $u$ is an \kl{infix} of $v$, this provides the existence
    of a $k \in \Nat$ such that $u = v_{k} \cdots v_{k + \card{u} - 1}$. In
    particular, $v_{k+i+p} = v_{k+i}$ for all $i \in \Nat$ such that $k+i+p < k
    + \card{u}$. Since we also have $v_{k+i+q} = v_{k+i}$ for all $1 \leq i
    \leq \card{v} - k - q$. We conclude that both $u$ and $v$ are of
    \kl{periodic length} the greatest common divisor of $p$ and $q$, and by
    minimality of $q$ this must be equal to $q$ and to $p$.
\end{proof}

The upcoming \cref{powers-infixes:cor} is based on the observation that if $S$
is a \kl{chain} for the \kl{suffix relation} and $P$ is a \kl{chain} for the
\kl{prefix relation}, then $SP$ is a \kl{chain} for the \kl{infix relation}.

\begin{corollary}
    \label{powers-infixes:cor}
    Let $u,v \in \Sigma^*$ and $k, \ell \in \Nat$
    such that $k \geq \factorial[p]{\card{u} \times \card{v}}$,
    $\ell \geq \factorial[p]{\card{v} \times \card{u}}$,
    and $u^k \infleq v^\ell$.
    Then, there exists $w \in \Sigma^*$ of size at most
    $\min \set{\card{u}, \card{v}}$ and a $p \in \Nat$
    such that
    $u^k \infleq v^\ell \infleq w^p$.
\end{corollary}

The reason why \kl{periodic words} built using a given period $x \in \Sigma^+$
are interesting for the \kl{infix relation} is that they naturally create
\kl{chains}. Indeed, if $x \in \Sigma^+$ is a finite word, then $\setof{x^p}{p
\in \Nat}$ is a \kl{chain} for the \kl{infix relation}. Note that in general,
the \kl{downwards closure} of a chain is \emph{not} a chain. However, for the
chains generated using periodic words, the \kl{downwards closure}
$\dwset[\infleq]{\setof{x^p}{p \in \Nat}}$ is a \emph{finite union} of
\kl{chains}. Because this set will appear in bigger equations, we introduce the
shorter notation $\intro*\InfPeriodChain{x}$ for the set of \kl{infixes} of
words of the form $x^p$, where $p$ ranges over $\Nat$.

\begin{lemma}
    \label{inf-period-chain:lem}
    Let $x \in \Sigma^+$ be a word, and
    Then $\InfPeriodChain{x}$ is a finite union of \kl{chains}
    for the \kl{infix}, \kl{prefix} and \kl{suffix} relations 
    simultaneously.
\end{lemma}
\begin{proof}
    Let $x \in \Sigma^+$ be a word, and let $P_x$ be the (finite) set 
    of all \kl{prefixes} of $x$, and $S_x$ be the (finite)
    set of all \kl{suffixes} of $x$.
    Assume that $w \in \InfPeriodChain{x}$, then $w = u x^p v$ for some
    $u \in S_x$, $v \in P_x$, and $p \in \Nat$.
    We have proven that
    \begin{equation*}
        \InfPeriodChain{x} \subseteq \bigcup_{u \in P_x} \bigcup_{v \in S_x} u x^* v
        \quad .
    \end{equation*}

    Let us now demonstrate that for all $(u,v) \in S_x \times P_x$, the
    language $u x^* v$ is a \kl{chain} for the \kl{infix}, \kl{suffix} and \kl{prefix} relations.
    To that end,
    let $(u,v) \in S_x \times P_x$ and $\ell, k \in \Nat$ be such that $\ell <
    k$, let us prove that $u x^\ell v \infleq u x^k  v$. Because $v \prefleq
    x$, we know that there exists $w$ such that $vw = x$. In particular,
    $ux^\ell vw = u x^{\ell + 1}$, and because $\ell < k$, we conclude that $u
    x^{\ell + 1} \prefleq u x^k v$. By transitivity, $u x^\ell v \prefleq u x^k
    v$, and \emph{a fortiori}, $u x^\ell v \infleq u x^k v$. 
    Similarly, because $u \suffleq x$,  there exists $w$ such that $wu  = x$, 
    and we conclude that $u x^{\ell} v \suffleq w u x^\ell v = x^{\ell + 1} v \suffleq u x^k v$.
    \qedhere
\end{proof}


\begin{corollary}
    \label{inf-period-union-chains:lem}
    Let $x,u,y \in \Sigma^*$ be words.  The following 
    are finite unions of \kl{grids} for the infix relation:
    $\InfPeriodChain{x}u$, $u \InfPeriodChain{x}$,
    and $\InfPeriodChain{x} u \InfPeriodChain{y}$.
\end{corollary}
\begin{proof}
    Because $\InfPeriodChain{x}$ is a finite union of \kl{chains} for the \kl{suffix}
    relation (rsing \cref{inf-period-chain:lem}), we conclude that $\InfPeriodChain{x}u$ is one too for
    the \kl{infix} relation. This holds similarly for $u \InfPeriodChain{x}$.
    In the case of $\InfPeriodChain{x} u \InfPeriodChain{y}$,
    we remark that $\InfPeriodChain{x} u$ is a finite union of chains
    for the \kl{suffix relation},
    and that $\InfPeriodChain{y}$ is one for the \kl{prefix relation}
    (using again \cref{inf-period-chain:lem}).
    As a consequence,
    $\InfPeriodChain{x}u \InfPeriodChain{y}$ is a finite union 
    of \kl{chains} for the \kl{infix} relation.
\end{proof}

The following combinatorial lemma connects the property of being
\kl{well-quasi-ordered} to a property of the \kl{periodic lengths} of words in
a language, based on the assumption that some factors can be iterated. It is
the core result that powers the analysis done in the upcoming
\cref{infix-finite-automata:thm,infix-amalgamation:thm}.

\begin{lemma}
    \label{pumping-periods:lem}
    Let $L \subseteq \Sigma^*$ be a language
    that is \kl{well-quasi-ordered} by the \kl{infix relation}.
    Let $k \in \Nat$, $u_1, \cdots, u_{k+1} \in \Sigma^*$,
    and $v_1, \cdots, v_{k} \in \Sigma^+$
    be such that
    $w[\vec{n}] \defined (\prod_{i = 1}^k u_i v_i^{n_i}) u_{k+1}$
    belongs to $L$
    for arbitrarily large values of $\vec{n} \in \Nat^k$.
    Then, 
    there exists $x,y \in \Sigma^+$ of size 
    at most $\max \setof{\card{v_i}}{1 \leq i \leq k}$
    such that 
    one of the following holds for all
    $\vec{n} \in \Nat^{k}$:
    \begin{enumerate}
        \item $w[\vec{n}] \in u_1 \InfPeriodChain{x}$,
        \item $w[\vec{n}] \in \InfPeriodChain{x} u_{k+1}$,
        \item $w[\vec{n}] \in \InfPeriodChain{x} u_i \InfPeriodChain{y}$
            for some $1 \leq i \leq k + 1$.

    \end{enumerate}
\end{lemma}
\begin{proof}
    Note that the result is obvious if $k = 0$, and therefore
    we assume $k \geq 1$ in the following proof.

    Let us construct a sequence of words $\seqof{w_i}[i \in \Nat]$, where $w_i
    \defined w[\vec{n_i}]$ for some well-chosen indices $\vec{n_i} \in \Nat^k$. The goal
    being that 
    if $w[\vec{n_i}]$ is an \kl{infix} of $w[\vec{n_j}]$,
    then it can intersect at most \emph{two} iterated words,
    with an intersection that is long enough to successfully apply
    \cref{periodic-infixes:lem}.
    In order to achieve this,
    let us first define $s$ as the maximal size of a word $v_i$
    ($1 \leq i \leq k$) and $u_j$ ($1 \leq j \leq k+1$).
    Then,
    we consider $\vec{n_0} \in \Nat^k$ such that $\vec{n_0}$ has all 
    its components greater than $\factorial{s}$ and such that
    $w[\vec{n_0}]$ belongs to $L$. 
    Then, we inductively define 
    $\vec{n_{i+1}}$  as the smallest vector of numbers greater than $\vec{n_i}$,
    such that $w[\vec{n_{i+1}}]$ belongs to $L$, 
    and with $\vec{n_i}$ having all components greater than
    $2\card{w[\vec{n_i}]}$.


    Let us assume that $k \geq 2$ in the following proof for symmetry purposes,
    and argue later on that when $k = 1$ the same argument goes through.
    Because $L$ is \kl{well-quasi-ordered} by the \kl{infix relation}, there
    exists $i < j$ such that $w[\vec{n_i}]$ is an \kl{infix} of $w[\vec{n_j}]$.
    Now, because of the chosen values for $\vec{n_j}$, there exists $1 \leq \ell \leq
    k-1$ such that $w[\vec{n_i}]$ is actually an \kl{infix} of $u_{\ell}
    v_{\ell}^{n_{j,\ell}} u_{\ell+1} v_{\ell+1}^{n_{j,\ell+1}} u_{\ell+2}$.
    Even more,
    one of the three following equations holds:
    \begin{itemize}
        \item $w[\vec{n_i}] \infleq v_{\ell}^{n_{j,\ell}} u_{\ell+1} v_{\ell+1}^{n_{j,\ell+1}}$,
        \item $w[\vec{n_i}] \infleq u_{\ell}
            v_{\ell}^{n_{j,\ell}}$,
        \item $w[\vec{n_i}] \infleq
            v_{\ell+1}^{n_{j,\ell+1}} u_{\ell+2}$.
    \end{itemize}
    In all those cases, we conclude using \cref{powers-infixes:cor}
    that there exists $x,y \in \Sigma^+$ of size at most $s$, and 
    a number $1 \leq t \leq k$ such that
    $v_i^{n_i} \in \InfPeriodChain{x}$ for all $1 \leq i \leq t$,
    and
    $v_i^{n_i} \in \InfPeriodChain{y}$ for all $t < i \leq k$.
    In particular,
    $w[\vec{n_i}] \in \InfPeriodChain{x} u_{t} \InfPeriodChain{y}$.

    
    When $k = 1$, the situation is a bit more specific since we only have two
    cases: either $w_i \infleq u_1 v_1^{n_j}$ or $w_i \infleq v_1^{n_j} u_2$,
    and we conclude with an identical reasoning.
\end{proof}


We are now ready to restate and prove our main theorem, proven by combining
\cref{pumping-periods:lem} together with classical pumping arguments on finite
state automata.

\begin{proofof}{infix-finite-automata:thm}
    Let $w \in L$, because $Q$ is finite, there exists
    a factorization of $w$
    into words $(\prod_{i = 1}^k u_i v_i) u_{k+1}$
    such that 
    for all $1 \leq i \leq k+1$, $\card{u_i} \leq \card{Q}$,
    for all $1 \leq i \leq k$, $1 \leq \card{v_i} \leq \card{Q}$,
    and satisfying 
    that $w[\vec{X}] \defined 
    (\prod_{i = 1}^k u_i v_i^{X_i}) u_{k+1}$
    belongs to $L$ for all choices of values $\vec{X} \in \Nat^k$.

    Applying \cref{pumping-periods:lem}, we conclude that 
    there exists $x,y \in \Sigma^+$ of size at most $\card{Q}$
    such that 
    $w \in u_1 \InfPeriodChain{x} \cup \InfPeriodChain{y} u_{k+1}
    \cup \bigcup_{1 \leq i \leq k+1} \InfPeriodChain{x} u_i \InfPeriodChain{y}$.
    In particular, we conclude that
    \begin{equation}
        \label{infix-automata:eq}
        L
        \subseteq
        \bigcup_{x,y,u\in \Sigma^{\leq \card{Q}}}
        u \InfPeriodChain{x}
        \cup 
        \InfPeriodChain{x} u
        \cup
        \InfPeriodChain{x} u \InfPeriodChain{y}
        \quad .
    \end{equation}

    Now, thanks to \cref{inf-period-union-chains:lem},
    this happens to be a finite union of \kl{chains}
    for the \kl{infix relation}.
\end{proofof}



\begin{corollary}
    \label{reg-wqo-decidable:cor}
    Given a regular language $L$, it is decidable whether
    $L$ is \kl{well-quasi-ordered} for the \kl{infix relation}.
\end{corollary}
\begin{proof}
    It suffices to decide whether 
    \cref{infix-automata:eq} holds for the language $L$.
    This is an inclusion of regular languages, which is decidable.
\end{proof}


Let $(X,\preceq)$ be a quasi-ordered set, and $L \subseteq X$ be such that $(L,
\preceq)$ is \kl{well-quasi-ordered}. It is not true in general that
$(\dwset{L}, \preceq)$ is \kl{well-quasi-ordered}. In the case of $(\Sigma^*,
\infleq)$ a typical example is to start from an infinite \kl{antichain} $A$,
together with an enumeration $\seqof{w_i}$ of $A$, and build the language $L
\defined \setof{ \prod_{i = 0}^n w_i }{ i \in \Nat }$. By definition, $L$ is a
\kl{chain} for the \kl{infix} ordering, hence \kl{well-quasi-ordered}. However,
$\dwset[\infleq]{L}$ contains $A$, and is therefore not
\kl{well-quasi-ordered}. As a fun corollary of our analysis, we conclude that
this particular example cannot be realized by a regular language $A$.


\begin{corollary}
    \label{reg-wqo-iff-dwclosed-wqo:cor}
    Given a regular language $L$, it is \kl{well-quasi-ordered} 
    for the \kl{infix relation} if and only if 
    $\dwset[\infleq] L$ is.
\end{corollary}

\section{Bounded Languages and Amalgamation Systems}
\label{infixes-amalgamation:sec}

When proving \cref{infix-finite-automata:thm}, we have leveraged a powerful
combinatorial argument on regular langugages stated in
\cref{pumping-periods:lem}. Our first remark is that this idea can
be generalized to the larger class of \intro{bounded languages}, i.e.,
languages $L \subseteq \Sigma^*$ such that there exists words $w_1, \dots, w_n$
satisfying $L \subseteq w_1^* \cdots w_n^*$.

\begin{theorem}
    \label{bounded-language:thm}
    Let $L$ be a \kl{bounded language} of $\Sigma^*$. Then,
    $L$ is a \kl{well-quasi-order} when endowed with the 
    \kl{infix relation} if and only if it is a finite union of \kl{grids}.
\end{theorem}
\begin{proof}
    Let $w_1, \dots, w_n$ be such that
    $L \subseteq w_1^* \cdots w_n^*$.
    Let us define $m \defined \max \setof{\card{w_i}}{1 \leq i \leq n}$

    Let $w[\vec{k}] \defined w_1^{k_1} \cdots w_n^{k_n}$ be a map from $\Nat^k$
    to $\Sigma^*$. We are interested in the intersection of the image of $w$
    with $L$. Let us assume for instance that for all $\vec{k} \in \Nat^n$,
    there exists $\vec{\ell} \geq \vec{k}$ such that $w[\vec{\ell}] \in L$.
    Then, leveraging \cref{pumping-periods:lem}, we conclude that there exists
    $x,y$ of size at most $\max\setof{\card{w_i}}{1 \leq i \leq n}$ such that
    $w[\vec{k}] \in \InfPeriodChain{x} \cup \InfPeriodChain{x}
    \InfPeriodChain{y}$, and we conclude that $L \subseteq \InfPeriodChain{x}
    \cup \InfPeriodChain{x} \InfPeriodChain{y}$.

    Now, it may be the case that one cannot simultaneously assume that two
    component of the vector $\vec{k}$ are unbounded. In general, given a set $S
    \subseteq \set{1, \dots, n}$ of indices, we say that $S$ is admissible if
    there exists a bound $N_0$ such that for all $\vec{b} \in \Nat^S$, there
    exists a vector $\vec{k} \in \Nat^n$, such that $\vec{k}$ is greater than
    $\vec{b}$ on the $S$ components, and the other components are below the
    bound $N_0$. The language of an admissible set $S$ is the set of words
    obtained by repeating $w_i$ at most $N_0$ times if it is not in $S$
    ($w_i^{\leq N_0}$) and arbitrarily many times otherwise ($w_i^*$).
    Note that $L \subseteq \bigcup_{S \text{ admissible }} L(S)$.

    Now, admissible languages are ready to be pumped according to
    \cref{pumping-periods:lem}. For every admissible language,
    the size of a word that is not iterated is at most
    $N_0 \times m$ by definition, and we conclude that:
    \begin{equation*}
        L \subseteq 
        \bigcup_{x,y \in \Sigma^{\leq n}}
        \bigcup_{u \in \Sigma^{\leq m \times N_0}}
        \InfPeriodChain{x} u \InfPeriodChain{y}
        \cup
        \InfPeriodChain{x} u
        \cup
        u \InfPeriodChain{x}
        \quad .
    \end{equation*}
\end{proof}

Note that this immediately generalizes \cref{reg-wqo-iff-dwclosed-wqo:cor}
to all \kl{bounded languages}.

\begin{corollary}
    \label{bounded-wqo-dwclosed:cor}
    Let $L$ be a \kl{bounded language} of $\Sigma^*$. Then,
    $L$ is a \kl{well-quasi-order} when endowed with the
    \kl{infix relation} if and only if $\dwset[\infleq]{L}$ is.
\end{corollary}


In the light of \cref{reg-wqo-iff-dwclosed-wqo:cor,bounded-wqo-dwclosed:cor}
one may wonder whether the \kl{downwards closed}
languages for the \kl{infix relation} satisfy a similar characterization in
terms of union of \kl{grids} as stated in
\cref{infix-finite-automata:thm,bounded-language:thm}. Let us
first remark that from our previous results we obtain the following lemma that
illustrates how all the previous characterizations of \kl{well-quasi-ordered}
languages for the \kl{infix relation} are collapsing in the case of
\kl{downwards closed} languages.

\begin{lemma}
    \label{dwclosed-infixes-wqo:lem}
    Let $L \subseteq \Sigma^*$ be a \kl{downwards closed} language for the
    \kl{infix relation} that is \kl{well-quasi-ordered}. Then, the following
    are equivalent:
    \begin{enumerate}
        \item $L$ is \kl{bounded},
        \item $L$ is \kl{regular},
        \item $L$ is a finite union of \kl{chains} and \kl{grids}.
    \end{enumerate}
\end{lemma}
\begin{proof}
    TODO.
\end{proof}

One may think that all \kl{downwards closed} languages for the \kl{infix
relation} that are \kl{well-quasi-ordered} are \kl{regular}. Note that this is
what happens in the case of the \kl{subword embedding}, where any \kl{downwards
closed} language for this relation is characterized by finitely many excluded
patterns, hence \kl{regular}. However, this is not the case for the \kl{infix
relation}, as we will now illustrate with the following two examples.

\begin{example}
    \label{dwclosed-wqo-not-finite-excl:ex}
    Let $L \defined a^* b^* \cup b^* a^*$. This language is \kl{downwards
    closed} for the \kl{infix relation}, is \kl{well-quasi-ordered} for the
    \kl{infix relation}, but is characterized by an \emph{infinite} number 
    of excluded infixes, respectively of the form $ab^ka$ and $ba^kb$ where $k \geq 1$.
\end{example}

To strengthen \cref{dwclosed-wqo-not-finite-excl:ex}, we will
leverage the \intro{Thue-Morse sequence} $\intro*\ThueMorse \in
\set{0,1}^{\Nat}$, which we will use as a black-box for its two main
characteristics: it is \kl{cube-free} and \kl{uniformly recurrent}. Being
\intro{cube-free} means that no (finite) word of the form $uuu$ is an
\kl{infix} of $\ThueMorse$, and being \intro{uniformly recurrent} means that
for every word $u$ that is an \kl{infix} of $\ThueMorse$, there exists $k \geq
1$ such that $u$ is an \kl{infix} of every word $v$ of size at least $k$. We
refer the reader to a nice survey of Allouche and Shallit for more information
on this sequence and its properties \cite{ALSHA99}.

\begin{lemma}
    \label{thue-morse:lemma}
    There exists a language $L$ that is \kl{downwards closed} for the \kl{infix
    relation}, \kl{well-quasi-ordered} for the \kl{infix relation}, but is not
    \kl{regular}.
\end{lemma}
\begin{proof}
    Let $\ThueMorse$ be the \kl{Thue-Morse sequence}
    $L$ be the set of infixes of $w$. By construction $L$ is an \emph{infinite}
    \kl{downwards closed} for the \kl{infix relation}. Let us argue that $L$ is
    \kl{well-quasi-ordered} for the \kl{infix relation}, but is not \kl{regular}.

    Assume by contradiction that $L$ is \kl{bounded}. In this case, there exist
    words $w_1, \dots, w_k \in \Sigma^*$ such that $L \subseteq w_1^* \cdots
    w_k^*$. Since $L$ is infinite and \kl{downwards closed}, there exists a
    word $u \in L$ such that $u = w_i^3$ for some $1 \leq i \leq k$. This is absurd
    because $u \infleq \ThueMorse$, which is \kl{cube-free}.

    Furthermore, $L$ is \kl{well-quasi-ordered} for the \kl{infix relation}.
    Indeed, consider a sequence $\seqof{u_i}$ of words in $L$. Without loss of
    generality, we may consider a subsequence such that $\card{u_i} <
    \card{u_{i+1}}$ for all $i \in \Nat$. Because $\ThueMorse$ is \kl{uniformly
    recurrent}, there exists $k \geq 1$ such that $u_1$ is an \kl{infix} of
    every word $v$ of size at least $k$. In particular, $u_1$ is an \kl{infix}
    of $u_k$, hence the sequence $\seqof{u_i}$ is \kl(wqo){good}.

    We conclude thanks to
    \cref{dwclosed-infixes-wqo:lem}, that $L$ is not \kl{regular}.
\end{proof}



\subsection{Amalgamation Systems}
\label{infixes-amalgamation:subsec}

It turns out that there is a rather large family of systems for which pumping
arguments based on so-called \emph{minimal runs} exist: they are called
\emph{amalgamation systems} and were recently introduced by Anad, Schmitz,
Sch\"{u}tze, and Zetzsche \cite{ASZZ24}. Having done the heavy lifting on
\kl{bounded languages}, the rest of this section is mostly devoted to the
introduction of \kl{amalgamation systems} and collecting the necessary pumping
argument they enjoy. Using this meta-proof, we will generalize
\cref{infix-finite-automata:thm} to context-free grammars, languages recognized
by vector addition systems with state (VASS), and more. The goal of this
section is not to introduce all of these systems, or justify their usefulness,
but to state and prove the following theorem.

\begin{theorem}[label=infix-amalgamation:thm,restate=infix-amalgamation:thm]
    \proofref{infix-amalgamation:thm}
    Let $L \subseteq \Sigma^*$ be a language recognized by an 
    \kl{amalgamation system}.
    Then $L$ is well-quasi-ordered by the infix relation if and only if $L$ is
    a finite union of chains for the \kl{infix relation}.
\end{theorem}

Let us now formally introduce the notion of \kl{amalgamation systems}, and
recall some results from \cite{ASZZ24} that will be useful for the proof of
\cref{infix-amalgamation:thm}. The notion of \kl{amalgamation system} is
tailored to produce \emph{pumping arguments}, which is exactly what our
\cref{pumping-periods:lem} talks about. At the core of a pumping argument,
there is a notion of \emph{run}, which could for instance be a sequence of
transitions taken in a finite state automaton. Continuing on the analogy with
finite automata, there is a natural ordering between runs, i.e., a run is
smaller than another one if one can ``unroll loops" of the first run to obtain
the second one. Typical pumping arguments then rely on the fact that
\emph{minimal} runs are of finite size, and that all other runs are
obtained by ``gluing" minimal runs together. This is exactly what
\kl{amalgamation systems} are about.

\AP Let us recall that over an alphabet $(\Sigma, =)$ a \kl{subword embedding}
between two words $u \in \Sigma^*$ and $v \in \Sigma^*$ is a function $\rho
\colon \range{\card{u}} \to \range{\card{v}}$ such that $u_i = v_{\rho(i)}$ for
all $i \in \range{\card{u}}$. We write $\intro*\HigEmb(u,v)$ the set of all
\kl{subword embeddings} between $u$ and $v$. It may be useful to notice that
the set of finite words over $\Sigma$ forms a category when we consider
\kl{subword embeddings} as morphisms, which is a fancy way to state that
$\mathrm{id} \in \HigEmb(u,u)$ and that $f \circ g \in \HigEmb(u,w)$ whenever
$g \in \HigEmb(u,v)$ and $f \in \HigEmb(v,w)$, for any choice of words
$u,v,w \in \Sigma^*$.

\AP Given a \kl{subword embedding} $f \colon u \to v$ between two words $u$ and
$v$, there exists a unique decomposition $v = \GapWord{f}{0} u_1 \GapWord{f}{1}
\cdots \GapWord{f}{k-1} u_k \GapWord{f}{k}$ where $\GapWord{f}{i} =
v[f(i)+1:f(i+1)-1]$ for all $1 \leq i \leq k-1$, $\GapWord{f}{k} =
v[f(k)+1:\card{v}]$, and $\GapWord{f}{0}   = v[1: f(1)-1]$. We say that
$\intro*\GapWord{f}{i}$ is the $i$-th \intro{gap word} of $f$. We encourage the
reader to look at \cref{gap-word-embedding:fig} to see an example of the
\kl{gap words} resulting from a \kl{subword embedding} between two words. These
\kl{gap words} will be useful to describe how and where runs of a system
(described by words) can be combined.

\begin{figure}
    \centering
    \includestandalone[width=\linewidth]{fig/gap-word-embedding-standalone}
    \caption{The \kl{gap words} resulting from a \kl{subword embedding} between two 
    finite words.}
    \label{gap-word-embedding:fig}
\end{figure}

\begin{figure}
    \centering
    \includestandalone[width=\linewidth]{fig/run-amalgamation-standalone}
    \caption{We illustrate how 
        embeddings $f$ and $g$ between runs of an
        \kl{amalgamation system} can be glued
        together, seen on their canonical decomposition.
    }
    \label{amalgamation-runs:fig}
\end{figure}


\begin{definition}
    An \intro{amalgamation system}
    is a triple $(\Sigma, R, E, \canrun)$ where
    $\Sigma$ is a finite alphabet,
    $R$ is a set of so-called \emph{runs},
    and 
    $E$ is a set of so-called \emph{run embeddings},
    and $\canrun \colon R \to (\Sigma \uplus \set{\varepsilon})^*$ is a 
    function computing a \emph{canonical decomposition} of a run.
    Given a run $r \in R$, and $i \in \range[0]{\card{\canrun(r)}}$, 
    the \intro{gap language} of $r$ at position $i$ is $\GapLanguage{r}{i} \defined
    \setof{\GapWord{f}{i}}{\exists s \in R. \exists f \in E(r,s) }$.
    An \kl{amalgamation system} furthermore satisfies the following 
    properties:
    \begin{enumerate}
        \item \emph{Word Embeddings as Morphisms.} For all $r, s \in R$,
            $E(r,s) \subseteq \HigEmb(\canrun(r), \canrun(s))$.
        \item \emph{$(R, E)$ Forms a Category.}
            For all $r,s,t \in R$,
            $\mathrm{id} \in E(r,r)$,
            and whenever $f \in E(r,s)$ and $g \in E(s,t)$,
            then $g \circ f \in E(r,t)$.
        \item \emph{Well-Quasi-Ordered System.}
            $(R, \leq_E)$ is a well-quasi-ordered set,
            where $r \leq_E s$ if and only if $E(r,s) \neq \emptyset$.
        \item \emph{Controlled Amalgamation.}
            For all $r, s, t \in R$,
            for all $f \in E(r,s)$,
            for all $g \in E(r,t)$,
            and for all $0 \leq i_0 \leq \card{\canrun(r)}$,
            there exists a run $z \in R$ and morphisms
            $h \in E(s,z)$ and $k \in E(t,z)$
            satisfying
            $h \circ f = k \circ g$,
            $\GapWord{h \circ f}{i} = \GapWord{f}{i} \GapWord{g}{i}$
            or 
            $\GapWord{h \circ f}{i} = \GapWord{g}{i} \GapWord{f}{i}$
            for every $0 \leq j \leq n$,
            and
            $\GapWord{h \circ f}{i_0} = \GapWord{f}{i_0} \GapWord{g}{i_0}$.
            We refer to \cref{amalgamation-runs:fig} for an illustration 
            of this property.
    \end{enumerate}

    The \intro{amalgamation language} of such a system
    is the set of all words $w \in \Sigma^*$ such that
    there exists a run $r \in R$
    such that the concatenation of the letters of its
    canonical decomposition, written $\intro*\yieldrun(r)$,
    equals $w$.
\end{definition}

Intuitively, the definition of an amalgamation system allows the comparison of
runs, and the proper ``gluing" of runs together to obtain new runs. Let us
recall some examples of languages that can be recognized by \kl{amalgamation
systems} given by the authors of the original paper: regular languages
\cite[Theorem 5.3]{ASZZ24}, VASS as a consequence of \cite[Theorem
5.5]{ASZZ24}, context-free languages as a consequence of \cite[Theorem
5.10]{ASZZ24}. For this paper to be self-contained, we will also recall how
runs of a finite state automaton can be understood as an \kl{amalgamation
system}.

\begin{example}[{\cite[Section 3.2]{ASZZ24}}]
    Let $A = (Q, \delta, q_0, F)$ be a finite state automaton over a finite
    alphabet $\Sigma$. Let $\Delta$ be the set of transitions $(q_1, a, q_2)
    \in Q \times \Sigma \times Q$,
    and $R \subseteq \Delta^*$ be the set of 
    words over transitions that start with the initial state $q_0$,
    end in a final state $q_f \in F$, and such that the end state of a
    letter is the start state of the following one.
    The canonical decomposition $\canrun \colon R \to \Sigma^*$
    is defined as a morphism from $\Delta^*$ to $\Sigma^*$
    that maps $(q,a,p)$ to $a$.
    Finally, given two runs $r$ and $s$ of the automaton,
    we say that an embedding $f \in \HigEmb(\canrun(r), \canrun(s))$
    belongs to $E(r,s)$ when
    $f$ is also defining an embedding from $r$ to $s$ as words in $\Delta^*$,
    where $\Delta$ is equipped with the equality relation.

    The system $(\Sigma, R, E, \canrun)$ is an \kl{amalgamation system},
    whose language is precisely the language of words recognized
    by the automaton $A$.
\end{example}
\begin{proof}
    By definition, the embeddings inside $E(r,s)$ are defined as elements
    of $\HigEmb(\canrun(r), \canrun(s))$, and they compose properly.
    Because $\Delta = Q \times \Sigma \times Q$ is finite, it is 
    a \kl{well-quasi-ordering} when equipped with the equality relation, and 
    we conclude that $\Delta^*$ with $\higleq$ is a \kl{well-quasi-order}
    according to Higman’s Lemma \cite{HIG52}.
    
    Let us now move to proving that the system satisfies the amalgamation
    property. Given three runs $r,s,t \in R$, and two embeddings $f \in E(r,s)$
    and $g \in E(r,t)$, we want to construct an amalgamated run $s \vee t$.
    Because letters in the run $r$ respect the transitions of the automaton
    (i.e., if the letter $i$ ends in state $q$, then the letter $i+1$ starts in
    state $q$), then the \kl{gap word} at position $i$ starts in state $q$ and
    ends in state $q$ too. This means that for both embeddings
    $f$ and $g$, the \kl{gap words} are read by the automaton by looping
    on a state. In particular, these loops can be taken in any order
    and continue to represent a valid run. That is, we can even select
    the order of concatenation in the amalgamation for \emph{all} 
    $0 \leq i \leq \card{\canrun(r)}$ and not just for one separately.

    Let us now remark that 
    the language of this amalgamation system is
    the set of $\yieldrun(r)$ when $r$ ranges over $R$,
    and because $R$ is the set of valid runs of the automaton,
    and $\yieldrun(r)$ is the word read along this run,
    we immediately conclude.
\end{proof}


Let us now introduce a combinatorial lemma that explains how \kl{gap languages}
can be pumped. Note that a single \kl{gap language} is always stable under
concatenation, and our concern will be on a potential description of
\emph{simultaneously} pumping different \kl{gap languages} to produce valid
runs.

\begin{lemma}
    \label{pumping-gap-languages:lem}
    Let $(\Sigma, R, E, \canrun)$ be an \kl{amalgamation system}, let $r \in R$
    be a run of size $n \in \Nat$, and let $s, t \in R$ be two runs
    with embeddings $f \in E(r,s)$ and $g \in E(r,t)$.
    Let $\seqof{u_i}[0 \leq i \leq n]$ be the sequence of \kl{gap words}
    for $f$, that is, $u_i =
    \GapWord{f}{i}$ for all $0 \leq i \leq n$. Similarly, let
    $\seqof{v_i}[0 \leq i \leq n]$ be the sequence of \kl{gap words} for $g$.

    Then, for all $k \in \Nat$, and $0 < \ell < n$, the following words belong to 
    the language of the \kl{amalgamation system}.
    \begin{enumerate}
        \item $(\prod_{i = 0}^{\ell - 1} u_i^{x_i} v_i u_i^{y_i} v_i u_i^{z_i}) 
               v_\ell u_\ell^k v_\ell
               (\prod_{i = \ell+1}^{n} u_i^{x_i} v_i u_i^{y_i} v_i u_i^{z_i})$,
        \item $(\prod_{i = 0}^{n - 1} u_i^{x_i} v_i u_i^{y_i}) 
               v_n u_n^k$,
        \item $u_0^k v_0 
            (\prod_{i = 1}^{n} u_i^{x_i} v_i u_i^{y_i})$.
    \end{enumerate}
    Where, for all $0 \leq i \leq n$, $x_i + y_i + z_i = k$, with $z_i = 0$
    in the two last items.
\end{lemma}
\begin{proof}
    The result immediately follows from the embedding property
    applied inductively on runs obtained by gluing 
    $k$ times $s$ with one or two times $t$ (depending on the item number).
\end{proof}

In the upcoming proof of \cref{infix-amalgamation:thm}, we will use the notion
of \intro{bounded language} for languages $L \subseteq \Sigma^*$ such that
there exists words $w_1, \dots, w_n$ satisfying $L \subseteq w_1^* \cdots
w_n^*$. This will be useful for us to derive the inclusion of a language $L$
into a union of \kl{chains}. Note that we isolate this intermediate result in a
separate theorem that could be applied beyond \kl{amalgamation systems}.

\begin{theorem}
    \label{bounded-language:thm}
    Let $L$ be a \kl{bounded language} of $\Sigma^*$. Then,
    $L$ is a \kl{well-quasi-order} when endowed with the 
    \kl{infix relation} if and only if it is a finite union of \kl{chains}.
\end{theorem}
\begin{proof}
    Let $w_1, \dots, w_n$ be such that
    $L \subseteq w_1^* \cdots w_n^*$.
    Let us define $m \defined \max \setof{\card{w_i}}{1 \leq i \leq n}$

    Let $w[\vec{k}] \defined w_1^{k_1} \cdots w_n^{k_n}$ be a map from $\Nat^k$
    to $\Sigma^*$. We are interested in the intersection of the image of $w$
    with $L$. Let us assume for instance that for all $\vec{k} \in \Nat^n$,
    there exists $\vec{\ell} \geq \vec{k}$ such that $w[\vec{\ell}] \in L$.
    Then, leveraging \cref{pumping-periods:lem}, we conclude that there exists
    $x,y$ of size at most $\max\setof{\card{w_i}}{1 \leq i \leq n}$ such that
    $w[\vec{k}] \in \InfPeriodChain{x} \cup \InfPeriodChain{x}
    \InfPeriodChain{y}$, and we conclude that $L \subseteq \InfPeriodChain{x}
    \cup \InfPeriodChain{x} \InfPeriodChain{y}$.

    Now, it may be the case that one cannot simultaneously assume that two
    component of the vector $\vec{k}$ are unbounded. In general, given a set $S
    \subseteq \set{1, \dots, n}$ of indices, we say that $S$ is admissible if
    there exists a bound $N_0$ such that for all $\vec{b} \in \Nat^S$, there
    exists a vector $\vec{k} \in \Nat^n$, such that $\vec{k}$ is greater than
    $\vec{b}$ on the $S$ components, and the other components are below the
    bound $N_0$. The language of an admissible set $S$ is the set of words
    obtained by repeating $w_i$ at most $N_0$ times if it is not in $S$
    ($w_i^{\leq N_0}$) and arbitrarily many times otherwise ($w_i^*$).
    Note that $L \subseteq \bigcup_{S \text{ admissible }} L(S)$.

    Now, admissible languages are ready to be pumped according to
    \cref{pumping-periods:lem}. For every admissible language,
    the size of a word that is not iterated is at most
    $N_0 \times m$ by definition, and we conclude that:
    \begin{equation}
    	\label{amalgamation-infix-wqo-inclusion:eqn}
        L \subseteq 
        \bigcup_{x,y \in \Sigma^{\leq n}}
        \bigcup_{u \in \Sigma^{\leq m \times N_0}}
        \InfPeriodChain{x} u \InfPeriodChain{y}
        \cup
        \InfPeriodChain{x} u
        \cup
        u \InfPeriodChain{x}
        \quad .
    \end{equation}
\end{proof}

To conclude this section and prove our main \cref{infix-amalgamation:thm}, it
suffices to demonstrate that languages recognized by \kl{amalgamation systems}
that are \kl{well-quasi-ordered} for the \kl{infix relation} are \kl{bounded
languages}.

\begin{lemma}
    \label{infix-amalgamation-bounded:lem}
    Let $L \subseteq \Sigma^*$ be a language recognized
    by an \kl{amalgamation system} that is \kl{well-quasi-ordered}
    for the \kl{infix relation}. Then $L$ is a \kl{bounded language}.
\end{lemma}
\begin{proof}
    Assume that $L$ is \kl{well-quasi-ordered} by the \kl{infix relation},
    and obtained by an \kl{amalgamation system}
    $(\Sigma, R, E, \canrun)$.

    Let us consider the set $M$ of minimal runs for the relation $\leq_E$,
    which is finite because the latter is a \kl{well-quasi-ordering}. By
    definition, for every word $w \in L$, there exists $r \in M$, $s \in R$,
    and $f \in E(r,s)$ such that $w = \canrun(s)$.
    In particular, we conclude that
    \begin{equation*}
        L \subseteq \bigcup_{r \in M} 
        \left(
        \GapLanguage{r}{0}
        \prod_{i = 1}^{\card{\canrun(r)}-1} \canrun(r)_i \GapLanguage{r}{i} 
        \right)
        \quad .
    \end{equation*}

    Let $u$ and $v$ be two words in the \kl{gap language}
    $\GapLanguage{r}{\ell}$ for some $r \in M$ and $0 < \ell <
    \card{\canrun(r)} \defined n$. One can apply \cref{pumping-gap-languages:lem}
    to conclude that  for all $k \geq 1$,
    \begin{equation*}
       \left(\prod_{i = 0}^{\ell - 1} u_i^{x_i} v_i u_i^{y_i} v_i u_i^{z_i}\right) 
       v u^k v
       \left(\prod_{i = \ell+1}^{n} u_i^{x_i} v_i u_i^{y_i} v_i u_i^{z_i}\right)
       \in 
       L
    \end{equation*}
    Where $x_i + y_i + z_i = k$ for all $0 \leq i \leq n$
    Let us assume without loss of generality that the sequence
    $\seqof{(x_i, y_i, n_i)}[i \geq 1]$ is pointwise increasing,
    and even increasing coordinate wise. This is possible because
    at least one of the coordinates tends to $+\infty$ and in the case
    of a variable that is bounded, one can extract the sequence to
    simply not use this variable.

    Now, leveraging \cref{pumping-periods:lem}, we get that for infinitely many
    $k \geq 1$, $u v^k u$ is an infix of some $x^p$ where $x$ has size smaller
    or equal than $u$ and $v$. This proves  that $u$ and $v$ are comparable for
    the prefix, infix, and suffix relations. In particular, we have can
    conclude that $\GapLanguage{r}{i}$ is a \kl{bounded language}, because of
    \cite[Lemma 4.1]{ASZZ24}, when $0 < i < \card{\canrun(r)}$. The proof goes
    on similarly for proving that $\GapLanguage{r}{0}$ and $\GapLanguage{r}{n}$
    are \kl{bounded languages}, using the other items of
    \cref{pumping-gap-languages:lem}.

    Because \kl{bounded languages} are stable under concatenation and finite
    unions, we conclude that $L$ itself must be a \kl{bounded language}.
\end{proof}

\begin{proofof}{infix-amalgamation:thm}
    We first apply \cref{infix-amalgamation-bounded:lem},
    and conclude with \cref{bounded-language:thm}.
\end{proofof}

For the next result, we impose a mild restriction by requiring the language class to be closed under rational transductions. This property holds for most common language classes, including all known \kl{amalgamation systems}.

\begin{theorem}
	\label{infix-wqo-is-emptiness:thm}
	Let $\mathcal{C}$ be a class of languages recognized by amalgamation systems and closed under rational transductions. Then the following are equivalent:
	
	\begin{enumerate}
		\item\label{wqo-infix-decidable} \kl[well-quasi-order]{Well-quasi-orderedness} of the \kl{infix relation} is decidable.
		\item\label{emptiness-decidable} Emptiness is decidable.
	\end{enumerate}
\end{theorem}

\begin{proof}
	\cref{emptiness-decidable} $\Rightarrow$ \cref{wqo-infix-decidable}. We aim to make the inclusion test of \cref{amalgamation-infix-wqo-inclusion:eqn} effective. 
	
	Let $R = \bigcup_{x,y \in \Sigma^{\leq n}} \bigcup_{u \in \Sigma^{\leq m \times N_0}} \InfPeriodChain{x} u \InfPeriodChain{y} \cup \InfPeriodChain{x}u \cup u\InfPeriodChain{x}$. For any concrete values of the bounds $n$, $m$, and $N_0$, this language is regular. As $\mathcal{C}$ is closed under rational transductions, we can therefore reduce the inclusion to emptiness of $L \cap \overline{R}$. However, we need to find these bounds first.
	
	To determine values for $n$ and $m$, we first test if $L$ is bounded. As the algorithm in \cite[Section 4.2]{ASZZ24} is effective, this yields words $w_1, \ldots w_n$ such that $L \subseteq w_1^* \cdots w_n^*$ and therefore yields also the bounds in questions.
	
	To determine the value for $N_0$, we first computer maximal subsets $S \subseteq [n]$ such that $w_i$ for $i \in S$ are simultaneously unbounded. We do this by applying a transduction mapping $w_i$ to some fresh letter $a_i$ if $i \in S$ and $\varepsilon$ otherwise, and testing for simultaneous unboundedness \cite[Section 4.1]{ASZZ24}. Given a candidate bound $n_0$ for $N_0$, we construct the languages $L_{S,n_0} = w_1^{\circ_1}\cdots w_n^{\circ_n}$ where $\circ_i = *$ if $i \in S$ and $\circ_i = \leq n_0$ otherwise. We then test if $L \subseteq \bigcup_{S \text{ maximal}} L_{S,n_0}$. If yes, $n_0$ is our value for $N_0$, otherwise, we increase it and repeat the construction. As we chose maximal sets $S$, this procedure will eventually terminate.
	
	\cref{wqo-infix-decidable} $\Rightarrow$ \cref{emptiness-decidable}. We 
	consider the transduction $\mathcal T = \Sigma^* \times \{a, b\}$. Then 
	for any language $L \in \mathcal C$, $\mathcal TL$ is well-quasi-ordered 
	by infix if and only if $L$ is empty.
\end{proof}

Let us conclude this section by fully characterizing what are
actually the chains for the \kl{infix} relation, without any 
regularity assumption.

\begin{lemma}
    \label{chains-infix:lem}
    Let $L$ be a \kl{chain} for the \kl{infix relation}. Then, there exists
    a \kl{chain} $P$ for the \kl{prefix relation} and a \kl{chain}
    $S$ for the \kl{suffix relation} such that $L \subseteq SP$.
\end{lemma}
\begin{proof}
    Because $L$ is a chain,
    we have an enumeration $L = \seqof{w_n}$,
    where $s_n \defined \card{w_n}$ is increasing, 
    and functions
    $f_{i,j} \colon \set{1, \dots, s_i} \to \set{1, \dots s_{j}}$
    satisfying
    that $f_{j,k} \circ f_{i, j} = f_{i, k}$,
    $f_{i,i} = \mathrm{id}$, and
    $w_i = w_j[f_{i,j}(1):f_{i,j}(s_i)]$ for all $i \leq j$.
    Now, we can build the infinite word $w_\infty$ indexed by $\Rel$, and
    embeddings $f_{i, \infty} \colon \set{1, \dots, s_i} \to \Rel$ such that
    $w_i = w_\infty[f_{i, \infty}(1):f_{i, \infty}(s_i)]$, and $f_{i, \infty}
    \circ f_{j, i} = f_{j, \infty}$ for all $i \leq j$.

    Let us define $w_l$ as the (infinite) word $w_\infty[-\infty: f_{0,
    \infty}(s_0)]$ and $w_r$ as the (infinite) word $w_\infty[f_{0,
    \infty}(s_0): +\infty]$. It is now clear that every word $w_i$ can be
    written as $w_i = w_{i,l} w_{i,r}$, where $w_{i,l}$ is a prefix of $w_r$
    and $w_{i,r}$ is a suffix of $w_l$. Furthermore, the functions $f_{i,
    \infty}$ can be used to notice that $\seqof{w_{i,r}}$ is a \kl{chain} for
    the \kl{prefix relation}, and $\seqof{w_{i,l}}$ is a \kl{chain} for the
    \kl{suffix relation}, which allows us to conclude.
\end{proof}




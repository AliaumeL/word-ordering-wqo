% lang: en-US

\section{Infixes}
\label{infixes:sec}


In this section, we study the well-quasi-ordering of languages under the infix
relation.

\begin{enumerate}
    \item examples $a^nb$, $a b^n a$
    \item no tree of infixes is available
\end{enumerate}

\subsection{Lower bound}

Approximative theorem.
\begin{theorem}
    Let $(X, \preceq)$ be a quasi-ordered set.
    Then the following are equivalent
    \begin{itemize}
        \item $X$ embeds into $(\Sigma^*, \preceq_{\text{infix}})$
        \item $X$ is countable, and for every $x \in X$,
            $\{y \in X \mid y \preceq x\}$ is finite.
    \end{itemize}

    Furthermore, if $X$ is recursively enumerable and $\preceq$ decidable,
    then
    the embedding is effective.
\end{theorem}


\subsection{Upper bound}

This section will use so-called \emph{amalgamation systems}.

\begin{itemize}
    \item Define amalgamation systems
    \item Pumping argument
    \item Characterization
    \item Decision procedure
\end{itemize}

\subsection{Monoid equations?}

\begin{itemize}
    \item What are the finite monoids $(M, \cdot)$ such that
    every language recognized by $M$ is well-quasi-ordered by the infix relation?
    \item What are the equations on $M$ and $P \subseteq M$ characterizing 
    the well-quasi-ordering of $L(M, P)$?
\end{itemize}


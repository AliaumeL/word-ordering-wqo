% LTeX: language=en-GB 
\section{Infixes and Regular Languages}
\label{infixes-regular:sec}

\AP
In this section, we study languages equipped with the \kl{infix relation}. As
opposed to the \kl{prefix} and \kl{suffix} relations, the \kl{infix relation}
can yield to very complicated \kl{well-quasi-ordered} languages. Formally, the
upcoming \cref{infix-embedding:thm} shows that \emph{any} countable
partial-ordering with finite initial segments can be embedded into the infix
relation of a language. To make the former statement precise,
Let us recall that an \intro{order embedding} from a quasi-ordered set $(X,
\preceq)$ into a quasi-ordered set $(Y, \preceq')$ is a function $f \colon X
\to Y$ such that for all $x, y \in X$, $x \preceq y$ if and only if $f(x)
\preceq' f(y)$. When such an embedding exists, we say that $X$ \reintro{embeds
into} $Y$. We say that a quasi-ordered set $(X, \preceq)$ is a \intro{partial
ordering} whenever the relation $\preceq$ is antisymmetric, that is $x \preceq
y$ and $y \preceq x$ implies $x = y$. 

\begin{theorem}
    \label{infix-embedding:thm}
    Let $(X, \preceq)$ be a \kl{partially ordered} set,
    and $\Sigma \defined \set{a,b,c}$.
    Then the following are equivalent:
    \begin{enumerate}
        \item \label{infix-embedding-embeds:item} 
            $X$ \kl{embeds into} $(\Sigma^*, \infleq)$,
        \item \label{infix-embedding-count:item}
            $X$ is countable, and for every $x \in X$,
            its \kl{downwards closure}
            $\dwset[\preceq]{x}$ is finite.
    \end{enumerate}
\end{theorem}
\begin{proof}
    Let us first prove that \cref{infix-embedding-embeds:item} implies
    \cref{infix-embedding-count:item}. Let $f \colon X \to \Sigma^*$ be an
    \kl{embedding}. Note that $f$ is injective, because $X$ and $\Sigma^*$
    are \kl{partially
    ordered}. In particular, because $\dwset[\infleq]{w}$ is finite
    for every $w \in \Sigma^*$, then so is $\dwset[\preceq]{x}$
    for every $x \in X$. Similarly, $X$ must be countable
    because $\Sigma^*$ is.

    Conversely, let us prove that \cref{infix-embedding-count:item} implies
    \cref{infix-embedding-embeds:item}. To that end, let us first define
    inductively a sequence $\seqof{X_n}$ of subsets of $X$ as follows: $X_0$ is
    the set of minimal elements of $X$, and $X_{n+1}$ is the set of minimal
    elements of $X \setminus \bigcup_{i \leq n} X_i$. Because initial segments
    of $X$ are finite, we know that $X = \bigcup_{n \in \Nat} X_n$. To
    simplify notations, we will also use $Y_n \defined \bigcup_{i \leq n} X_i$.
    We also let $L_0 \defined a b^* a$, and 
    $L_{n+1} \defined L_n \cup c^{n+1} b^* c^{n+1} (L_n c^{n+1})^*$.
    Now, let us inductively define \kl{embeddings} $f_n \colon (Y_n,\preceq) \to
    (L_n, \infleq)$.

    For the base case, let $\seqof{x_i}[i \in \Nat]$ be an enumeration of
    $X_0$, which is possible because $X$ is countable. We define $f_0 \colon
    X_0 \to \Sigma^*$ as follows: $f_0(x_i) \defined a b^i a$. It is immediate
    that $f_0$ is an \kl{embedding}. For the inductive step, let $\seqof{x_i}[i
    \in \Nat]$ be an enumeration of $X_{n+1}$, and assume that $f_n$ is an
    \kl{embedding} from $Y_n$ to $\Sigma^*$.
    Let us define $D_i \defined \setof{ y \in Y_n }{ y \preceq x_i }$
    for $i \in \Nat$, all of which are finite by assumption.
    We then let for all $i \in \Nat$:
    \begin{equation*}
        f_{n+1}(x_i) \defined 
        \underbrace{c^{n+1} b^i c^{n+1} \prod_{y \in D_i} \overbrace{f_n(y)}^{\in L_n} c^{n+1}}_{\in L_{n+1}}
        \quad .
    \end{equation*}
    And let $f_{n+1}(y) \defined f_n(y)$ for all $y \in Y_n$.

    To check that $f_{n+1}$ is an \kl{embedding}, let us first notice that
    words $f_{n+1}(x_i)$ and $f_{n+1}(x_j)$ are incomparable for the \kl{infix
    relation} when $i \neq j$, because the only factor of the form $c^{n+1}
    a^{n+1} b^t a^{n+1} c^{n+1}$ appears at the beginning of these words, and
    precisely encode $i$ (resp. $j$) in the value of $t$. Now, let us consider
    $y \in Y_n$ and $x_i \in X_{n+1}$ such that $y \preceq x_i$. Because $y \in
    D_i$, we have $f_n(y) \infleq f_{n+1}(x_i)$ by construction, and since
    $f_n(y) = f_{n+1}(y)$, we conclude that $f_{n+1}(y) \infleq f_{n+1}(x_i)$.
    Conversely, assume that $f_{n+1}(y) \infleq f_{n+1}(x_i)$. Again, this
    means that $f_n(y) \infleq f_{n+1}(x_i)$. Now, the encoding is robust
    enough so that we can conclude $f_{n}(y) \infleq f_{n}(z)$ for some $z \in
    D_i$ By induction hypothesis, $y \preceq z$, hence, $y \preceq x_i$.

    We conclude by noticing that $\bigcup_{n \in \Nat} f_n$ is an \kl{embedding}
    of $X$ into $\Sigma^*$.
\end{proof}

As a consequence of \cref{infix-embedding:thm}, we cannot replay proofs of
\cref{prefixes:sec}, and will actually need to leverage some regularity of the
languages to obtain a characterization of \kl{well-quasi-ordered} languages
under the \kl{infix relation}. Let us first play this game for languages that
are recognized by finite automata. We assume that the reader is familiar with
the notions of deterministic finite automaton and regular languages, and refer
to the book of Thomas for a comprehensive introduction to the subject
\cite{THOM97}. The main goal of the remainder of this section is to prove the
following \cref{infix-finite-automata:thm}.

\begin{theorem}[restate=infix-finite-automata:thm,label=infix-finite-automata:thm]
    Let $L \subseteq \Sigma^*$ be a language recognized by a finite automaton.
    Then $L$ is \kl{well-quasi-ordered} by the \kl{infix relation} if and only if $L$ is
    a finite union of \kl{chains} for the \kl{infix relation}.
\end{theorem}

\AP In order to prove \cref{infix-finite-automata:thm}, we will perform some
preliminary analysis on the structure of an automaton recognizing a
well-quasi-ordered language under the infix relation, which will be powered by
folklore results on \emph{periodic} words. Let us recall that a non-empty word
$w \in \Sigma^+$ is \intro(word){periodic} with period $x \in \Sigma^*$ if
there exists a $p \in \Nat$ such that $w \infleq x^p$. The \intro{periodic
length} of a word $u$ is the minimal length of a word $x$ such that $u$ is an
\kl{infix} of $x^p$ for some $p \in \Nat$ and $x \in \Sigma^+$. We will
essentially rely on the following result on periodic words.

\begin{lemma}
    \label{periodic-infixes:lem}
    Let $u,v \in \Sigma^*$ be two (non-empty) \kl{periodic words}
    having \kl{periodic lengths} $p$ and $q$ respectively.
    Then, if $u \infleq v$ and $\card{u} \geq \factorial[p]{p \times q}$,
    then $u$ and $v$ share the same \kl{periodic length}
    $p = q$.
\end{lemma}
\begin{proof}
    The fact that $u$ and $v$ are \kl{periodic length}
    respectively $p$ and $q$ translates into the fact that $u_{i+p} = u_i$ and
    $v_{i+q} = v_i$ for all indices $i \in \Nat$ such that those letters are
    well-defined.

    Now, assume that $u$ is an \kl{infix} of $v$, this provides the existence
    of a $k \in \Nat$ such that $u = v_{k} \cdots v_{k + \card{u} - 1}$. In
    particular, $v_{k+i+p} = v_{k+i}$ for all $i \in \Nat$ such that $k+i+p < k
    + \card{u}$. Since we also have $v_{k+i+q} = v_{k+i}$ for all $1 \leq i
    \leq \card{v} - k - q$. We conclude that both $u$ and $v$ are of
    \kl{periodic length} the greatest common divisor of $p$ and $q$, and by
    minimality of $q$ this must be equal to $q$ and to $p$.
\end{proof}

The upcoming \cref{powers-infixes:cor} is based on the observation that if $S$
is a \kl{chain} for the \kl{suffix relation} and $P$ is a \kl{chain} for the
\kl{prefix relation}, then $SP$ is a \kl{chain} for the \kl{infix relation}.

\begin{corollary}
    \label{powers-infixes:cor}
    Let $u,v \in \Sigma^*$ and $k, \ell \in \Nat$
    such that $k \geq \factorial[p]{\card{u} \times \card{v}}$,
    $\ell \geq \factorial[p]{\card{v} \times \card{u}}$,
    and $u^k \infleq v^\ell$.
    Then, there exists $w \in \Sigma^*$ of size at most
    $\min \set{\card{u}, \card{v}}$ and a $p \in \Nat$
    such that
    $u^k \infleq v^\ell \infleq w^p$.
\end{corollary}

The reason why \kl{periodic words} built using a given period $x \in \Sigma^+$
are interesting for the \kl{infix relation} is that they naturally create
\kl{chains}. Indeed, if $x \in \Sigma^+$ is a finite word, then $\setof{x^p}{p
\in \Nat}$ is a \kl{chain} for the \kl{infix relation}. Note that in general,
the \kl{downwards closure} of a chain is \emph{not} a chain. However, for the
chains generated using periodic words, the \kl{downwards closure}
$\dwset[\infleq]{\setof{x^p}{p \in \Nat}}$ is a \emph{finite union} of
\kl{chains}. Because this set will appear in bigger equations, we introduce the
shorter notation $\intro*\InfPeriodChain{x}$ for the set of \kl{infixes} of
words of the form $x^p$, where $p$ ranges over $\Nat$.

\begin{lemma}
    \label{inf-period-chain:lem}
    Let $x \in \Sigma^+$ be a word, and
    Then $\InfPeriodChain{x}$ is a finite union of \kl{chains}
    for the \kl{infix}, \kl{prefix} and \kl{suffix} relations 
    simultaneously.
\end{lemma}
\begin{proof}
    Let $x \in \Sigma^+$ be a word, and let $P_x$ be the (finite) set 
    of all \kl{prefixes} of $x$, and $S_x$ be the (finite)
    set of all \kl{suffixes} of $x$.
    Assume that $w \in \InfPeriodChain{x}$, then $w = u x^p v$ for some
    $u \in S_x$, $v \in P_x$, and $p \in \Nat$.
    We have proven that
    \begin{equation*}
        \InfPeriodChain{x} \subseteq \bigcup_{u \in P_x} \bigcup_{v \in S_x} u x^* v
        \quad .
    \end{equation*}

    Let us now demonstrate that for all $(u,v) \in S_x \times P_x$, the
    language $u x^* v$ is a \kl{chain} for the \kl{infix}, \kl{suffix} and \kl{prefix} relations.
    To that end,
    let $(u,v) \in S_x \times P_x$ and $\ell, k \in \Nat$ be such that $\ell <
    k$, let us prove that $u x^\ell v \infleq u x^k  v$. Because $v \prefleq
    x$, we know that there exists $w$ such that $vw = x$. In particular,
    $ux^\ell vw = u x^{\ell + 1}$, and because $\ell < k$, we conclude that $u
    x^{\ell + 1} \prefleq u x^k v$. By transitivity, $u x^\ell v \prefleq u x^k
    v$, and \emph{a fortiori}, $u x^\ell v \infleq u x^k v$. 
    Similarly, because $u \suffleq x$,  there exists $w$ such that $wu  = x$, 
    and we conclude that $u x^{\ell} v \suffleq w u x^\ell v = x^{\ell + 1} v \suffleq u x^k v$.
    \qedhere
\end{proof}


\begin{corollary}
    \label{inf-period-union-chains:lem}
    Let $x,u,y \in \Sigma^*$ be words.  The following 
    are finite unions of \kl{chains} for the infix relation:
    $\InfPeriodChain{x}u$, $u \InfPeriodChain{x}$,
    and $\InfPeriodChain{x} u \InfPeriodChain{y}$.
\end{corollary}
\begin{proof}
    Because $\InfPeriodChain{x}$ is a finite union of \kl{chains} for the \kl{suffix}
    relation (rsing \cref{inf-period-chain:lem}), we conclude that $\InfPeriodChain{x}u$ is one too for
    the \kl{infix} relation. This holds similarly for $u \InfPeriodChain{x}$.
    In the case of $\InfPeriodChain{x} u \InfPeriodChain{y}$,
    we remark that $\InfPeriodChain{x} u$ is a finite union of chains
    for the \kl{suffix relation},
    and that $\InfPeriodChain{y}$ is one for the \kl{prefix relation}
    (using again \cref{inf-period-chain:lem}).
    As a consequence,
    $\InfPeriodChain{x}u \InfPeriodChain{y}$ is a finite union 
    of \kl{chains} for the \kl{infix} relation.
\end{proof}

The following combinatorial lemma connects the property of being
\kl{well-quasi-ordered} to a property of the \kl{periodic lengths} of words in
a language, based on the assumption that some factors can be iterated. It is
the core result that powers the analysis done in the upcoming
\cref{infix-finite-automata:thm,infix-amalgamation:thm}.

\begin{lemma}
    \label{pumping-periods:lem}
    Let $L \subseteq \Sigma^*$ be a language
    that is \kl{well-quasi-ordered} by the \kl{infix relation}.
    Let $k \in \Nat$, $u_1, \cdots, u_{k+1} \in \Sigma^*$,
    and $v_1, \cdots, v_{k} \in \Sigma^+$
    be such that
    $w[\vec{n}] \defined (\prod_{i = 1}^k u_i v_i^{n_i}) u_{k+1}$
    belongs to $L$
    for arbitrarily large values of $\vec{n} \in \Nat^k$.
    Then, 
    there exists $x,y \in \Sigma^+$ of size 
    at most $\max \setof{\card{v_i}}{1 \leq i \leq k}$
    such that 
    one of the following holds for all
    $\vec{n} \in \Nat^{k}$:
    \begin{enumerate}
        \item $w[\vec{n}] \in u_1 \InfPeriodChain{x}$,
        \item $w[\vec{n}] \in \InfPeriodChain{x} u_{k+1}$,
        \item $w[\vec{n}] \in \InfPeriodChain{x} u_i \InfPeriodChain{y}$
            for some $1 \leq i \leq k + 1$.

    \end{enumerate}
\end{lemma}
\begin{proof}
    Note that the result is obvious if $k = 0$, and therefore
    we assume $k \geq 1$ in the following proof.

    Let us construct a sequence of words $\seqof{w_i}[i \in \Nat]$, where $w_i
    \defined w[\vec{n_i}]$ for some well-chosen indices $\vec{n_i} \in \Nat^k$. The goal
    being that 
    if $w[\vec{n_i}]$ is an \kl{infix} of $w[\vec{n_j}]$,
    then it can intersect at most \emph{two} iterated words,
    with an intersection that is long enough to successfully apply
    \cref{periodic-infixes:lem}.
    In order to achieve this,
    let us first define $s$ as the maximal size of a word $v_i$
    ($1 \leq i \leq k$) and $u_j$ ($1 \leq j \leq k+1$).
    Then,
    we consider $\vec{n_0} \in \Nat^k$ such that $\vec{n_0}$ has all 
    its components greater than $\factorial{s}$ and such that
    $w[\vec{n_0}]$ belongs to $L$. 
    Then, we inductively define 
    $\vec{n_{i+1}}$  as the smallest vector of numbers greater than $\vec{n_i}$,
    such that $w[\vec{n_{i+1}}]$ belongs to $L$, 
    and with $\vec{n_i}$ having all components greater than
    $2\card{w[\vec{n_i}]}$.


    Let us assume that $k \geq 2$ in the following proof for symmetry purposes,
    and argue later on that when $k = 1$ the same argument goes through.
    Because $L$ is \kl{well-quasi-ordered} by the \kl{infix relation}, there
    exists $i < j$ such that $w[\vec{n_i}]$ is an \kl{infix} of $w[\vec{n_j}]$.
    Now, because of the chosen values for $\vec{n_j}$, there exists $1 \leq \ell \leq
    k-1$ such that $w[\vec{n_i}]$ is actually an \kl{infix} of $u_{\ell}
    v_{\ell}^{n_{j,\ell}} u_{\ell+1} v_{\ell+1}^{n_{j,\ell+1}} u_{\ell+2}$.
    Even more,
    one of the three following equations holds:
    \begin{itemize}
        \item $w[\vec{n_i}] \infleq v_{\ell}^{n_{j,\ell}} u_{\ell+1} v_{\ell+1}^{n_{j,\ell+1}}$,
        \item $w[\vec{n_i}] \infleq u_{\ell}
            v_{\ell}^{n_{j,\ell}}$,
        \item $w[\vec{n_i}] \infleq
            v_{\ell+1}^{n_{j,\ell+1}} u_{\ell+2}$.
    \end{itemize}
    In all those cases, we conclude using \cref{powers-infixes:cor}
    that there exists $x,y \in \Sigma^+$ of size at most $s$, and 
    a number $1 \leq t \leq k$ such that
    $v_i^{n_i} \in \InfPeriodChain{x}$ for all $1 \leq i \leq t$,
    and
    $v_i^{n_i} \in \InfPeriodChain{y}$ for all $t < i \leq k$.
    In particular,
    $w[\vec{n_i}] \in \InfPeriodChain{x} u_{t} \InfPeriodChain{y}$.

    
    When $k = 1$, the situation is a bit more specific since we only have two
    cases: either $w_i \infleq u_1 v_1^{n_j}$ or $w_i \infleq v_1^{n_j} u_2$,
    and we conclude with an identical reasoning.
\end{proof}


We are now ready to restate and prove our main theorem, proven by combining
\cref{pumping-periods:lem} together with classical pumping arguments on finite
state automata.

\begin{proofof}{infix-finite-automata:thm}[main]
    Let $w \in L$, because $Q$ is finite, there exists
    a factorization of $w$
    into words $(\prod_{i = 1}^k u_i v_i) u_{k+1}$
    such that 
    for all $1 \leq i \leq k+1$, $\card{u_i} \leq \card{Q}$,
    for all $1 \leq i \leq k$, $1 \leq \card{v_i} \leq \card{Q}$,
    and satisfying 
    that $w[\vec{X}] \defined 
    (\prod_{i = 1}^k u_i v_i^{X_i}) u_{k+1}$
    belongs to $L$ for all choices of values $\vec{X} \in \Nat^k$.

    Applying \cref{pumping-periods:lem}, we conclude that 
    there exists $x,y \in \Sigma^+$ of size at most $\card{Q}$
    such that 
    $w \in u_1 \InfPeriodChain{x} \cup \InfPeriodChain{y} u_{k+1}
    \cup \bigcup_{1 \leq i \leq k+1} \InfPeriodChain{x} u_i \InfPeriodChain{y}$.
    In particular, we conclude that
    \begin{equation}
        \label{infix-automata:eq}
        L
        \subseteq
        \bigcup_{x,y,u\in \Sigma^{\leq \card{Q}}}
        u \InfPeriodChain{x}
        \cup 
        \InfPeriodChain{x} u
        \cup
        \InfPeriodChain{x} u \InfPeriodChain{y}
        \quad .
    \end{equation}

    Now, thanks to \cref{inf-period-union-chains:lem},
    this happens to be a finite union of \kl{chains}
    for the \kl{infix relation}.
\end{proofof}



\begin{corollary}
    Given a regular language $L$, it is decidable whether
    $L$ is \kl{well-quasi-ordered} for the \kl{infix relation}.
\end{corollary}
\begin{proof}
    It suffices to decide whether 
    \cref{infix-automata:eq} holds for the language $L$.
    This is an inclusion of regular languages, which is decidable.
\end{proof}


Let $(X,\preceq)$ be a quasi-ordered set, and $L \subseteq X$ be such that $(L,
\preceq)$ is \kl{well-quasi-ordered}. It is not true in general that
$(\dwset{L}, \preceq)$ is \kl{well-quasi-ordered}. In the case of $(\Sigma^*,
\infleq)$ a typical example is to start from an infinite \kl{antichain} $A$,
together with an enumeration $\seqof{w_i}$ of $A$, and build the language $L
\defined \setof{ \prod_{i = 0}^n w_i }{ i \in \Nat }$. By definition, $L$ is a
\kl{chain} for the \kl{infix} ordering, hence \kl{well-quasi-ordered}. However,
$\dwset[\infleq]{L}$ contains $A$, and is therefore not
\kl{well-quasi-ordered}. As a fun corollary of our analysis, we conclude that
this particular example cannot be realized by a regular language $A$.


\begin{corollary}
    Given a regular language $L$, it is \kl{well-quasi-ordered} 
    for the \kl{infix relation} if and only if 
    $\dwset[\infleq] L$ is.
\end{corollary}

\section{Infixes and Amalgamation Systems}
\label{infixes-amalgamation:sec}

When proving \cref{infix-finite-automata:thm}, we have leveraged a powerful
combinatorial argument from \cref{pumping-periods:lem}. It turns out that there
is a rather large family of systems for which pumping arguments based on
so-called \emph{minimal runs} exist: they are called \emph{amalgamation
systems} and were recently introduced by Anad, Schmitz, Sch\"{u}tze, and
Zetzsche \cite{ASZZ24}. Having done the heavy lifting on finite automata, the
rest of this section is mostly devoted to the introduction of \kl{amalgamation
systems} and collecting the necessary pumping argument they enjoy. Using this
meta-proof, we will generalize \cref{infix-finite-automata:thm} to context-free
grammars, languages recognized by vector addition systems with state (VASS),
and more. The goal of this section is not to introduce all of these systems, or
justify their usefulness, but to state and prove the following theorem.

\begin{theorem}[label=infix-amalgamation:thm,restate=infix-amalgamation:thm]
    Let $L \subseteq \Sigma^*$ be a language recognized by an 
    \kl{amalgamation system}.
    Then $L$ is well-quasi-ordered by the infix relation if and only if $L$ is
    a finite union of chains for the \kl{infix relation}.
\end{theorem}

Let us now formally introduce the notion of \kl{amalgamation systems}, and
recall some results from \cite{ASZZ24} that will be useful for the proof of
\cref{infix-amalgamation:thm}. The notion of \kl{amalgamation system} is
tailored to produce \emph{pumping arguments}, which is exactly what our
\cref{pumping-periods:lem} talks about. At the core of a pumping argument,
there is a notion of \emph{run}, which could for instance be a sequence of
transitions taken in a finite state automaton. Continuing on the analogy with
finite automata, there is a natural ordering between runs, i.e., a run is
smaller than another one if one can ``unroll loops" of the first run to obtain
the second one. Typical pumping arguments then rely on the fact that
\emph{minimal} runs are of finite size, and that all other runs are
obtained by ``gluing" minimal runs together. This is exactly what
\kl{amalgamation systems} are about.

\AP Let us recall that over an alphabet $(\Sigma, =)$ a \kl{subword embedding}
between two words $u \in \Sigma^*$ and $v \in \Sigma^*$ is a function $\rho
\colon \range{\card{u}} \to \range{\card{v}}$ such that $u_i = v_{\rho(i)}$ for
all $i \in \range{\card{u}}$. We write $\intro*\HigEmb(u,v)$ the set of all
\kl{subword embeddings} between $u$ and $v$. It may be useful to notice that
the set of finite words over $\Sigma$ forms a category when we consider
\kl{subword embeddings} as morphisms, which is a fancy way to state that
$\mathrm{id} \in \HigEmb(u,u)$ and that $f \circ g \in \HigEmb(u,w)$ whenever
$g \in \HigEmb(u,v)$ and $f \in \HigEmb(v,w)$, for any choice of words
$u,v,w \in \Sigma^*$.

\AP Given a \kl{subword embedding} $f \colon u \to v$ between two words $u$ and
$v$, there exists a unique decomposition $v = \GapWord{f}{0} u_1 \GapWord{f}{1}
\cdots \GapWord{f}{k-1} u_k \GapWord{f}{k}$ where $\GapWord{f}{i} =
v[f(i)+1:f(i+1)-1]$ for all $1 \leq i \leq k-1$, $\GapWord{f}{k} =
v[f(k)+1:\card{v}]$, and $\GapWord{f}{0}   = v[1: f(1)-1]$. We say that
$\intro*\GapWord{f}{i}$ is the $i$-th \intro{gap word} of $f$. These \kl{gap
words} will be useful to describe how and where runs of a system (described by
words) can be combined.


\begin{definition}
    An \intro{amalgamation system}
    is a triple $(\Sigma, R, E, \canrun)$ where
    $\Sigma$ is a finite alphabet,
    $R$ is a set of so-called \emph{runs},
    and 
    $E$ is a set of so-called \emph{run embeddings},
    and $\canrun \colon R \to (\Sigma \uplus \set{\varepsilon})^*$ is a 
    function computing a \emph{canonical decomposition} of a run.
    Given a run $r \in R$, and $i \in \range[0]{\card{\canrun(r)}}$, 
    the \intro{gap language} of $r$ at position $i$ is $\GapLanguage{r}{i} \defined
    \setof{\GapWord{f}{i}}{\exists s \in R. \exists f \in E(r,s) }$.
    An \kl{amalgamation system} furthermore satisfies the following 
    properties:
    \begin{enumerate}
        \item \emph{Word Embeddings as Morphisms.} For all $r, s \in R$,
            $E(r,s) \subseteq \HigEmb(\canrun(r), \canrun(s))$.
        \item \emph{$(R, E)$ Forms a Category.}
            For all $r,s,t \in R$,
            $\mathrm{id} \in E(r,r)$,
            and whenever $f \in E(r,s)$ and $g \in E(s,t)$,
            then $g \circ f \in E(r,t)$.
        \item \emph{Well-Quasi-Ordered System.}
            $(R, \leq_E)$ is a well-quasi-ordered set,
            where $r \leq_E s$ if and only if $E(r,s) \neq \emptyset$.
        \item \emph{Controlled Amalgamation.}
            For all $r, s, t \in R$,
            for all $f \in E(r,s)$,
            for all $g \in E(r,t)$,
            and for all $0 \leq i_0 \leq \card{\canrun(r)}$,
            there exists a run $z \in R$ and morphisms
            $h \in E(s,z)$ and $k \in E(t,z)$
            satisfying
            $h \circ f = k \circ g$,
            $\GapLanguage{h \circ f}{i} = \GapLanguage{f}{i} \GapLanguage{g}{i}$
            or 
            $\GapLanguage{h \circ f}{i} = \GapLanguage{g}{i} \GapLanguage{f}{i}$
            for every $0 \leq j \leq n$,
            and
            $\GapLanguage{h \circ f}{i_0} = \GapLanguage{f}{i_0} \GapLanguage{g}{i_0}$.
    \end{enumerate}

    The \intro{amalgamation language} of such a system
    is the set of all words $w \in \Sigma^*$ such that
    there exists a run $r \in R$
    such that the concatenation of the letters of its
    canonical decomposition, written $\intro*\yieldrun(r)$,
    equals $w$.
\end{definition}

Intuitively, the definition of an amalgamation system allows the comparison of
runs, and the proper ``gluing" of runs together to obtain new runs. Let us
recall some examples of languages that can be recognized by \kl{amalgamation
systems} given by the authors of the original paper: regular languages
\cite[Theorem 5.3]{ASZZ24}, VASS as a consequence of \cite[Theorem
5.5]{ASZZ24}, context-free languages as a consequence of \cite[Theorem
5.10]{ASZZ24}. For this paper to be self-contained, we will also recall how
runs of a finite state automaton can be understood as an \kl{amalgamation
system}.

\begin{example}
    Let $A = (Q, \delta, q_0, F)$ be a finite state automaton over a finite
    alphabet $\Sigma$. Let $\Delta$ be the set of transitions $(q_1, a, q_2)
    \in Q \times \Sigma \times Q$,
    and $R \subseteq \Delta^*$ be the set of 
    words over transitions that start with the initial state $q_0$,
    end in a final state $q_f \in F$, and such that the end state of a
    letter is the start state of the following one.

    The canonical decomposition $\canrun \colon R \to \Sigma^*$
    is defined as a morphism from $\Delta^*$ to $\Sigma^*$
    that maps $(q,a,p)$ to $a$.

    Finally, given two runs $r$ and $s$ of the automaton,
    let $E(r,s) = \HigEmb(\canrun(r), \canrun(s))$, that is, we do not
    remove morphisms.

    The system $(\Sigma, R, E, \canrun)$ is an \kl{amalgamation system},
    whose language is precisely the language of words recognized
    by the automaton $A$.
\end{example}
\begin{proof}
    TODO.
\end{proof}

\begin{proofof}{infix-amalgamation:thm}
    Assume that $L$ is \kl{well-quasi-ordered} by the \kl{infix relation},
    and obtained by an \kl{amalgamation system}
    $(\Sigma, R, E, \canrun)$.

    Let us consider the set $M$ of minimal runs for the relation $\leq_E$,
    which is finite because the latter is a \kl{well-quasi-ordering}. By
    definition, for every word $w \in L$, there exists $r \in M$, $s \in R$,
    and $f \in E(r,s)$ such that $w = \canrun(s)$.
    In particular, we conclude that
    \begin{equation*}
        L \subseteq \bigcup_{r \in M} 
        \yieldrun \left(
        \GapLanguage{r}{0}
        \prod_{i = 1}^{\card{\canrun(r)}-1} \canrun(r)_i \GapLanguage{r}{i} 
        \right)
        \quad .
    \end{equation*}

    Let $u$ and $v$ be two words in the \kl{gap language}
    $\GapLanguage{r}{\ell}$ for some $r \in M$ and $0 \leq \ell \leq
    \card{\canrun(r)}$. We have corresponding runs $s_1, s_2 \in R$ and
    embeddings $f_1 \in E(r,s_1)$ and $f_2 \in E(r,s_2)$ such that
    $\GapWord{f_1}{\ell} = u$ and $\GapWord{f_2}{\ell} = v$. Using the
    amalgamation property, we can inductively construct runs $r_k \in R$ and
    embeddings $f_k \in E(r,s_k)$ such that $\GapWord{f_k}{\ell} = u v^k u$,
    and $\GapWord{f_k}{i}$ is obtained by concatenating in some order two
    occurrences of $\GapWord{f_1}{i}$ and $k$ occurrences of
    $\GapWord{f_2}{i}$. Because $L$ is \kl{well-quasi-ordered} for the
    \kl{infix relation}, there exists $i < j$ such that $\yieldrun(r_i) \infleq
    \yieldrun(r_j)$. Assuming that $\card{\canrun(r)}$ is greater than $2$, and
    that we have extracted an increasing sequence of indices so that
    $\alpha(i+1) > \factorial{\card{\yieldrun(r_i)}}$, we conclude that $u
    v^{\alpha(i)} u$ is an \kl{infix} of $\yieldrun(r_j)$.
    Because of the number of repetitions, a \emph{large} infix of $v^{\alpha(i)}$ must be
    an infix of some $(\GapWord{f_2}{p})^{\alpha(j)}$.
    We conclude that $(\GapWord{f_2}{p})^{\alpha(j)}$ and $v$
    have a common period. Refining the analysis, we also conclude
    that $uv^{q}$ and $(\GapWord{f_2}{p})^{\alpha(j)}$ have the same period,
    which proves that $uv^{q}u$ belongs to some $\InfPeriodChain{x}$
    \textbf{todo}.

    \begin{quote}
        Idea: when pumping like this, we can always find
        large blocks of $\GapWord{f_2}{p}$, so this looks like
        something suitable for \cref{pumping-periods:lem}.
        This should allow us to conclude that every pair of words
        $u,v$ in a given gap language
        are infixes of \emph{the same} periodic word repeated.
    \end{quote}


    Then, we can leverage \cref{pumping-periods:lem}
    to conclude that $L$ is a finite union of \kl{chains}
    for the \kl{infix relation}.
\end{proofof}

\begin{conjecture}
    Let $L$ be given by an \kl{effective amalgamation system},
    then it is decidable whether $L$ 
    is \kl{well-quasi-ordered} by the \kl{infix relation}.
\end{conjecture}

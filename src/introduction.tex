% LTeX: language=en-GB 
% !TeX root=../wqo-on-words.lncs.tex
\section{Introduction}
\label{introduction:sec}

\AP A \intro{well-quasi-ordered} set is a set $X$ equipped with a quasi-order
$\preceq$ such that every infinite sequence $\seqof{x_n}$ of elements taken in
$X$ contains an increasing pair $x_i \preceq x_j$ with $i < j$. Well-quasi-orderings serve
as a core combinatorial tool powering many termination arguments, and was
successfully applied to the verification of infinite state transition
systems~\cite{ABDU96,ABDU98}. One of the appealing properties of
well-quasi-orderings is that they are closed under many operations, such as
taking products, finite unions, and finite powerset
constructions~\cite{SCSC12}. Perhaps more surprisingly, the class of
well-quasi-ordered sets is also stable under the operation of taking finite
words and finite trees labelled by elements of a well-quasi-ordered set~\cite{HIG52,KRU72}.

\AP Note that in the case of finite words and finite trees, the precise choice
of ordering is crucial to ensure that the resulting structure is
\kl{well-quasi-ordered}. The celebrated result of Higman states that the set of
finite words over an ordered alphabet $(X, \preceq)$ is \kl{well-quasi-ordered}
by the so-called \kl{subword embedding relation}~\cite{HIG52}. Let us recall
that the \kl{subword relation} for words over $(X, \preceq)$ is defined as
follows: a word $u$ is a \intro{subword} of a word $v$, written $u
\intro*\higleq v$, if there exists an increasing function $f \colon \{1,
\ldots, |u|\} \to \{1, \ldots, |v|\}$ such that $u_i \preceq v_{f(i)}$ for all
$i \in \{1, \ldots, |u|\}$.

\AP However, there are many other natural orderings on words that could be
considered in the context of \kl{well-quasi-orderings}, even in the simplified
setting of a finite alphabet $\Sigma$ equipped with the equality relation. In
this setting, the three alternatives we consider are the \intro{prefix
relation} ($u \intro*\prefleq v$ if there exists $w$ with $uw = v$), the
\intro{suffix relation} ($u \intro*\suffleq v$ if there exists $w$ such that
$wu = v$), and the \intro{infix relation} ($u \intro*\infleq v$ if there exists
$w_1,w_2$ such that $w_1 u w_2 = v$). Note that these three relations
straightforwardly generalize to infinite quasi-ordered alphabets.
Unfortunately, it is easy to see that none of these constructions are 
well-quasi-ordered as soon as the alphabet contains two distinct letters:
for instance, the infinite sequence $a b^n a$ is \kl{well-quasi-ordered} by
the \kl{subword relation} but by neither the \kl{prefix relation}, nor the
\kl{suffix relation}, nor the \kl{infix relation}.

\AP While this dooms \kl[well-quasi-ordered]{well-quasi-orderedness} of these
relations in the general case, there may be \emph{subsets} of $\Sigma^*$ which
are \kl{well-quasi-ordered} by these relations. As a simple example, take the
case of finite sets of (finite) words which are all \kl{well-quasi-ordered}
regardless of the ordering considered. This raises the question of
characterizing exactly which subsets $L \subseteq \Sigma^*$ are
\kl{well-quasi-ordered} with respect to the \kl{prefix relation} (respectively,
the \kl{suffix relation} or the \kl{infix relation}), and designing
suitable decision procedures.

\AP Let us argue that these decision procedures fit a larger picture in the
research area of well-quasi-orderings.
Indeed, there has been recent breakthroughs in deciding whether a given order
is a \kl{well-quasi-order}, for instance in the context of the verification of
infinite state transition systems~\cite{DBLP:conf/fsttcs/FinkelG19} or in the
context of logic~\cite{DBLP:journals/pacmpl/BergstrasserGLZ24}.
In the graph theory community, recent works have studied classes of graphs 
that are \kl{well-quasi-ordered} by the induced subgraph relation
using similar language theoretic techniques~\cite{DRT10,lopez25,ALM17}.
Furthermore, a previous work by Kuske shows that any
\emph{reasonable}\footnote{ This will be made precise in
\cref{infix-embedding:thm}. } partially ordered set $(X, \leq)$ can
be embedded into $\set{a,b}^*$ with the \kl{infix relation}~\cite[Lemma
5.1]{DBLP:journals/ita/Kuske06}. Phrased differently, one can encode a large
class of partially ordered sets as subsets of $\set{a,b}^*$. As a consequence,
the following decision problem provides a reasonable abstract framework for
deciding whether a given partially ordered set is \kl{well-quasi-ordered}:
given a language $L \subseteq \Sigma^*$, decide whether $L$ is
\kl{well-quasi-ordered} by the \kl{infix relation}.

\AP The runtime of an algorithm based on \kl{well-quasi-orderings} is deeply
related to the ``complexity'' of the underlying quasi-order~\cite{SCHMITZ17}.
One way to measure this complexity is to consider its so-called \kl{ordinal
invariants}: for instance, the \kl{maximal order type} (or \kl{m.o.t.}),
originally defined by De Jongh and Parikh \cite{dejongh77}, is the order type
of the maximal linearization of a well-quasi-ordered set. In the case of a
finite set, the \kl{m.o.t.} is precisely the size of the set. Better runtime
bounds were obtained by considering two other parameters~\cite{SCHMITZ19}: the
\kl{ordinal height} introduced by Schmidt \cite{schmidt81}, and the \kl{ordinal
width} of Kříž and Thomas \cite{kriz90b}. Therefore, when characterizing
\kl{well-quasi-ordered} languages, we will also be interested in deriving upper
bounds on their \kl{ordinal invariants}. This analysis also allows us to better
compare the \kl{well-quasi-orderings}. We refer to
\cref{ordinal-invariants:subsec} for a more detailed introduction to these
parameters and ordinal computations in general.

\paragraph*{Contributions} We focus on languages over a finite alphabet
$\Sigma$. In this setting, we first characterize languages that are
\kl{well-quasi-ordered} by the \kl{prefix relation} (and symmetrically, by the
\kl{suffix relation}), and derive tight bounds on their \kl{ordinal
invariants}. These generic results are then used to devise a decision procedure
for checking whether a language is \kl{well-quasi-ordered} by the \kl{prefix
relation}, provided the language is given as input as a finite automaton
(\cref{prefix-wqo-reg-decidable:cor}). A
summary of these results can be found in \cref{prefixes-summary:fig}.

\begin{figure}[h]
    \centering
    \setlength{\tabcolsep}{6pt}
    \begin{tabular}{c|p{6cm}|c|c}
        \toprule
        $L$ & \textbf{Characterisation} & $\oWidth{L}$ & $\oType{L}$ \\
        \midrule
        arbitrary & \cref{prefixes:thm}: finite unions of \kl{chains}
                  & $< \omegaOrd$ & $< \omegaOrd^2$ \\
        \addlinespace
        regular   & \cref{prefix-wqo-reg-decidable:cor}: finite unions of \kl(language){regular} \kl{chains} 
                  & $< \omegaOrd$ & $< \omegaOrd^2$ \\
        \bottomrule
    \end{tabular}
    \caption{Summary of results for the \kl{prefix relation} (and symmetrically, for the \kl{suffix relation}).}
    \label{prefixes-summary:fig}
\end{figure}

We then turn our attention to the \kl{infix relation}. In this case, we notice
that Lemma 5.1 from \cite{DBLP:journals/ita/Kuske06} imply that there are
\kl{well-quasi-ordered} languages for the \kl{infix relation} that have
arbitrarily large \kl{ordinal invariants} (except for the \kl{ordinal height},
which is always at most $\omegaOrd$). Therefore, we focus on two natural
semantic restrictions on languages: on the one hand, we consider \kl{bounded
languages}, that is, languages included in some $w_1^* \cdots w_k^*$ for some
finite choice of words $w_1, \ldots, w_k$; on the other hand, we consider
\kl{downwards closed} languages, that is, languages closed under taking
\kl{infixes}. In both cases, we provide a very precise characterization of
\kl{well-quasi-ordered} languages by the \kl{infix relation}, and derive tight
bounds on their \kl{ordinal invariants}. These results are summarized in
\cref{infixes-summary:fig}. We furthermore notice that for \kl{downwards
closed} languages that are \kl{well-quasi-ordered} by the \kl{infix relation},
being \kl(language){bounded} is the same as being \kl(language){regular}
(\cref{dwclosed-infixes-wqo:lem}), and that a \kl{bounded language} is
\kl{well-quasi-ordered} by the \kl{infix relation} if and only if its
\kl{downwards closure} is \kl{well-quasi-ordered} by the \kl{infix relation}
(\cref{bounded-wqo-dwclosed:cor}). This shows that, for \kl{bounded languages},
being \kl{well-quasi-ordered} implies that their \kl{downwards closure} is a
\kl{regular language}, which is a weakening of the usual result that the
\kl{downwards closure} of \emph{any language} for the \kl{scattered subword
relation} is always a \kl{regular language}.


\begin{figure}[h]
  \centering
  \setlength{\tabcolsep}{6pt}
  \begin{tabular}{c|p{6cm}|c|c}
      \toprule
      $L$ & \textbf{Characterisation} & $\oWidth{L}$ & $\oType{L}$ \\
      \midrule
      arbitrary & \cref{infix-embedding:thm}: countable well-quasi orders
      with \kl{finite initial segments} & $< \omegaOne$  & $< \omegaOne$ \\
      \addlinespace
      \kl(language){bounded} & \cref{bounded-language:thm}: 
      finite union of products of \kl{chains} for the
      \kl{prefix} and \kl{suffix} relations
                             & $< \omegaOrd^2$ & $< \omegaOrd^3$ \\
      \addlinespace
      \kl{downwards closed}  & \cref{dw-closed-wqo-charac:thm}: finite union of infixes of \kl{ultimately uniformly recurrent words} & $< \omegaOrd^2$ & $< \omegaOrd^3$ \\
      \bottomrule
    \end{tabular}
    \caption{Summary of results for the \kl{infix relation},
    the bounds on $\oWidth{L}$ and $\oType{L}$ are tight, and respectively 
    proven in
    \cref{ordinal-invariants-bounded:cor}
    and 
  \cref{small-ordinal-invariants:thm}.}
    \label{infixes-summary:fig}
\end{figure}

Turning our attention to decision procedures, we consider two computational
models respectively tailored to \kl{downwards closed} languages and to
\kl{bounded languages}. For \kl{downwards closed} languages, we consider a
model based on representations of infinite words
(\cref{decision-procedures:sec}), for which we provide a decision procedure
(\cref{automatic-wqo:thm}). The model used to represent these infinite words is
based on \kl{automatic sequences} and \kl{morphic sequences} \cite{CAKA97},
which are well-studied in the context of symbolic dynamics. For \kl{bounded
languages}, we consider the model of \kl{amalgamation systems} \cite{ASZZ24},
which is an abstract computational model that encompasses many classical ones,
such as finite automata, context-free grammars, and Petri nets \cite{ASZZ24}.
We show that if a language recognized by an \kl{amalgamation system} is
\kl{well-quasi-ordered} by the \kl{infix relation}, then it is a \kl{bounded
language} (\cref{infix-amalgamation:thm}), and is therefore
\kl(language){regular}. Furthermore, we show that we can decide whether a given
language recognized by an \kl{amalgamation system} is \kl{well-quasi-ordered}
by the \kl{infix relation} (\cref{infix-wqo-is-emptiness:thm}). We defer the
introduction of \kl{amalgamation systems} to
\cref{amalgamation-systems:subsec}.

\paragraph*{Related work} The study of alternative \kl{well-quasi-ordered}
relations over finite words is far from new. For instance, orders obtained by
so-called \emph{derivation relations} where already analysed by Bucher,
Ehrenfeucht, and Haussler \cite{BUEUD85}, and were later extended by
D'Alessandro and Varricchio \cite{ALVA03,ALVA06}. However, in all those cases
the orderings are \emph{multiplicative}, that is, if $u_1 \preceq v_1$ and $u_2
\preceq v_2$ then $u_1u_2 \preceq v_1v_2$. This assumption does not hold for
the \kl{prefix}, \kl{suffix}, and \kl{infix} relations.

A similar question was studied by Atminas, Lozin, and Moshkov \cite{ALM17}, in
the hope of finding characterizations of classes of \emph{finite graphs} that
are \kl{well-quasi-ordered} by the \emph{induced subgraph relation}
\cite[Section 7]{ALM17}. In this setting, it is common to refer to classes of
graphs via a list of \emph{forbidden patterns}, which are finite graphs that
cannot be found as induced subgraphs in the class. Applying this reasoning to
finite words with the \kl{infix relation}, they provide an efficient decision
procedure for checking whether a language $L \subseteq \Sigma^*$ is
\kl{well-quasi-ordered} by the \kl{infix relation} whenever said language is
given as input via a list of \emph{forbidden factors} \cite[Theorem 1, Theorem
2]{ALM17}. The key construction of their paper is to study languages $L$ that
are \emph{regular} (recognized by some finite deterministic automata), for
which they can decide whether $L$ is \kl{well-quasi-ordered} by the \kl{infix
relation} \cite[Theorem 1]{ALM17}. Because it is easy to transform a list of
forbidden factors into a regular language \cite[Theorem 1]{ALM17}, this yields
the desired decision procedure. Our work extends this result in several ways:
first, we also consider the \kl{prefix relation} and the \kl{suffix relation},
then we consider non-regular languages, and finally, we provide very precise
descriptions of the \kl{well-quasi-ordered} languages, as well as tight bounds
on their \kl{ordinal invariants}. 

\paragraph*{Outline} We introduce in \cref{prelims:sec} the necessary
background on \kl{well-quasi-orders} and \kl{ordinal invariants}. In
\cref{prefixes:sec}, which is relatively self-contained, we study the
\kl{prefix relation} and prove in \cref{prefixes:thm} the characterization of
\kl{well-quasi-ordered} languages by the \kl{prefix relation}. In
\cref{infixes-bounded:sec}, we obtain the \kl[infix relation]{infix} analogue
of \cref{prefixes:thm} specifically for \kl{bounded languages}
(\cref{bounded-language:thm}). In \cref{infixes-dwclosed:sec}, we study the
\kl{downwards closed} languages, characterize them using a notion of
\kl{ultimately uniformly recurrent words} borrowed from symbolic dynamics
(\cref{dw-closed-wqo-charac:thm}), and compute bounds on their \kl{ordinal
invariants} in \cref{small-ordinal-invariants:thm}. Finally, we generalize
these results to all \kl{amalgamation systems} in
\cref{infixes-amalgamation:sec} in (\cref{infix-amalgamation:thm}), and provide
a decision procedure for checking whether a language is \kl{well-quasi-ordered}
by the \kl{infix relation} (resp. \kl{prefix} and \kl{suffix}) in this context
(\cref{infix-wqo-is-emptiness:thm}).

\paragraph*{Acknowledgements} We would like to thank participants of the 2024
edition of \kl{Autobóz} for their helpful comments and discussions.
We would also like to thank Vincent Jugé for his pointers on word combinatorics.

% LTeX: language=en-GB 
% !TeX root=../wqo-on-words.tex
\section{Introduction}
\label{introduction:sec}

\AP A \intro{well-quasi-ordered} set is a set $X$ equipped with a quasi-order
$\preceq$ such that every infinite sequence $\seqof{x_n}$ of elements taken in
$X$ contains an increasing pair $x_i \preceq x_j$ with $i < j$. Well-quasi-orderings serve
as a core combinatorial tool powering many termination arguments, and was
successfully applied to the verification of infinite state transition
systems~\cite{ABDU96,ABDU98}. One of the appealing properties of
well-quasi-orderings is that they are closed under many operations, such as
taking products, finite unions, and finite powerset
constructions~\cite{SCSC12}. Perhaps more surprisingly, the class of
well-quasi-ordered sets is also stable under the operation of taking finite
words and finite trees labelled by elements of a well-quasi-ordered set
\cite{HIG52,KRU72}.

\AP Note that in the case of finite words and finite trees, the precise choice
of ordering is crucial to ensure that the resulting structure is
\kl{well-quasi-ordered}. The celebrated result of Higman states that the set of
finite words over an ordered alphabet $(X, \preceq)$ is \kl{well-quasi-ordered}
by the so-called \kl{subword embedding relation}~\cite{HIG52}. Let us recall
that the \kl{subword relation} for words over $(X, \preceq)$ is defined as
follows: a word $u$ is a \intro{subword} of a word $v$, written $u
\intro*\higleq v$, if there exists an increasing function $f \colon \{1,
\ldots, |u|\} \to \{1, \ldots, |v|\}$ such that $u_i \preceq v_{f(i)}$ for all
$i \in \{1, \ldots, |u|\}$.

\begin{figure}
    \centering
    \begin{subfigure}[t]{0.48\textwidth}
    	\centering
    	\includestandalone[width=\linewidth]{fig/prefix-embedding-standalone}
    	\caption{Prefix Relation}
   	\end{subfigure}%
   	\hfill%
   	\begin{subfigure}[t]{0.48\textwidth}
   		\centering
   		\includestandalone[width=\linewidth]{fig/infix-embedding-standalone}
   		\caption{Infix Relation}
   	\end{subfigure}
   	\begin{subfigure}[t]{0.48\textwidth}
   		\centering
   		\includestandalone[width=\linewidth]{fig/subword-embedding-standalone}
   		\caption{Subword Relation}
   	\end{subfigure}
   	
   	\caption{A simple representation of the \kl{subword relation},
        \kl{prefix relation},
        and \kl{infix relation},
        on the alphabet $\set{a,b}$ for words of
        length at most $3$. The figures are Hasse Diagrams,
        representing the successor relation of the order.
        Furthermore, we highlight in dashed red relations that are added
        when moving from the prefix relation to the infix one,
        and to the infix relation to the subword one.}
    \label{word-embeddings:fig}
\end{figure}

\AP However, there are many other natural orderings on words that could be
considered in the context of \kl{well-quasi-orderings}, even in the simplified
setting of a finite alphabet $\Sigma$ equipped with the equality relation. In
this setting, the three alternatives we consider are the \intro{prefix
relation} ($u \intro*\prefleq v$ if there exists $w$ with $uw = v$), the
\intro{suffix relation} ($u \intro*\suffleq v$ if there exists $w$ such that
$wu = v$), and the \intro{infix relation} ($u \intro*\infleq v$ if there exists
$w_1,w_2$ such that $w_1 u w_2 = v$). Note that these three relations
straightforwardly generalize to infinite quasi-ordered alphabets.
Unfortunately, it is easy to see that none of these constructions are 
well-quasi-ordered as soon as the alphabet contains two distinct letters:
for instance, the infinite sequence $a^n b^n a^n$ is \kl{well-quasi-ordered} by
the \kl{subword relation} but by neither the \kl{prefix relation}, nor the
\kl{suffix relation}, nor the \kl{infix relation}.

\AP While this dooms \kl[well-quasi-ordered]{well-quasi-orderedness} of these
relations in the general case, there may be \emph{subsets} of $\Sigma^*$ which
are \kl{well-quasi-ordered} by these relations. As a simple example, take the
case of finite sets of (finite) words which are all \kl{well-quasi-ordered}
regardless of the ordering considered. This raises the question of
characterizing exactly which subsets $L \subseteq \Sigma^*$ are
\kl{well-quasi-ordered} with respect to the \kl{prefix relation} (respectively,
the \kl{suffix relation} or the \kl{infix relation}), and designing
suitable decision procedures.

\AP Let us argue that these decision procedures fit a larger picture in the
research area of well-quasi-orderings.
Indeed, there has been recent breakthroughs in deciding whether a given order
is a \kl{well-quasi-order}, for instance in the context of the verification of
infinite state transition systems~\cite{DBLP:conf/fsttcs/FinkelG19} or in the
context of logic~\cite{DBLP:journals/pacmpl/BergstrasserGLZ24}.
Furthermore, a previous work by Kuske shows that any
\emph{reasonable}\footnote{ This will be made precise in
\cref{infix-embedding:thm}. } partially ordered set $(X, \leq)$ can
be embedded into $\set{a,b}^*$ with the \kl{infix relation}~\cite[Lemma
5.1]{DBLP:journals/ita/Kuske06}. Phrased differently, one can encode a large
class of partially ordered sets as subsets of $\set{a,b}^*$. As a consequence,
the following decision problem provides a reasonable abstract framework for
deciding whether a given partially ordered set is \kl{well-quasi-ordered}:
given a language $L \subseteq \Sigma^*$, decide whether $L$ is
\kl{well-quasi-ordered} by the \kl{infix relation}.

\AP When considering an algorithm based on a \kl{well-quasi-ordering}, the
runtime of the algorithm is deeply related to the ``complexity'' of the
underlying quasi-order~\cite{SCHMITZ17}. One way to measure this complexity is
to consider its so-called \kl{ordinal invariants}: for instance, the
\kl{maximal order type} (or \kl{m.o.t.}), originally defined by De Jongh and Parikh
\cite{dejongh77}, is the order type of the maximal linearization of a
well-quasi-ordered set. In the case of a finite set, the \kl{m.o.t.} is precisely
the size of the set. Better runtime bounds were obtained by considering two
other parameters~\cite{SCHMITZ19}: the \kl{ordinal height} introduced by
Schmidt \cite{schmidt81}, and the \kl{ordinal width} of Kříž and Thomas
\cite{kriz90b}. Therefore, when characterizing \kl{well-quasi-ordered}
languages, we will also be interested in deriving upper bounds on their
\kl{ordinal invariants}. We refer to \cref{ordinal-invariants:subsec} for a
more detailed introduction to these parameters and ordinal computations in
general.

\paragraph*{Contribution} In this paper, we focus on words over a finite
alphabet $\Sigma$, and characterize subsets $L \subseteq \Sigma^*$ that are
\kl{well-quasi-ordered} by the \kl{prefix relation}, the \kl{suffix relation},
and the \kl{infix relation}. Furthermore, we devise decision algorithms
whenever the languages are given by reasonable computational models. Finally,
we provide precise bounds on the possible \kl{ordinal invariants} of such
languages.

In the case of the \kl{prefix} and \kl{suffix} relations, we show that a
language $L \subseteq \Sigma^*$ is \kl{well-quasi-ordered} by the \kl{prefix
relation} (resp. \kl{suffix}) if and only if it is a finite union of
\kl{chains} of the \kl{prefix relation} (resp. \kl{prefix}), where a
\intro{chain} is a totally ordered set that is well-founded. Note that
\kl{chains} are the simplest possible \kl{well-quasi-ordered} sets (they are
\kl{totally ordered} and \kl{well-founded}) and it is known that finite unions
of \kl{well-quasi-ordered} sets are \kl{well-quasi-ordered}. As a consequence,
the above characterization states that only the simplest possible
\kl{well-quasi-ordered} sets are \kl{well-quasi-ordered} by the \kl{prefix
relation} (\cref{prefixes:thm}). Let us furthermore highlight that this
characterization holds without any restriction on the decidability of the
language $L$ itself, but heavily relies on the assumption that $\Sigma$ is
finite. This allows us to derive tight bounds on the \kl{ordinal invariants} of
such \kl{well-quasi-ordered} languages, which can be interpreted in two dual
ways: first, these languages are extremely simple, which means that one could
have hoped for using directly another well-quasi-ordered set without resorting
to finite words; second, any time one encounters a \kl{well-quasi-ordered}
language, it is going to provide relatively efficient algorithms. Indeed, we
prove that all languages $L \subseteq \Sigma^*$ are \kl{well-quasi-ordered} by
the \kl{prefix relation} (resp. \kl{suffix relation}) have an \kl{ordinal
height} of at most $\omegaOrd$, a finite \kl{ordinal width}, and a \kl{maximal
order type} strictly below $\omegaOrd^2$
(\cref{prefixes-ordinal-invariants:cor}).

\AP The straightforward generalization of the results for the \kl{prefix
relation} and the \kl{suffix relation} to the \kl{infix relation} is not
possible. Indeed, it follows from Kuske's result that $\Sigma^*$ equipped with
the \kl{subword relation} can be embedded into $\set{a,b}^*$ with the \kl{infix
relation}~\cite[Lemma 5.1]{DBLP:journals/ita/Kuske06}. This implies that there
are \kl{well-quasi-ordered} languages for the \kl{infix relation} that have
very large \kl{ordinal invariants}: for instance, the \kl{maximal order type}
of the \kl{subword relation} is $\omegaOrd^{\omegaOrd^{\card{\Sigma} - 1}}$,
which equals its \kl{ordinal width} \cite[Corollary 3.9, Theorem
4.21]{DZSCSC20}. We show that in two situations, this can be avoided: when the
language is \kl{downwards closed} (i.e., when it is closed under taking
\kl{infixes}), and when the language is \kl(language){bounded} (i.e., 
when it is included in some $w_1^* \cdots w_k^*$ for some finite choice of
words $w_1, \ldots, w_k$). In those cases, we are able to characterize
\kl{well-quasi-ordered} languages by the \kl{infix relation} and derive tight
bounds on their \kl{ordinal invariants}.

In the case of \kl{bounded languages}, we prove that a \kl{bounded language} $L
\subseteq \Sigma^*$ is \kl{well-quasi-ordered} by the \kl{infix relation} if
and only if it is (included in) a finite union of languages $S_i \cdot P_i$,
where each $S_i$ is a \kl{chain} for the \kl{suffix relation}, and where each
$P_i$ is a \kl{chain} for the \kl{prefix relation}
(\cref{bounded-language:thm}) This result directly translates into upper bounds
on the possible \kl{ordinal invariants} of such languages similarly as for the
\kl{prefix relation}. Notice that these upper bounds are significantly smaller
than \kl{ordinal invariants} of the \kl{subword relation}: they have an \kl{ordinal
height} of at most $\omegaOrd$, an \kl{ordinal width} strictly below
$\omegaOrd^2$, and a \kl{maximal order type} strictly below $\omegaOrd^3$
(\cref{ordinal-invariants-bounded:cor}).

In the case of \kl{downwards closed} languages, we prove that they are deeply
related to the notion of \kl{uniformly recurrent words}, borrowed from the
study of word combinatorics. As an intermediate result, we prove that an
infinite word $w$ is \kl{ultimately uniformly recurrent} if and only if the set
of all \kl{infixes} of $w$ is \kl{well-quasi-ordered} by the \kl{infix
relation} (\cref{ultimately-uniformly-recurrent:lem}). We show that every
language $L \subseteq \Sigma^*$ that is \kl{downwards closed} and
\kl{well-quasi-ordered} by the \kl{infix relation} is a finite union of the
sets of finite \kl{infixes} of some \kl{ultimately uniformly recurrent}
bi-infinite words. This also proves that such languages have an \kl{ordinal
height} of at most $\omegaOrd$, an \kl{ordinal width} strictly below
$\omegaOrd^2$, and a \kl{maximal order type} strictly below $\omegaOrd^3$
(\cref{small-ordinal-invariants:thm}).

Then, we turned our attention to decision procedures. To that end, we need to
choose a computational model representing languages. For \kl{downwards closed}
languages, because of their close connection with infinite words, we considered
a model based on \kl{automatic sequences}. Using this model, we can decide
whether a language is \kl{well-quasi-ordered} by the \kl{infix relation}
(\cref{automatic-wqo:thm}). We also studied another representation, where
languages are recognized by \kl{amalgamation systems} \cite{ASZZ24}. Such
systems will be formally introduced in \cref{amalgamation-systems:subsec}, but
for the moment let us just say that they include many classical computational
models such as finite automata, context-free grammars, and Petri nets
\cite{ASZZ24}. This provides us with a \emph{meta-algorithm} for deciding
whether a given language is \kl{well-quasi-ordered} by the \kl{prefix
relation}, the \kl{suffix relation}, or the \kl{infix relation} under mild
effective restrictions on the computational model that we call an \kl{effective
amalgamative class}, the formal definition of which we defer to
\cref{infixes-amalgamation-effective:subsec}. Given a class $\mathcal{C}$ that
is a \kl{strongly effective amalgamative class} of languages, we designed a
decision procedure that takes as input a language $L \in \mathcal{C}$, and
decides whether $L$ is \kl{well-quasi-ordered} by the \kl{prefix relation}, the
\kl{suffix relation}, or the \kl{infix relation}
(\cref{infix-wqo-is-emptiness:thm}). Quite surprisingly, we also showed that,
if a language recognized by an \kl{amalgamation system} is
\kl{well-quasi-ordered} for the \kl{infix relation}, then it is a \kl{bounded
language} (\cref{infix-amalgamation:thm}), which automatically bounds the
\kl{ordinal invariants} of the language. Let us point out that the above result
implies that the hypothesis of \kl{bounded languages} on the theoretical side
is not a restriction in practice. As a down-to-earth and more easily
understandable consequence, our generic decision procedure applies to the class
$\mathcal{C}_\text{aut}$ of all languages recognized by finite automata, and to
the class $\mathcal{C}_\text{cfg}$ of all languages recognized by context-free
grammars, which are both \kl{effective amalgamative classes}
(\cref{aut-cfg-infix:cor}).

Finally, we noticed that for \kl{downwards closed} languages that are
\kl{well-quasi-ordered} by the \kl{infix relation}, being
\kl(language){bounded} is the same as being \kl(language){regular}.
Furthermore, a \kl{bounded language} is \kl{well-quasi-ordered} by the
\kl{infix relation} if and only if its \kl{downwards closure} is
\kl{well-quasi-ordered} by the \kl{infix relation}
(\cref{bounded-wqo-dwclosed:cor}). This shows that, for \kl{bounded languages}
(and therefore, for all languages recognized by \kl{amalgamation systems}) that
are \kl{well-quasi-ordered} by the \kl{infix relation}, their \kl{downwards
closure} is a \kl{regular language}. This is a weak version of the usual result
that the \kl{downwards closure} for the \kl{scattered subword relation} is
always a \kl{regular language}.


\todo[inline]{
  Add a figure to recap our complexity results: 
  bounded / downwards closed / amalgamative / morphic | width.
  And only talk about WQO classes, collapse everything and say regular.
}

\paragraph*{Related work} The study of alternative \kl{well-quasi-ordered}
relations over finite words is far from new. For instance, orders obtained by
so-called \emph{derivation relations} where already analysed by Bucher,
Ehrenfeucht, and Haussler \cite{BUEUD85}, and were later extended by
D'Alessandro and Varricchio \cite{ALVA03,ALVA06}. However, in all those cases
the orderings are \emph{multiplicative}, that is, if $u_1 \preceq v_1$ and $u_2
\preceq v_2$ then $u_1u_2 \preceq v_1v_2$. This assumption does not hold for
the \kl{prefix}, \kl{suffix}, and \kl{infix} relations.

A similar question was studied by Atminas, Lozin, and Moshkov \cite{ALM17}, in
the hope of finding characterizations of classes of \emph{finite graphs} that
are \kl{well-quasi-ordered} by the \emph{induced subgraph relation}
\cite[Section 7]{ALM17}. In this setting, it is common to refer to classes of
graphs via a list of \emph{forbidden patterns}, which are finite graphs that
cannot be found as induced subgraphs in the class. Applying this reasoning to
finite words with the \kl{infix relation}, they provide an efficient decision
procedure for checking whether a language $L \subseteq \Sigma^*$ is
\kl{well-quasi-ordered} by the \kl{infix relation} whenever said language is
given as input via a list of \emph{forbidden factors} \cite[Theorem 1, Theorem
2]{ALM17}. The key construction of their paper is to study languages $L$ that
are \emph{regular} (recognized by some finite deterministic automata), for
which they can decide whether $L$ is \kl{well-quasi-ordered} by the \kl{infix
relation} \cite[Theorem 1]{ALM17}. Because it is easy to transform a list of
forbidden factors into a regular language \cite[Theorem 1]{ALM17}, this yields
the desired decision procedure. Our work extends this result in several ways:
first, we also consider the \kl{prefix relation} and the \kl{suffix relation},
then we consider non-regular languages, and finally, we provide very precise
descriptions of the \kl{well-quasi-ordered} languages, as well as tight bounds
on their \kl{ordinal invariants}. In particular, we highlight the role of
\kl{bounded languages} as the one that are well-behaved with respect to being
\kl{well-quasi-ordering} by the \kl{infix relation} \cref{infixes-bounded:sec}.
These non-trivial extensions continue to rely
on classical word combinatorial techniques that are already present in the work
of Atminas, Lozin, and Moshkov \cite[Section 3]{ALM17}.

\paragraph*{Outline} 
We introduce in \cref{prelims:sec} the
necessary background on \kl{well-quasi-orders} and \kl{ordinal invariants}.
In
\cref{prefixes:sec}, which is relatively
self-contained, we study the \kl{prefix relation} and prove in
\cref{prefixes:thm} the characterization of \kl{well-quasi-ordered}
languages by the \kl{prefix relation}. In
\cref{infixes-bounded:sec}, we
obtain the \kl[infix relation]{infix} analogue of \cref{prefixes:thm}
specifically for \kl{bounded languages}
(\cref{bounded-language:thm}). 
In \cref{infixes-dwclosed:sec}, we study the \kl{downwards closed}
languages, and compute bounds on their \kl{ordinal invariants} in \cref{small-ordinal-invariants:thm}.
Finally, 
we generalize these results to all
\kl{amalgamation systems} in \cref{infixes-amalgamation:sec}
in
(\cref{infix-amalgamation:thm}),
and provide a decision procedure for checking whether a language is
\kl{well-quasi-ordered} by the \kl{infix relation} (resp. \kl{prefix} and \kl{suffix}) in
this context (\cref{infix-wqo-is-emptiness:thm}).

\paragraph*{Acknowledgements} We would like to thank participants of the 2024
edition of \kl{Autobóz} for their helpful comments and discussions.
We would also like to thank Vincent Jugé for his pointers on word combinatorics.

% LTeX: language=en-GB 
\section{Introduction}
\label{introduction:sec}

A \intro{well-quasi-ordered} set is a set $X$ equipped with a quasi-order
$\preceq$ such that every infinite sequence $\seqof{x_n}$ of elements taken in
$X$ contains a pair $x_i \preceq x_j$ with $i < j$. Well-quasi-orderings serve
as a core combinatorial tool powering many termination arguments, and was
successfully applied to the verification of infinite state transition systems
\cite{ABDU96,ABDU98}. One of the appealing properties of well-quasi-orderings
is that they are closed under many operations, such as taking products, finite
unions, and finite powersets \cite{SCSC12}. Perhaps more surprisingly, the
class of well-quasi-ordered sets is also stable under the operation of taking
finite words and finite trees labelled by elements of a well-quasi-ordered set
\cite{HIG52,KRU72}.

\AP
Note that in the case of finite words and finite trees, the precise choice of
ordering on words and trees is crucial to ensure that the resulting structure
is \kl{well-quasi-ordered}. The celebrated result of Higman states that the set
of finite words over an ordered alphabet $(X, \preceq)$ is
\kl{well-quasi-ordered} by the so-called \kl{subword embedding relation}
\cite{HIG52}. Let us recall that the \kl{subword relation} for words over $(X,
\preceq)$ is defined as follows: a word $u$ is a \intro{subword} of a word $v$,
written $u \intro*\higleq v$,
if there exists an increasing function $f \colon \{1, \ldots, |u|\} \to \{1,
\ldots, |v|\}$ such that $u_i \preceq v_{f(i)}$ for all $i \in \{1, \ldots,
|u|\}$.

\begin{figure}
    \centering
    \begin{tabular}{p{3cm}p{8cm}}
    Prefix Relation
    &
    \includestandalone[width=\linewidth]{fig/prefix-embedding-standalone}
    \\
    Infix relation
    &
    \includestandalone[width=\linewidth]{fig/infix-embedding-standalone}
    \\
    Subword relation
    &
    \includestandalone[width=\linewidth]{fig/subword-embedding-standalone}
    \end{tabular}
    \caption{A simple representation of the \kl{subword relation},
        \kl{prefix relation},
        and \kl{infix relation},
        on the alphabet $\set{a,b}$ for words of
        length at most $3$. The figures are Hasse Diagrams,
        representing the successor relation of the order.
        Furthermore, we highlight in red relations that are added
        when moving from the prefix relation to the infix one,
        and to the infix relation to the subword one.}
    \label{word-embeddings:fig}
\end{figure}

\AP However, there are many other natural orderings on words that could be
considered in the context of \kl{well-quasi-orderings}. Let us restrict
ourselves to a simplified case where the alphabet is finite and equipped with
the equality relation. In this setting, the three alternatives we consider are
the \intro{prefix relation} --- $u \intro*\prefleq v$ if there exists $w$ with
$uw = v$ ---, the \intro{suffix relation} --- $u \intro*\suffleq v$ if there
exists $w$ such that $wu = v$ ---, and the \intro{infix relation} --- $u
\intro*\infleq v$ if there exists $w_1,w_2$ such that $w_1 u w_2 = v$ ---. Note
that these three relations straightforwardly generalize to infinite
quasi-ordered alphabets. Unfortunately, it is an easy check that all of these
constructions are not well-quasi-ordered as soon as the alphabet contains two
non-equivalent letters. This is not the end of the road, because some
\emph{subsets} of $\Sigma^*$ might still be \kl{well-quasi-ordered} by these
relations, the simplest example being the case of finite sets of (finite) words
which are all \kl{well-quasi-ordered} regardless of the ordering considered.
This raises the question of characterizing the subsets $L \subseteq \Sigma^*$
that are \kl{well-quasi-ordered} with respect to the \kl{prefix relation}
(respectively, the \kl{suffix relation} or the \kl{infix relation}), and
providing suitable decision procedures.


\subparagraph{Contribution} \AP In this paper, we focus on words over a finite
alphabet $\Sigma$, and characterize subsets $L \subseteq \Sigma^*$ that are
\kl{well-quasi-ordered} by the \kl{prefix relation}, the \kl{suffix relation},
and the \kl{infix relation}. Let us remark that these word orderings are all
\intro{well-founded}, i.e., they do not contain infinite \intro{decreasing
sequences} $\seqof{x_n}$ such that $x_{n+1} \prec x_n$ for all $n \in \Nat$.
Let us recall that the key difference between \kl{well-quasi-orderings} and the
maybe more usual notion of well-founded orderings is that the latter allows the
existence of infinite \intro{antichains}, that is, sets of pairwise
incomparable elements. In fact, a set $X$ is \kl{well-quasi-ordered} if and
only if it is \kl{well-founded} and does not contain infinite \kl{antichains}
\cite{KRU72}.  As a consequence, our study focuses on detecting infinite
\kl{antichains} in languages $L \subseteq \Sigma^*$. Remark that for the
relations we consider (including the \kl{subword relation}), the set of all
words of a fixed size $k$ is always an \kl{antichain}, and there are therefore
no bound on the maximal length of an antichain. 

In the case of the \kl{prefix relation}, we provide a complete characterization
of languages $L \subseteq \Sigma^*$ that are \kl{well-quasi-ordered}, which are
the simplest possible languages: they are precisely the \emph{finite} unions of
\intro{chains}, that is, sets of words that are all comparable for the
\kl{prefix relation}. Note that \kl{chains} are the simplest possible
\kl{well-quasi-ordered} sets (they are totally ordered and well-founded) and it
is known that finite unions of \kl{well-quasi-ordered} sets are
\kl{well-quasi-ordered}. As a consequence, the above characterization states
that only the simplest possible \kl{well-quasi-ordered} sets are
\kl{well-quasi-ordered} by the \kl{prefix relation}. Let us furthermore
highlight that this characterization holds without any restriction on the
decidability of the language $L$ itself, but heavily relies on the assumption
that $\Sigma$ is finite. By symmetry, we obtain an analogue statement for the
\kl{suffix relation}.

In the case of the \kl{infix relation}, the situation is more complex, and we
show that \emph{any} countable quasi-ordering with finite initial segments can
be \kl{embedded} into the \kl{infix relation} of a language over a binary
alphabet $\Sigma \defined \set{a,b}$. As an example, on can identify
$(\Sigma^*, \higleq)$ with a subset of $(\Sigma^*, \infleq)$ where $\higleq$ is
the \kl{subword relation} and $\infleq$ is the \kl{infix relation}. This
strongly contrasts with the \kl{prefix relation}, and suggests that a
meaningful characterization of \kl{well-quasi-ordered} languages under the
\kl{infix relation} is unlikely to be found. 

We circumvent this issue by adding some regularity assumptions on the languages
$L \subseteq \Sigma^*$ that we consider. This can be done by assuming that $L$
is recognized by a finite automaton, or more generally that $L$ is the language
of an \kl{amalgamation system}, a recently introduced notion that abstracts the
notion of systems whose runs can be suitably ``pumped'' \cite{ASZZ24}. As
examples \kl{amalgamation systems} covers the case of regular languages,
context free languages, or even languages recognized by ``vector addition
systems with state" \cite{ASZZ24}. Under this relatively mild assumption, we
also prove that \kl{well-quasi-ordered} languages under the \kl{infix relation}
are precisely the finite unions of \kl{chains}.

\subparagraph{Related work} The study of alternative \kl{well-quasi-ordered}
relations over finite words is far from new. For instance, orders obtained by
so-called \emph{derivation relations} where already analysed by Bucher,
Ehrenfeucht, and Haussler \cite{BUEUD85}, and were later extended by
D'Alessandro and Varricchio \cite{ALVA03,ALVA06}. However, in all those cases
the orderings are \emph{multiplicative}, that is, if $u_1 \preceq v_1$ and $u_2
\preceq v_2$ then $u_1u_2 \preceq v_1v_2$. This assumption does not hold for
the \kl{prefix}, \kl{suffix}, and \kl{infix} relations.

\subparagraph{Outline} In \cref{prefixes:sec}, which is relatively
self-contained, we study the \kl{prefix relation} prove in \cref{prefixes:thm}
the characterization of \kl{well-quasi-ordered} languages by the \kl{prefix
relation}. In \cref{infixes-regular:sec}, we prove that any countable quasi-ordering
with finite initial segments can be embedded into the \kl{infix relation} of a
language (\cref{infix-embedding:thm}), we obtain the analogue of
\cref{prefixes:thm} for languages recognized by finite automata
(\cref{infix-finite-automata:thm}). Finally, in \cref{infixes-amalgamation:sec}, we
introduce \kl{amalgamation systems} and generalize
\cref{infix-finite-automata:thm} to this setting
(\cref{infix-amalgamation:thm}).

\subparagraph{Acknowledgements} We would like to thank participants of the 2024
edition of \kl{Autobóz} for their helpful comments and discussions.

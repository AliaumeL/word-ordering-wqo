% LTeX: language=en-GB 
\section{Introduction}
\label{introduction:sec}

A \intro{well-quasi-ordered} set is a set $X$ equipped with a quasi-order
$\preceq$ such that every infinite sequence $\seqof{x_n}$ of elements taken in
$X$ contains a pair $x_i \preceq x_j$ with $i < j$. Well-quasi-orderings serve
as a core combinatorial tool powering many termination arguments, and was
successfully applied to the verification of infinite state transition systems
\cite{ABDU96,ABDU98}. One of the appealing properties of well-quasi-orderings
is that they are closed under many operations, such as taking products, finite
unions, and finite powersets \cite{SCSC12}. Perhaps more surprisingly, the
class of well-quasi-ordered sets is also stable under the operation of taking
finite words and finite trees labelled by elements of a well-quasi-ordered set
\cite{HIG52,KRU72}.

\AP Note that in the case of finite words and finite trees, the precise choice
of ordering is crucial to ensure that the resulting structure is
\kl{well-quasi-ordered}. The celebrated result of Higman states that the set of
finite words over an ordered alphabet $(X, \preceq)$ is \kl{well-quasi-ordered}
by the so-called \kl{subword embedding relation} \cite{HIG52}. Let us recall
that the \kl{subword relation} for words over $(X, \preceq)$ is defined as
follows: a word $u$ is a \intro{subword} of a word $v$, written $u
\intro*\higleq v$, if there exists an increasing function $f \colon \{1,
\ldots, |u|\} \to \{1, \ldots, |v|\}$ such that $u_i \preceq v_{f(i)}$ for all
$i \in \{1, \ldots, |u|\}$.

\begin{figure}
    \centering
    \begin{subfigure}[t]{0.48\textwidth}
    	\centering
    	\includestandalone[width=\linewidth]{fig/prefix-embedding-standalone}
    	\caption{Prefix Relation}
   	\end{subfigure}%
   	\hfill%
   	\begin{subfigure}[t]{0.48\textwidth}
   		\centering
   		\includestandalone[width=\linewidth]{fig/infix-embedding-standalone}
   		\caption{Infix Relation}
   	\end{subfigure}
   	\begin{subfigure}[t]{0.48\textwidth}
   		\centering
   		\includestandalone[width=\linewidth]{fig/subword-embedding-standalone}
   		\caption{Subword Relation}
   	\end{subfigure}
   	
   	\caption{A simple representation of the \kl{subword relation},
        \kl{prefix relation},
        and \kl{infix relation},
        on the alphabet $\set{a,b}$ for words of
        length at most $3$. The figures are Hasse Diagrams,
        representing the successor relation of the order.
        Furthermore, we highlight in red relations that are added
        when moving from the prefix relation to the infix one,
        and to the infix relation to the subword one.}
    \label{word-embeddings:fig}
\end{figure}

\AP However, there are many other natural orderings on words that could be
considered in the context of \kl{well-quasi-orderings}. Let us restrict
ourselves to a simplified case where the alphabet is finite and equipped with
the equality relation. In this setting, the three alternatives we consider are
the \intro{prefix relation} ($u \intro*\prefleq v$ if there exists $w$ with $uw
= v$), the \intro{suffix relation} ($u \intro*\suffleq v$ if there exists $w$
such that $wu = v$), and the \intro{infix relation} ($u \intro*\infleq v$ if
there exists $w_1,w_2$ such that $w_1 u w_2 = v$). Note that these three
relations straightforwardly generalize to infinite quasi-ordered alphabets.
Unfortunately, it is easy to see that all of these constructions are not
well-quasi-ordered as soon as the alphabet contains two non-equivalent letters:
for instance, the infinite sequence $a^n b^n a^n$ is \kl{well-quasi-ordered} by
the \kl{subword relation} but by neither the \kl{prefix relation}, nor the
\kl{suffix relation}, nor the \kl{infix relation}.


\AP While this dooms \kl[well-quasi-ordered]{well-quasi-orderedness} of these relations in the general case, there may be
\emph{subsets} of $\Sigma^*$ which are \kl{well-quasi-ordered} by these
relations. As a simple example, take the case of finite sets of (finite) words
which are all \kl{well-quasi-ordered} regardless of the ordering considered.
This raises the question of characterizing exactly which subsets $L \subseteq \Sigma^*$
are \kl{well-quasi-ordered} with respect to the \kl{prefix relation}
(respectively, the \kl{suffix relation} or the \kl{infix relation}), and
providing suitable decision procedures.



\subparagraph{Contribution} In this paper, we focus on words over a finite
alphabet $\Sigma$, and characterize subsets $L \subseteq \Sigma^*$ that are
\kl{well-quasi-ordered} by the \kl{prefix relation}, the \kl{suffix relation},
and the \kl{infix relation}. Specifically, for the \kl{prefix relation} and the \kl{suffix relation}, we are able to show the following decomposition into \intro{chains}, that is, totally ordered ascending sequences.

{
\renewcommand{\labelenumi}{R\arabic{enumi}}
\begin{enumerate}
	\item A language $L \subseteq \Sigma^*$ is \kl{well-quasi-ordered} by the \kl{prefix relation} if and only if it is a finite union of \kl{chains} of the \kl{prefix relation}.
	\item A language $L \subseteq \Sigma^*$ is \kl{well-quasi-ordered} by the \kl{suffix relation} if and only if it is a finite union of \kl{chains} of the \kl{suffix relation}.
\end{enumerate}
}

Note that \kl{chains} are the simplest possible
\kl{well-quasi-ordered} sets (they are totally ordered and well-founded) and it
is known that finite unions of \kl{well-quasi-ordered} sets are
\kl{well-quasi-ordered}. As a consequence, the above characterization states
that only the simplest possible \kl{well-quasi-ordered} sets are
\kl{well-quasi-ordered} by the \kl{prefix relation}. Let us furthermore
highlight that this characterization holds without any restriction on the
decidability of the language $L$ itself, but heavily relies on the assumption
that $\Sigma$ is finite. 

\AP One way to measure the complexity of a \kl{well-quasi-ordered} set is to
consider its so-called \intro{ordinal invariants}. Several of these ordinal
measures have been developed for well-quasi-orders. The \intro{maximal order
type} (or m.o.t.) was originally defined by \cite{dejongh77} as the order type
of the maximal linearization of a well-quasi-ordered set. Then,
\cite{schmidt81} introduced the \intro{ordinal height} as the order type of a
maximal chain of a well-quasi-ordered set. Finally, \cite{kriz90b} crafted the
notion of \intro{ordinal width}, standardized the definitions of these three
\kl{ordinal invariants}, and proved numerical relationships between them. For
the \kl{prefix relation}, the \kl{suffix relation}, and the \kl{infix
relation}, the \kl{ordinal height} is always bounded by $\omega$, and the key
parameter to describe its complexity will be the \kl{ordinal width}, i.e.,
``how large the antichains are.''

\todo{rephrase}
\AP In the case of the \kl{prefix relation} and the \kl{suffix relation}, we
conclude from our characterization that the \kl{ordinal width} of a language
$L$ is either finite or $L$ is not \kl{well-quasi-ordered}. The straightforward
generalization of the results for the \kl{prefix relation} and the \kl{suffix
relation} to the \kl{infix relation} is not possible. For example, Kuske has
shown that any countable partial order with finite downward closures can be
embedded into it~\cite{DBLP:journals/ita/Kuske06}. In particular, this means
that the \kl{subword relation} can be embedded into the \kl{infix relation}.
The former being a \kl{well-quasi-ordered} with a very large \kl{ordinal width}
(see for example https://arxiv.org/pdf/2312.14587), this implies that such
"easy characterizations" will not be possible in general. However, not all is
lost: we show that with some mild restrictions we can recover much of the
simple structure of the \kl[prefix relation]{prefix} and \kl{suffix relation}:

{
\renewcommand{\labelenumi}{R\arabic{enumi}}
\begin{enumerate}
	\setcounter{enumi}{2}
    \item A \emph{bounded} language $L \subseteq \Sigma^*$ is \kl{well-quasi-ordered} by the \kl{infix relation} if and only if it is a finite union of languages $S_i \cdot P_i$, where each $S_i$ is \kl{well-quasi-ordered} by the \kl{suffix relation} and each $P_i$ is \kl{well-quasi-ordered} by the \kl{prefix relation}.
\end{enumerate}
}

We illustrate how this result can be used to study languages recognized by
finite automata, context-free grammars, petri nets, and other systems. This is
possible by leveraging the recently introduced notion of \kl{amalgamation
systems} \cite{ASZZ24}.

{
\renewcommand{\labelenumi}{R\arabic{enumi}}
\begin{enumerate}
	\setcounter{enumi}{2}
	\item A language $L \subseteq \Sigma^*$ recognized by an \kl{amalgamation system} is \kl{well-quasi-ordered} by the \kl{infix relation} if and only if it is a finite union of languages $S_i \cdot P_i$, where each $S_i$ is \kl{well-quasi-ordered} by the \kl{suffix relation} and each $P_i$ is \kl{well-quasi-ordered} by the \kl{prefix relation}.
\end{enumerate}
}


Finally, there has been recent interest in deciding whether a given order is a
well-quasi-order. These questions have been investigated for example in the
context of the verification of counter
machines~\cite{DBLP:conf/fsttcs/FinkelG19} and in the context of
logic~\cite{DBLP:journals/pacmpl/BergstrasserGLZ24}. We conclude our results
with a contribution in this continuity.

\todo{verify this result :)}
{
\renewcommand{\labelenumi}{R\arabic{enumi}}
\begin{enumerate}
	\setcounter{enumi}{3}
	\item For languages of amalgamation systems, it is decidable whether they are \kl{well-quasi-ordered} by the \kl[prefix relation]{prefix}, \kl[suffix relation]{suffix}, or \kl{infix relation}.
\end{enumerate}
}

\subparagraph{Related work} The study of alternative \kl{well-quasi-ordered}
relations over finite words is far from new. For instance, orders obtained by
so-called \emph{derivation relations} where already analysed by Bucher,
Ehrenfeucht, and Haussler \cite{BUEUD85}, and were later extended by
D'Alessandro and Varricchio \cite{ALVA03,ALVA06}. However, in all those cases
the orderings are \emph{multiplicative}, that is, if $u_1 \preceq v_1$ and $u_2
\preceq v_2$ then $u_1u_2 \preceq v_1v_2$. This assumption does not hold for
the \kl{prefix}, \kl{suffix}, and \kl{infix} relations.

\subparagraph{Outline} In
\cref{prefixes:sec}, which is relatively
self-contained, we study the \kl{prefix relation} and prove in
\cref{prefixes:thm} the characterization of \kl{well-quasi-ordered}
languages by the \kl{prefix relation}. In
\cref{infixes-regular:sec}, we
obtain the \kl[infix relation]{infix} analogue of \cref{prefixes:thm}
specifically for regular languages
(\cref{infix-finite-automata:thm}). Finally, we generalize this result to all
\kl{amalgamation systems} in \cref{infixes-amalgamation:sec}
in
(\cref{infix-amalgamation:thm}).

\subparagraph{Acknowledgements} We would like to thank participants of the 2024
edition of \kl{Autobóz} for their helpful comments and discussions.

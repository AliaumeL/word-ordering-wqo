% LTeX: language=en-GB 
\section{Introduction}
\label{introduction:sec}

A \intro{well-quasi-ordered} set is a set $X$ equipped with a quasi-order
$\preceq$ such that every infinite sequence $x_0, x_1, x_2 , \ldots$ contains a
pair $x_i \preceq x_j$ with $i < j$. Well-quasi-orderings serve as a core
combinatorial tool powering many termination arguments, and in particular
applies to the verification of infinite state transition systems
\cite{ABDU96,ABDU98}. 
In this paper, we will use an equivalent definition of
well-quasi-orderings based on the non-existence of infinite \intro{antichains}
-- sets of pairwise incomparable elements -- as well as the non-existence of
infinite \intro{decreasing sequences} -- sequences of elements $x_0, x_1, x_2,
\ldots$ such that $x_{i+1} \prec x_{i}$ --. 


One of the appealing properties of well-quasi-orderings is that they are closed
under many operations, such as taking products, finite unions, and finite
powersets \cite{SCSC12}. More surprisingly, the class of well-quasi-ordered
sets is also stable under the operation of taking finite words and finite trees
labelled by elements of a well-quasi-ordered set \cite{HIG52,KRU72}. 

Note that in these latter cases, the precise choice of ordering on words and
trees is crucial to ensure that the resulting structure is well-quasi-ordered.
A celebrated result of Higman implies that the set of finite words over a
finite alphabet is well-quasi-ordered by the so-called \kl{subword relation}
\cite{HIG52}. The subword relation for words over $(X, \preceq)$ is defined as
follows: a word $u$ is a \intro{subword} of a word $v$ if there exists an
increasing function $f \colon \{1, \ldots, |u|\} \to \{1, \ldots, |v|\}$ such
that $u_i \preceq v_{f(i)}$ for all $i \in \{1, \ldots, |u|\}$.

There are many other natural orderings on words that could be considered in the
context of well-quasi-orderings. For instance, one could consider the
\kl{prefix relation}, \kl{suffix relation} or the \kl{infix relation}, all of
which are not well-quasi-ordered as soon as the alphabet contains two distinct
letters. Let us remark that these other word orderings are however
well-founded, and therefore the (only) obstruction to being well-quasi-ordered
is not due to the existence of infinite sets of incomparable elements.


\subparagraph{Contribution} In this paper, we characterize for which languages
$L \subseteq \Sigma^*$ the other natural orderings on word are
well-quasi-ordered. When languages are decided by simple enough machines (e.g.,
finite automata or pushdown automata), we will also provide \emph{decision
procedures} to check whether the language is well-quasi-ordered by the
considered relation. Quite surprisingly, for the \kl{prefix relation} and the
\kl{suffix relation}, the only languages that are \kl{well-quasi-ordered} are
the \emph{finite} unions of \kl{chains}, that is, the simplest possible
well-quasi-ordered sets. 

\subparagraph{Outline}

% LTeX: language=en-GB 
\section{Introduction}
\label{introduction:sec}

\AP A \intro{well-quasi-ordered} set is a set $X$ equipped with a quasi-order
$\preceq$ such that every infinite sequence $\seqof{x_n}$ of elements taken in
$X$ contains a pair $x_i \preceq x_j$ with $i < j$. Well-quasi-orderings serve
as a core combinatorial tool powering many termination arguments, and was
successfully applied to the verification of infinite state transition
systems~\cite{ABDU96,ABDU98}. One of the appealing properties of
well-quasi-orderings is that they are closed under many operations, such as
taking products, finite unions, and finite powerset
constructions~\cite{SCSC12}. Perhaps more surprisingly, the class of
well-quasi-ordered sets is also stable under the operation of taking finite
words and finite trees labelled by elements of a well-quasi-ordered set
\cite{HIG52,KRU72}.

\AP Note that in the case of finite words and finite trees, the precise choice
of ordering is crucial to ensure that the resulting structure is
\kl{well-quasi-ordered}. The celebrated result of Higman states that the set of
finite words over an ordered alphabet $(X, \preceq)$ is \kl{well-quasi-ordered}
by the so-called \kl{subword embedding relation}~\cite{HIG52}. Let us recall
that the \kl{subword relation} for words over $(X, \preceq)$ is defined as
follows: a word $u$ is a \intro{subword} of a word $v$, written $u
\intro*\higleq v$, if there exists an increasing function $f \colon \{1,
\ldots, |u|\} \to \{1, \ldots, |v|\}$ such that $u_i \preceq v_{f(i)}$ for all
$i \in \{1, \ldots, |u|\}$.

\begin{figure}
    \centering
    \begin{subfigure}[t]{0.48\textwidth}
    	\centering
    	\includestandalone[width=\linewidth]{fig/prefix-embedding-standalone}
    	\caption{Prefix Relation}
   	\end{subfigure}%
   	\hfill%
   	\begin{subfigure}[t]{0.48\textwidth}
   		\centering
   		\includestandalone[width=\linewidth]{fig/infix-embedding-standalone}
   		\caption{Infix Relation}
   	\end{subfigure}
   	\begin{subfigure}[t]{0.48\textwidth}
   		\centering
   		\includestandalone[width=\linewidth]{fig/subword-embedding-standalone}
   		\caption{Subword Relation}
   	\end{subfigure}
   	
   	\caption{A simple representation of the \kl{subword relation},
        \kl{prefix relation},
        and \kl{infix relation},
        on the alphabet $\set{a,b}$ for words of
        length at most $3$. The figures are Hasse Diagrams,
        representing the successor relation of the order.
        Furthermore, we highlight in red relations that are added
        when moving from the prefix relation to the infix one,
        and to the infix relation to the subword one.}
    \label{word-embeddings:fig}
\end{figure}

\AP However, there are many other natural orderings on words that could be
considered in the context of \kl{well-quasi-orderings}, even in the simplified
setting of a finite alphabet $\Sigma$ equipped with the equality relation. In
this setting, the three alternatives we consider are the \intro{prefix
relation} ($u \intro*\prefleq v$ if there exists $w$ with $uw = v$), the
\intro{suffix relation} ($u \intro*\suffleq v$ if there exists $w$ such that
$wu = v$), and the \intro{infix relation} ($u \intro*\infleq v$ if there exists
$w_1,w_2$ such that $w_1 u w_2 = v$). Note that these three relations
straightforwardly generalize to infinite quasi-ordered alphabets.
Unfortunately, it is easy to see that all of these constructions are not
well-quasi-ordered as soon as the alphabet contains two distinct letters:
for instance, the infinite sequence $a^n b^n a^n$ is \kl{well-quasi-ordered} by
the \kl{subword relation} but by neither the \kl{prefix relation}, nor the
\kl{suffix relation}, nor the \kl{infix relation}.

\AP While this dooms \kl[well-quasi-ordered]{well-quasi-orderedness} of these
relations in the general case, there may be \emph{subsets} of $\Sigma^*$ which
are \kl{well-quasi-ordered} by these relations. As a simple example, take the
case of finite sets of (finite) words which are all \kl{well-quasi-ordered}
regardless of the ordering considered. This raises the question of
characterizing exactly which subsets $L \subseteq \Sigma^*$ are
\kl{well-quasi-ordered} with respect to the \kl{prefix relation} (respectively,
the \kl{suffix relation} or the \kl{infix relation}), and together with
suitable decision procedures.

\AP Let us argue that this question is not only of theoretical interest.
Indeed, there has been recent breakthroughs in deciding whether a given order
is a \kl{well-quasi-order}, for instance in the context of the verification of
infinite state transition systems~\cite{DBLP:conf/fsttcs/FinkelG19} or in the
context of logic~\cite{DBLP:journals/pacmpl/BergstrasserGLZ24}. Our decision
procedures clearly contribute to this line of research in the context of words.
Furthermore, a previous work by Kuske shows that any
\emph{reasonable}\footnote{ This will be made precise in
\cref{infix-embedding:thm}. } partially ordered set $(X, \leq)$ can
be embedded into $\set{a,b}^*$ with the \kl{infix relation}~\cite[Lemma
5.1]{DBLP:journals/ita/Kuske06}. Phrased differently, one can encode a large
class of partially ordered sets as subsets of $\set{a,b}^*$. As a consequence,
the following decision problem provides a reasonable abstract framework for
deciding whether a given partially ordered set is \kl{well-quasi-ordered}:
given a language $L \subseteq \Sigma^*$, decide whether $L$ is
\kl{well-quasi-ordered} by the \kl{suffix relation}, or the \kl{infix
relation}.

\AP When considering an algorithm based on a \kl{well-quasi-ordering}, the
runtime of the algorithm is deeply related to the ``complexity'' of the
underlying quasi-order \todo{CITE???}. One way to measure this complexity is to
consider its so-called \intro{ordinal invariants}: for instance, the
\intro{maximal order type} (or m.o.t.), originally defined by De Jongh and
Parikh \cite{dejongh77}, is the order type of the maximal linearization of a
well-quasi-ordered set. In the case of a finite set, the m.o.t. is precisely
the size of the set. Better runtime bounds were obtained by considering two
other parameters\todo{cite sylvain width/ HDR?}: the \intro{ordinal height}
introduced by Schmidt \cite{schmidt81}, and the \intro{ordinal width} of Kříž
and Thomas \cite{kriz90b}. Therefore, when characterizing
\kl{well-quasi-ordered} languages, we will also be interested in computing
bounds on their \kl{ordinal invariants}. We refer to the
\cref{ordinal-invariants:subsec} for a more detailed introduction to these
parameters and ordinal computations in general. For the moment, let us just
remark that all languages $L \subseteq \Sigma^*$ have an \kl{ordinal height} of
at most $\omega$.

\subparagraph{Contribution} In this paper, we focus on words over a finite
alphabet $\Sigma$, and characterize subsets $L \subseteq \Sigma^*$ that are
\kl{well-quasi-ordered} by the \kl{prefix relation}, the \kl{suffix relation},
and the \kl{infix relation}. Furthermore, we devise decision algorithms
whenever the languages are given by reasonable computational models. Finally,
we provide relatively low bounds on the possible \kl{ordinal invariants} of
such languages.

Specifically, for the \kl{prefix relation} and the
\kl{suffix relation}, we are able to show the following decomposition into
\intro{chains}, that is, totally ordered ascending sequences.

{
\renewcommand{\labelenumi}{R\arabic{enumi}}
\begin{enumerate}
	\item A language $L \subseteq \Sigma^*$ is \kl{well-quasi-ordered} by the \kl{prefix relation} if and only if it is a finite union of \kl{chains} of the \kl{prefix relation}.
    \hfill (\cref{prefixes:thm} p.\pageref{prefixes:thm})
	\item A language $L \subseteq \Sigma^*$ is \kl{well-quasi-ordered} by the \kl{suffix relation} if and only if it is a finite union of \kl{chains} of the \kl{suffix relation}.
    \hfill (\cref{prefixes:thm} p.\pageref{prefixes:thm})
\end{enumerate}
}

Note that \kl{chains} are the simplest possible \kl{well-quasi-ordered} sets
(they are totally ordered and well-founded) and it is known that finite unions
of \kl{well-quasi-ordered} sets are \kl{well-quasi-ordered}. As a consequence,
the above characterization states that only the simplest possible
\kl{well-quasi-ordered} sets are \kl{well-quasi-ordered} by the \kl{prefix
relation}. Let us furthermore highlight that this characterization holds
without any restriction on the decidability of the language $L$ itself, but
heavily relies on the assumption that $\Sigma$ is finite. This allows us to
derive the following bounds on the \kl{ordinal invariants} of such
\kl{well-quasi-ordered} languages, which can be interpreted in two dual ways:
first, these languages are extremely simple, which means that one could have
hoped for using directly another well-quasi-ordered set without resorting to
finite words; second, any time one encounters a \kl{well-quasi-ordered}
language, it is going to provide relatively efficient algorithms.
{
\renewcommand{\labelenumi}{R\arabic{enumi}}
\begin{enumerate}
    \setcounter{enumi}{2}
    \item All languages $L \subseteq \Sigma^*$ are 
        \kl{well-quasi-ordered} by the \kl{prefix relation} 
        (resp. \kl{suffix relation}) have an \kl{ordinal height} of at most $\omega$,
        a finite \kl{ordinal width}, and a \kl{maximal order type} strictly
        below $\omega^2$.
        \hfill (\cref{prefixes-ordinal-invariants:cor} p.\pageref{prefixes-ordinal-invariants:cor})
\end{enumerate}
}



\AP The straightforward generalization of the results for the \kl{prefix
relation} and the \kl{suffix relation} to the \kl{infix relation} is not
possible. Indeed, it follows from Kuske's result that $\Sigma^*$ equipped with
the \kl{subword relation} can be embedded into $\set{a,b}^*$ with the \kl{infix
relation}~\cite[Lemma 5.1]{DBLP:journals/ita/Kuske06}. This implies that there
are \kl{well-quasi-ordered} languages for the \kl{infix relation} that have
very large \kl{ordinal invariants}: for instance, the \kl{maximal order type}
of the \kl{subword relation} is $\omega^{\omega^{\card{\Sigma} - 1}}$, which
equals its \kl{ordinal width} \cite[Corollary 3.9, Theorem 4.21]{DZSCSC20}.
However, not all is lost if we restrict ourselves to \kl{bounded languages},
i.e., languages that are included in some $w_1^* \cdots w_k^*$ for some finite
choice of words $w_1, \ldots, w_k$. We are able to show the following:

{
\renewcommand{\labelenumi}{R\arabic{enumi}}
\begin{enumerate}
	\setcounter{enumi}{3}
    \item A \kl{bounded language} $L \subseteq \Sigma^*$ is \kl{well-quasi-ordered} by the \kl{infix relation} if and only if it is a finite union of languages $S_i \cdot P_i$, where each $S_i$ 
        is a \kl{chain} for the \kl{suffix relation},
        and where
        each $P_i$ is a \kl{chain} for the \kl{prefix relation}.
\end{enumerate}
}

This result directly translates into bounds on the possible \kl{ordinal invariants}
of such languages similarly as for the \kl{prefix relation}.
{
\renewcommand{\labelenumi}{R\arabic{enumi}}
\begin{enumerate}
    \setcounter{enumi}{4}
    \item All \kl{bounded languages}
            $L \subseteq \Sigma^*$ that are \kl{well-quasi-ordered} by the \kl{infix relation}
            have an \kl{ordinal height} of at most $\omega$,
            an \kl{ordinal width} strictly below $\omega^2$, and
            a \kl{maximal order type} strictly below $\omega^3$.
\end{enumerate}
}

\AP
Finally, we provide a decision procedure for checking whether a language is
\kl{well-quasi-ordered}, which takes as inputs languages recognized
by \kl{amalgamation systems} \cite{ASZZ24}. Such systems will be
formally introduced in \cref{amalgamation-systems:subsec}, but for the moment
let us just say that they include many classical computational models such as
finite automata, context-free grammars, and petri nets \cite{ASZZ24}. 
{
\renewcommand{\labelenumi}{R\arabic{enumi}}
\begin{enumerate}
    \setcounter{enumi}{5}
    \item There is a decision procedure for checking whether a language $L$ recognized by an \kl{amalgamation system} is \kl{well-quasi-ordered} by the \kl{prefix relation}, the \kl{suffix relation}, or the \kl{infix relation}. Furthermore, if it is \kl{well-quasi-ordered}, then $L$ is a \kl{bounded language}.
\end{enumerate}
}
Notice that the above result implies that the hypothesis of
\kl{bounded languages} on the theoretical side is not a restriction at all in
practice.


\subparagraph{Related work} The study of alternative \kl{well-quasi-ordered}
relations over finite words is far from new. For instance, orders obtained by
so-called \emph{derivation relations} where already analysed by Bucher,
Ehrenfeucht, and Haussler \cite{BUEUD85}, and were later extended by
D'Alessandro and Varricchio \cite{ALVA03,ALVA06}. However, in all those cases
the orderings are \emph{multiplicative}, that is, if $u_1 \preceq v_1$ and $u_2
\preceq v_2$ then $u_1u_2 \preceq v_1v_2$. This assumption does not hold for
the \kl{prefix}, \kl{suffix}, and \kl{infix} relations.

\subparagraph{Outline} We introduce in \cref{prelims:sec} the
necessary background on \kl{well-quasi-orders} and \kl{ordinal invariants}.
In
\cref{prefixes:sec}, which is relatively
self-contained, we study the \kl{prefix relation} and prove in
\cref{prefixes:thm} the characterization of \kl{well-quasi-ordered}
languages by the \kl{prefix relation}. In
\cref{infixes-regular:sec}, we
obtain the \kl[infix relation]{infix} analogue of \cref{prefixes:thm}
specifically for regular languages
(\cref{infix-finite-automata:thm}). Finally, we generalize this result to all
\kl{amalgamation systems} in \cref{infixes-amalgamation:sec}
in
(\cref{infix-amalgamation:thm}).

\subparagraph{Acknowledgements} We would like to thank participants of the 2024
edition of \kl{Autobóz} for their helpful comments and discussions.
